% Presentation for dissertation defense on October 4, 2011 at Howard
% University
% author: John H. Johnson
% email: john.j.jr.gmail.com
% started: September 29, 2011

\documentclass{beamer}

\mode<presentation>{
    \usetheme{Warsaw}
    \setbeamercovered{transparent}
}

\title{Some Differences Between an Ideal in the Stone-\v{C}ech Compactification of Commutative and Noncommutative Semigroups}
\author{John H.~Johnson}
\institute{
  Howard University\\
  Washington, DC
}

\date{
  Dissertation Defense\\
  October 4, 2011
}

\newcommand{\la}{\langle}
\newcommand{\ra}{\rangle}
\newcommand{\bbN}{\mathbb{N}}
\newcommand{\bbZ}{\mathbb{Z}}
\newcommand{\calG}{\mathcal{G}}
\newcommand{\calI}{\mathcal{I}}
\newcommand{\calJ}{\mathcal{J}}
\newcommand{\calP}{\mathcal{P}}
\newcommand{\calT}{\mathcal{T}}
\newcommand{\Pf}{\mathcal{P}_f}

\font\bigmath=cmsy10 scaled \magstep 3
\newcommand{\bigtimes}{\hbox{\bigmath \char'2}}


\begin{document}

\begin{frame}
  \titlepage
\end{frame}

\section{Basic Semigroup Theory and Ramsey Theory}
\begin{frame}
  \begin{definition}
    A pair $(S, \cdot)$ is a \alert{semigroup} iff $S$ is a nonempty set and $\cdot$ is a binary operation on $S$ such that for all $x$, $y$, and $z \in S$, 
    \[
      x\cdot ( y \cdot z) = (x \cdot y) \cdot z.
    \]
  \end{definition}
  
  \pause

  \begin{definition}
    A semigroup $S$ is \alert{commutative} iff for all $x$ and $y \in S$, $x \cdot y = y \cdot x$. 
  \end{definition}

  \pause
  
  \begin{definition}
    Let $\emptyset \ne J \subseteq S$. 
    We say $J$ is a \alert{left ideal of $S$} iff $S\cdot J \subseteq J$.
    \pause
    We say $J$ is a \alert{right ideal of $S$} iff $J\cdot S \subseteq J$.
    \pause
    We say $J$ is a \alert{(two-sided) ideal of $S$} iff $J$ both a left and right ideal. 
  \end{definition}  
\end{frame}

\begin{frame}
  \begin{block}{Prototypical Statement}
    Let $(S, \cdot)$ be a semigroup with property $P$ and let $r$ be a positive integer.
    If $S = \bigcup_{i=1}^r C_i$, then there exists $i \in \{1, 2, \ldots, r\}$ such that $C_i$ contains a set with property $P$. 
  \end{block}

  \pause

  \begin{block}{Rough Statement}
    Studying such ``partition regular'' sets is equivalent to studying
    certain ultrafilters on $S$. 
  \end{block}

  \pause

  \begin{block}{Algebra in the Stone-\v{C}ech Compactification}
    Given a discrete semigroup $S$, we let $\beta S$ be the collection
    of all ultrafilters on $S$.
    We can extend the semigroup operation on $S$ to turn $\beta S$
    into a compact Hausdorff right-topological semigroup. 
  \end{block}
\end{frame}

\section{Van der Waerden's Theorem}
\begin{frame}
  \begin{block}{Van der Waerden's Theorem}
    \textsl{Let $r$ be a positive integer.
    If $\bbN = \bigcup_{i=1}^r C_i$, then there exists $i \in \{1, 2, \ldots, r\}$ such that for all $\ell \in \bbN$, there exist $a$, $d \in \bbN$ with $\{\, a, a+d, \ldots, a + \ell d \,\} \subseteq
    C_i$.} 
  \end{block}

  \pause

  \begin{definition}
    \vspace{-1em}
    \begin{align*}
      \mathcal{AP} &= \{\, p \in \beta \bbN : \mbox{for all $A \in p$, $A$ contains an} \\
      &\hspace{5em}\mbox{ $\ell$-term arithmetic progression for every $\ell \in \bbN$} \,\}.
    \end{align*}
  \end{definition}

  \pause

  \begin{block}{General Facts}
    \begin{itemize}
      \item Van der Waerden's Theorem implies that $\mathcal{AP} \ne  \emptyset$. 
        \pause

      \item $\mathcal{AP}$ is closed in $\beta\bbN$. 
        \pause

      \item $\mathcal{AP}$ is a two-sided ideal, that is, $\beta\bbN +       \mathcal{AP} \subseteq \mathcal{AP}$ and $\mathcal{AP} + \beta\bbN \subseteq \mathcal{AP}$.
    \end{itemize}
  \end{block}
\end{frame}

\section{VDW in Semigroups}
\begin{frame}
  \begin{theorem}
    Let $r \in \bbN$ and $S$ a semigroup.
    If $S = \bigcup_{i=1}^r C_i$, then there exists $i \in \{1, 2,
    \ldots, r \}$ such that for all $\ell \in \bbN$, there exist $a$,
    $d \in S$ with $\{\, a, ad, \ldots, ad^\ell \,\} \subseteq C_i$. 
  \end{theorem}

  \pause

  \begin{definition}
    Let $S$ be a semigroup and $\ell \in \bbN$. 
    If $a$, $d \in S$, we call the set $\{\, a, ad, \ldots, ad^\ell
    \,\}$ an \textsl{\mbox{$\ell$-term} algebraic line}.
  \end{definition}
\end{frame}

\begin{frame}
  \begin{definition}
    Let $S$ be a semigroup and $\ell \in \bbN$. 
    If $a$, $d \in S$, we call the set $\{\, a, ad, \ldots, ad^\ell
    \,\}$ an \textsl{\mbox{$\ell$-term} algebraic line}.
  \end{definition}
\pause
  \begin{definition}
    Let $S$ be a semigroup. 
    Define
    \vspace{-1em}
    \begin{align*}
      \mathcal{AL}(S) &= \{\, p \in \beta S : \mbox{for all $A \in
        p$, $A$ contains an} \\
      &\hspace{5em}\mbox{ $\ell$-term algebraic 
        line for every $\ell \in \bbN$} \,\}.
    \end{align*}
  \end{definition}

  \pause

  \begin{block}{General Facts}
    \begin{itemize}
      \item The above theorem implies that $\mathcal{AL}(S) \ne
        \emptyset$. 
        \pause

      \item  $\mathcal{AL}(S)$ is closed.
        \pause

      \item  $\mathcal{AL}(S)$ is a left ideal, that is, $\beta S \cdot       \mathcal{AL}(S) \subseteq \mathcal{AL}(S)$. 
        \pause

      \item \alert{$\mathcal{AL}(S)$ may not be a right ideal if $S$ is          noncommutative.}
    \end{itemize}
  \end{block}
\end{frame}

\section{Hales-Jewett Theorem and $J$-sets}
\begin{frame}
  \begin{definition}
    Let $S$ be the free semigroup on the finite alphabet $A$, and let
    $\star$ denote an element not in $A$.
    \begin{itemize}
      \item[(a)] A \textsl{variable word (on $A$)} $w(\star)$ is a word in the
      free semigroup on the alphabet $A \cup \{\star\}$ in which
      $\star$ occurs.

      \item[(b)] Given a variable word $w(\star)$ on $A$ and $a \in A$,
      we let $w(a)$ denote the word in $S$ where each occurrence of
      $\star$ in $w(\star)$ is replaced by $a$.

      \item[(c)] Given a variable word $w(\star)$ on $A$ we call the set
      $\{\, w(a) : a \in A \,\}$ a \textsl{combinatorial line}.
    \end{itemize}
  \end{definition}
  
  \pause

  \begin{example}
    Let $A = \{a, b, c\}$ and $w(\star) = aa\star bc \star \star ca$.
    Then $\{aa a bc aa ca, aa b bc bb ca, aa c bc cc ca\}$ is a
    combinatorial line. 
  \end{example}
\end{frame}

\begin{frame}
  \begin{block}{Hales-Jewett Theorem}
    \textsl{Let $S$ be the free semigroup on the finite alphabet $A$
      and let $r \in \bbN$.
      If $S = \bigcup_{i=1}^r C_i$, then there exist $i \in \{1, 2,
      \ldots, r\}$ and a variable word $w(\star)$ such that $\{\, w(a)
      : a \in A \,\} \subseteq C_i$.
    }
  \end{block}

  \pause

  \begin{block}{Goal}
    We will define a special type of ``combinatorial line'' in a
    semigroup.
    \pause
    \alert{$J$-sets are one way to define such an object.}
  \end{block}
\end{frame}

\begin{frame}
  \begin{definition}
    Let $S$ be a semigroup.
    \begin{itemize}
      \item For each $m \in \bbN$ define
        \[
          \calJ_m = \{\, (t_1, t_2, \ldots, t_m) \in \bbN^m : t_1 <
          t_2 < \cdots < t_m \,\}.
        \]
        \pause

      \item For each $m \in \bbN$, $a \in S^{m+1}$, $t \in \calJ_m$,
        and $f$ a sequence in $S$, define
        \vspace{-1em}
        \[
          x(m, a, t, f) = \biggl( \prod_{i=1}^m \bigl( a(i) f(t_j)
          \bigr) \biggr) a(m+1).
        \]
        \pause

      \item We call $A \subseteq S$ a \textsl{$J$-set} if and only if
        for every $F \subseteq \mbox{$^{\bbN}{S}$}$ finite, there exists $m \in
        \bbN$, $a \in S^{m+1}$, and $t \in \calJ_m$ such that for all
        $f \in F$, $x(m, a, t, f) \in A$.      
    \end{itemize}
  \end{definition}
\end{frame}

\begin{frame}
  \begin{definition}
    Let $S$ be a semigroup and define 
    \[
      J(S) = \{\, p \in \beta S : \mbox{for all $A \in p$, $A$ is a
        $J$-set} \,\}.
    \]
  \end{definition}

  \pause

  \begin{block}{General Facts}
    \begin{itemize}
      \item The Hales-Jewett theorem implies that $J(S) \ne
        \emptyset$. 
        \pause

      \item  $J(S)$ is closed.
        \pause

      \item  $J(S)$ is an ideal.
        \pause

      \item $J(S)$ contains idempotents, that is, there exists $p \in
        J(S)$ such that $p \cdot p = p$.
    \end{itemize}
  \end{block}
\end{frame}

\begin{frame}
  \begin{theorem}[Dissertation result]
    Let $S$ be a commutative semigroup and let $T \subseteq S$ be a subsemigroup. 
    The following statements are equivalent.
    \begin{itemize}
      \item[(a)] $J(T) = \beta T \cap J(S)$.
      \item[(b)] $\beta T \cap J(S) \ne \emptyset$.
      \item[(c)] $T$ is a $J$-set in S.
    \end{itemize}
  \end{theorem}

  \pause

  \begin{theorem}[Dissertation result]
    Let $S$ be a semigroup and let $T \subseteq S$ be a subsemigroup. 
    The following statements are equivalent.
    \begin{itemize}
      \item[(a)] \alert{$J(T) \subseteq \beta T \cap J(S)$.}
      \item[(b)] $\beta T \cap J(S) \ne \emptyset$.
      \item[(c)] $T$ is a $J$-set in S.
    \end{itemize}
  \end{theorem}
\end{frame}

\section{Central Sets Theorem}
\begin{frame}
  \begin{definition}
    Members of idempotents in $J(S)$ are called \textsl{$C$-sets}.
  \end{definition}
  
  \pause

  \begin{theorem}
    Let $S$ be a semigroup.
    Then $C \subseteq S$ is a $C$-set if and only if there exist
    functions $m \colon \Pf(\mbox{$^{\bbN}{S}$}) \to \bbN$, $\alpha
    \in \bigtimes_{F \in \Pf(\mbox{$^{\bbN}{S}$})} S^{m(F) + 1}$, and
    $t \in \bigtimes_{\mbox{$^{\bbN}{S}$}} \calJ_m$ such that 
    \pause
    \begin{enumerate}
      \item if $F$, $G \in \Pf(\mbox{$^{\bbN}{S}$})$ with $F \subsetneq
        G$, then $t(F)\bigl( m(F) \bigr) < t(G)(1)$; and,
        \pause
      
      \item whenever $n \in \bbN$, $G_1$, $G_2$, \ldots, $G_n
          \in \Pf(\calT)$ with $G_1 \subsetneq G_2 \subsetneq \cdots
          \subsetneq G_n$, and $\la f_i \ra_{i=1}^n \in
          \bigtimes_{i=1}^n G_i$, we have
          $\prod_{i=1}^n x(m(G_i), \alpha(G_i), t(G_i), f_i) \in C$. 
    \end{enumerate}
  \end{theorem}
\end{frame}

\begin{frame}
  \begin{theorem}[Central Sets Theorem]
    Let $S$ be a semigroup.
    If $C \subseteq S$ is a central set, then $C$ is a $C$-set. 
  \end{theorem}

  \pause

  \begin{definition}
    Central sets are members of minimal idempotents.
  \end{definition}
\end{frame}

\section{Thank you}
\begin{frame}
  \begin{block}{Thanks!}
    Thank you for listening! \\
  \end{block}
\end{frame}
\end{document}
