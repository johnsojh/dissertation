% This chapter proves a combinatorial characterization of
% $C$-sets. [6/19/2011 John]

\newcommand{\ds}{(X, \la T_s \ra_{s \in S})}
\chapter{Dynamical Characterization of $C$-sets}
\section{Dynamical Systems and $\beta S$}
Furstenburg in \cite[Chapter 8]{Furstenberg:1981fk}, using notions of topological dynamics, originally defined a central subset of positive integers as follows. 
We call $A \subseteq \bbN$ a \emph{central set} if and only if there exist a compact metric space $(X, d)$, a continuous function $T \colon X \to X$, points $x$, $y \in X$, and a neighborhood $U$ of $y$ such that
\begin{itemize}
  \item[(a)]
    $\{\, n \in \bbN : T^n(y) \in U \,\}$ is syndetic;

  \item[(b)]
    $\displaystyle\lim_{n\to\infty} d\bigl(T^n(x), T^n(y) \bigr) = 0$; and

  \item[(c)]
    $A = \{\, n \in \bbN : T^n(x) \in U \,\}$. 
\end{itemize}
Where for each $n \in \bbN$, $T^n$ is the composition of $T$ with itself $n$-times.

The dynamical definition for central sets is simpler than the combinatorial definition but it is still nontrivial. 
(For instance, it is not obvious that central sets are partition regular with the dynamical definition.)
However since central sets originated in this topological dynamics context, it is natural to investigate whether $C$-sets admit a dynamical characterization; and, in this chapter we prove that such a characterization exists. 

Before we can begin we will need to define what we mean by a dynamical
system and provide a connection to $\beta S$.
  \begin{defn}
    \label{defn:semiact}
    Let $S$ be a set. 
    A triple $(X, S, \pi)$ is a \textsl{semigroup action of $S$ on
      $X$} (or more shortly, \textsl{$S$ acts on $X$}) if and only if 
      \begin{itemize}
        \item[(1)] $S$ is a semigroup; and
        
        \item[(2)] $\pi : S \times X \to X$ is a function%
          \endnote{
            The usual (and sometimes confusing) convention is to write the value
    $\pi(s,x)$ as $s \cdot x$.
    Following this convention the condition on $\pi$ becomes $s \cdot (t
    \cdot x) = st \cdot x$.
    We will not follow this convention since we will soon introduce
    better notation that will suit our purposes.
          }
 such that
          for all $s$, $t \in S$ and for every $x \in X$,
          \[ \pi\bigl(s, \pi(t,x)\bigr) = \pi(st, x). \]
          
      \end{itemize}
  \end{defn}
  

Eventually we will want to extend our action of $S$ on $X$ to an
action of $\beta S$ on $X$.
To produce this extension we will be using \cite[Lemma
3.30]{Hindman:1998fk} (and also \cite[Corollary
4.22]{Hindman:1998fk}). 
Before we can use these results directly we will need to characterize
semigroup actions into a more convenient form. 

  \begin{prop} 
    \label{prop:semiact}
    Let $X$ be a set, $(S,\cdot)$ a semigroup and $\pi : S \times X \to X$.
    The triple $(X, S, \pi)$ is a semigroup action if and only if
    there exists a semigroup homomorphism $T : S \to \setfunc{X}{X}$ such
    that for all $s \in S$ and every $x \in X$,
      \[ T(s)(x) = \pi(s,x). \]
  \end{prop}
  \begin{proof}
    First observe that $\setfunc{X}{X}$ is a semigroup under function composition.
    Now assume that the triple $(X, S, \pi)$ is a semigroup action. 
    Define the map $T : S \to \setfunc{X}{X}$ by $T(s)(x) =
    \pi(s,x)$.
    To see that $T$ is a semigroup homomorphism, let $s$, $t \in S$
    and $x \in X$. 
    Then 
      \begin{align*}
        T(st)(x) = \pi(st,x) &= \pi\bigl(s, \pi(t,x)\bigr), \\
        &= \pi\bigl(s, T(t)(x)\bigr), \\
        &= T(s)\bigl(T(t)(x)\bigr), \\
        &= \bigl(T(s) \circ T(t)\bigr) (x).
      \end{align*}
    Hence $T(st) = T(s) \circ T(t)$.

    Conversely, suppose $T$ is a semigroup homomorphism. 
    Let $s$, $t \in S$ and $x \in X$.
    Then 
      \begin{align*}
        \pi(st, x) &= T(st)(x), \\
        &= \bigl(T(s) \circ T(t)\bigr) (x), \\
        &= \pi\bigl(s, T(t)(x)\bigr), \\
        &= \pi\bigl(s, \pi(t,x)\bigr).
      \end{align*}
    Hence $(X, S, \pi)$ is a semigroup action.
  \end{proof}

With this Proposition we can now effectively forgot about our original Definition \ref{defn:semiact} and simply consider semigroup actions as homomorphisms of $S$ into $\setfunc{X}{X}$. 

% However, in we are not just concerned with any type of semigroup action. 
% After all we are taking $\beta S$ to be a compact right-topological semigroup, and so we would like to apply this topological algebra to our action in some way. 
% We shall soon see that the notion of a dynamical system is one way to accomplish this goal.

  \begin{defn}
    A pair $(X, \la T_s \ra_{s \in S})$ is a \textsl{dynamical system} if and only if
      \begin{itemize}
        \item[(1)] $X$ is a compact Hausdorff space;
        \item[(2)] $S$ is a semigroup;
        \item[(3)] $T_s : X \to X$ is continuous for all $s \in S$;
          and
        \item[(4)] $T_s \circ T_t = T_{st}$ for all $s$, $t \in S$.
      \end{itemize}
  \end{defn}

  \begin{rmk}
    Let $(X, \la T_s \ra_{s \in S})$ be a dynamical system and define $T : S \to
    \setfunc{X}{X}$ by $T(s) = T_s$.
    Then by Proposition \ref{prop:semiact} we are justified in saying
    that \textsl{$S$ acts on $X$ via $\la T_s \ra_{s \in S}$}.
  \end{rmk}

Using this remark we can extend the action of the dynamical system to
$\beta S$ as follows.
Giving $\setfunc{X}{X}$ the product topology and taking $S$ to be 
discrete, we have that the function $T :
S \to \setfunc{X}{X}$ is a continuous semigroup
homomorphism into a compact space.
Therefore by \cite[Theorem 3.27]{Hindman:1998fk} we can produce a
continuous extension $\widetilde{T}$ of $T$.
(More directly, $\widetilde{T}$ is defined for each $p \in \beta S$ by
$\widetilde{T}(p) \in \bigcap \{\, c\ell\bigl( T[A] \bigr) : A \in p \,\}$, where
the closure is in $\setfunc{X}{X}$.)
Now by \cite[Theorem 2.22(a)]{Hindman:1998fk} the space $\setfunc{X}{X}$ is
a compact right-topological semigroup.
In order to show that $\widetilde{T}$ is a semigroup homomorphism it
suffices by, \cite[Corollary 4.22]{Hindman:1998fk}, to show that for all
$s \in S$, the
map $\lambda_{T(s)}$ is continuous.
However by \cite[Theorem 2.2(b)]{Hindman:1998fk} we have that
$\lambda_{T(s)}$ is continuous if and only if $T(s)$ is continuous. 
Since $(X, \la T_s \ra_{s \in S})$ is a dynamical system and $T(s) = T_s$ we know by
definition that $T(s)$ is continuous. 
Hence $\widetilde{T} : \beta S \to \setfunc{X}{X}$ is a continuous semigroup
homomorphism.

  \begin{rmk}
    By using the map $\widetilde{T} : \beta S \to X$, we can define
    $T_p : X \to X$, for $p \in \beta S$, as $T_p =
    \widetilde{T}(p)$. 
    Since $\widetilde{T}$ is a semigroup homomorphism we immediately
    conclude that $T_p \circ T_q = T_{pq}$ for all $p$, $q \in \beta
    S$.
   \end{rmk}

It's important to note that in general $(\beta S, \la T_p \ra_{p
  \in \beta S})$ is not a dynamical system, and 
the next example shows that $T_p$ may not be continuous for $p \in S^*$.
 

  \begin{example}
    We have that $(\beta\bbN, \la \lambda_s \ra_{s\in\bbN})$ is a
    dynamical system, but if $p \in \bbN^*$, then $\lambda_p$ is not
    continuous.
    This is proved in \cite[Theorem 6.10 and Remark 6.11]{Hindman:1998fk}.
   \end{example}

This example is somewhat disappointing since we lose the ``dynamical
part'' when extending the dynamical system to $\beta S$.
However even with this lost we will be able to prove something
intelligible about certain dynamical systems. 
Intuitively, the points of $S^*$ can be thought of as points at
infinity. 
Since in dynamical systems we are often concerned with the
``long-run'' behavior of our maps $T_s$ we will often be able to
correspond any interesting long run behavior with a `point at infinity'
in $S^*$.

To make this idea precise, we will be using the notion of a limit
along an ultrafilter. 

  \begin{defn}
    \label{defn:plim}
    Let $S$ be a discrete space, $p \in \beta S$, $X$ a compact
    Hausdorff%
    \endnote{
      Our definition of a \hbox{$p$-limit} is a bit more restrictive
    then usual.
    Normally the definition only takes $X$ be any topological space.
    We placed these extra restrictions on $X$ to ensure that every
    \hbox{$p$-limit} exists and is unique.
    }
 topological space, $\la x_s : s \in S \ra$ a family
    of points in $X$, and $y \in X$.
    Then \hbox{$p$--$\displaystyle\lim_{s \in S} x_s = y$} if and only
    if for every
    neighborhood $U$ of $y$ we have $\{\, s \in S : x_s \in U \,\} \in p$.
  \end{defn}

  \begin{prop}
    \label{prop:dsplim}
    Let $(X, \la T_s \ra_{s \in S})$ be a dynamical system.
    Then for every $p \in \beta S$ and each $x \in X$ we have $T_p(x)
    = \hbox{$p$--$\lim_{s \in S} T_s(x)$}$.
  \end{prop}
  \begin{proof}
    By definition, we need to show that for all neighborhoods $U$ of
    $T_p(x)$, we have $\{\, s \in S : T_s(x) \in U \,\} \in p$. 
    Define $\pi_x : \setfunc{X}{X} \to X$ by $\pi_x(f) = f(x)$ and let
    $U$ be a neighborhood of $T_p(x)$.
    Note that $\pi_x^{-1}[U]$ is a neighborhood of $T_p$.
    By definition of $T_p$, we have that for all $A \in p$,
    $\pi_x^{-1}[U] \cap \{\, T_s : s \in A \,\} \ne \emptyset$. 
    Hence for all $A \in p$, there exists $s \in A$ such that $T_s(x)
    \in U$. 

    Suppose there exists a neighborhood $U$ of $T_p(x)$ such that
    $\{\, s \in S : T_s(x) \in U \,\} \not\in p$. 
    Put $A = \{\, s \in S : T_s(x) \not\in U\,\}$.
    Then $A \in p$.
    Therefore there exists $s \in A$ such that $T_s(x) \in U$ (by our
    first paragraph) and $T_s(x) \not\in U$ (by our definition of
    $A$), a contradiction.
  \end{proof}

With these preliminaries out of the way, we can now provide a
dynamical characterization of $C$-sets.

\section{Dynamical Characterization of $C$-sets}
\label{sec:dyncsets}
 \begin{defn}
    \label{defn:JSUR}
    Let $(X, \la T_s \ra_{s \in S})$ be a dynamical system, and let $x$, $y \in X$. 
    The pair $(x,y)$ is \textsl{jointly sparsely uniformly recurrent}
    (we'll abbreviate this to JSUR) if and only if $\{\, s \in S :
    \hbox{$T_s(x) \in U$ and $T_s(y) \in U$} \,\}$ is a $J$-set for every
    neighborhood $U$ of $y$.%
    \endnote{
      In this section Definition \ref{defn:JSUR}, Lemma \ref{lem:JSUR},
      and Theorem \ref{thm:dyncsets} and their proofs are all, essentially,
      minor modifications of \cite[Definition 3.1]{Burns:2007uq},
      \cite[Lemma 3.3]{Burns:2007uq}, \cite[Theorem 3.4]{Burns:2007uq}
      respectively. 
    }
  \end{defn}


  \begin{lem}
    \label{lem:JSUR}
    Let $(X, \la T_s \ra_{s \in S})$ be a dynamical system, and let $x$, $y \in X$.
    The following statements are equivalent.
    \begin{itemize}
      \item[(a)] The pair $(x, y)$ is JSUR.
      \item[(b)] There exists $r \in J(S)$ such that $T_r(x) = y = T_r(y)$.
      \item[(c)] There exists an idempotent $r \in J(S)$ such that $T_r(x)
        = y = T_r(y)$. 
    \end{itemize}
  \end{lem}
  \begin{proof}
    (a) $\Rightarrow$ (b). 
    For each neighborhood $U$ of $y$, put 
      \[  
        B_U = \{\, s \in S : \hbox{$T_s(x) \in U$ and $T_s(y) \in U$}
        \,\}.
      \]
    By assumption each $B_U$ is a $J$-set. 
    We now show that the collection 
    \[
      \{\, B_U : \hbox{$U$ is a neighborhood of $y$} \,\}
    \] 
    is closed under finite intersection
    by showing that, for all neighborhoods $U$ and $V$ of $y$ we have $B_{U
      \cap V} = B_U \cap B_V$.
    Let $s \in S$, then 
      \begin{align*}
        s \in B_{U \cap V} &\iff \hbox{$T_s(x) \in U \cap V$ and $T_s(y) \in U
        \cap V$}, \\
      &\iff \hbox{$T_s(x) \in U$, $T_s(x) \in V$, $T_s(y) \in U$, and
        $T_s(y) \in V$}, \\
      &\iff s \in B_U \cap B_V.
      \end{align*}
    
    By Lemma \ref{lem:pr-jsets}, we know that every $J$-set
    of $S$ is partition regular. 
    Therefore by \cite[Theorem 3.11 (b)]{Hindman:1998fk},  we can pick 
    $r \in J(S)$ such that $\{\, B_U : \hbox{$U$ is a neighborhood of
      $y$}\,\} \subseteq r$. 
    
    Now for all neighborhoods $U$ of $y$, we have $B_U \subseteq \{\,
    s \in S : T_s(x) \in U \,\}$ and $B_U \subseteq \{\, s \in S :
    T_s(y) \in U \,\}$. 
    Therefore $\{\, s \in S : T_s(x) \in U\,\} \in r$ and $\{\, s \in
    S : T_s(y) \in U \,\} \in r$. 
    By Definition \ref{defn:plim}, we can conclude that
    $r$-$\displaystyle\lim_{s\in S} T_s(x) = y$ and
    $r$-$\displaystyle\lim_{s \in S} T_s(y) = y$. 
    Hence by Proposition \ref{prop:dsplim}, we have $T_r(x) =
    r$-$\displaystyle\lim_{s \in S} T_s(x) = y = r$-$\displaystyle\lim_{s \in S}
    T_s(y) = T_r(y)$.
  
    (b) $\Rightarrow$ (c).
    Put $M  = \{\, r \in J(S) : T_r(x) = y = T_r(y) \,\}$. 
    We'll show that $M$ is a nonempty compact subsemigroup of $J(S)$. 
    If we can show this, then we can pick an idempotent in $M$ and our
    result follows.
    The fact that $M \ne \emptyset$ follows from our assumption. 
    To see that $M$ is compact it suffices to show that $M$ is
    closed. 
    Let $r \not\in M$, then either $T_r(x) \ne y$ or $T_r(y) \ne y$. 
    First, assume that $T_r(x) \ne y$. 
    By Definition \ref{defn:plim} and Proposition \ref{prop:dsplim},
    pick a
    neighborhood $U$ of $y$ such that 
    $\{\, s \in S : T_s(x) \in U \,\} \not\in r$.
    Put $A = \{\, s \in S : T_s(x) \in U \,\}$ and note that $S
    \setminus A \in r$ and $\overline{S \setminus A} \cap M =
    \emptyset$.
    Now assume that $T_r(y) \ne y$. 
    By Definition \ref{defn:plim} and \ref{prop:dsplim}, pick a
    neighborhood $U$ of $y$ such that
    $\{\, s \in S : T_s(y) \in U \,\} \not\in r$.
    Put $A = \{\, s \in S : T_s(y) \in U \,\}$ and note that $S
    \setminus A \in r$ and $\overline{S \setminus A} \cap M =
    \emptyset$.
    Hence $M$ is a nonempty closed subset of $J(S)$.

    To see that $M$ is a subsemigroup, let $q$, $r \in M$.
    Then $T_{qr}(x) = T_q\bigl(T_r(x)\bigr) = T_q(y) =
    T_q\bigl(T_r(y)\bigr) = T_{qr}(y)$ and $T_q(y) = y$. 
    Hence $qr \in M$. 
    
    (c) $\Rightarrow$ (a).
    Pick $r$ as guaranteed in (c). 
    Let $U$ be a neighborhood of $y$. 
    Then $\{\, s \in S : T_s(x) \in U \,\} \in r$ and $\{\,  s \in S :
    T_s(y) \in U \,\} \in r$.
    Hence $\{\, s \in S : \hbox{$T_s(x) \in U$ and $T_s(y) \in U$}
    \,\} \in r$.  
  \end{proof}

  \begin{thm}
    \label{thm:dyncsets}
    Let $(S,\cdot)$ be a semigroup and $A \subseteq S$. 
    Then $A$ is a $C$-set if and only if there exist a dynamical
    system $(X, \la T_s \ra_{s \in S})$ with points $x$, $y \in X$ where $(x,y)$ is JSUR, and
    a neighborhood $U$ of $y$ such that $A = \{\, s \in S : T_s(x) \in
    U \,\}$.
  \end{thm}
  \begin{proof}
    ($\Rightarrow$) Let $A \subseteq S$ be our $C$-set, and by
    Theorem \ref{thm:csets} pick an idempotent $r \in J(S)$ such that
    $A \in r$. 
    Let $R = S \cup \{e\}$ be the semigroup with an identity $e$
    adjoined to S. 
    (For expository convenience, we still add this new identity even
    if $S$ already contains an identity.)
    Give $\{0,1\}$ the discrete topology, and take
    $\setfunc{R}{\{0,1\}}$ to have
    the product topology, and put $X = \setfunc{R}{\{0,1\}}$.
    Hence $X$ is a compact Hausdorff space.
    For each $s \in S$, define $T_s : X \to X$ by $T_s(f) = f \circ
    \rho_s$. 
    % Perhaps I should sketch the argument here?
    By \cite[Theorem 19.14]{Hindman:1998fk}, $(X, \la T_s \ra_{s \in S})$ is a dynamical
    system. 
    
    Now let $x = \cchi_A$ be the characteristic function of $A$, and
    put $y = T_r(x)$.
    % Perhaps I should provide a proof since there is not one
    % explicitly given in the book?
    Then by \cite[Remark 19.13]{Hindman:1998fk}, we have that $T_r(y)
    = T_r\bigl(T_r(x)\bigr) = T_{rr}(x) = T_r(x) = y$.
    Therefore by (c) in Lemma \ref{lem:JSUR}, the pair $(x, y)$ is JSUR.

    Put $U = \{\, w \in X : w(e) = y(e) \,\}$, and note that $U =
    \pi^{-1}\bigl[\{y(e)\}\bigr]$ and so $y \in U$. 
    Hence $U$ is a (subbasic) open neighborhood of $y$.
    To help us show that $U$ is the neighborhood of $y$ we are looking
    for we will show that $y(e) = 1$.
    Since $y = T_r(x)$ we have that $\{\, s \in S : T_s(x) \in U \,\}
    \in r$ by Definition \ref{defn:plim} and Proposition \ref{prop:dsplim}.
    Since $A \in r$, we can pick $s \in A$ such that $T_s(x) \in U$. 
    Then by definition of $U$ and our choice of $T_s(x) \in U$, we
    have $y(e) = T_s(x)(e) = x\bigl(\rho_s(e)\bigr) = x(es) = x(s) =
    \chi_{A}(s) = 1$. 
    Finally, given $s \in S$, we have
      \begin{align*}
        s \in A &\iff \chi_{A}(s) = 1, \\
                &\iff x(s) = 1, \\
                &\iff x(es) = 1, \\
                &\iff (x \circ \rho_s)(e) = 1, \\
                &\iff T_s(x)(e) = 1 = y(e), \\
                &\iff T_s(x) \in U.
      \end{align*}
   Hence $A = \{\, s \in S : T_s(x) \in U \,\}$. 
   
   ($\Leftarrow$) Let $(X, \la T_s \ra_{s \in S})$ and let the points $x$, $y \in X$ be given as
   guaranteed.
   By Theorem \ref{thm:csets}, pick an idempotent $r \in J(S)$
   such that $T_r(x) = y = T_r(y)$. 
   Since $U$ is a neighborhood of $y$ and $T_r(x) = y$, we have that
   $A \in r$ by Definition \ref{defn:plim} and Proposition \ref{prop:dsplim}. 
 \end{proof}

\theendnotes
