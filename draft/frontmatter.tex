% The following is the title page as required by the Graduate School.
\newcommand{\thetitlepage}{
  \clearpage
  \thispagestyle{empty} 
  \begin{center}
    HOWARD UNIVERSITY \\ \vspace{1em}
    \textbf{Some Differences Between an Ideal in the Stone-\v{C}ech Compactification of Commutative and Noncommutative Semigroups} \\ \vspace{4em}

    A Dissertation \\
    Submitted to the Faculty of the \\
    Graduate School \\ \vspace{4em}
    
    of \\ \vspace{2.5em}

    \textbf{HOWARD UNIVERSITY} \\ \vspace{4em}
    
    in partial fulfillment of \\
    the requirements for the \\
    degree of \\ \vspace{2.5em}

    \textbf{DOCTOR OF PHILOSOPHY} \\ \vspace{2.5em}

    Department of Mathematics \\ \vspace{2em}
    
    by \\ \vspace{2em}

    \textbf{John H.~Johnson} \\ \vspace{1em}
    
    Washington, DC \\
    September 2011 \\
    \vfill
  \end{center}
}

% This is a template for the Committee Approval Form.
\newcommand{\approval}{
  \clearpage
  \begin{center}
    \textbf{HOWARD UNIVERSITY} \\
    \textbf{GRADUATE SCHOOL} \\
    \textbf{DEPARTMENT OF MATHEMATICS} \\ \vspace{1em}
    
    DISSERTATION COMMITTEE
  \end{center}

  % Code copied to produced a horizontal line for signatures. 
  \newcommand{\sigline}{\makebox[3in]{\hrulefill}}

  \vspace{4em}
  
  \begin{tabular}{@{}l @{}l}
    \hspace{15em} & \sigline \\
    \hspace{15em} & Abdul-Aziz Yakubu, Ph.D. \\
    \hspace{15em} & Chairperson \vspace{4em} \\
    \hspace{15em} & \sigline \\
    \hspace{15em} & Alexander Burstein, Ph.D.  \vspace{4em} \\
    \hspace{15em} & \sigline \\
    \hspace{15em} & Neil Hindman, Ph.D.  \vspace{4em} \\
    \hspace{15em} & \sigline \\
    \hspace{15em} & Arthur Grainger, Ph.D. \vspace{4em} \\
    \hspace{15em} & \sigline \\
    \hspace{15em} & Fran\c{c}ois Ramaroson, Ph.D. \vspace{4em} \\
    
    \sigline & \\
    Neil Hindman, Ph.D. & \\
    Dissertation Advisor & \vspace{2em} \\

    Candidate: John H.~Johnson & \vspace{2em} \\
    Date of Defense: $\infty$
  \end{tabular}
  
  \vfill
}

\newcommand{\dedication}{
  \clearpage
  
  \begin{center}
    \textbf{DEDICATION}

    \vspace{3em}

    \textsl{This dissertation is dedicated to my parents for their unending love and tireless support.}
  \end{center}
}

\newcommand{\acknowledgements}{
  \clearpage
  \begin{center}
    \textbf{ACKNOWLEDGEMENTS}
  \end{center}

  I wish to thank my advisor, Dr.~Hindman, for his saintly patience, open, honest and quick communication, and his phenomenal mathematical ideas and guidance. 
  Without his support, this dissertation would not have been written.  
  I hope that I have proved a few theorems here that Dr.~Hindman can be proud of.

  I also want to thank the wonderful (and often unappreciated!) Howard Mathematics Department. 
  (Particularly, Dr.~Ramaroson who gave me my first little taste of mathematics research several years ago.)

  I'm eternally grateful to the Bridge to the Doctorate (Dr.~Lee and his great staff helped ease my transition into Howard); GK12 (Dr.~Alfred miraculously ran this excellent program himself!); and Preparing Future Faculty programs for tuition and (great!) stipend funding throughout the years.  
  Without these programs, I would not have been able to participate in graduate education!

  I also wish to thank the mathematics department at James Madison University for welcoming me into their department during the 2010--2011 academic year, and providing me with my first `real' job for the 2011--2012 academic year.  

  I also want to thank my fellow Howard graduate students. 
  You all provided many laughs and much help throughout the years.

  Thank you Camelia for your wonderful and loving support (and patience!) these last three years.

  Finally, I wish to thank my family for just being themselves!
}

\thetitlepage
\dedication
\acknowledgements

\newpage
\begin{abstract}
Furstenberg, using topological dynamics, defined the notion of a central set of positive integers, and proved a powerful combinatorial theorem about central sets that has since come to be called the Central Sets Theorem.
Since Furstenberg's original investigations, the Central Sets Theorem has been generalized and extended, via the algebraic structure of the Stone-\v{C}ech compactification, to apply to any semigroup.

Through this research it was discovered that there are many sets, beside central sets, that satisfy the conclusion of the Central Sets Theorem.
Since many results on central sets are derived solely from the Central Sets Theorem, the focus has recently shifted to those sets, called $C$-sets, that satisfy the conclusion of the Central Sets Theorem.

The main motivation behind the research for this dissertation is the question: How analogous are $C$-sets with central sets?
More precisely, what properties of central sets fail to hold for $C$-sets.
The main result of this dissertation shows that, when the underlying semigroup is noncommutative, the ideal associated with central sets and the ideal associated with $C$-sets has slightly different `behavior'.  

Along the way we prove a topological dynamical characterization of $C$-sets; and, we give a new and simpler definition of a $C$-set, and prove that this new definition is equivalent to the old definition.
As an immediate result, we also get a new and simpler version of the Central Sets Theorem.
\end{abstract}

\approval

