% The following is the title page as required by the Graduate School.
\newcommand{\thetitlepage}{
  \clearpage
  \thispagestyle{empty} 
  \begin{center}
    HOWARD UNIVERSITY \\ \vspace{1em}
    \textbf{Some Differences Between an Ideal in the Stone-\v{C}ech Compactification of Commutative and Noncommutative Semigroups} \\ \vspace{4em}

    A Dissertation \\
    Submitted to the Faculty of the \\
    Graduate School \\ \vspace{4em}
    
    of \\ \vspace{2.5em}

    \textbf{HOWARD UNIVERSITY} \\ \vspace{4em}
    
    in partial fulfillment of \\
    the requirements for the \\
    degree of \\ \vspace{2.5em}

    \textbf{DOCTOR OF PHILOSOPHY} \\ \vspace{2.5em}

    Department of Mathematics \\ \vspace{2em}
    
    by \\ \vspace{2em}

    \textbf{John H.~Johnson} \\ \vspace{1em}
    
    Washington, DC \\
    September 2011 \\
    \vfill
  \end{center}
}

% This is a template for the Committee Approval Form.
\newcommand{\approval}{
  \clearpage
  \begin{center}
    \textbf{HOWARD UNIVERSITY} \\
    \textbf{GRADUATE SCHOOL} \\
    \textbf{DEPARTMENT OF MATHEMATICS} \\ \vspace{1em}
    
    DISSERTATION COMMITTEE
  \end{center}

  % Code copied to produced a horizontal line for signatures. 
  \newcommand{\sigline}{\makebox[3in]{\hrulefill}}

  \vspace{4em}
  
  \begin{tabular}{@{}l @{}l}
    \hspace{15em} & \sigline \\
    \hspace{15em} & Abdul-Aziz Yakubu, Ph.D. \\
    \hspace{15em} & Chairperson \vspace{4em} \\
    \hspace{15em} & \sigline \\
    \hspace{15em} & Alexander Burstein, Ph.D.  \vspace{4em} \\
    \hspace{15em} & \sigline \\
    \hspace{15em} & Neil Hindman, Ph.D.  \vspace{4em} \\
    \hspace{15em} & \sigline \\
    \hspace{15em} & Arthur Grainger, Ph.D. \vspace{4em} \\
    \hspace{15em} & \sigline \\
    \hspace{15em} & Fran\c{c}ois Ramaroson, Ph.D. \vspace{4em} \\
    
    \sigline & \\
    Neil Hindman, Ph.D. & \\
    Dissertation Advisor & \vspace{2em} \\

    Candidate: John H.~Johnson & \vspace{2em} \\
    Date of Defense: $\infty$
  \end{tabular}
  
  \vfill
}

\newcommand{\dedication}{
  \clearpage
  
  \begin{center}
    \textbf{DEDICATION}

    \vspace{3em}

    \textsl{This dissertation is dedicated to my parents for their unending love and tireless support.}
  \end{center}
}

\newcommand{\acknowledgements}{
  \clearpage
  \begin{center}
    \textbf{ACKNOWLEDGEMENTS}
  \end{center}

  I wish to thank my advisor, Dr.~Hindman, for his saintly patience, open, honest and quick communication, and his phenomenal mathematical ideas and guidance. 
  Without his support, this dissertation would not have been written.  
  I hope that I have proved a few theorems here that Dr.~Hindman can be proud of.

  I also want to thank the wonderful (and often unappreciated!) Howard Mathematics Department. 
  (Particularly, Dr.~Ramaroson who gave me my first little taste of mathematics research several years ago.)

  I'm eternally grateful to the Bridge to the Doctorate (Dr.~Lee and his great staff helped ease my transition into Howard); GK12 (Dr.~Alfred miraculously ran this excellent program himself!); and Preparing Future Faculty programs for tuition and (great!) stipend funding throughout the years.  
  Without these programs, I would not have been able to participate in graduate education!

  I also wish to thank the mathematics department at James Madison University for welcoming me into their department during the 2010--2011 academic year, and providing me with my first `real' job for the 2011--2012 academic year.  

  I also want to thank my fellow Howard graduate students. 
  You all provided many laughs and much heap throughout the years.

  Thank you Camelia for your wonderful and loving support (and patience!) these last three years.

  Finally, I wish to thank my family for just being themselves!
}

\thetitlepage
\dedication
\acknowledgements

\newpage
\begin{abstract}
  Furstenburg, using the notions of topological dynamics, defined the
concept of a central set of positive integers, and proved a powerful
combinatorial theorem concerning central sets called the Central Sets
Theorem.  Since Furstenburg's original statement of this theorem, the
Central Sets Theorem has been generalized and extended to apply to any
semigroup.  

Despite this theorem's name however, the Central Sets Theorem does not
provide a combinatorial characterization for central sets. (But, the
theorem often serves as a useful combinatorial substitute.)  Recently
the focus has shifted from central sets to those sets that satisfy the
Central Sets Theorem.  Such sets are called C-sets.  C-sets, similar
to central sets, interact nicely with the algebraic structure of the
Stone-Cech compactification.

In this dissertation, we give a new and simpler definition of a C-set,
and prove that this new definition is equivalent to the old
definition.  We also provide a proof topological dynamics
characterization of C-sets.  The main result of this dissertation is a
proof that an ideal (associated with C-sets) in the algebraic
structure of the Stone-Cech compactification of a discrete semigroup
interacts differently with subsemigroups depending on if the
underlying semigroup is commutative or noncommutative. 

  Van der Waerden's Theorem states that given any finite partition of the positive integers, at least one cell of the partition will contain arbitrarily long finite arithmetic progressions.
  We can roughly say that some algebraic structure, of the positive integers, is always
persevered under finite partitions.
  And, more generally, we can inquire on which algebraic structures of a semigroup will be
persevered under finite partitions.
  For instance, we state a straightforward analogue of van der Waerden's Theorem that applies to any semigroup.

  However, it turns out that, by using the algebraic structure of the Stone-\v{C}ech compactification, we can show this straightforward analogue is not optimal for noncommutative semigroups.

  Using a more general theorem as inspiration, we define a J-set of a semigroup. 
 
  The main problem we focus on in this dissertation is when the underlying semigroup is noncommutative.  

\end{abstract}

\approval

