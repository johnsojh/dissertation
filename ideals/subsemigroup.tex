% Draft of Ideals and Subsemigroup chapter

\documentclass[12pt]{article}

\usepackage{amsthm, amssymb, amsmath}
\usepackage{endnotes}
% \usepackage[doublespacing]{setspace}

\newtheoremstyle{plain}{3mm}{3mm}{\slshape}{}{\bfseries}{.}{.5em}{}
\theoremstyle{plain}
\newtheorem{thm}{Theorem}[section]
\newtheorem{CSTv1}{Original Central Sets Theorem}
\newtheorem{vdw}[thm]{Van der Waerden's Theorem}
\newtheorem{FST}[thm]{Hindman's Theorem}
\newtheorem{MBR}[thm]{Multiple Birkhoff Recurrence Theorem}
\newtheorem{recur}[thm]{Recurrence Theorem}
\newtheorem{OCST}[thm]{Furstenburg's Original Central Sets Theorem}
\newtheorem{cst}[thm]{Central Sets Theorem}
\newtheorem{cor}[thm]{Corollary}
\newtheorem{prop}[thm]{Proposition}
\newtheorem{lem}[thm]{Lemma}
\newtheorem{claim}[thm]{Claim}
\newtheorem{ques}[thm]{Question}
\newtheorem{conj}[thm]{Conjecture}
\newtheorem{fact}[thm]{Fact}

\theoremstyle{definition}
\newtheorem{defn}[thm]{Definition}
\newtheorem{example}[thm]{Example}
\newtheorem{rmk}[thm]{Remark}


\newcommand{\la}{\langle}
\newcommand{\ra}{\rangle}
\newcommand{\bbN}{\mathbb{N}}
\newcommand{\bbZ}{\mathbb{Z}}
\newcommand{\calA}{\mathcal{A}}
\newcommand{\calF}{\mathcal{F}}
\newcommand{\calH}{\mathcal{H}}
\newcommand{\calI}{\mathcal{I}}
\newcommand{\calP}{\mathcal{P}}
\newcommand{\calT}{\mathcal{T}}
\newcommand{\Pf}{\mathcal{P}_f}

\newcommand{\dom}{\mathrm{dom}}

\newcommand{\setfunc}[2]{\hbox{${}^{\hbox{$#1$}}\hskip -1 pt #2$}}

\font\bigmath=cmsy10 scaled \magstep 3
\newcommand{\bigtimes}{\hbox{\bigmath \char'2}}

\newcommand{\cchi}{\raise 2 pt \hbox{$\chi$}}


\begin{document}
\section{Introduction}
Our main motivating factor has been to find certain analogues of
theorems that involve the smallest ideal.
The reason for this line of research is that there are many nice
algebraic and combinatorial results that utilize the smallest ideal. 
(Arguably, through these results, the smallest ideal is perhaps the
best understood ideal in the Stone-\v{C}ech compactification.)
It is then natural to wonder how much of this ``niceness'' can be
imported to other ideals.

In particular, we shall prove an analogue to the following theorem. 
\begin{thm}
  Let $(S,\cdot)$ be a semigroup and let $T \subseteq S$ be a
  subsemigroup. 
  The following statements are equivalent.
  \begin{itemize}
    \item[(a)] $K(\beta T) = \beta T \cap K(\beta S)$.
    \item[(b)] $\beta T \cap K(\beta S)$ is nonempty.
    \item[(c)] $T$ is piecewise syndetic in $S$.
  \end{itemize}
\end{thm}
\begin{proof}
  Using \cite[Theorems 1.65 and 4.40]{Hindman:1998fk}, we have that
  \mbox{(a) $\iff$ (b)} and \mbox{(b) $\iff$ (c)}, respectively. 
\end{proof}

In this chapter we shall prove the following two analogue theorems.
\begin{thm}
  Let $(S, +)$ be a commutative semigroup and let $T \subseteq S$ be a
  subsemigroup. 
  The following statements are equivalent.
  \begin{itemize}
    \item[(a)] $J(\beta T) = \beta T \cap J(\beta S)$.
    \item[(b)] $\beta T \cap J(\beta S)$ is nonempty.
    \item[(c)] $T$ is a $J$-set in $S$.
  \end{itemize}
\end{thm}

\begin{thm}
  Let $(S, \cdot)$ be a commutative semigroup and let $T \subseteq S$ be a
  subsemigroup. 
  The following statements are equivalent.
  \begin{itemize}
    \item[(a)] $J(\beta T) \subseteq \beta T \cap J(\beta S)$.
    \item[(b)] $\beta T \cap J(\beta S)$ is nonempty.
    \item[(c)] $T$ is a $J$-set in $S$.
  \end{itemize}
\end{thm}

We shall also show that statement (a) in this theorem cannot be
strengthen to ``$J(T) = \beta T \cap J(S)$'' without some additional
assumptions on $S$. 

\section{Commutative Semigroups}
\begin{prop}
  Let $(S, +)$ be a commutative semigroup and $A \subseteq S$. 
  Then $A$ is a $J$-set in $S$ if and only if for every $F \in
  \Pf(\calT)$, there exist $a \in A$ and $H \in \Pf(\bbN)$ such that
  for all $f \in F$, $x(a, H, f) \in A$. 
\end{prop}
\begin{proof}
  ($\Rightarrow$)
  Let $F \in \Pf(\calT)$ and let $\overline{c} \in \calT$ be a
  constant sequence.
  Put $G = (F + \{\overline{c}\}) \cup \{\overline{c}\} $ and pick $b
  \in S$ and $H \in \Pf(\bbN)$ such that for all $g \in G$, $x(b,H,g)
  \in A$.
  Put $a = x(b, H, \overline{c})$, then $a \in A$ and for all $f \in
  F$, we have $a + \sum_{t \in H}f(t) = b + \sum_{t \in H} (f(t) +
  \overline{c}(t)) \in A$.

  ($\Leftarrow$) Obvious.
\end{proof}

\begin{prop}
  Let $(S, +)$ be a commutative semigroup and $A \subseteq S$.
  Then $A$ is a $J$-set in $S$ if and only if for every $F \in
  \Pf(\calT)$ there exist a sequence $\la a_n \ra_{n=1}^\infty$ in $S$
  and a sequence $\la H_n \ra_{n=1}^\infty$ in $\Pf(\bbN)$ with the
  following properties:
  \begin{itemize}
    \item[(1)] For all $n \in \bbN$, $\max H_n < \min H_{n+1}$. 

    \item[(2)] For all $f \in F$ and for every $n \in \bbN$, $x(a_n,
      H_n, f) \in A$.
  \end{itemize}
\end{prop}
\begin{proof}
  ($\Rightarrow$)
  Let $F \in \Pf(\calT)$.
  We shall recursively construct our sequences $\la a_n
  \ra_{n=1}^\infty$ and $\la H_n \ra_{n=1}^\infty$ such that following
  hypotheses are satisfied for $m\in \bbN$.
  \begin{itemize}
    \item[(i)] For all $k < m$, $\max H_k < \min_{k+1}$.
    \item[(ii)] For all $f \in F$ and for all $k \le n$, $x(a_k, H_k,
    f) \in A$.
  \end{itemize}
  
  Since $A$ is a $J$-set, pick $a_1 \in S$ and $H_1 \in \Pf(\bbN)$
  such that for all $f \in F$, $x(a_1, H_1, f) \in A$. 
  Clearly our hypotheses (i) and (ii) are satisfied for this choice.
  Now let $m > 1$ and assume $\la a_n \ra_{n=1}^{m-1}$ and $\la H_n
  \ra_{n=1}^{m-1}$ have been chosen to satisfy hypotheses (i) and
  (ii).
  Put $L = \max H_{m-1} + 1$. 
  Pick $a_m \in S$, and $H_m \in \Pf(\bbN)$ such that $\min H_m \ge L$
  and for all $f \in F$, $x(a_m, H_m, f) \in A$.
  Again, a simple check shows that these choices are consistent with
  our hypotheses (i) and (ii).

  ($\Leftarrow$) Obvious.
\end{proof}

\begin{prop}
  Let $(S, +)$ be a commutative semigroup, and let $T \subseteq S$
  be a subsemigroup which is also a $J$-set in $S$. 
  If $A \subseteq T$ is a $J$-set in $T$, then $A$ is a $J$-set in $S$.
\end{prop}
\begin{proof}
  Let $F \in \Pf\bigl(\calT(S)\bigr)$.
  Since $T$ is a $J$-set in $S$, pick sequences $\la a_n
  \ra_{n=1}^\infty$ in $S$ and $\la H_n \ra_{n=1}^\infty$ in
  $\Pf(\bbN)$ as guaranteed by the above Proposition such that for
  all $f \in F$ and $n \in \bbN$, $x(a_n, H_n, f) \in T$.
  For each $f \in F$, define $g_f \in \calT(T)$ by $g_f(n) = x(a_n,
  H_n, f)$. 
  Since $A$ is a $J$-set in $T$, pick $a \in T$, and $H \in
  \Pf(\bbN)$ such that for all $f \in F$, $x(a, H, g_f) \in A$. 
  Put $G = \bigcup_{t \in H} H_t$ and $b = a + \sum_{t \in H} a_t$. 
  Then for all $f \in F$ we have
  \begin{align*}
    b + \sum_{t \in G} f(t) &= a + \sum_{t \in H} a_t + \sum_{s \in
      \bigcup_{t \in H} H_t} f(s), \\
    &= a + \sum_{s \in H}(a_s + \sum_{t \in H_s} f(t)), \\
    &= a + \sum_{s \in H} g_f(s) = x(a, H, g_f).
  \end{align*}
  Hence it follows that $A$ is a $J$-set in $S$. 
\end{proof}

\begin{thm}
  Let $(S, +)$ be a commutative semigroup and let $T \subseteq S$ be a
  subsemigroup.
  The following statements are equivalent. 
  \begin{itemize}
    \item[(a)] $J(\beta T) = \beta T \cap J(\beta S)$.
    \item[(b)] $\beta T \cap J(\beta S)$ is nonempty.
    \item[(c)] $T$ is a $J$-set in $S$.
  \end{itemize}
\end{thm}
\begin{proof}
  (\mbox{a $\Rightarrow$ b})
  $J(T) \ne \emptyset$ by \mbox{$\la$citation$\ra$}.

  (\mbox{b $\Rightarrow$ c})
  Since $J(S)$ is closed by \mbox{$\la$citation$\ra$} and by the
  definition of $J(S)$ \mbox{$\la$citation$\ra$}, this follows.

  (\mbox{c $\Rightarrow$ a})
  ($\subseteq$) 
  Let $p \in J(T)$.
  Let $A \in p$ and $F \in \Pf\bigl(\calT(S)\bigr)$.
  Since $A$ is a $J$-set in $T$ and $T$ is a $J$-set in $S$, from our
  above proposition, we have that $A$ is a $J$-set in $T$. 
  Hence $p \in \beta T \cap J(\beta S)$.

  ($\supseteq$)
  Let $p \in \beta T \cap J(S)$ and $A \in p$. 
  Since $T \in p$, we have $A \cap T \in p$.
  Therefore we may assume that $A \subseteq T$.
  Let $F \in \Pf\bigl(\calT(T)\bigr)$.
  Since $F \in \Pf\bigl(\calT(S)\bigr)$ and $A$ is a $J$-set in $S$,
  by the above proposition we may pick $a \in A$ and $H \in \Pf(\bbN)$
  such that for all $f \in F$, $x(a, H, f) \in A$. 
  Since $a \in A \subseteq T$, we have shown that $A$ is a $J$-set in
  $T$, hence $p \in J(T)$. 
\end{proof}

\section{Arbitrary Semigroups}
Our first goal shall be to develop a generalization of the above
Proposition. 
However, given the definition of a $J$-set we will need the following
technical lemmas.

\begin{lem}
  \label{lem:prodrewrt}
  Let $(S, \cdot)$ be  a semigroup, $F \in \Pf(\calT)$, $\la m(k)
  \ra_{k=1}^\infty$ a sequence in $\bbN$, $\la a_k \ra_{k=1}^\infty
  \in \bigtimes_{k=1}^\infty S^{m(k)+1}$, and $\la H_k
  \ra_{k=1}^\infty \in \bigtimes_{k=1}^\infty \calI_{m(k)}$ with $\max
  H_k\bigl(m(k)\bigr) < \min H_{k+1}(1)$ for every $k \in \bbN$.
  For each $f \in F$ define $g_f \in \calT$ by $g_f(k) = x(m(k), a_k,
  H_k, f)$. 

  If $G \in \Pf(\bbN)$, then, with $n = \sum_{k \in G} m(k)$, there
  exists $c \in S^{n+1}$ and $D \in \calI_n$ such that $\max D(n) =
  \max H_{\max G}\bigl(m(\max G)\bigr)$ and for all $f \in F$, 
  \[
    \prod_{k \in G} g_f(k) = x(n, c, D, f).
  \]
\end{lem}
\begin{proof}
  We prove our implication by induction on the size of $G$. 
  If $|G| = 1$, then $G = \{k\}$ for some $k \in \bbN$.
  Therefore by definition of $g_f$,
  \[
    \prod_{k \in G} g_f(k) = g_f(k) = x(m(k), a_k, H_k, f).
  \]
  Hence we may put $c = a_k$ and $D = H_k$.

  Now assume that $|G| > 1$ and our implication is true for all
  nonempty finite subsets of $\bbN$ with size less than $|G|$. 
  Put $K = G \setminus \{\max G\}$, and for notation convenience, put $s
  = \max G$. 
  By our induction hypothesis, with $l = \sum_{k \in K} m(k)$, pick $d
  \in S^{l+1}$ and $E \in \calI_l$ as guaranteed. 
  Then 
  \begin{align*}
    \prod_{k \in G} g_f(k) &= \prod_{k \in K} g_f(k) g_f(s), \\
    &= x(l, d, E, f) g_f(s) \\
    &= x(l, d, E,f) x(m(s), a_s, H_s, f).
  \end{align*}
  Put $n = \sum_{k \in G} m(k)$ and define $c \in S^{n+1}$ by 
  \[
    c(j) = 
    \begin{cases}
      d(j) & \mbox{if $j \in \{1, 2, \ldots, l\}$,} \\
      d(l+1)a_s(1) & \mbox{if $j = l+1$,} \\
      a_s(j-l) & \mbox{if $j \in \{l+2, l+3, \ldots, l+m(s)+ 1\}$.}
    \end{cases}
  \]
  By our induction hypothesis, $\max E(l) = \max H_{\max
    K}\bigl(m(\max K)\bigr)$.
  Moreover by definition of $\la H_k \ra_{k=1}^\infty$, $\max H_{\max
    K}\bigl(m(\max K)\bigr) < \min H_s(1)$. 
  Therefore we can define $D \in \calI_n$ by
  \[
    D(j) = 
    \begin{cases}
      E(j) & \mbox{if $j \in \{1, 2, \ldots, l\}$,} \\
      H_s(j-l) & \mbox{if $j \in \{l+1, l+2, \ldots, l+m(s) \}$.}
    \end{cases}
  \]
  We'll show that $\prod_{k \in G} g_f(k) = x(n, c, D, f)$. 
  
  We have that
  \begin{align*}
    x(n, c, D, f) &= \prod_{j=1}^n\Bigl( c(j) \prod_{t \in D(J)} f(t)
  \Bigr) c(n+1), \\
  &= \prod_{j=1}^l\Bigl( c(j) \prod_{t \in D(J)} f(t) \Bigr) c(l+1)
  \prod_{t \in D(l+1)} f(t) \Bigl( \prod_{j=l+2}^n c(j) \prod_{t \in D(J)}
  f(t) \Bigr) c(n+1), \\
  &= \prod_{j=1}^l\Bigl( c(j) \prod_{t \in D(J)} f(t) \Bigr)
  d(l+1)a_s(1) \cdot \\
  &\hspace{5em} \prod_{t \in H_s(1)} f(t) \Bigl( \prod_{j=l+2}^n
  a_s(j-l) \prod_{t \in H_s(j-l)} f(t) \Bigr) a_s(n+1-l), \\
  &= x(l, d, E, f) a_s(1) \prod_{t \in H_s(1)} f(t) \Bigl(
  \prod_{j=l+2}^{l+m(s)} a_s(j-l) \prod_{t \in H_s(j-l)} f(t) \Bigr) a_s(n+1-l), \\
  &= x(l, d, E, f) \prod_{j=1}^{m(s)}\Bigl( a_s(j) \prod_{t \in H_s(j)} f(t)
  \Bigr) a_s(m(s)+1), \\
  &= x(l, d, E, f) x(m(s), a_s, H_s, f) = \prod_{k \in G} g_f(k).
  \end{align*}
\end{proof}

\begin{lem}
  Let $(S, \cdot)$ be  a semigroup, $F \in \Pf(\calT)$, $\la m(k)
  \ra_{k=1}^\infty$ a sequence in $\bbN$, $\la a_k \ra_{k=1}^\infty
  \in \bigtimes_{k=1}^\infty S^{m(k)+1}$, and $\la H_k
  \ra_{k=1}^\infty \in \bigtimes_{k=1}^\infty \calI_{m(k)}$ with $\max
  H_k\bigl(m(k)\bigr) < \min H_{k+1}(1)$ for every $k \in \bbN$.
  For each $f \in F$ define $g_f \in \calT$ by $g_f(k) = x(m(k), a_k,
  H_k, f)$. 
  
  If $l \in \bbN$, $b \in S^{l+1}$, and $G \in \calI_l$, then there
  exist $n \in \bbN$, $c \in S^{n+1}$, and $D \in \calI_n$ such that
  for all $f \in F$, 
  \[
    x(l, b, G, g_f) = x(n, c, D, f).
  \]
\end{lem}
\begin{proof}
  We prove our implication by induction on $l$.
  Let $l = 1$, $b \in S^{2}$, and $G \in \calI_1$, then
  \[
    x(1, b, G, g_f) = b(1) \prod_{t \in G(1)} g_f(t) b(2).
  \]
  Apply Lemma \ref{lem:prodrewrt} to pick $n \in \bbN$, $d \in
  S^{n+1}$, and $D \in \calI_n$ such that $\prod_{t \in G(1)} g_f(t) =
  x(n, d, D, f)$.
  Define $c \in S^{n+1}$ by
  \[
    c(i) =
    \begin{cases}
      b(1)d(1) & \mbox{if $i = 1$,} \\
      d(i) & \mbox{if $i \in \{2, 3, \ldots, n \}$,} \\
      d(n+1)b(2) & \mbox{if $i = n+1$.}
    \end{cases}
  \]
  We show that $x(1, b, G, g_f) = x(n, c, D, f)$.
  We have that
  \begin{align*}
    x(n, c, D, f) &= \prod_{i=1}^n\Bigl(c(i) \prod_{t \in D(i)} f(t)
    \Bigr)c(n+1), \\
    &= c(1)\prod_{t \in D(1)} f(t) \prod_{i=2}^n\Bigl(c(i) \prod_{t
      \in D(i)} f(t) \Bigr) c(n+1), \\
    &= b(1)d(1) \prod_{t \in D(1)} f(t)
    \prod_{i=2}^n\Bigl(d(i)\prod_{t \in D(i)} f(t) \Bigr) d(n+1)b(2),
    \\
    &= b(1) \prod_{i=1}^n\Bigl(d(i) \prod_{t \in D(i)} f(t) \Bigr)
    d(n+1) b(2), \\
    &= b(1) x(n, d, D, f)b(2) = b(1) \prod_{t \in G(1)} g_f(t) b(2),
    \\
    &= x(1, b, G, g_f).
  \end{align*}

  Now let $l > 1$ and assume our implication is true for $l-1$ with
  \mbox{$\la$more conditions$\ra$}. 
  Let $b \in S^{l+1}$ and $G \in \calI_l$.
  Then 
  \[
    x(l, b, G, g_f) = \prod_{i=1}^{l-1}\Bigl(b(i) \prod_{t \in G(i)}
    g_f(t) \Bigr) b(l) \cdot \prod_{t \in G(l)} g_f(t) b(l+1).
  \]
  By our induction hypothesis pick $p \in \bbN$, $d \in S^{p+1}$, and
  $D \in \calI_p$ such that $\prod_{i=1}^{l-1}\bigl(b(i) \prod_{t \in
    G(i)} g_f(t)\bigr)b(l) = x(p, d, D, f)$.
  By Lemma \ref{lem:prodrewrt}, pick $m \in \bbN$, $e \in S^{m+1}$,
  and $E \in \calI_m$ such that $\prod_{t \in G(l)} g_f(t) = x(m, e,
  E, f)$ and $\max E(m) = \max H_{\max G(m(\max G))}$.
  Hence
  \[
    x(l, b, G, g_f) = x(p, d, D, f)x(m,e,E,f)b(l+1).
  \]
  Put $n = p+m$. 
  Define $c \in S^{n+1}$ by
  \[
    c(i) = 
    \begin{cases}
      d(i) & \mbox{if $i \in \{1, 2, \ldots, p\}$,} \\
      d(p+1)b(l)e(1) & \mbox{if $i=p+1$,} \\
      e(i-p) & \mbox{if $i \in \{p+2, P+3, \ldots, p+m\}$,} \\
      e(m+1)b(l+1) & \mbox{if $i = p + m + 1$.}
    \end{cases}
  \]
  Define $L \in \calI_n$ by
  \[
    L(i) = 
    \begin{cases}
      D(i) & \mbox{if $i \in \{1, 2, \ldots, p\}$}, \\
      E(i-p) & \mbox{if $i \in \{p+1, p+2, \ldots, p+m \}$}.
    \end{cases}
  \]
  [Need to show $\max D(p) < \min E(1)$ by weaken my induction
  hypothesis.] 
  We show that $x(l, b, G, g_f) = x(n, c, L, f)$.
  We have that
  \begin{align*}
    x(n, c, L, f) &= \prod_{i=1}^n\Bigl( c(i) \prod_{t \in L(i)} f(t)
    \Bigr) c(n+1), \\
    &= \prod_{i=1}^p\Bigl( c(i) \prod_{t \in L(i)} f(t) \Bigr) c(p+1)
    \prod_{t \in L(p+1))} f(t) \prod_{i=p+2}^{p+m}\Bigl( c(i) \prod_{t \in L(i)} f(t)
    \Bigr) c(p+m+1),
    &= \prod_{i=1}^p\Bigl( d(i) \prod_{t \in D(i)} f(t) \Bigr) d(p+1)b(l)e(1)
    \prod_{t \in E(1)} f(t) \prod_{i=p+2}^{p+m}\Bigl( e(i-p) \prod_{t \in E(i-p)} f(t)
    \Bigr) e(m+1)b(l+1), \\
    &= x(p, d, D, f) b(l) e(1)
    \prod_{t \in E(1)} f(t) \prod_{i=2}^{m}\Bigl( e(i) \prod_{t \in E(i)} f(t)
    \Bigr) e(m+1)b(l+1), \\
    &= x(p, d, D, f) b(l) x(m, e, E, f) b(l+1) = x(l, b, G, g_f).
  \end{align*}

\section{An Example}
 \begin{defn}
  Let $(S,\cdot)$ be a semigroup.
    \begin{itemize}
    \item[(a)] For each $m \in \mathbb{N}$,
      \begin{align*}
      \mathcal{I}_m = \bigl\{\,\bigl(H(1) &, H(2), \ldots, 
      H(m)\bigr) : \hbox{ each $H(i) \in 
      \mathcal{P}_f(\mathbb{N})$} \\ 
      & \hbox{ and for any $i \in \{1, 2,\ldots, m-1\}$, 
      $\max H(i) < \min H(i+1)$} \,\}.
      \end{align*}

    \item[(b)] Let $A \subseteq S$.
      We say that $A$ is a \textbf{$J$-set in $S$} if and only if
      for all $F \in \mathcal{P}_f(^{\mathbb{N}}{\!S})$, there exist
      $m \in \mathbb{N}$, $c \in S^{m+1}$, and $H \in \mathcal{I}_m$
      such that for every $f \in F$,
      \[
      \prod_{i=1}^m\bigl(c(1) \prod_{t \in H(i)} f(t)\bigr)c(m+1) \in
      A.
      \]

    \item[(c)] $J(S) = \{\, p \in \beta S : \hbox{$A$ is a 
      $J$-set in $S$ for every $A \in p$} \,\}$.
    \end{itemize}
  \end{defn}

  \begin{thm}
  Let $S$ be a semigroup.
  Then $J(S)$ is a closed ideal of $\beta S$, and if $A$ is 
  a $J$-set in $S$, then there exists $p \in J(S)$ such that
  $A \in p$
  \end{thm}
  \begin{proof}[Proof Outline]
  The fact that $J(S)$ is a closed ideal follows from 
  \cite[Theorem 3.5]{NewCST}.
  The second assertion follows from \cite[Theorem 2.14]{CarProdCST}.
  \end{proof}

The following result should be clear after a moment's thought.

  \begin{lem}\label{lem:seqs}
  Let $S$ be the free semigroup on two generators $a$ and $b$.
  Enumerate $\mathcal{P}_f(S)$ as $\la A_n \ra_{n=1}^\infty$. 
   Then there exist two sequences $\la v_n \ra_{n=1}^\infty$ and 
  $\la w_n \ra_{n=1}^\infty$ in $S$ satisfying the following 
  three conditions:
    \begin{itemize}
    \item[(1)] $\{\, v_n : n \in \mathbb{N} \,\} \subseteq
      \{\, (ab)^t : t \in \mathbb{N} \,\}$ and $\{\, w_n :
      n \in \mathbb{N} \,\} \subseteq \{\, b^t : t \in \mathbb{N} 
      \,\}$.

    \item[(2)] For each $n$, the length of each of the words $v_n$ 
      and $w_n$ is longer than the length of every word in $B_n$.

    \item[(3)] For each $n > 1$, for every $k \in \{1, 2, \ldots,
      n-1\}$, and for all $x \in B_k$, the length of each of the 
      words $v_n$ and $w_n$ is longer than the length of 
      $v_kxw_k$.
    \end{itemize}
  \end{lem}

  \begin{defn}\label{jset}
  Pick Pick $\la v_n \ra_{n=1}^\infty$ and $\la w_n \ra_{n=1}^\infty$ as
  guaranteed by the above lemma. 
  Let
  \[ A = \bigcup_{n=1}^\infty v_nA_nw_n. \]
  \end{defn}

  \begin{prop}\label{prop:jset}
  $A$ is a $J$-set in the free semigroup on two generators.
  \end{prop}
  \begin{proof}
  Let $F \in \mathcal{P}_f(^{\mathbb{N}}{\!S})$.
  Pick $H \in \mathcal{I}_1$, then $\{\, \prod_{t \in H(1)} f(t) :
  f \in F \,\} \in \mathcal{P}_f(S)$.
  Therefore pick $n \in \mathbb{N}$ such that $A_n = \{\,
  \prod_{t \in H(1)} f(t) : f \in F \,\}$.
  Then for every $f \in F$,
  \[
  v_n\prod_{t \in H(1)} f(t)w_n \in A.
  \]
  \end{proof}

Now that we have our $J$-set, we will use it to form our 
required subsemigroup of the free semigroup on two generators 
$a$ and $b$.

  \begin{defn}\label{subsgrp}
  \[
  T = \bigl\la A \cup \{a\} \bigr\ra = \bigl\{\,u_1u_2 \cdots u_s : 
  \hbox{$s \in \mathbb{N}$ and for every $i \in \{1, 2, \ldots, s\}$,
  $u_i \in A \cup \{a\}$}\,\bigr\}
  \]
  \end{defn}

  \begin{prop}\label{prop:subsgrp}
  $T$ is a proper subsemigroup of $S$, the free semigroup on two 
  generators, with $A \subseteq T$.
  \end{prop}
  \begin{proof}
  The fact that $T$ is a subsemigroup of $S$ with $A \subseteq T$
  follows directly from the definition of $T$.
  To see that $T \subsetneq S$ it suffices to show that $b \not\in
  T$.
  First, note that $(\forall x \in A)(\ell(x) \ge 3)$ and so
  $b \not\in A$.
  It follows that if $x \in T$ and $\ell(x) = 1$, then $x=a$. 
  Hence $b \not\in T$.
  \end{proof}

\section{Main Result}
We are now ready to prove our main result.
Instead of proving Theorem \ref{thm:main} directly, we will
prove the following slightly stronger result from which this  
Theorem will follow directly.

  \begin{thm}
  There exists a noncommutative semigroup $S$, a subsemigroup
  $T$, and a subset $A \subseteq T$ such that $A$ is a $J$-set
  in $S$ but there exists $F \in \mathcal{P}_f(^{\mathbb{N}}{T})$
  such that for all $m \in \mathbb{N}$, for each $c \in S^{m+1}$,
  and for any $H \in \mathcal{I}_m$, there exists $f \in F$ such
  that if $c(1) \in T$, then $\prod_{i=1}(c(i)\prod_{t \in H(i)} f(t))
  c(m+1) \not\in A.$
  \end{thm}
  \begin{proof}
  Let $S$ be the free semigroup on two generators $a$ and $b$,
  Let $A$ and $T$  be as in Definitions \ref{jset} and 
  \ref{subsgrp}, respectively. 
  The fact that $A$ is a $J$-set in $S$ is Proposition 
  \ref{prop:jset}, and the fact that $T$ is a subsemigroup of $S$
  with $A \subseteq T$ is Proposition \ref{prop:subsgrp}.

  Put $F = \{\overline{a}\}$, and let $m \in \mathbb{N}$, $c \in
  S^{m+1}$, and $H \in \mathcal{I}_m$. 
  For each $j \in \{1, 2, \ldots, m\}$ put $n_j = |H(j)|$. 
  Then
  \[
  \prod_{i=1}^m\Bigl(c(i) \prod_{t \in H(i)} f(t)\Bigr)c(m+1) =
  c(1)a^{n_1}c(2)a^{n_2} \cdots c(m)a^{n_m}c(m+1).
  \]
  Suppose that $c(1) \in T$ and $c(1)a^{n_1}c(2)a^{n_2} \cdots 
  c(m)a^{n_m}c(m+1) \in A$. 
  Since $c(1) \in T$, there exists $s \in \mathbb{N}$ such that
  $c(1) = u_1u_2 \cdots u_s$ and for all $i \in \{1, 2, \ldots, s\}$,
  $u_i \in A \cup \{a\}$. 
  Since $c(1)a^{n_1}c(2)a^{n_2} \cdots c(m)a^{n_m}c(m+1) \in A$, pick
  $k \in \mathbb{N}$ such that $c(1)a^{n_1}c(2)a^{n_2} \cdots 
  c(m)a^{n_m}c(m+1) \in v_kA_kw_k$, and pick $y \in A_k$ such that
  $c(1)a^{n_1}c(2)a^{n_2} \cdots c(m)a^{n_m}c(m+1) = v_kyw_k$. 
  We have two cases and three sub-cases to consider. 
  
  \textsl{Case 1 ($u_1=a$).} 
  First, note that by condition (1) of Lemma \ref{lem:seqs} the
  word $v_kyw_k$ starts with the substring ``$ab$''.
  If $s=1$, then the word $c(1)a^{n_1}c(2) \allowbreak a^{n_2} \cdots 
  c(m) a^{n_m} c(m+1)$ starts with the substring ``$aa$'', a
  contradiction.
  If $s>1$, then regardless of whether $u_2 \in A$ or $u_2 = a$,
  we still have that the word $c(1)a^{n_1}c(2) \allowbreak a^{n_2}
  \cdots c(m) a^{n_m} c(m+1)$ starts with the substring ``$aa$'',
  a contradiction.

  \textsl{Case 2 ($u_1 \in A$)}. 
  In this case, we have three sub-cases to consider. 
  Pick $n \in \mathbb{N}$ such that $u_1 \in v_nA_nw_n$, and pick
  $x \in A_n$ such that $u_1 = v_nxw_n$.
  
  \textsl{Sub-case I ($k<n$)}. 
  It follows from condition (3) of Lemma \ref{lem:seqs} that
  $\ell\bigl(c(1)\bigr) \ge \ell(u_1) > \ell(v_kyw_k)$.
  However, since $c(1)$ is a substring of $v_kyw_k$, this is a
  contradiction. 

  \textsl{Sub-case II ($k=n$)}.
  We have that 
    \begin{align*}
    v_kyw_k &= c(1)a^{n_1}c(2)a^{n_2} \cdots c(m)a^{n_m}c(m_1), \\
           &= v_kxw_ku_2 \cdots u_sa^{n_1}c(2)a^{n_2} \cdots 
           c(m)a^{n_m}c(m_1).
    \end{align*}
  Therefore $yw_k = xw_ku_2 \cdots u_sa^{n_1}c(2)a^{n_2} \cdots 
  c(m)a^{n_m}c(m+1)$.
  By condition (1) of Lemma \ref{lem:seqs} and the form of the 
  word $a^{n_m}c(m+1)$ it follows that $w_k$ is a substring of
  $c(m+1)$.
  So pick $v \in S\cup\{\emptyset\}$ such that $c(m+1) = vw_k$.
  Therefore $y = xw_ku_2 \cdots u_s a^{n_1}c(2)a^{n_2} \cdots 
  c(m)a^{n_m}v$, and so $w_k$ is a substring of $y$.
  In particular, $\ell(y) > \ell(w_k)$.
  However, by condition (2) of Lemma \ref{lem:seqs}, 
  $\ell(w_k) > \ell(y)$ also, a contradiction. 

  \textsl{Sub-case III ($k>n$)}. 
  Put $t = \ell(v_nx)$. 
  By condition (3) of Lemma \ref{lem:seqs}, $\ell(v_k) > t + 
  \ell(w_n) \ge t+1$. 
  In fact, by condition (2) we can sharpen this last inequality
  to $t + \ell(w_n) \ge t+2$. 
  Therefore the letter in position $t+1$ and $t+2$ of the word
  $v_nxw_nu_2 \cdots u_sa^{n_1}c(2)a^{n_2} \cdots c(m)a^{n_m}c(m+1)$
  is ``$b$''. 
  If $t+1$ is odd, then the letter in position $t+1$ of 
  $v_kyw_k$ is ``$a$'', a contradiction.
  If $t+1$ is even, then the letter in position $t+2$ of 
  $v_kyw_k$ is ``$a$'', a contradiction.
  \end{proof}

From this result we immediately get Theorem \ref{thm:main} 
as a corollary. 


\end{proof}

\begin{lem}
  Let $(S, \cdot)$ be a semigroup, $A \subseteq S$ a $J$-set, and $F
  \in \Pf(\calT)$.
  Then there exist a sequence on natura numbers $\la m(k)
  \ra_{k=1}^\infty$, $\la a_k \ra_{k=1}^\infty \in
  \bigtimes_{k=1}^\infty S^{m(k)+1}$, and $\la H_k \ra_{k=1}^\infty
  \in \bigtimes_{k=1}^\infty \calI_{m(k)}$ such that
  \begin{itemize}
    \item[(1)] for all $k \in \bbN$, $\max H_k\bigl(m(k)\bigr) < \min
      H_{k+1}(1)$; and,

    \item[(2)] for all $k \in \bbN$ and $f \in F$, $x(m(k), a_k, H_k,
      f) \in A$.
  \end{itemize}
\end{lem}

\begin{prop}
  Let $(S, \cdot)$ be  semigroup, and let $T \subseteq S$ be a
  subsemigroup which is also a $J$-set in $S$.

  If $A \subseteq T$ is a $J$-set in $T$, then $A$ is also a $J$-set
  in $S$.
\end{prop}

% Notes section produced by the 'endnotes' package
%\theendnotes

% Things referenced in the introduction. Eventually this will placed
% in a separate file so the References appear at the end.
\bibliographystyle{amsplain}
\bibliography{../references}
\end{document}
