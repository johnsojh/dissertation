\documentclass[12pt]{article}

\usepackage{fancyhdr}
\pagestyle{fancy}

\chead{\textbf{DISSERTATION SUMMARY}}
\cfoot{}

\addtolength{\baselineskip}{6pt}

\begin{document}
  \begin{center}
    \textbf{``Some Differences Between an Ideal in the Stone-\v{C}ech Compactification of Commutative and Noncommutative Semigroups''}
    \vspace{3em}

    \textbf{John H.~Johnson}
    \vspace{3em}
  \end{center}

Furstenberg, using topological dynamics, defined the notion of a central set of positive integers, and proved a powerful combinatorial theorem about central sets that has since come to be called the Central Sets Theorem.
Since Furstenberg's original investigations, the Central Sets Theorem has been generalized and extended, via the algebraic structure of the Stone-\v{C}ech compactification, to apply to any semigroup.

Through this research it was discovered that there are many sets, beside central sets, that satisfy the conclusion of the Central Sets Theorem.
Since many results on central sets are derived solely from the Central Sets Theorem, the focus has recently shifted to those sets, called $C$-sets, that satisfy the conclusion of the Central Sets Theorem.

The main motivation behind the research for this dissertation is the question: How analogous are $C$-sets with central sets?
More precisely, what properties of central sets fail to hold for $C$-sets.
The main result of this dissertation shows that, when the underlying semigroup is noncommutative, the ideal associated with central sets and the ideal associated with $C$-sets has slightly different `behavior'.  

Along the way we prove a topological dynamical characterization of $C$-sets; and, we give a new and simpler definition of a $C$-set, and prove that this new definition is equivalent to the old definition.
As an immediate result, we also get a new and simpler version of the Central Sets Theorem.
\end{document}
