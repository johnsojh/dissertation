\documentclass[12pt]{report}
\usepackage{amsfonts}
\usepackage{amsthm} % needed for \begin{proof}
\usepackage{graphicx}% needed to insert pictures with yap


%\usepackage{sectsty}
%\usepackage{amsthm}
%\usepackage{amsmath}
\usepackage{amssymb}
\usepackage{latexsym}
\usepackage{longtable}    % may not have to use this package
\usepackage{graphpap}     % may not have to use this package

\textheight=7.968 in
\textwidth=5.5 in
\hoffset=-.06 in
\voffset=-.3125 in

\newfont{\bb}{msbm10 at 12pt} %Blackboard characters outside of equations
\font\sf=cmss10 scaled \magstep 1
\font\bext=cmex10 scaled\magstep 1
\newcommand{\NN}{{\bb N}}
\newcommand{\QQ}{{\bb Q}}
\newcommand{\RR}{{\bb R}}
\newcommand{\DD}{{\bb D}}
\newcommand{\mod}{{\rm \ mod\ }}
\newcommand{\pf}{{\mathcal P}_f}
\def\onerp{\hbox{\vtop{\vskip -10 pt\hbox{\bext\char'001}}}}
\def\onelp{\hbox{\vtop{\vskip -10 pt\hbox{\bext\char'000}}}}
\def\bigrp{\hbox{\vtop{\vskip -10 pt\hbox{\bext\char'001}}}}
\def\biglp{\hbox{\vtop{\vskip -10 pt\hbox{\bext\char'000}}}}
\def\biglb{\hbox{\vtop{\vskip -10 pt\hbox{\bext\char'010}}}}
\def\bigrb{\hbox{\vtop{\vskip -10 pt\hbox{\bext\char'011}}}}
\overfullrule=5pt
%\newcounter{theorem}[chapter]
%\newtheorem{thm}[theorem]{Theorem}
%\newtheorem{lemma}[theorem]{Lemma}
%\newtheorem{cor}[theorem]{Corollary}
%\newtheorem{pro}[theorem]{Proposition}
%\theoremstyle{definition}
%\newtheorem{defin}[theorem]{Definition}
%\newcommand{\supp}{\hbox{\rm supp}}
%\newcommand{\subsupp}{\hbox{\scriptsize\rm supp}}
%\newcounter{parno}

\newtheorem{thm}{Theorem}[chapter]
\newtheorem{lemma}[thm]{Lemma}
\newtheorem{cor}[thm]{Corollary}
\newtheorem{pro}[thm]{Proposition}
\theoremstyle{definition}
\newtheorem{defin}[thm]{Definition}
\newcommand{\supp}{\hbox{\rm supp}}
\newcommand{\subsupp}{\hbox{\scriptsize\rm supp}}
\newcounter{parno}





\addtolength{\voffset}{-.5in} %- .75in is what you use for grad school margins!
\addtolength{\textheight}{1in} %1.25in is what you use for grad school margins!
\addtolength{\textwidth}{.5in}
\renewcommand{\baselinestretch}{1.5}\small\normalsize

\begin{document}

\renewcommand{\baselinestretch}{1}\small\normalsize\bigskip

\pagenumbering{roman}

\begin{titlepage}
\begin{center}
HOWARD UNIVERSITY\\\ \\
{\bf Separating Milliken-Taylor Systems and Variations 
Thereof in the Dyadics and the Stone-\v Cech Compactification 
of ${\bf\mathbb{N}}$}\\\
\\\ \\\ \\A Dissertation\\ Submitted to the Faculty of the\\
Graduate School\\\ \\\ \\\ \\ of \\\ \\\ \\{\bf HOWARD UNIVERSITY}\\\
\\\ \\\  \\in partial fulfillment of\\ the requirements for the \\degree of\\\ \\\ \\\ \\{\bf DOCTOR OF
PHILOSOPHY}\\\ \\\ \\\ \\Department of Mathematics\\\ \\\ \\by\\\ \\\ \\{\bf Kendall Williams}\\\ \\Washington, D.C.\\July 2010
\end{center}
\end{titlepage}





\setcounter{page}{2}

\thispagestyle{plain}




\begin{center}

{\bf HOWARD UNIVERSITY\\GRADUATE SCHOOL\\DEPARTMENT OF MATHEMATICS}\\\ \\DISSERTATION COMMITTEE

\end{center}

\vspace{.5in}

\begin{tabular}{@{}l @{}l}

\hspace{2.5in} & \hspace{2.5in}\ \\ \cline{2-2}

\hspace{2.5in} & Toka Diagana, Ph.D.\\

\hspace{2.5in} & Chairperson\\[.55in] \cline{2-2}

\hspace{2.5in} & Alexander Burstein, Ph.D.\\[.55in]\cline{2-2}

\hspace{2.5in} & Neil Hindman, Ph.D.\\[.55in]\cline{2-2}

\hspace{2.5in} & Paul Peart, Ph.D.\\[.55in]\cline{2-2}

\hspace{2.5in} & Arthur Grainger, Ph.D.\\

\hspace{2.5in} & Associate Professor\\

\hspace{2.5in} & Morgan State University\\[.55in] \cline{1-1}



Neil Hindman, Ph.D.\\

Dissertation Advisor\\

\ \\

Candidate:  Kendall Williams\\

\ \\

Date of Defense:  July 8, 2010

\end{tabular}

\renewcommand{\baselinestretch}{1.5}\small\normalsize\bigskip

\newpage
\begin{center}
{\bf ACKNOWLEDGEMENTS}
\end{center}

I cannot begin to thank Professor Neil Hindman enough for 
all of the contributions that he has made towards my receipt
of my Ph.D. in Mathematics.  Oftentimes, he effortlessly 
bestowed both his mathematical genius and general knowledge 
of life upon me; nonetheless, even when it took him a great
deal of effort, he never failed or hesitated to do so.  Dr.
Hindman has been nothing but patient with me and I will be 
thankful for this as long as I have a Ph.D.  Even when I 
did not understand the simplest things and could not bear to
ask him the same question for the tenth time, he patiently 
waited days, weeks, or even months for me to figure it out on my
own.  He gave a 
speech at his self-proclaimed ``obituary" conference where
he not-so-subtly encouraged his flock of Ph.D. students to
hurry up with our research and finish up our degrees.  (Of
course, anyone who knows Dr. Hindman knows that he pays close
attention to detail and no matter how quickly something was
done, once he puts his stamp on it, it is certified correct.)     
Of all the math I should have been writing down at that 
conference, his comment about our timeliness was the one 
thing I took note of.  Both he and I know that while I may 
not be the brightest crayon in the box, although it did not 
take me long, I could have finished my Ph.D. before I did.  But, 
Dr. Hindman's words did not fall upon deaf ears.  They did 
help me ``get my butt in gear" as he would so eloquently 
put it.  Please do not misconstrue my comments:  Dr. Hindman 
never once seemed especially bothered or even impatient 
about the matriculation time of his students (or anything
else for that matter).
      
The fact that Dr. Hindman takes so many students under his 
wing and provides us with the guidance that we need in so
many aspects of our lives speaks volumes towards his character.

So, I would like to extend my sincerest apologies for my
tardiness (even to classes) and my most heartfelt 
gratitude for Dr. Neil Hindman's patience and most of all 
for his kindness and help.

I would also like to acknowledge the National Science Foundation's
Bridge to the Doctorate Program, the Alfred P. Sloan Foundation's
Scholarship Program, and the GAANN Fellowship for the funding
provided in order for me to complete my degree.


\newpage
\begin{center}
{\bf ABSTRACT}
\end{center}

Given a discrete topological space $S$, we define $\beta S$
to be the set of ultrafilters on $S$.  In this dissertation, we will
be working with the set of dyadic rational numbers.  
Using the notation $\omega = \{0,1,2,\ldots\}$, the set of 
dyadic rationals is defined as follows $\mathbb{D}
=\{{m \over 2^t}:m \in \mathbb{Z} \hbox{ and } t \in \omega\}$.
We shall consider $\mathbb{D}$ as a discrete topological space.  
Namely, we will be considering $\mathbb{D}^+$,
the set of positive numbers contained in $\mathbb{D}$. Note
that $\mathbb{N} = \{1,2,3,...\} \subseteq \mathbb{D}^+$. 
Any number in $\mathbb{D}^+$ (and in turn
in $\mathbb{N}$) has a terminating binary expansion.  Thus,
for any $x \in \mathbb{D}^+$, there exists a unique finite 
nonempty subset $F$ of the integers
such that $x=\sum_{t \in F} {2^t}$.  We will investigate the
algebraic structure of $\beta\mathbb{D}^+$ and also derive
several partition results in $\mathbb{D}^+$.


There is a close knit relationship between the algebraic 
structure of $\beta\mathbb{N}$ and combinatorial or Ramsey 
theoretical partition results in $\mathbb{N}$.  Our focus 
will lie on Milliken-Taylor systems (and variations thereof) 
with respect to both their combinatorial structure and 
algebraic structure in $\beta\mathbb{D}^+$.  Namely, we will
prove results similar to that used in the algebraic proof of
the Finite Sums Theorem.  The difference being that in this
dissertation our ultrafilters will be sums and products of
idempotents and we will be dealing with multiple sequences.  
We will begin with the subsets of elements of our ultrafilters 
being variations of Milliken-Taylor Systems.  As we progress
with our results, these subsets will be progressively more complex.
Further, as Hindman used idempotent results in his algebraic 
proof of the Finite Sums Theorem, we shall use our idempotent 
results (as well as others) to prove analogues of the Finite 
Sums Theorem. 

    



\newpage
%\tableofcontents

\hrule height 0 pt depth 0 pt
\vskip 1 in
\begin{center}
{\bf \Large CONTENTS}
\end{center}
\vskip 1 in
\hbox to \hsize {{\bf Dissertation Committee}\dotfill {\bf ii}}
\bigskip
\hbox to \hsize {{\bf Acknowledgements}\dotfill {\bf iii}}
\bigskip
\hbox to \hsize {{\bf Abstract}\dotfill {\bf iv}}
\bigskip
\hbox to \hsize {{\bf List of Figures}\dotfill {\bf vi}}
\bigskip
\hbox to \hsize {{\bf 1 Introduction}\dotfill {\bf 1}}
\bigskip
\hbox to \hsize {{\bf 2 Separating Milliken-Taylor Systems in ${\mathbb D}$}\dotfill {\bf 10}}
\bigskip
\hbox to \hsize {{\bf 3 Polynomials in $\beta{\mathbb N}$}\dotfill {\bf 28}}
\bigskip
\hbox to \hsize {{\bf Bibliography}\dotfill {\bf 53}}

\newpage


\hrule height 0 pt depth 0 pt
\vskip 1 in
\begin{center}
{\bf \Large List of Figures}
\end{center}
\vskip 1 in
\hbox to \hsize {{\bf Figure 2.1}\dotfill {\bf 15}}

\newpage
\pagenumbering{arabic}

\chapter{Introduction}

\indent

The foundation for the work herein lies in Ramsey
Theory.  Namely, the motivation came from the notion of
finite sums and the Finite Sums Theorems proven by N. Hindman.
There are two versions: the finite Finite Sums Theorem and 
the infinite Finite Sums Theorem better known as Hindman's
Theorem.  Our direct motivation was the infinite version 
which we shall state later.  But, the infinite version was 
motivated by the finite version \cite[Corollary 2.4]{H} 
whose proof relied upon R. Rado's famous results on kernel 
partition regularity.  A 
matrix $A$ is said to be kernel partition regular over a
semigroup $S$ if and only if whenever $S\setminus \{0\}$ is
finitely colored, there exists a monochromatic solution $\vec x$
such that $A\vec x=\vec 0$.  Rado characterized the kernel 
partition regularity of a matrix $A$ by a condition known 
as the columns condition.  See \cite{H} for more detail on 
the finite Finite Sums Theorem and \cite{R} and \cite{R2}
for more detail on Rado's Theorem.

From here on, when referring to the Finite Sums Theorem, we
will be considering the infinite version.  Hindman's original
proof was combinatorial in nature and while elementary
techniques were utilized, the combinatorial proof is quite 
complicated.  There is a much simpler algebraic proof which 
uses the fact that given an element $A$ of an idempotent 
ultrafilter $p$, there exists a sequence whose finite sums
are contained in $A$.  The 
Finite Sums Theorem was utilized by both K. Milliken and A. 
Taylor in proving their separate versions of what is known
as the Milliken-Taylor Theorem.  We are especially interested 
in these results due to the fact that they both were 
motivation for and used in the proofs of some of our results.  
Further, they inspired sets known as Milliken-Taylor Systems,
which we shall utilize in our results. 

Below, we introduce several crucial definitions and
results of \cite{HS} which will be used throughout this dissertation.  
One may refer to \cite{HS} for any necessary supplemental 
information.

 %Define $f: \mathbb{D}^+
%\to \mathbb{D}^+$ by $f(x)=f(\sum_{t \in
%\subsupp(x)}{2^t})=\sum_{t \in -\subsupp(x)}{2^{t}}$.  
%We are especially interested in the fact that $f=f^{-1}$,
%$f[\mathbb{N}]=\mathbb{D} \cap (0,2)$, and $f[\mathbb{D} 
%\cap (0,2)]=\mathbb{N}$.  
%It is easily seen that for each 
%$x_{1},x_{2} \in \mathbb{D}^+$ such that $\min\supp(x_1)>\max\supp(x_2)$,
%$f(x_{1}+x_{2})=f(x_1)+f(x_2)$.  
%We will frequently be working with
%sequences that converge to $0$.  Thus the requirement that
%there is no carrying when adding two terms of a sequence can easily be met by 
%``thinning" our original sequence by passing to a subsequence.  
%Our function $f$ will be utilized in the proofs of some of our
%combinatorial results.  



%A finite compressed sequence $\vec {a}=\langle a_{1},a_{2},
%...,a_{\ell(\vec a)}\rangle$ in $\mathbb{R}$ is a sequence
%where no two consecutive terms are equal.  For a set $A$,
%we will let $\pf(A)$ denote the finite nonempty subsets of
%$A$.  Given a finite compressed sequence $\vec a$ in 
%$\mathbb{R}$ and a sequence $\langle x_t \rangle_{t=1}^\infty$
%in $\mathbb{R}$, the Milliken-Taylor system generated by
%$\vec a$ and $\langle x_t \rangle_{t=1}^\infty$, denoted by
%$MT(\vec {a}, \langle x_t \rangle_{t=1}^\infty)$ is 
%$\{\sum_{i=1}^{\ell(\vec a)}{a_i}\sum_{t \in F_i}{x_t}:
%F_{1}<F_{2}<\ldots<F_{\ell(\vec a)}\}$.

%Let $\vec a$ and $\vec b$ be finite compressed sequences 
%in $\mathbb{N}$ and assume that there is an $s \in \mathbb{Q}^+$
%such that $\vec {a}=s\vec {b}$.  Then, by \cite[Theorem 17.34]{HS},
%whenever $r \in \mathbb{N}$ and $\mathbb{N}=\bigcup_{i=1}^{r}{A_i}$, 
%there exist $i \in \{1,2,...,r\}$ and sequences $\langle x_t \rangle_{t=1}^\infty$
%and $\langle y_t \rangle_{t=1}^\infty$ such that $MT(\vec {a},
%\langle x_t \rangle_{t=1}^\infty) \subseteq A_i$ and $MT(\vec
%{b}, \langle y_t \rangle_{t=1}^\infty) \subseteq A_i$.
%Also, let $\vec a$ and $\vec b$ be finite compressed
%sequences with entries from $\mathbb{Z}\setminus\{0\}$.  Then, 
%by \cite[Theorem 3.1]{HLS}, the following statements are equivalent:
%whenever $r \in \mathbb{N}$ and $\mathbb{Z}=
%\bigcup_{i=1}^{r}{B_i}$, there exists $i \in \{1,2,...,r\}$
%and sequences $\langle x_n \rangle_{n=1}^\infty$ and
%$\langle y_n \rangle_{n=1}^\infty$ in $\mathbb{N}$ with
%$MT(\vec {a}, \langle x_n \rangle_{n=1}^\infty) \cup MT
%(\vec {b}, \langle y_n \rangle_{n=1}^\infty) \subseteq B_i;$ 
%and there is a positive rational $\alpha$ such that
%$\vec {a}=\alpha\vec {b}$.  


%By \cite[Theorems 4.10a, 4.12b, 4.13, 4.14, 5.11]{HS}, respectively,
%the following results hold.
%Let $(S,+)$ be a semigroup.   
%with $A \subseteq S$.  For $s \in
%S$, we define $s^{-1}A$ as $\{t \in S:s+t \in A\}$.
%For any $p,q \in \beta S$, $A \in p+q$ if and only if $\{t
%\in S:t^{-1}A \in q\} \in p$.  For $p \in \beta S$,
%$A^{\star}(p)=\{t \in A:t^{-1}A \in p\}$.  Now let
%$p+p=p \in \beta S$.  Then for every $A \in p$, $A^{\star}
%(p) \in p$.  Further, for each $s \in A^{\star}(p)$, 
%$s^{-1}A^{\star}(p) \in p$.  
%Given a sequence 
%$\langle x_n \rangle_{n=1}^\infty$ in $S$, the finite sums
%of $\langle x_n \rangle_{n=1}^\infty$ is $FS(\langle x_n
%\rangle_{n=1}^\infty)=\{\sum_{t \in F}{x_t}:F \in 
%\pf(\mathbb{N})\}$.  We know that given a semigroup $S$, if
%we let $\langle x_n \rangle_{n=1}^\infty$ be a sequence in
%$S$, there exists an idempotent $p \in \beta S$ such that
%for each $m \in \mathbb{N}$, $FS(\langle x_n 
%\rangle_{n=1}^\infty) \in p$, i.e. $p \in \bigcap_{m=1}^{\infty}
%\overline{FS(\langle x_n \rangle_{n=m}^\infty)}$.  

%From \cite[Theorem 17.31]{HS}, we have the following result: Let $\vec a$ be a
%finite compressed sequence and $\ell(\vec a)$ be the length
%of $\vec a$.  Let $\langle y_t \rangle_{t=1}^\infty$
%be a sequence in $\mathbb{N}$, $p+p=p \in \bigcap_{k=1}^{\infty}
%FS(\langle y_t \rangle_{t=k}^\infty)$, and $A \in a_{1}p+
%a_{2}p+...+a_{\ell(\vec a)}p;$ then there is a sum 
%subsystem $\langle x_t \rangle_{t=1}^\infty$ of $\langle 
%y_t \rangle_{t=1}^\infty$ such that $MT(\vec{a}, \langle 
%x_t \rangle_{t=1}^\infty) \subseteq A$.  



%\begin{defin}\label{HS 17.30}{}\hfill\break
%\begin{itemize}
%\item[(a)] 
%\item[(b)] Given $\vec a\in\mathbb{A}$ with length $m$ and a 
%sequence $\langle x_t\rangle_{t=1}^\infty$ in $\mathbb{N}$, 
%$MT(\vec a,\langle x_t\rangle_{t=1}^\infty)=$
%\begin{eqnarray*}
%\textstyle\{\sum_{i=1}^ma_i\sum_{t\in F_i}~x_t:
%F_1,F_2,\ldots,F_m\in\pf(\mathbb{N})
%\hbox{ and } F_1<F_2<\ldots<F_m\}\ .
%\end{eqnarray*}
%\end{itemize}
%\end{defin}

 

\begin{defin}\label{HS 3.1}
Let $D$ be any set.  A filter on $D$ is a nonempty set $\mathcal U$ 
of subsets of $D$ with the following properties:
\begin{itemize}
\item[(a)] If $A,B \in \mathcal U$, then $A\cap B \in 
\mathcal U$.
\item[(b)] If $A\in\mathcal U$ and $A\subseteq B\subseteq D$,
then $B\in\mathcal U$.
\item[(c)] $\emptyset\notin\mathcal U$.
\end{itemize} 
\end{defin}

An ultrafilter on a set $D$ is a maximal filter on $D$.  A 
topology can be defined on $\beta D$, the set of all 
ultrafilters on a discrete topological space $D$.

\begin{defin}\label{HS 3.15}
Let $D$ be a discrete topological space.
\begin{itemize}
\item[(a)] $\beta{D}=\{p:p \hbox{ is an ultrafilter on } D\}$.
\item[(b)] Given $A\subseteq D$, $\bar{A}=\{p\in\beta{D}:A
\in p\}$
\end{itemize}
\end{defin}

\begin{thebibliography}{9}

\smallskip
\renewcommand{\baselinestretch}{1}\small\normalsize\bigskip

\bibitem{BH} N. Hindman and V. Bergelson, {\it Quotient sets
and density recurrent sets}, preprint.

%\bibitem{DH} D. De, N. Hindman, {\it Image partition regularity
%near zero}, Discrete Math. {\bf 309} (2009), 3219-3232.

\bibitem{DHLL} W. Deuber, N. Hindman, I. Leader, and H. Lefmann, 
{\it Infinite partition regular
matrices}, Combinatorica {\bf 15} (1995), 333-355.

\bibitem{E} R. Ellis, {\it Distal transformation groups},
Pacific J. Math. {\bf 8} (1958), 401-405.

\bibitem{GJ} L. Gillman, M. Jerison, {\it Rings of continuous
functions}, D. Van Nostrand Co., Inc., Princeton, 1960.

\bibitem{H} N. Hindman, {\it Partition regularity of matrices},
Integers {\bf 7(2)} (2007), A-18.

\bibitem{HLS} N. Hindman, I. Leader, D. Strauss, {\it Separating 
Milliken-Taylor Systems with Negative Entries\/}, Proc. Edinburgh Math. 
Soc. {\bf 46} (2003), 45-61.

\bibitem{HS} N. Hindman and D. Strauss, {\it Algebra in the Stone-\v Cech
compactification\/}, De Gruyter, Berlin, 1998.

\bibitem{M} K. Milliken, {\it Ramsey's Theorem with sums or unions},
J. Combin. Theory Ser. A {\bf 18} (1975), 276-290.

\bibitem{R} R. Rado, {\it Studien zur Kombinatorik}, Math. Zeit.
{\bf 36} (1933), 242-280.

\bibitem{R2} R. Rado, {\it Note on combinatorial analysis}, Proc.
London Math. Soc. {\bf 48} (1943), 122-160.

\bibitem{T} A. Taylor, {\it A canonical partition relation for finite
subsets of $\omega$}, J.Combin. Theory Ser. A {\bf 21} (1976), 137-146.

\end{thebibliography}


\end{document}
