% The following is a dissertation template that meets the Howard
% University specifications.

% Howard University requires a 12 point type Times New Roman (or
% Helvetica) font. Before submitting a final draft, change the DRAFT
% option to FINAL and remove the SHOWTRIMS option.

\documentclass[letterpaper, 12pt, oneside, showtrims, draft]{memoir}

% Top, right, bottom, and left margins must be 1 inch each from the
% edge. Using the geometry package with the options 'margin=1in' is a
% convenient way to accomplish this.
\usepackage[margin=1in]{geometry}  

% The amsthm package is used to provide the theorem-type and proof
% environments that make typesetting theorems and definitions easy.
\usepackage{amsthm}

% This modification of the plain theorem style template uses slanted
% fonts instead of italic fonts in the theorem statement. I prefer
% this style since in the older style you can confuse the variables
% with the "regular" text.
\newtheoremstyle{plain}
  {3mm}                         % Space above theorem and previous line.
  {3mm}                         % Space below theorem box and next line.
  {\slshape}                    % Use slanted font in body of theorem.
  {}                            % Indent amount from margin. (Here no indent.)
  {\bfseries}                   % Theorem head font.
  {.}                           % Punctation after theorem head.
  {.5em}                        % Space after theorem head.
  {}                            % Theorem head specification. 

\theoremstyle{plain}

\usepackage{amssymb}
\usepackage{amsmath}
\usepackage{amsfonts}

% Short name for the angle bracket symbols.
\newcommand{\la}{\langle}
\newcommand{\ra}{\rangle}

\usepackage{hu}

\begin{document}

% Howard University requires dissertations to be double-spaced. In
% (La)TeX this looks bad and wastes paper. The following spacing using
% a baselineskip value of 1.7pt is a good compromise. 
\addtolength{\baselineskip}{1.7pt}

\renewcommand{\title}{A new and simpler Central Sets Theorem} 
\renewcommand{\author}{John H.~Johnson}
\newcommand{\monthyear}{July 2011}

\pagenumbering{roman}
% The following is the title page as required by the Graduate School.
\newcommand{\thetitlepage}{
  \clearpage
  \begin{center}
    HOWARD UNIVERSITY \\ \vspace{1em}
    \textbf{\title} \\ \vspace{4em}

    A Dissertation \\
    Submitted to the Faculty of the \\
    Graduate School \\ \vspace{4em}
    
    of \\ \vspace{4em}

    \textbf{HOWARD UNIVERSITY} \\ \vspace{4em}
    
    in partial fulfillment of \\
    the requirements for the \\
    degree of \\ \vspace{4em}

    \textbf{DOCTOR OF PHILOSOPHY} \\ \vspace{4em}

    Department of Mathematics \\ \vspace{2em}
    
    by \\ \vspace{2em}

    \textbf{\author} \\ \vspace{1em}
    
    Washington, DC \\
    \monthyear
    \vfill
  \end{center}
}

% This is a template for the Committee Approval Form.
\newcommand{\approval}{
  \clearpage
  \begin{center}
    \textbf{HOWARD UNIVERSITY} \\
    \textbf{GRADUATE SCHOOL} \\
    \textbf{DEPARTMENT OF MATHEMATICS} \\ \vspace{1em}
    
    DISSERTATION COMMITTEE
  \end{center}

  % Code copied to produced a horizontal line for signatures. 
  \newcommand{\sigline}{\makebox[3in]{\hrulefill}}

  \vspace{4em}
  
  \begin{tabular}{@{}l @{}l}
    \hspace{15em} & \sigline \\
    \hspace{15em} & Toka Diagana, Ph.D. \\
    \hspace{15em} & Chairperson \vspace{4em} \\
    \hspace{15em} & \sigline \\
    \hspace{15em} & Alexander Burstein, Ph.D.  \vspace{4em} \\
    \hspace{15em} & \sigline \\
    \hspace{15em} & Neil Hindman, Ph.D.  \vspace{4em} \\
    \hspace{15em} & \sigline \\
    \hspace{15em} & Arthur Grainger, Ph.D. \vspace{4em} \\
    \hspace{15em} & \sigline \\
    \hspace{15em} & Paul Peart, Ph.D. \\
  \end{tabular}
  
  
  \vfill
}

\thetitlepage
\approval


Bartel Leendert van der Waerden in \cite{Van-der-Waerden:1927fk} proved the following remarkable assertion about arithmetic progressions in the positive integers.


  Let $r$ and $k$ be positive integers.
  Then there exists a positive integer $N$ such that if $\{1, 2, \ldots, N\} = \bigcup_{i=1}^r C_i$, then there exists $i \in \{1, 2, \ldots, r\}$ such that $C_i$ contains a $k$-term arithmetic progression, that is, there exist positive integers $a$ and $d$ with $\{a, a+d, \ldots, a + (k-1)d\} \subseteq C_i$.



\end{document}

