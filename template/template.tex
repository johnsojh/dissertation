% The following is a dissertation template that meets the Howard
% University specifications.

% Howard University requires a 12 point type Times New Roman (or
% Helvetica) font. Before submitting a final draft, change the DRAFT
% option to FINAL and remove the SHOWTRIMS option.

\documentclass[letterpaper, 12pt, oneside, showtrims, draft]{memoir}

% Top, right, bottom, and left margins must be 1 inch each from the
% edge. Using the geometry package with the options 'margin=1in' is a
% convenient way to accomplish this.
\usepackage[margin=1in]{geometry}  

% The amsthm package is used to provide the theorem-type and proof
% environments that make typesetting theorems and definitions easy.
\usepackage{amsthm}

% This modification of the plain theorem style template uses slanted
% fonts instead of italic fonts in the theorem statement. I prefer
% this style since in the older style you can confuse the variables
% with the "regular" text.
\newtheoremstyle{plain}
  {3mm}                         % Space above theorem and previous line.
  {3mm}                         % Space below theorem box and next line.
  {\slshape}                    % Use slanted font in body of theorem.
  {}                            % Indent amount from margin. (Here no indent.)
  {\bfseries}                   % Theorem head font.
  {.}                           % Punctation after theorem head.
  {.5em}                        % Space after theorem head.
  {}                            % Theorem head specification. 

\theoremstyle{plain}

\usepackage{amssymb}
\usepackage{amsmath}
\usepackage{amsfonts}

% Short name for the angle bracket symbols.
\newcommand{\la}{\langle}
\newcommand{\ra}{\rangle}

\usepackage{hu}

\begin{document}

% Howard University requires dissertations to be double-spaced. In
% (La)TeX this looks bad and wastes paper. The following spacing using
% a baselineskip value of 1.7pt is a good compromise. 
\addtolength{\baselineskip}{1.7pt}
\pagenumbering{roman}
\pagestyle{plain}

% Includes title page and committee approval form.
% The following is the title page as required by the Graduate School.
\newcommand{\thetitlepage}{
  \clearpage
  \thispagestyle{empty} 
  \begin{center}
    HOWARD UNIVERSITY \\ \vspace{1em}
    \textbf{Some Differences Between an Ideal in the Stone-\v{C}ech Compactification of Commutative and Noncommutative Semigroups} \\ \vspace{4em}

    A Dissertation \\
    Submitted to the Faculty of the \\
    Graduate School \\ \vspace{4em}
    
    of \\ \vspace{2.5em}

    \textbf{HOWARD UNIVERSITY} \\ \vspace{4em}
    
    in partial fulfillment of \\
    the requirements for the \\
    degree of \\ \vspace{2.5em}

    \textbf{DOCTOR OF PHILOSOPHY} \\ \vspace{2.5em}

    Department of Mathematics \\ \vspace{2em}
    
    by \\ \vspace{2em}

    \textbf{John H.~Johnson} \\ \vspace{1em}
    
    Washington, DC \\
    August 2011 \\
    \vfill
  \end{center}
}

% This is a template for the Committee Approval Form.
\newcommand{\approval}{
  \clearpage
  \begin{center}
    \textbf{HOWARD UNIVERSITY} \\
    \textbf{GRADUATE SCHOOL} \\
    \textbf{DEPARTMENT OF MATHEMATICS} \\ \vspace{1em}
    
    DISSERTATION COMMITTEE
  \end{center}

  % Code copied to produced a horizontal line for signatures. 
  \newcommand{\sigline}{\makebox[3in]{\hrulefill}}

  \vspace{4em}
  
  \begin{tabular}{@{}l @{}l}
    \hspace{15em} & \sigline \\
    \hspace{15em} & Abdul-Aziz Yakubu, Ph.D. \\
    \hspace{15em} & Chairperson \vspace{4em} \\
    \hspace{15em} & \sigline \\
    \hspace{15em} & Alexander Burstein, Ph.D.  \vspace{4em} \\
    \hspace{15em} & \sigline \\
    \hspace{15em} & Neil Hindman, Ph.D.  \vspace{4em} \\
    \hspace{15em} & \sigline \\
    \hspace{15em} & Arthur Grainger, Ph.D. \vspace{4em} \\
    \hspace{15em} & \sigline \\
    \hspace{15em} & Fran\c{c}ois Ramaroson, Ph.D. \vspace{4em} \\
    
    \sigline & \\
    Neil Hindman, Ph.D. & \\
    Dissertation Advisor & \vspace{2em} \\

    Candidate: John H.~Johnson & \vspace{2em} \\
    Date of Defense: $\infty$
  \end{tabular}
  
  \vfill
}

\newcommand{\dedication}{
  \clearpage
  
  \begin{center}
    \textbf{DEDICATION}

    \vspace{3em}

    \textsl{This dissertation is dedicated to my parents for their unending love and tireless support.}
  \end{center}
}

\newcommand{\acknowledgements}{
  \clearpage
  \begin{center}
    \textbf{ACKNOWLEDGEMENTS}
  \end{center}

  I wish to thank my advisor, Dr.~Hindman, for his saintly patience, open, honest and quick communication, and his phenomenal mathematical ideas and guidance. 
  Without his support, this dissertation would not have been written.  
  I hope that I have proved a few theorems here that Dr.~Hindman can be proud of.

  I also want to thank the wonderful (and often unappreciated!) Howard Mathematics Department. 
  (Particularly, Dr.~Ramaroson who gave me my first little taste of mathematics research several years ago.)

  I'm eternally grateful to the Bridge to the Doctorate (Dr.~Lee and the excellent staff help ease my transition into Howard); GK12 (Dr.~Alfred miraculously ran this excellent program himself!); and Preparing Future Faculty programs for tuition and (great!) stipend funding throughout the years.  
  Without these programs, I would not have been able to participate in graduate education!

  I also wish to thank the mathematics department at James Madison University for welcoming me into their department during the 2010--2011 academic year, and providing me with my first `real' job for the 2011--2012 academic year.  

  I also want to thank my fellow Howard graduate students. 
  You all provided many laughs and much heap throughout the years.

  Thank you Camelia for your wonderful and loving support (and patience!) these last three years.

  Finally, I wish to thank my family for just being themselves!
}

\thetitlepage
\dedication
\acknowledgements
\approval

\include{dedication}
\include{acknowledgments}
\include{abstract}

% Include table of contents
\include{toc}

% Include list of figures.
\include{lofig}

% Include list of abbreviations.
\include{loabbr}

% Include list of symbols.
\include{losym}

\pagenumbering{arabic}
% \documentclass[12pt]{article}

\usepackage{amsthm, amssymb, amsmath}
\usepackage{color}
\usepackage{endnotes}
\usepackage{url}


\usepackage[margin=1in]{geometry}

\usepackage[doublespacing]{setspace}
\usepackage{url}

\newtheoremstyle{plain}{3mm}{3mm}{\slshape}{}{\bfseries}{.}{.5em}{}
\theoremstyle{plain}

% Numbered theorems 
\newtheorem{thm}{Theorem}[section]
\newtheorem{lem}[thm]{Lemma}
\newtheorem{prop}[thm]{Proposition}
\newtheorem{cor}[thm]{Corollary}
\newtheorem{up}[thm]{Ultrafilter Principle}
\newtheorem{radoSelect}[thm]{Rado's Selection Lemma}
\newtheorem{php}[thm]{Pigeonhole Principle}

% Unnumbered named theorems or results
\newtheorem*{fact}{Fact}
\newtheorem*{hj}{Hales-Jewett Theorem}

\newtheorem*{ramsey}{Ramsey's Theorem}
\newtheorem*{vdw}{Van der Waerden's Theorem}
\newtheorem*{schur}{Schur's Theorem}

\newtheorem{FST}[thm]{Hindman's Theorem}
\newtheorem{MBR}[thm]{Multiple Birkhoff Recurrence Theorem}
\newtheorem{recur}[thm]{Recurrence Theorem}
\newtheorem{OCST}[thm]{Furstenburg's Original Central Sets Theorem}
\newtheorem{cst}[thm]{Central Sets Theorem}



\newtheorem{claim}[thm]{Claim}
\newtheorem{ques}[thm]{Question}
\newtheorem{conj}[thm]{Conjecture}


\theoremstyle{definition}

% Numbered "definition" style theorem environments
\newtheorem{defn}[thm]{Definition}
\newtheorem{rmk}[thm]{Remark}
\newtheorem{example}[thm]{Example}

\newcommand{\la}{\langle}
\newcommand{\ra}{\rangle}
\newcommand{\bbN}{\mathbb{N}}
\newcommand{\bbZ}{\mathbb{Z}}
\newcommand{\bbR}{\mathbb{R}}
\newcommand{\AP}{\mathcal{AP}}
\newcommand{\AL}{\mathcal{AL}}

% Short names for calligraphic math letters.
\newcommand{\calA}{\mathcal{A}}
\newcommand{\calB}{\mathcal{B}}
\newcommand{\calC}{\mathcal{C}}
\newcommand{\calE}{\mathcal{E}}
\newcommand{\calF}{\mathcal{F}}
\newcommand{\calG}{\mathcal{G}}
\newcommand{\calH}{\mathcal{H}}
\newcommand{\calI}{\mathcal{I}}
\newcommand{\calJ}{\mathcal{J}}
\newcommand{\calP}{\mathcal{P}}
\newcommand{\calR}{\mathcal{R}}
\newcommand{\calS}{\mathcal{S}}
\newcommand{\calT}{\mathcal{T}}
\newcommand{\calU}{\mathcal{U}}

\newcommand{\Pf}{\mathcal{P}_f}


\newcommand{\setfunc}[2]{\hbox{${}^{\hbox{$#1$}}\hskip -1 pt #2$}}

\font\bigmath=cmsy10 scaled \magstep 3
\newcommand{\bigtimes}{\hbox{\bigmath \char'2}}

\newcommand{\cchi}{\raise 2 pt \hbox{$\chi$}}

\begin{document}
Cardinality is commonly thought of as a ``measure'' on the size of sets.
Intuitively this ``notion of largeness'' provides a good mathematical formalization of size since every set has a unique cardinal number; and under ZFC (the usual Zermelo-Fraenkel axioms of set theory along with the axiom of choice) there is a fixed nontrivial order relation such that every set of cardinals is wellordered.
However the concept of cardinality has its own mathematical peculiarities as a notion of largeness.
One peculiarity is starkly illustrated by the formal independence of the Continuum Hypothesis from ZFC (provided ZFC is a consistent theory).
Recall that the Continuum Hypothesis is the assertion that $|\bbR|$ is the first cardinal after $|\bbN|$.
The independence of this statement means that, without adding extra set-theoretical axioms, we cannot determine the precise location of the cardinal $|\bbR|$ in the class of all cardinals.

Of course by using the well-known diagonal argument of Cantor%
\endnote{
  Interestingly Grattan-Guinness's observation in \cite[page 134, footnote 1]{Grattan-Guinness:1978kx} implies that Paul de Bois-Reymond was the first to publish a diagonal-type argument. 
}
we can easily prove the weaker assertion that $|\bbR|$ is strictly greater than $|\bbN|$.
The point we wish to emphasize is that questions about relative sizes are often easier to study than questions about absolute sizes.
(We can summarize this observation into an elliptic misleading proverb: qualitative questions are easier to answer than quantitive questions.)
Unfortunately, another peculiarity of cardinality is that the concept is a somewhat blunt tool to use in the study of relative sizes.
Therefore in this chapter we start by introducing two different (but ultimately related) notions of largeness that are more amendable to answering certain interesting questions on relative sizes.

To get an idea of what we will be studying in this chapter, let's start by considering some fixed set $X$.
We want to produce a reasonable definition for a collection $\calF$ of subsets of $X$ such that members of $\calF$ are considered as ``relatively large'' subsets of $X$.
We ask what properties should $\calF$ possess?
To answer this question we perform a bit of intuitive reasoning.

Naturally, we would like to consider $X$ to be a large subset of itself, and therefore it seems reasonable to require that $X \in \calF$.
If a subset of $X$ contains a large subset, then we should also consider the containing subset as large itself.
In terms of $\calF$ this says that if $A \in \calF$ and $A \subseteq B \subseteq X$, then $B \in \calF$. 
The requirement that $\calF$ be closed under supersets places some restrictions on any reasonable properties $\calF$ may have.
For instance, if $\emptyset \in \calF$, then $\calF = \calP(X)$. 
($\calP(X)$ is the usual power set of $X$, that is, the collection of all subsets of $X$.)
Since it seems slightly perverse to have a collection of large subsets of $X$ to potentially be equal to $\calP(X)$, we further require that $\emptyset \not\in \calF$. 
Furthermore, since $X \in \calF$ and $\emptyset \not\in \calF$, we also require that $X$ be nonempty.

The requirements that $\calF$ be nonempty, doesn't contain the empty set, and is closed under supersets forms the core of several definitions we shall consider throughout this chapter. 

\section{Filters and Ultrafilters}
Since the material in this section is mostly standard and can be found in several places (for instance see \cite[Chapter 5]{Schechter:1997fk} or \cite[Chapter 3]{Hindman:1998fk}) we choose to leave most of the proofs of our assertions to the reader.
% TODO: Add some references to the results we state.
\begin{defn}
  \label{defn:filters}
  Let $X$ be a nonempty set.
  We call $\calF \subseteq \calP(X)$ a \textsl{filter on $X$} if and only if $\calF$ satisfies the following three conditions:
  \begin{itemize}
    \item[(1)] $\emptyset \ne \calF$ and $\emptyset \not\in\calF$.
    \item[(2)] If $A \in \calF$ and $A \subseteq B \subseteq X$, then $B \in \calF$.
    \item[(3)] If $A$ and $B$ are elements of $\calF$, then $A \cap B \in \calF$.
  \end{itemize}
\end{defn}
\begin{rmk}
  Intuitively we may think of elements of a filter as simply a collection of relatively large subsets of some fixed set.
  Conditions (1) and (2) of a filter nicely align with this intuition, but condition (3) may at first look a litte strange.
  This condition may be thought of as saying that we require our large sets to interlock in a highly nontrivial way.%
  \endnote{
    Filters, filter bases, and ultrafilters were introduced in two notes of Cartan, \cite{Cartan:1937vn} and \cite{Cartan:1937ys}, as one way to generalize the use of arguments based on sequences in metric spaces to topological spaces. 
    However, see Sundstr\"{o}m's article \cite[Section 4.2]{Sundstrom:2010zr} for some history and references to others that discovered the concepts of filters and ultrafilters independently. 
  }
\end{rmk}

\begin{example}
  \label{ex:prinFilt}
  Let $X$ be a nonempty set.
  If $\emptyset \ne A \subseteq X$, then the set $\calE(A) = \{\, B \subseteq X : A \subseteq B \,\}$ is a filter on $X$.
  If $A$ is a singleton set, say $A = \{a\}$, then we write $\calE(A)$ as $\calE(a)$. 
  We call such filters \textsl{principal filters}.
\end{example}

Using condition (3) in Definition \ref{defn:filters}, a simple argument shows that every filter on a nonempty finite set is necessarily a principal filter.
In this dissertation we wil adopt the bias that principal filters are essentially trivial or well-known objects.%
\endnote{
  The mathematical truth of the matter is that in fact principal filters are highly \textsl{non-trivial} objects.
  The easiest way to see this is to observe that if $\calF$ is a filter on a discrete space $X$ and $\overline{\calF} = \{\, p \in \beta X : \calF \subseteq p \,\}$, then the set $\{\, \overline{\calF} : \mbox{$\calF$ is a principal filter} \,\}$ is a clopen basis for $\beta X$.
}
Given this bias we shall primarily be concerned with filters on infinite sets.
Accordingly, our next example shows that there is at least one nonprincipal filter%
\endnote{
  To avoid the negative adjective `nonprincipal' to filters some mathematicians call nonprincipal filters \textsl{free filters}.
}
on every infinite set.

\begin{example}
  Let $X$ be an infinite set, then $\calC =\{\, A \subseteq X : \mbox{$X \setminus A$ is finite} \,\}$ is a nonprincipal filter on $X$.
  (To see that this filter is nonprincipal simply observe that for all $x \in X$, $X \setminus \{x\} \in \calC$ and hence $\bigcap\calC = \emptyset$.) 
  We call this filter the \textsl{cofinite filter} or \textsl{Fr\'{e}chet filter}.
\end{example}
\begin{rmk}
  It is a fact that all nonprincipal filters contain the cofinite filter. 
  See \cite[Section 5.5(e), page 103]{Schechter:1997fk} for a collection of equivalent statements to this fact. 
  (In the referenced statements it's probably easier to prove $(B) \implies (A)$, $(A) \implies (C)$, $(C) \implies (D)$, and $(D) \implies (B)$.)
\end{rmk}

The relation $\subseteq$ is a natural partial ordering on the collection of all filters on a nonempty set. 
In analogy with coarser and finer topologies on a set, we make the following definition about coarser and finer filters.

\begin{defn}
  Let $\calF_1$ and $\calF_2$ be filters on $X$.
  We say that $\calF_1$ is \textsl{coarser than} $\calF_1$ or $\calF_2$ is \textsl{finer than} $\calF_1$ if and only if $\calF_1 \subseteq \calF_2$.
\end{defn}

Since the set of all filters on a set $X$ is partially ordered, it is natural to wonder about minimal and maximal elements with respect to this partial ordering.
There is only one minimal filter, and happily it is contained in every filter, $\{X\}$.
There is no largest filter, that is, there is no filter that contains every filter%
\endnote{
  Some mathematicians define a filter as follows:
  \begin{defn}
    Let $X$ be a nonempty set and $\calF \subseteq \calP(X)$.
    We call $\calF$ a \textsl{filter} if and only if $\calF$ satisfies the following three conditions:
    \begin{itemize}
      \item[(1)] $\emptyset \ne \calF$.
      \item[(2)] If $A$ and $B$ are elements of $\calF$, then $A cap B \in \calF$.
      \item[(3)] If $A \in \calF$ and $A \subseteq B \subseteq X$, then $B \in \calF$.
    \end{itemize}
    We call a filter $\calF$ \textsl{proper} if $\emptyset \in \calF$.
  \end{defn}
  With this definition the only improper filter is $\calP(X)$ and this filter is also the largest filter in the poset of all filters on $X$.
  Of course if we consider the poset of all proper filters on $X$, then there is no largest proper filter.
}%
, but the situation for the existence of maximal filters is mathematically interesting (or mathematically worrying depending on your point-of-view).
To befit this added complication we give maximal filters a special name.

\begin{defn}
  A filter $\calU$ on $X$ is called an \textsl{ultrafilter on $X$} if and only if $\calU$ is a \mbox{$\subseteq$-maximal} filter.
\end{defn}

To even show that any ultrafilters exist we take a small, but important, detour to show how various collections of sets can be used to generate filters.

\begin{defn}
  Let $X$ be a nonempty set.
  \begin{itemize}
    \item[(a)] We call $\calB \subseteq \calP(X)$ a \textsl{filter base on $X$} if and only if $\calB$ satisfies the following two conditions:
    \begin{itemize}
      \item[(1)] $\emptyset \ne \calB$ and $\emptyset \not\in \calB$.
        
      \item[(2)] If $A$ and $B$ are elements of $\calB$, then there exists $C \in \calB$ such that $C \subseteq A \cap B$.
    \end{itemize}

    \item[(b)] We call $\calS \subseteq \calP(X)$ a \textsl{filter subbase on $X$} or $\calS$ has the \textsl{finite intersection property} (abbreviated f.i.p.) if and only if for every nonempty finite subset $\calA \subseteq \calS$ we have $\bigcap \calA \ne \emptyset$.

    \item[(c)] Given a collection $\calA \subseteq \calP(X)$, put 
      \[
         \calA^\uparrow = \{\, B \subseteq X : \mbox{$A \subseteq B$ for some $A \in \calA$} \,\}.
      \]
  \end{itemize}
\end{defn}

\begin{prop}
  \label{prop:fltBase}
  Let $\calB$ be a filter base on $X$, then the set
  \[
    \calB^\uparrow = \{\, A \subseteq X : \mbox{$B \subseteq A$ for some $B \in \calB$} \,\}
  \]
  is a filter on $X$.
  Moreover, this is the coarsest filter that contains $\calB$.
  We call this filter the \textsl{filter generated by the base $\calB$}.
\end{prop}

\begin{prop}
  Let $\calS$ be a filter subbase on $X$, then the set
  \[
     \calB = \bigl\{\, \bigcap \calA : \mbox{$\emptyset \ne \calA \subseteq \calS$ is finite} \,\bigr\}
  \]
  is a filter base on $X$. 
\end{prop}
\begin{cor}
  \label{cor:fltSubbase}
  Let $\calS$ be a filter subbase on $X$, then the set
  \[
    \{\, A \subseteq X : \mbox{$\bigcap\calA \subseteq A$ for some finite $\emptyset \ne \calA \subseteq \calS$} \,\}
  \]
  is a filter on $X$.
  Moreover, this is the coarsest filter that contains $\calS$.
  We call this filter the \textsl{filter generated by the subbase $\calS$}.
\end{cor}
\begin{rmk}
  Since a filter subbbase is often easier to describe than a filter base, we often only use Corollary \ref{cor:fltSubbase} directly when constructing new filters.
\end{rmk}

With this little detour done, we now resume our study of ultrafilters.

\begin{thm}
  \label{thm:equivUf}
  Let $\calU$ be a filter on $X$.
  The following statements are equivalent.
  \begin{itemize}
    \item[(a)] $\calU$ is an ultrafilter on $X$.
    \item[(b)] For every $A \subseteq X$, either $A \in \calU$ or $X \setminus A \in \calU$.
    \item[(c)] For every $A$, $B \subseteq X$, if $A \cup B \in \calU$, then either $A \in \calU$ or $B \in \calU$.
  \end{itemize}
\end{thm}
\begin{proof}
  (a) $\Rightarrow$ (b)
  Suppose that $A \not\in \calU$.
  If $\calU \cup \{A\}$ is a filter subbase, then by Corollary
  \ref{cor:fltSubbase} we can generate a filter that contains $\calU$
  and $A$.
  However since $\calU$ is maximal, this would imply that $A \in
  \calU$, a contradiction.
  Therefore $\calU \cup \{A\}$ is not a filter subbase and so there
  must exists $B \in \calU$ such that $B \cap A = \emptyset$, that is,
  $B \subseteq X \setminus A$.
  It follows that $X \setminus A \in \calU$.
  
  (b) $\Rightarrow$ (c)
  If $A \not\in \calU$ and $B \not\in \calU$, then by assumption $X
  \setminus A \in \calU$ and $X \setminus B \in \calU$.
  Then $X \setminus (A \cup B) = (X \setminus A) \cap (X \setminus B)$
  would be an element of $\calU$.
  However this implies that $\emptyset = \bigl(X \setminus (A \cup
  B)\bigr) \cap (A \cup B) \in \calU$, a contradiction.

  (c) $\Rightarrow$ (a)
  Let $\calF$ be a filter on $X$ with $\calU \subseteq \calF$.
  Let $A \in \calF$.
  Since $\calU \subseteq \calF$, it follows that $X \setminus A
  \not\in \calU$.
  However, $A \cup (X \setminus A) = X \in \calU$ and by assumption it
  follows that $A \in \calU$.
  Therefore $\calU = \calF$ and hence $\calU$ is a maximal filter.
\end{proof}
\begin{rmk}
  A more complete list of statements equivalent to the definition of an ultrafilter can be found in \cite[Theorem 3.6]{Hindman:1998fk}.
\end{rmk}

\begin{example}
  Let $X$ be a nonempty set and $a \in X$.
  Recall that $\calE(a) = \{\, A \subseteq X : a \in A \,\}$.
  Then $\calE(a)$ is an ultrafilter on $X$.
  The fact that $\calE(a)$ is a filter was first mentioned in Example  \ref{ex:prinFilt}, and the fact that $\calE(a)$ is an ultrafilter
  follows easily from Theorem \ref{thm:equivUf}(b).
  We call such ultrafilters \textsl{principal ultrafilters}.
\end{example}

We will also adopt the bias that principal ultrafilters are essentially trivial or well-known objects.
In contrast to the situation with nonprincipal filters the existence of nonprincipal ultrafilters is sensitive to the underlying axioms of our particular set theory.
Without some reasonable strong version of the axiom of choice there may be no nontrivial ultrafilters at all.
It is a fact that there are models of ZF where no nonprincipal ultrafilters exists \cite{Blass:1977fk}.
The stronger statement that every filter can be extended to an ultrafilter follows from our next result and Zorn's Lemma.%
\endnote{
  More precisely the Ultrafilter Principle is stronger than the statement that every infinite set has a nonprincipal ultrafilter. 
  Rav's paper \cite[Section 2]{Rav:1977ys} contains several interesting collection of statements equivalent to the Ultrafilter Principle which are essentially combinatorial in character. 
  All of these statements are variants of Rado's Selection Lemma.
  (Rado's Selection Lemma is a combinatorial result used by Rado in \cite[Lemma 1]{Rado:1949fk}. 
  See \cite{Gottschalk:1951uq} for an easy proof of this selection lemma via Tychonoff's Theorem.)
}

\begin{lem}
  \label{lem:chainFlt}
  Let $\Phi$ be a collection of filters on a set $X$.
  \begin{itemize}
    \item[(a)] $\bigcap\Phi$ is a filter on $X$.

    \item[(b)] If $\Phi$ is a \mbox{$\subseteq$-chain}, then $\bigcup\Phi$ is a filter on $X$.
  \end{itemize}
\end{lem}

\begin{up}
  Every filter is contained in an ultrafilter.
\end{up}
\begin{proof}[Proof Sketch]
  Let $X$ be a set and $\calF$ a filter on $X$.
  Let $\Phi$ be the collection of all filters on $X$ that are finer than $\calF$.
  By Lemma \ref{lem:chainFlt} and Zorn's Lemma, $\Phi$ contains a maximal element.
\end{proof}
\begin{rmk}
  To produce a nonprincipal ultrafilters we simply apply the Ultrafilter Principle to the cofinite filter.
  It is a fact that every ultrafilter is nonprincipal if and only if it contains the cofinite filter.
\end{rmk}

In order to produce a filter it is necessary and sufficient to produce a collection with f.i.p.
However as a practical matter it may be inconvenient or difficult to verify that a collection of sets has f.i.p.
Therefore in the next section we give some of the basic definitions and results around a concept ``dual'' to the notion of filters that will essentially make producing filters trivial.

\section{Grills}
The dual concept to filters that we will introduce in this section is due to a note of Choquet in \cite{Choquet:1947uq}. 
Since Choquet's note doesn't contain any proofs, and since the concepts he lays out appear to be slightly less well-known than the concept of a filter, in this subsection we choose to give complete proofs for (most of)%
\endnote{
  We have been unable to verify (or falsify!) two of Choquet's assertions.
  One assertion is in the last paragraph of section II, and the other assertion is in section IV of \cite{Choquet:1947uq}.
}
Choquet's assertions.

\begin{defn}
  \label{defn:grill}
  Let $X$ be a nonempty set.
  We call $\calG \subseteq \calP(X)$ a \textsl{grill on $X$} if and only if $\calG$ satisfies the following three conditions:
      \begin{itemize}
        \item[(1)] $\emptyset \ne \calG$ and $\emptyset \not\in \calG$.

        \item[(2)] If $A \in \calG$ and $A \subseteq B \subseteq X$, then $B \in \calG$.

        \item[(3)] If $A \in \calG$ and $B \not\in \calG$, then $A \setminus B \in \calG$. 
      \end{itemize}
\end{defn}
\begin{rmk}
  Again we can regard a grill as a collection of relatively large subsets of $X$. 
  Conditions (1) and (2) align with our intuition, but condition (3) can be thought of as stating that we can throw away a portion of a large set provided that the part we throw away is not large itself.
  Proposition \ref{prop:alt3} shows that condition (3) is equivalent to saying that large sets cannot be finitely decomposed into a union of non-large sets.
\end{rmk}

\begin{prop}
  \label{prop:alt3}
  Let $X$ be a nonempty set and $\calG \subseteq \calP(X)$.
  Then $\calG$ is a grill on $X$ if and only if $\calG$ satisfies the following three conditions:
  \begin{itemize}
    \item[(1)] $\emptyset \ne \calG$ and $\emptyset \not\in \calG$.

    \item[(2)] If $A \in \calG$ and $A \subseteq B \subseteq X$, then $B \in \calG$.

    \item[(3)] For all $A$, $B \subseteq X$, if $A \cup B \in \calG$, then either $A \in \calG$ or $B \in \calG$. 
  \end{itemize}
\end{prop}
\begin{proof}
  We only focus on proving that condition (3) in our Proposition is equivalent to condition (3) in Definition \ref{defn:grill}(a).

  ($\Rightarrow$)
  Let $A$ and $B$ be subsets of $X$ with $A \cup B \in \calG$.
  Suppose that $A \not\in \calG$, then by Definition \ref{defn:grill}(a) $(A \cup B) \setminus A \in \calG$.
  Since $(A \cup B) \setminus A \subseteq B$, condition (2) of Definition \ref{defn:grill}(a) implies that $B \in \calG$.

  ($\Leftarrow$)
  Let $A$ and $B$ be subsets of $X$ with $A \in \calG$ and $B \not\in \calG$.
  Observe that since $B \not\in \calG$ we have $A \cap B \not\in G$.
  Now $(A \setminus B) \cup (A \cap B) = A \in \calG$ and so by assumption we must have $A \setminus B \in \calG$.
\end{proof}

\begin{prop}
  \label{prop:FltGrl}
  Let $\calF$ be a filter on a set $X$ and put $G(\calF) = \{\, A \subseteq X : X \setminus A \not\in \calF \,\}$.
  \begin{itemize}
    \item[(a)] $G(\calF)$ is a grill on $X$.
      We call the grill $G(\calF)$ the grill \textsl{associated with the filter $\calF$}.
    \item[(b)] $G(\calF) = \{\, A \subseteq X : A \cap B \ne \emptyset \mbox{ for all $B \in \calF$} \,\}$.
    \item[(c)] $G(\calF) = \bigcup\{\, \calF' : \mbox{$\calF'$ is a finer filter than $\calF$} \,\}$.
  \end{itemize}
\end{prop}
\begin{proof}
  For notational convenience put $\calG = G(\calF)$.
 
  (a)
  To see that $\emptyset \ne \calG$ observe that $\calF \subseteq \calG$.
  Also $\emptyset \not\in \calG$ since $X \setminus \emptyset = X \in \calF$.

  Now let $A \in \calG$ and $A \subseteq B \subseteq X$.
  If $X \setminus B \in \calF$, then we would have $X \setminus A \in \calF$ too.
  Therefore we have that $B \in \calG$.

  Instead of showing condition (3) of Definition \ref{defn:grill}(a) we prove that $\calG$ satisfies condition (3) of Proposition \ref{prop:alt3}.
  Let $A \cup B \in \calG$, then $X \setminus (A \cup B) \not\in \calF$.
  If $X \setminus A \in \calF$ and $X \setminus B \in \calF$, then $(X \setminus A) \cap (X \setminus B) \in \calF$.
  Therefore either $X \setminus A \not\in \calF$ or $X \setminus B \not\in \calF$, that is, either $A \in \calG$ or $B \in \calG$.

  (b)
  Let $A \in \calG$ and $B \in \calF$.
  If $A \cap B = \emptyset$, then $B \subseteq X \setminus A$ and so $X \setminus A \in \calF$, a contradiction. 
  Therefore $A \cap B \ne \emptyset$.

  Now let $A \subseteq X$ such that for all $B \in \calF$, $A \cap B \ne \emptyset$.
  Then it follows immediately that $X \setminus A \not\in \calF$, that is, $A \in \calG$.

  (c)
  Let $A \in \calG$, then by (b) $\calF \cup \{A\}$ has f.i.p. and so there exists a filter $\calF'$ that contains $\calF \cup \{A\}$.
  
  Now let $\calF'$ be a finer filter than $\calF$ and let $A \in \calF'$. 
  Since $\calF \subseteq \calF'$ and $\emptyset \ne \calF'$ we have that for all $B \in \calF$, $A \cap B \ne \emptyset$.
  Hence by applying (b) we are done.
\end{proof}
\begin{rmk}
  From Proposition \ref{prop:FltGrl} (b) or (c) we see that the grill associated with a given filter is simply the collection of all sets we can use to form a filter finer than our given filter.
  However it follows from Proposition \ref{prop:duality}, and working with simply examples shows, that as we generate finer and finer filters, the corresponding grills become coarser.
\end{rmk}

\begin{prop}
  \label{prop:GrlFlt}
  Let $\calG$ be grill on $X$ and put $F(\calG) = \{\, A \in \calG : X \setminus A \not\in \calG \,\}$.
  \begin{itemize}
    \item[(a)] Then $F(\calG)$ is a filter on $X$.
      We call the filter $F(\calG)$ the \textsl{filter associated with the grill $\calG$}.
    \item[(b)] $F(\calG) = \{\, A \in \calG : \mbox{$A \cap B \in \calG$ for all $B \in \calG$} \,\}$.
  \end{itemize}
\end{prop}
\begin{proof}
  For notational convenience put $\calF = F(\calG)$.

  (a)
  Since $X \setminus X = \emptyset \not\in \calG$ we have that $X \in \calF$ and so $\emptyset \ne \calF$.
  Also since $X \setminus \emptyset = X \in \calG$, we have that $\emptyset \not\in \calF$.

  Now let $A$ and $B$ be elements of $\calF$, that is, $X \setminus A$ and $X \setminus B$ are not in $\calG$.
  From Proposition \ref{prop:alt3} it follows that $X \setminus (A \cap B) \not\in \calG$, that is, $A \cap B \in \calF$. 

  Let $A \in \calF$ and $A \subseteq B \subseteq X$.
  Since $X \setminus B \subseteq X \setminus A$ and $X \setminus A \not\in \calG$, it follows from Definition \ref{defn:grill}(a) that $X \setminus B \not\in \calG$, that is, $B \in \calF$.

  (b)
  Let $A \in \calF$ and $B \in \calG$.
  Suppose that $A \cap B \not\in \calG$, then $X \setminus (A \cap B) \in \calF$.
  Hence $\bigl( (X \setminus A) \cup (X \setminus B) \bigr) \cap A \in \calF$, that is, $\emptyset \cup \bigl( (X \setminus B) \cap A) \bigr) \in \calF$.
  However $(X \setminus B) \cap A \in \calF$ if and only if $X \setminus \bigl( X \setminus (B \cap A) \bigr) \not\in \calG$, that is, $B \cup (X \setminus A) \not\in \calG$.
  But this is a contradiction since $B \in \calG$ and $B \subseteq B \cup (X \setminus A)$.
  
  Now let $A \in \calG$ such that $A \cap B \in \calG$ for all $B \in \calG$.
  Since $\emptyset \not\in \calG$ it follows that $X \setminus A \not\in \calG$, that is, $A \in \calF$.
\end{proof}

We now give an easy result that asserts that the functions $G$ and $F$ implicitly defined in Propositions \ref{prop:FltGrl} and \ref{prop:GrlFlt} are one-to-one from the set of all filters onto the set of grills.

\begin{prop}
  Let $X$ be a nonempty set.
  \begin{itemize}
    \item[(a)] If $\calG$ is a grill on $X$, then $\calG = G\bigl(F(\calG)\bigr)$.
    
    \item[(b)] If $\calF$ be a filter on $X$, then $\calF = F\bigl(G(\calF)\bigr)$.
  \end{itemize}
\end{prop}

\begin{defn}
  Let $\calF$ be a filter on $X$ and let $\calG$ be a grill on $X$.
  We say $\calF$ and $\calG$ are \textsl{associated} if and only if $\calF = F(\calG)$ or $\calG = G(\calF)$. 
\end{defn}

With the relation of association we have the following duality type result for filters and grills.

\begin{prop}
  \label{prop:duality}
  Let the filter $\calF_1$ be associated with the grill $\calG_1$, and let the filter $\calF_2$ be associated with the grill $\calG_2$.
  Then $\calF_1 \subseteq \calF_2$ if and only if $\calG_2 \subseteq \calG_1$.
\end{prop}
\begin{proof}
  ($\Rightarrow$)
  Let $A \in \calG_2$, that is, $X \setminus A \not\in \calF_2$.
  Since $\calF_1 \subseteq \calF_2$ we have $X \setminus A \not\in \calF_1$ too.
  Therefore $A \in \calG_1$.

  ($\Leftarrow$)
  Let $A \in \calF_1$.
  If $A \not\in \calF_2$, then $X \setminus A \in \calG_2$ and (by assumption) $X \setminus A \in \calG_1$, that is, $A \not\in \calF_1$, a contradiction.
  Therefore $A \in \calF_2$.
\end{proof}

\begin{prop}
  A grill $\calG$ is a filter if and only if $F(\calG)$ is an ultrafilter.
\end{prop}
\begin{proof}
  ($\Rightarrow$)
  Let $A \subseteq X$.
  We show that either $A \in F(\calG)$ or $X \setminus A \in F(\calG)$.
  Suppose that $A \not\in F(\calG)$, that is, assume that either $A \not\in \calG$ or $A \in \calG$ and $X \setminus A \in \calG$.
  However by hypothesis, $\calG$ is a filter and so we cannot have both $A \in \calG$ and $X \setminus A \in \calG$.
  Since $A \not\in \calG$ and $X \in \calG$ we have that $X \setminus A \in \calG$ and so $X \setminus A \in F(\calG)$.

  ($\Leftarrow$)
  It suffices to show that $\calG$ is closed under finite intersections.
  To this end let $A$ and $B$ be elements of $\calG$.
  However suppose that $A \cap B \not\in \calG$.
  If $A \cap B \in F(\calG)$, then $A \cap B \in \calG$.
  Hence we can assume that $A \cap B \not\in F(\calG)$.
  Therefore since $F(\calG)$ is an ultrafilter we have that $(X \setminus A) \cup (X \setminus B) = X \setminus (A \cap B) \in F(\calG)$.
  Also since $F(\calG)$ is an ultrafilter either $X \setminus A \in F(\calG)$ or $X \setminus B \in F(\calG)$. 
  This last sentence implies that either $A \not\in \calG$ or $B \not\in \calG$, a contradiction.
\end{proof}
\begin{cor}
  Let $\calU$ be a filter.
  $\calU$ is an ultrafilter if and only if $\calU = G(\calU)$.
\end{cor}
\begin{proof}
  Let $\calU$ be an ultrafilter.
  It's immediate that $\calU \subseteq G(\calU)$.
  Let $A \in G(\calU)$, that is, $X \setminus A \not\in \calU$. 
  Since $X \setminus A \not\in \calU$ we must have $A \in \calU$.

  Now suppose that $\calU$ is a filter with $\calU = G(\calU)$.
  Then $\calU = F\bigl(G(\calU)\bigr)$ is an ultrafilter.
\end{proof}

We now shift our attention to grill bases.
While grill bases are easier to describe then grills we shall see that in Proposition \ref{prop:GrillB}(c) that there is generally no coarsest grill that contains a grill base.
(This negative result is in contrast to the fact that there is a coarsest filter that contains a filter base).

\begin{defn}
  Let $X$ be a nonempty set.
  \begin{itemize}
    \item[(b)] We call $\calB \subseteq \calP(X)$ a \textsl{grill base on $X$} if and only if $\calB$     satisfies the following two conditions:
      \begin{itemize}
        \item[(1)] $\emptyset \ne \calB$ and $\emptyset \not\in \calB$.

        \item[(2)] If $A \in \calB$ and $A \subseteq B \subseteq X$, then $B \in \calB$.
      \end{itemize}

    \item[(c)] We call $\calS \subseteq \calP(X)$ a \textsl{grill subbase} if and only if $\emptyset \ne \calS$ and $\emptyset \not\in \calS$. 
  \end{itemize}
\end{defn}

\begin{prop}
  \label{prop:grlBasePhi}
  Let $\calB$ be a grill base on $X$ and put $\Phi = \{\, \calF : \mbox{$\calF \subseteq \calB$ is a filter} \,\}$.
  Then $\Phi$ is nonempty, partially ordered by $\subseteq$, and every \mbox{$\subseteq$-chain} in $\Phi$ has an upper bound in $\Phi$.
\end{prop}
\begin{proof}
  Since $X \in \calB$ and $\{X\}$ is a filter contained in $\calB$ we have $\{X\} \in \Phi$.
  $\Phi$ is obviously partially ordered by $\subseteq$.
  Let $\calC$ be a chain in $\Phi$, then $\bigcup\calC$ is a filter by Lemma \ref{lem:chainFlt} and $\bigcup\calC \subseteq \calB$.
\end{proof}

\begin{prop}
  Let $\calB$ be a grill base on $X$ and let $\Phi$ be as in Proposition \ref{prop:grlBasePhi}.
  Put 
  \[
    F(\calB) = \bigcap\{\, \calF : \mbox{$\calF$ is a maximal filter in $\Phi$} \,\}.
  \]
  Then $F(\calB)$ is a filter on $X$.
  We call the filter $F(\calB)$ the \textsl{filter associated with the base $\calB$}.
\end{prop}
\begin{proof}[Proof Sketch]
  By Zorn's Lemma the set $\Phi$ has maximal elements, and by Lemma \ref{lem:chainFlt} $F(\calB)$ is a filter on $X$.
\end{proof}

\begin{cor}
  $G\bigl(F(\calB)\bigr)$ is a grill on $X$.
  We call the grill $G\bigl(F(\calB)\bigr)$ the \textsl{grill associated with the base $\calB$}.
\end{cor}

\begin{prop}
  \label{prop:GrillB}
  Let $\calB$ be a grill base on $X$.
  \begin{itemize}
    \item[(a)] $F(\calB) = \{\, A \in \calB : \mbox{$A \cap B \in \calB$ for all $B \in \calB$} \,\}$.

    \item[(b)] $\calB \subseteq G\bigl(F(\calB)\bigr)$.

    \item[(c)] $\calB = \bigcap\{\, \calG : \mbox{$\calB \subseteq \calG$ and $\calG$ is a grill} \,\}$.
  \end{itemize}
\end{prop}
\begin{proof}
  (a) 
  Let $A \in F(\calB)$, then for every maximal filter $\calF \subseteq \calB$ we have $A \in \calF$. 
  Let $B \in \calB$, then the set $\{\, C \in \calB : B \subseteq C \,\}$ is a (principal) filter contained in $\calB$.
  By applying Zorn's Lemma, it follows that this filter is contained in a maximal filter $\calF \subseteq \calB$.
  In particular, $A \cap B \in \calF$ and so $A \cap B \in \calB$.

  Now let $A \in \calB$ such that for all $B \in \calB$, $A \cap B \in \calB$.
  Suppose that there exists a maximal filter $\calF \subseteq \calB$ with $A \not\in \calF$. 
  If $A \cap C \ne \emptyset$ for all $C \in \calF$, then $\calF \cup \{A\}$ is a filter subbase contained in $\calB$ that strictly contains the maximal filter $\calF$ in $\calB$.
  Therefore there exists $C \in \calF$ such that $A \cap C = \emptyset$.
  However in this case, $C \in \calB$, $\emptyset = A \cap C \in \calB$, a contradiction since $\emptyset \not\in \calB$.
  Therefore we must have that $A$ is a member of every maximal filter contained in $\calB$.

  (b)
  Observe that $A \in G\bigl(F(\calB)\bigr)$ if and only if $X \setminus A \not\in F(\calB)$ if and only if $X \setminus A \not\in \calB$ or $X \setminus A \in \calB$ and there exists $B \in \calB$ such that $(X \setminus A) \cap B \not\in \calB$.
  Let $A \in \calB$. 
  If $X \setminus A \not\in \calB$, then we're done.
  If $X \setminus A \in \calB$, then $(X \setminus A) \cap A = \emptyset \not\in \calB$.

  (c)
  Put
  \[
    \Gamma = \{\, \calG : \mbox{$\calB \subseteq \calG$ and $\calG$ is a grill} \,\}.
  \]
  First note that $\Gamma \ne \emptyset$ since by (b) $G\bigl(F(\calB)\bigr) \in \Gamma$.
  The fact that $\calB \subseteq \bigcap\Gamma$ is clear.

  Now suppose that $(\bigcap\Gamma) \setminus \calB \ne \emptyset$ and let $A \in (\bigcap\Gamma) \setminus \calB$.
  Since $A \not\in \calB$, we have that $A \not\in F(\calB)$, and moreover $X \setminus A \in G\bigl(F(\calB)\bigr)$.
  Let $\calF$ be the filter generated by $F(\calB) \cup \{X \setminus A\}$. 
  Now $X \setminus A \in \calF$ and so $A \not\in G(\calF)$. 
  To arrive at a contradiction we prove that $\calB \subseteq G(\calF)$.
  To show that $\calB \subseteq G(\calF)$ it suffices to show that for every $B \in \calB$ and $C \in \calF$ we have $B \cap C \ne \emptyset$.
  In order to show to this it suffices to show that for every $B \in \calB$ and every finite nonempty set $\calA \subseteq F(\calB) \cup \{X \setminus A\}$ we have $B \cap \bigcap\calA \ne \emptyset$.
  So let $\calA \subseteq F(\calB) \cup \{X \setminus A\}$ be a finite nonempty subset with $|\calA| = n$.
  Enumerate $\calA = \{D_1, D_2, \ldots, D_n\}$.
  If $D_i \in F(\calB)$ for each $i \in \{1, 2, \ldots, n\}$, then we have that $B \cap \bigcap_{i=1}^n D_i \ne \emptyset$. 
  Now without loss of generality suppose $D_n = X \setminus A$ and $D_i \in F(\calB)$ for each $i \in \{1, 2, \ldots, n-1\}$. 
  Put $D = \bigcap_{i=1}^{n-1} D_i \in F(\calB)$.
  Suppose that $B \cap D \cap (X \setminus A) = \emptyset$, then $B \cap D \subseteq A$.
  Since $D \in F(\calB)$ we have that $B \cap D \in \calB$ and hence $A \in \calB$, a contradiction.
  Therefore $B \cap D \cap (X \setminus A) \ne \emptyset$ and so $\calB \subseteq G(\calF)$.
  However $G(\calF)$ is grill that contains $\calB$ with $A \not\in G(\calF)$, a contradiction. 
\end{proof}

\begin{prop}
  Let $\calS$ be a grill subbase on a set $X$.
  Then $\calS^\uparrow$ and
  \[
    B(\calS) = \bigcap\{\, \calB : \mbox{$\calS \subseteq \calB$ and $\calB$ is a grill base} \,\}
  \]
  are both grill bases on $X$.
  We call $\calS^\uparrow$ or $B(\calS)$ the \textsl{grill base associated with the grill subbase $\calS$}.
\end{prop}
\begin{proof}
  The fact that $\calS^\uparrow$ is a grill base is clear.
  Put $\Psi = \{\, \calB : \mbox{$\calS \subseteq \calB$ and $\calB$ is a grill base} \,\}$ and observe that $\Psi$ is nonempty since it has $\{\, B \subseteq X : \mbox{$A \subseteq B$ for some $A \in \calS$} \,\}$ as a member.
  Hence let $B(\calS) = \bigcap\Psi$.

  Now $B(\calS)$ is nonempty since $X \in \calB$ for every grill base and so in particular $X \in B(\calS)$.
  Also since the empty set is not an element of any grill base we have that $\emptyset \not\in \calB$ for every grill base we have $\emptyset \not\in B(\calS)$.

  Let $A \in B(\calS)$ and $A \subseteq B \subseteq X$. 
  For every grill base $\calB$ with $\calS \subseteq \calB$ we have $A \in \calB$ and hence $B \in \calB$.
  Therefore $B \in B(\calS)$.
\end{proof}

\begin{cor}
  Let $\calS$ be a grill subbase on a set $X$.
  Then $F\bigl(B(\calS)\bigr)$ is a filter on $X$ which we call the \textsl{filter associated with the grill subbase $\calS$}; and $G\Bigl(F\bigl(B(\calS)\bigr)\Bigr)$ with $\calS \subseteq G\Bigl(F\bigl(B(\calS)\bigr)\Bigr)$ is a grill on $X$ which we call the \textsl{grill associated with the grill subbase $\calS$}. 
\end{cor}
\begin{proof}
  The assertions that $F\bigl(B(\calS)\bigr)$ is a filter and $G\Bigl(F\bigl(B(\calS)\bigr)\Bigr)$ is a grill is follows immediately.
  To see that $\calS \subseteq G\Bigl(F\bigl(B(\calS)\bigr)\Bigr)$ observe that $\calS \subseteq B(\calS)$.
\end{proof}

With a grill subbase we have a simple and trivial way to generate filters.
Therefore if we are trying to study a nonempty collection of sets that doesn't contain the empty set, then it may be easier to study the associated filter.
In the next section we give a definition for a special type of grill subbase for which studying the corresponding filters has been successful.


\section{Partition Regularity}

\begin{defn}
  Let $\calS$ be a grill subbase on some nonempty set.
  We say that the $\calS$ is \textsl{partition regular} if and only if whenever $\calA$ is a finite collection of sets with $\bigcup\calA \in \calS$, then there exist $A \in \calA$ and $B \in \calS$ such that $B \subseteq A$.
\end{defn}

\begin{prop}
  \label{prop:ParRegUp}
  Let $\calS$ be a grill subbase on some nonempty set.
  Put 
  \[
    \calB = \calS^\uparrow =\{\, B : \mbox{$A \subseteq B$ for some $A \in \calS$} \,\}.
  \]
  The following statements are equivalent.
  \begin{itemize}
    \item[(a)] $\calS$ is partition regular.
    \item[(b)] $\calB$ is a partition regular.
  \end{itemize}
\end{prop}
\begin{proof}
  ($\Rightarrow$)
  Clearly $\calB$ is a grill base. 
  Let $\calA$ be a finite collection of sets with $\bigcup\calA \in \calB$.
  Then there exists $A \in \calS$ such that $A \subseteq \bigcup\calA$.
  Therefore $\bigcup_{B \in \calA} (A \cap B) = A \in \calS$ and so we may pick $B \in \calA$ and $C \in \calS$ such that $C \subseteq A \cap B$. 
  Hence $A \cap B \in \calB$ and so $A \in \calB$.

  ($\Leftarrow$)
  Let $\calA$ be finite collection of sets with $\bigcup\calA \in \calS$, then $\calA \in \calB$.
  Since $\calB$ is a partition regular pick $A \in \calA$ and $B \in \calB$ such that $B \subseteq A$.
  Pick $C \in \calS$ such that $C \subseteq B$.
  Therefore $C \subseteq A$ and this completes the proof.
\end{proof}
\begin{cor}
  Let $\calS$ be a grill subbase on some nonempty set.
  Put 
  \[
    \calB = \calS^\uparrow =\{\, B : \mbox{$A \subseteq B$ for some $A \in \calS$} \,\}.
  \]
  The following statements are equivalent.
  \begin{itemize}
    \item[(a)] $\calS$ is partition regular.
    \item[(b)] $\calB$ is a grill.
  \end{itemize}
\end{cor}
\begin{rmk}
Partition regularity roughly asserts that some property of $X$, here represented by members of $\calR$, occurs a ``large'' number of times in $X$.
In fact so large, that no matter how we finitely divide up $X$, at least one cell in the division has our specified property. 
One of the easiest partition regular property to observe is the infinite form of the pigeonhole principle.
\end{rmk}

The following theorem, taken from \cite[Theorem 3.11]{Hindman:1998fk} shows the intimate connection between partition regular sets and ultrafilters.

\begin{thm}
  Let $X$ be a set and $\calS$ a grill subbase on $X$.
  Put $\calB = \calS^\uparrow$.
  The following statements are equivalent.
  \begin{itemize}
    \item[(a)] $\calS$ is partition regular.
    \item[(b)] $\calS^\uparrow$ is partition regular.
    \item[(c)] $\calS^\uparrow$ is a grill.
    \item[(d)] If $\calA \subseteq \calP(X)$ has the property that every finite nonempty subfamily of $\calA$ has an intersection in $\calS^\uparrow$, then there is an ultrafilter $\calU$ on $X$ with $\calA \subseteq \calU \subseteq \calS^\uparrow$.
    \item[(e)] If $A \in \calS$, then there is some ultrafilter $\calU$ on $X$ such that $A \in \calU \subseteq \calS^\uparrow$.
  \end{itemize}
\end{thm}
\begin{proof}
  $(a) \iff (b)$ and $(b) \iff (c)$ 
  This is Proposition \ref{prop:ParRegUp}.

  $(c) \Rightarrow (d)$ 
  Since $\calA$ is a filter subbase, we let $\calF$ be the filter generated by $\calA$.
  By construction we have that $\calF \subseteq \calS^\uparrow$. 
  Since $\calS^\uparrow$ is a grill, there exists an ultrafilter $\calU$ on $X$ with $\calF \subseteq \calU \subseteq \calS^\uparrow$.

  $(d) \Rightarrow (e)$
  Put $\calA = \{A\}$.

  $(d) \Rightarrow (c)$.
  Let $A \cup B \in \calS^\uparrow$. 
  Pick an ultrafilter on $X$ such that $A \cup B \in \calU \in \calS^\uparrow$.
  Then either $A \in \calU$ or $B \in \calU$, that is, either $A \in \calU$ or $B \in \calU$.  
\end{proof}


\section{Algebraic Semigroup Theory}
\section{Stone-\v{C}ech Compactification of a Discrete Space}
\section{Algebra in the Stone-\v{C}ech Compactification of a Discrete Semigroup}
\section{Ramsey Theoretic Motivation}
\section*{Open Problems}
\section{Ultrafilters and Partition Regularity}

% While Ramsey's Theorem is important and provides a good example of a
% partition regular pair $(X, \calR)$, in this dissertation we will
% mainly be considering partition regular sets where the underlying set
% $X$ has an algebraic structure of a semigroup and elements in $\calR$
% are defined in terms of this structure. 

\begin{schur}
  Let $r \in \bbN$ and $\bbN = \bigcup_{i=1}^r C_i$.
  Then there exist $i \in \{1, 2, \ldots, r\}$ and $x$, $y$, and $z
  \in \bbN$ such that $\{\, x, y, x+y \,\} \subseteq C_i$.
\end{schur}

\begin{vdw}
  Let $r \in \bbN$ and $\bbN = \bigcup_{i=1}^r C_i$.
  Then for every $\ell \in \bbN$, there exist $i \in \{1, 2, \ldots,
  r\}$ and $a$, $d \in \bbN$ such that $\{\, a, a+d, \ldots, a+\ell d
  \,\} \subseteq C_i$.
\end{vdw}

% Endnotes 
\theendnotes

% Things referenced in the preliminaries chapter. Eventually this will
% placed in a separate file so the References appear at the end.
\bibliographystyle{amsplain}
\bibliography{../references}

\end{document}
% % This is a chapter in my dissertation tentatively entitled "A New and
% Simpler Central Sets Theorem".  The new and simpler part is due to a
% different definition of J-sets, which in turns leads to a different
% definition of C-sets.  Happily the new definition is equivalent to
% the old definition.  Therefore no proofs need to be rewritten; but
% there is value in rewriting the proofs, since as this chapter shows,
% the rewritten proofs are often simpler than the old proofs.
%
% N.B. The new definition of J-sets essentially first appears in an
% article, "Ramsey Theory in Noncommutative Semigroups", by Bergelson
% and Hindman in their Theorem 2.6. [6/19/2011 John]

\chapter{Central Sets Theorem}

\section{Hales-Jewett Theorem}
In the article \cite{Hales:1963fk}, Alfred W.~Hales and Robert I.~Jewett abstracted van der Waerden's combinatorial argument into a powerful theorem which implies (among other things) van der Waerden's Theorem itself. 
The vast applicability of the Hales-Jewett Theorem lies in the fact that it relates finite coloring of free semigroups and `combinatorial lines'. 
In this chapter, one of our first tasks shall be to import the notion of a special type of combinatorial line into other types of semigroups. 
However before we can get started we must first state some terminology related to the Hales-Jewett Theorem itself.

\begin{defn}
  Let $A$ be a nonempty finite set, let $\star$ denote an element not in $A$, and let $S$ be the free semigroup generated by $A$.
  \begin{itemize}
    \item[(a)] 
      A \emph{variable word (on $A$)} is a word $w(\star)$ in the free semigroup on the alphabet $A \cup \{\star\}$ in which $\star$ occurs.

    \item[(b)] 
      Given a variable word $w(\star)$ on $A$ and $a \in A$, we let $w(a)$ denote the word in $S$ where each occurrence of $\star$ in $w(\star)$ is replaced by $a$.

    \item[(c)] 
      Given a variable word $w(\star)$ on $A$ we call the set $\{\, w(a) : a \in A \,\}$ a \emph{combinatorial line}.
  \end{itemize}
\end{defn}

\begin{hj}
  Let $A$ be a nonempty finite set, let $S$ be the free semigroup generated by $A$, and let $r$ be a positive integer.
  If $S = \bigcup_{i=1}^r C_i$, then there exists $i \in \{1, 2, \ldots, r\}$ such that $C_i$ contains a combinatorial line, that is, there exists a variable word $w(\star)$ with $\bigl\{\, w(a) : a \in A \,\bigr\} \subseteq C_i$.
\end{hj}

To import the notion of a combinatorial line into other semigroups we shall specialize the notion of a combinatorial line to require that it begins and ends with a constant letter, and that there are no two adjacent variable letters.
The following theorem shows that the conclusion of the Hales-Jewett Theorem may be strengthened to these specialize combinatorial lines.  

\begin{thm}
  \label{thm:special-hj}
  Let $r$ be a positive integer and $A$ a nonempty finite set.
  If $S$ is the free semigroup generated by $A$ and $S = \bigcup_{i=1}^r C_i$, then there exist $i \in \{1, 2, \ldots, r\}$ and a variable word $w(\star)$ which begins and ends with a constant letter and has no two adjacent variable letters such that $\bigl\{\, w(a) : a \in A \,\bigr\} \subseteq C_i$.
\end{thm}
\begin{proof}
  Define the function $\varphi \colon S \to \{1, 2, \ldots, r\}$ by $\varphi(w) = \min\{i : w \in C_i\}$. 
  Fix $b \in A$ and define $\psi \colon S \to S$ as follows: if $w \in S$ and $w = a_1a_2 \cdots a_n$ with each $a_i \in A$, then $\psi(w) = ba_1ba_2b\cdots a_nb$.
  We have that $\varphi \circ \psi \colon S \to \{1, 2, \ldots, r\}$.
  By the Hales-Jewett Theorem, pick $i \in \{1, 2, \ldots, r\}$ and a variable word $w(\star)$ such that $\{\, w(a) : a \in A \,\} \subseteq (\varphi \circ \psi)^{-1}[\{i\}]$.
  Observe that $\psi\bigl(w(\star)\bigr)$ is a variable word that begins and ends with a constant, doesn't have any adjacent variable letters, and $\{\, \psi\bigl(w(a)\bigr) : a \in A \} \subseteq C_i$.
\end{proof}

\begin{thm}
  \label{thm:finitary-shj}
  Let $k$ be a positive integer.
  We consider $\{1, 2, \ldots, k\}$ as a finite alphabet on $k$ symbols, and for every $n \in \bbN$, we consider $\{1, 2, \ldots, k\}^n$ as the set of all length $n$ words generated by $\{1, 2, \ldots, k\}$.

  Then for each $k$ and $r$ in $\bbN$ there exists $n \in \bbN$ such that whenever $\{1, 2, \ldots, k\}^n$ is $r$-colored, that is, $\varphi \colon \{1, 2, \ldots, k\}^n \to \{1, 2, \ldots, r\}$, there exists a variable word $w(\star)$ which begins and ends with a constant letter and has no two adjacent variable letters such that $\bigl\{\, w(i) : i \in \{1, 2, \ldots, k\} \,\bigr\}$ is constant on $\varphi$.
\end{thm}
\begin{proof}
  Let $k$ and $r$ be given and suppose that the conclusion fails.
  That is for each $n \in \bbN$ choose an $r$-coloring $\varphi_n \colon \{1, 2, \ldots, k\}^n \to \{1, 2, \ldots, r\}$ such that no variable word which begins and ends with a constant and has no two adjacent variable letters is constant on $\varphi_n$.
  
  Put $S = \bigcup_{n=1}^\infty \{1, 2, \ldots, k\}^n$ and define $\varphi \colon S \to \{1, 2, \ldots, r\}$ as follows: If $w$ is a length $n$ word with $w = (x_1, x_2, \ldots, x_n)$ and each $x_i \in \{1, 2, \ldots, k\}$, then $\varphi(w) = \varphi_n(w)$.
  By Theorem \ref{thm:special-hj} pick $i \in \{1, 2, \ldots, r\}$ and a variable word $w(\star)$ which beings and ends with a constant letter and has no two adjacent variable letters with $\bigl\{\, w(i) : i \in \{1, 2, \ldots, k\} \,\bigr\} \subseteq \varphi^{-1}[\{i\}]$.
  Suppose that the length of $w(\star)$ is $n$, then, $\varphi_n$ is constant on $\bigl\{\, w(i) : i \in \{1, 2, \ldots, k\} \,\bigr\}$, a contradiction. 
\end{proof}

\section{$J$-sets}
The way we shall import the notion of our specialized combinatorial line into other semigroups is by defining the concept of a $J$-set. 
In what follows if $A$ and $B$ are sets, we let $\setfunc{A}{B}$ represent the collection of all functions with domain $A$ and codomain $B$.

\begin{defn}
  Let $(S, \cdot)$ be a semigroup.
  \begin{itemize}
    \item[(b)] 
      For each $m \in \bbN$, define
      \[
        \calJ_m = \{\, (t_1, t_2, \ldots, t_m) \in \bbN^m : t_1 < t_2 < \cdots < t_m \,\}.
      \]

    \item[(b)] 
      Put $\calT(S) = \setfunc{\bbN}{S}$.
      If the semigroup is clear from context, we write $\calT$ instead of $\calT(S)$.

    \item[(c)] 
      For each $m \in \bbN$, $a \in S^{m+1}$, $t \in \calJ_m$, and $f \in \calT$, put
      \[
        \textstyle
        x(m, a, t, f) = \Bigl( \prod_{i=1}^m \bigl( a(i)f(t_i) \bigr) \Bigr)a(m+1).
      \]

    \item[(d)] 
      We call a subset $A \subseteq S$ a \emph{$J$-set (in $S$)} if and only if for every $F \in \Pf(\calT)$ there exist $m \in \bbN$, $a \in S^{m+1}$, and $t \in \calJ_m$ such that for each $f \in F$ we have $x(m, a, t, f) \in A$.
  \end{itemize}
\end{defn}
\begin{rmk}
  Despite appearances to the contrary, I must point out that $J$-sets are \textsl{not} named after the author!
  In particular, the term $J$-set is derived from the term $J_Y$ set which first appeared as a definition in \cite[Definition 2.4(b)]{Hindman:1996fk}.
\end{rmk}

Our definition of a $J$-set looks different from the definitions given in \cite[Definition 2.2(a)]{Hindman:2009vn} and \cite[Definition 3.3(d)]{De:2008uq}.
The first results we prove about $J$-sets shall be that all of these definitions are equivalent.
However before proving this result we introduce some notation that is analogous to $\calJ_m$.

\begin{defn}
  For each $m \in \bbN$ define
  \begin{align*}
    \calI_m &= \Bigl\{\, \bigl( H(1), H(2), \ldots, H(m) \bigr) : \hbox{$H(i) \in \mathcal{P}_f(\bbN)$ for all $i \in \{1, 2, \ldots, m\}$} \\
        &\hspace{3em} \hbox{ and $\max H(i) < \min H(i+1)$ for all $i \in \{1, 2, \ldots, m-1\}$} \Bigr\}.
  \end{align*}
\end{defn}

Given a $A$ a subset of some semigroup $S$, in \cite[Definition 2.2(a)]{Hindman:2009vn} $A$ is considered a $J$-set if and only if for every $F \in \Pf(\calT)$, there exists $m \in \bbN$, $a \in S^{m+1}$, and $H \in \calI_m$ such that for all $f \in F$, $\Bigl( \prod_{i=1}^n\bigl( a(i) \prod_{t \in H(i)} f(t) \bigr) \Bigr) a(m+1) \in A$.
To show that our definition of a $J$-set is equivalent to \cite[Definition 2.2(a)]{Hindman:2009vn} we will use the following technical `rewriting' lemma.

\begin{lem}
  \label{lem:rewrite-jset}
  Let $(S, \cdot)$ be a semigroup, $m \in \bbN$, $a \in S^{m+1}$, $H \in \calI_m$, and $F \in \Pf(\calT)$. 
  Fix $b \in S$ and for each $f \in F$ define $g_f \in \calT$ by $g_f(t) = f(t)b$. 
  Then there exists $n \in \bbN$, $c \in S^{n+1}$, and $t \in \calJ_n$ such that for all $f \in F$, $x(n, c, t, f) = \prod_{i=1}^m \bigl( a(i) \prod_{t \in H(i)} g_f(t)\big) a(m+1)$.
\end{lem}
\begin{proof}
  Put $H(0) = \emptyset$ and for each $s \in \{0, 1, \ldots, m\}$ define $h_s = \sum_{i=0}^s |H(i)|$.
  Put $n = h_m$ and enumerate $\bigcup_{i=1}^m H(i)$ as a strictly increasing sequence $t_1 < t_2 < \cdots < t_n$ in $\bbN$. 
  We will adopt some temporary terminology and say that for $f \in F$, $\prod_{i=1}^m \bigl( a(i) \prod_{t \in H(i)} g_f(t) \bigr) a(m+1)$ has \emph{proper representation} if and only if $\prod_{i=1}^m \bigl( a(i) \prod_{t \in H(i)} g_f(t) \bigr) a(m+1) = x(n, c, t, f)$ where $c \in S^{n+1}$ is defined as follows:
  \[
    c(j) = 
    \begin{cases}
      a(1) & \mbox{if $j = 1$;} \\
      b & \mbox{if $s \in \{0, 1, \ldots, m-1\}$ and} \\
      &   \hspace{2em}\mbox{$2+h_s \le j \le h_{s+1}$; and} \\
      ba(s+1) & \mbox{if $s \in \{1, 2, \ldots, m\}$ and $j = 1+h_s$.}
    \end{cases}
  \]
  {
    (To see how our $c$ was derived let's consider a reasonable small example with $m = 3$, $H(1) = \{3, 5\}$, $H(2) = \{7\}$, and $H(3) = \{9, 11, 15\}$.
    With these numbers we have
    \begin{align*}
      \textstyle
      \prod_{i=1}^3 \bigl( a(i) \prod_{t \in H(i)} g(t) \bigr) a(4) &=
      a(1) g(3)g(5) a(2) g(7) a(3) g(9)g(11)g(15) a(4) \\
      &= a(1) f(3)bf(5)b a(2) f(7)b a(3) f(9)bf(11)bf(15)b a(4).
    \end{align*}
    Therefore $n = 6$, $c(1) = a(1)$, $c(2) = b$, $c(3) = ba(2)$, $c(4) = ba(3)$, $c(5) = b$,  $c(6) = b$, and $c(7) = ba(4)$.)
  }

  We prove that for $f \in F$, $\prod_{i=1}^m \bigl( a(i) \prod_{t \in H(i)} g_f(t) \bigr) a(m+1)$ has proper representation by induction on $m$. 
  First suppose that $m = 1$, then 
  \[
    \textstyle
    \prod_{i=1}^1 \bigl( a(i) \prod_{t \in H(i)} g_f(t) \bigr) a(2) = a(1) f(t_1)b  f(t_2)b \cdots f(t_n)b a(2).
  \]
  In this case $h_0 = 0$, $h_1 = n$, and so $s$ can only be 0 or 1.
  If $2+h_0 = 2 \le j \le h_1 = n$, then by definition of $c$ we have $c(j) = b$ for all $j \in \{2, 3, \ldots, n\}$. 
  Also since $1 + h_1 = n+1$, we have $c(n+1) = ba(2)$. 
  Therefore
  \[
    \textstyle
    a(1) f(t_1)b  f(t_2)b \cdots f(t_n)b a(2) = c(1) f(t_1) c(2) f(t_2) c(3) \cdots f(t_n) c(n+1),
  \]
  and so $\prod_{i=1}^1 \bigl( a(i) \prod_{t \in H(i)} g_f(t) \bigr) a(2)$ has proper representation. 

  Now let $m > 1$ and assume that $\prod_{i=1}^{m-1} \bigl( a(i) \prod_{t \in H(i)} g_f(t) \bigr) a(m)$ has proper representation with $\prod_{i=1}^{m-1} \bigl( a(i) \prod_{t \in H(i)} g_f(t) \bigr) a(m) = x(n, c, t, f)$. 
  Then we have 
  \begin{align*}
    \textstyle
    \prod_{i=1}^{m} \bigl( a(i) \prod_{t \in H(i)} g_f(t) \bigr) a(m+1) &= \textstyle
    \prod_{i=1}^{m-1} \bigl( a(i) \prod_{t \in H(i)} g_f(t) \bigr) a(m) \prod_{t \in H(m)} g_f(t) a(m+1), \\
    &= \textstyle 
    x(n, c, t, f) \prod_{t \in H(m)} g_f(t) a(m+1).
  \end{align*}
  Now $c(n+1) \prod_{t \in H(m)} g_f(t) a(m+1)$ has proper representation, say with, 
  \[
    \textstyle
    \prod_{t \in H(m)} g_f(t) a(m+1) = x(p, d, u, f).
  \]
  By translating the indices for $u$ and $d$ it follows that 
  \[
    \textstyle
    \prod_{i=1}^{m} \bigl( a(i) \prod_{t \in H(i)} g_f(t) \bigr) a(m+1)
  \]
  has proper representation.
\end{proof}

\begin{thm}
  Let $(S, \cdot)$ be a semigroup and $A$ a subset of $S$.
  The following are equivalent. 
  \begin{itemize}
    \item[(a)] $A$ is a $J$-set.
    \item[(b)] For all $F \in \Pf(\calT)$, there exist $m \in \bbN$, $a \in S^{m+1}$, and $H \in \calI_m$ such that for all $f \in F$ we have $\prod_{i=1}^n\bigl( a(i) \prod_{t \in H(i)} f(t) \bigr) a(m+1) \in A$.
  \end{itemize}
\end{thm}
\begin{proof}
  \textsl{(a) $\Rightarrow$ (b)}
  Let $F \in \Pf(\calT)$ and pick $m \in \bbN$, $a \in S^{m+1}$, and $t \in \calJ_m$ as guaranteed. 
  For each $i \in \{1, 2, \ldots, m\}$ put $H(i) = \{ t_i\}$. 
  Then $H \in \calI_m$ and the conclusion follows. 

  \textsl{(b) $\Rightarrow$ (a)}
  For all $F \in \Pf(\calT)$ by Lemma \ref{lem:rewrite-jset} there exist $m \in \bbN$, $a \in S^{m+1}$, and $t \in \calJ_m$ such that for all $f \in F$, $x(m, a, t, f) = \prod_{i=1}^n\bigl(a(i)\prod_{t \in H(i)} f(t)\bigr)a(m+1) \in A$.
  Therefore $A$ is a $J$-set.
\end{proof}

With this theorem all of the results previously proved about $J$-sets apply  with our definition.
However in this chapter, we choose to reprove these results primarily to show how the proofs can be simplified with the `new' definition of $J$-set.

The first result we (re)prove is \cite[Lemma 2.4]{Hindman:2010fk} which shows that when the underlying semigroup is commutative the definition of a $J$-set is considerably simplified. 

\begin{lem}
  \label{lem:comm-jsets}
  Let $(S, +)$ be a commutative semigroup and let $A \subseteq S$. 
  Then $A$ is a $J$-set if and only if for every $F \in \Pf(\calT)$, there exist $a \in S$ and $H \in \Pf(\bbN)$ such that for all $f \in F$, $a + \sum_{t \in H} f(t) \in A$.
\end{lem}
\begin{proof}
  ($\Rightarrow$)
  Let $F \in \Pf(\calT)$.
  Pick $m \in \bbN$, $a \in S^{m+1}$, and $t \in \calJ_m$ such that $x(m, a, t, f) \in A$ for each $f \in F$.
  Put $b = \sum_{i=1}^{m+1} a(i)$ and $H = \{t_1, t_2, \ldots, t_m\}$.
  Since $S$ is commutative we have that $b + \sum_{t \in H} f(t) = x(m, t, a, f) \in A$.

  ($\Leftarrow$)
  Let $F \in \Pf(\calT)$ and fix $b \in S$.
  For each $f \in F$ define $g_f \in \calT$ by $g_f(t) = f(t)+b$. 
  Pick $a \in S$ and $H \in \Pf(\bbN)$ such that for all $f \in F$, $a + \sum_{t \in H} g_f(t) \in A$. 
  Put $m = |H|$ and enumerate $H$ as a strictly increasing sequence $t_1 < t_2 < \cdots < t_m$. 
  Define $c \in S^{m+1}$ by $c(1) = a$ and $c(i) = b$ for every $i \in \{2, 3, \ldots, m+1\}$.
  Then $x(m, c, t, f) = a + \sum_{t \in H} g_f(t) \in A$.
\end{proof}

We next show that $J$-sets satisfy a superficially stronger condition.
(In \cite[Definition 3.3(e)]{De:2008uq} this superficially stronger condition was originally used to define a $J$-set.)

\begin{lem}
  \label{lem:jset-start}
  Let $(S, \cdot)$ be a semigroup and $A$ a $J$-set in $S$.
  Then for every $F \in \Pf(\calT)$ and each positive integer $n$ there exist $m \in \bbN$, $a \in S^{m+1}$, and $t \in \calJ_m$, with $t_1 > n$ such that for all $f \in F$, $x(m, a, t, f) \in A$.
\end{lem}
\begin{proof}
  For each $f \in F$ define $g_f \in \calT$ by $g_f(t) = f(n+t)$.
  Since $A$ is a $J$-set, pick $m \in \bbN$, $a \in S^{m+1}$, and $t \in \calJ_m$ such that for all $f \in F$, $x(m, a, t, g_f) \in A$. 
  Define $u \in \calJ_m$ by $u_i =  n + t_i$.
  Then $x(m, a, u, f) = x(m, a, t, g_f) \in A$ for every $f \in F$.
\end{proof}

The next result we reprove is \cite[Theorem 3.4]{De:2008uq} which states that the collection of all ultrafilters on a semigroup, each of whose member is a $J$-set, is a closed ideal.
To state this result we introduce the following notation.

\begin{defn}
  Let $(S, \cdot)$ be a semigroup. 
  Define
  \[
    J(S) = \{\, p \in \beta S : \mbox{for every $A \in p$, $A$ is a $J$-set} \,\}.
  \]
\end{defn}

\begin{thm}
  Let $(S, \cdot)$ be a semigroup.
  If $J(S)$ is nonempty, then $J(S)$ is a closed two-sided ideal of $\beta S$.
\end{thm}
\begin{proof}
  Let $p \not\in J(S)$ and pick $A \in p$ such that $A$ is not a $J$-set.
  By definition of $J(S)$ we must have $\overline{A} \cap J(S) = \emptyset$.
  Since $\overline{A}$ is a (basic) open neighborhood of $p$, it follows that $J(S)$ is topologically closed in $\beta S$.

  Now let $p \in J(S)$ and $q \in \beta S$.
  To see that $J(S)$ is an ideal, we show that $pq \in J(S)$ and $qp \in J(S)$. 

  We first show that $pq \in J(S)$.
  Let $F \in \Pf(\calT)$, let $A \in pq$, and put $B = \{\, x \in S : x^{-1}A \in q \,\}$.
  Then $B \in p$ and so $B$ is a $J$-set.
  Pick $m \in \bbN$, $a \in S^{m+1}$, and $t \in \calJ_m$ such that for all $f \in F$ we have $x(m, a, t, f) \in B$.
  By definition of $B$ this means that for all $f \in F$, $x(m, a, t, f)^{-1}A \in q$. 
  Since $q$ is an ultrafilter and $F$ is finite, we have $\bigcap_{f \in F} x(m, a, t, f)^{-1}A \in q$, and therefore, $\bigcap_{f \in F} x(m, a, t, f)^{-1}A \ne \emptyset$.
  Pick $b \in \bigcap_{f \in F} x(m, a, t, f)^{-1}A$ and define $c \in S^{m+1}$ by
  \[
    c =
    \begin{cases}
      a(j) & \mbox{if $j \in \{1, 2, \ldots, m\}$,} \\
      a(m+1)b & \mbox{if $j = m+1$.}
    \end{cases}
  \]
  Therefore, for each $f \in F$, $x(m, c, t, f) \in A$, that is, $A$ is a $J$-set and so $pq \in J(S)$.

  Now we show that $qp \in J(S)$.
  Again let $F \in \Pf(\calT)$, let $A \in qp$, and put $B = \{\, x \in S : x^{-1}A \in p \,\}$.
  Then $B \in q$, and therefore $B \ne \emptyset$.
  Pick $b \in B$, then $b^{-1}A \in p$ and so $b^{-1}A$ is a $J$-set.
  Pick $m \in \bbN$, $a \in S^{m+1}$, and $t \in \calJ_m$ such that for all $f \in F$, we have $x(m, a, t, f) \in b^{-1}A$.
  Define $c \in S^{m+1}$ by
  \[
    c =
    \begin{cases}
      ba(1) & \mbox{if $j =1$,} \\
      a(j) & \mbox{if $j \in \{2, \ldots, m, m+1\}$.}
    \end{cases}
  \]
  Therefore, for each $f \in F$, $x(m, c, t, f) \in A$, that is, $A$ is a $J$-set and so $qp \in J(S)$.
\end{proof}

It is clear that every semigroup $S$ is a $J$-set in itself, but this easy fact doesn't necessarily imply that $J(S) \ne \emptyset$.
It's true that $J(S)$ \emph{is} nonempty but this nontrivial fact follows either from the fact that piecewise syndetic sets are $J$-sets, Lemma \ref{lem:pr-jsets} and \cite[Theorem 3.11]{Hindman:1998fk}, or the Central Sets Theorem.%
\endnote{
  The fact that $J(S)$ is nonempty was first proven via the Central Sets Theorem in \cite[Theorem 3.8]{De:2008uq}.
  The proof that $J$-sets are partition regular was first proved in \cite[Theorem 2.14]{Hindman:2010fk}. 
  The proof that piecewise syndetic sets are $J$-sets is new.
}

We start by proving that piecewise syndetic sets are $J$-sets.
The following result is modeled after \cite[Theorems 14.1 and 14.7]{Hindman:1998fk}. 

\begin{thm}
  Let $(S, \cdot)$ be a semigroup and $A \subseteq S$ piecewise syndetic.
  Then $A$ is also a $J$-set.
\end{thm}
\begin{proof}
  Let $F \in \calT$, put $k = |F|$, and enumerate $F$ as $\{f_1, f_2, \ldots, f_k\}$.
  Put $Y = \bigtimes_{t=1}^k \beta S$. 
  Then with the product topology $Y$ is a compact right-topological semigroup by \cite[Theorem 2.22]{Hindman:1998fk}
  For each $i \in \bbN$ define
  \begin{align*}
    I_i &= \Bigl\{\, \bigl( x(m, a, t, f_1), x(m, a, t, f_2), \ldots,
    x(m, a, t, f_k) \bigr) : \mbox{$m \in \bbN$, $a \in S^{m+1}$,} \\
    &\hspace{5em} \mbox{$t \in \calJ_m$, and $t_1 > i$}
    \,\Bigr\},
  \end{align*}
  and put $E_i = I_i \cup \{\, (a, a, \ldots, a) : a \in S \,\}$.
  Let $I = \bigcap_{i=1}^\infty \overline{I_i}$ and let $E = \bigcap_{i=1}^\infty \overline{E_i}$.

  Observe that $I$ and $E$ are nonempty closed subsets of $Y$.
  (Since each $I_i \ne \emptyset$ and $I_{i+1} \subseteq I_i$.)

  We claim that $E$ is a subsemigroup of $Y$ and $I$ is an ideal of $E$.
  Let $p$, $q \in E$, $U$ be an open neighborhood of $pq$, and let $i \in \bbN$. 
  Since $\rho_q$ is continuous, pick $V$ a neighborhood of $p$ such that $Vq \subseteq U$. 
  If $p \in I$, then pick $\vec{x} \in I_i \cap V$; otherwise pick $\vec{x} \in E_i \cap V$.
  If $\vec{x} \in I_i$, then pick $m \in \bbN$, $a \in S^{m+1}$, and $t \in \calJ_m$ with $t_1 > i$ such that
  \[
    \vec{x} = \bigl( x(m, a, t, f_1), x(m, a, t, f_2), \ldots, x(m,
    a, t, f_l) \bigr)
  \]
  In this case put $j = t_m$, otherwise put $j=i$. 

  Since $\lambda_{\vec{x}}$ is continuous, pick $W$ a neighborhood of $q$ such that $xW \subseteq U$. 
  If $q \in I$, then pick $\vec{y} \in I_j \cap W$, otherwise pick $\vec{y} \in E_j \cap W$.
  Then $\vec{x} \vec{y} \in E_i \cap U$, and if $p \in I$ or $q \in I$, then $\vec{x} \vec{y} \in I_i \cap U$. 
  Hence, it follows that $E$ is a subsemigroup of $Y$ and $I$ is an ideal of $E$.

  Pick $p \in K(\beta S) \cap \overline{A}$ and put $\overline{p} = (p, p, \ldots, p)$. 
  Since $K(Y) = \bigtimes_{t=1}^l K(\beta S)$ (by \cite[Theorem 2.23]{Hindman:1998fk}), we have that $\overline{p} \in K(Y)$. 
  We show that $\overline{p} \in E$.
  Let $U$ be a neighborhood of $\overline{p}$ and pick $B_1$, $B_2$, \dots, $B_l \in p$ such that $\bigtimes_{t=1}^l \overline{B_t} \subseteq U$. 
  Pick $a \in \bigcap_{t=1}^l B_t$, then $(a, a, \ldots, a) \in U \cap E_i$ for all $i \in \bbN$. 
  Hence $\overline{p} \in E$ and moreover $\overline{p} \in K(Y) \cap E$.

  Since $K(Y) \cap E \ne \emptyset$, by Theorem \ref{thm:smallest-subsemigrp} we have $K(E) = K(Y) \cap E$.
  Therefore $\overline{p} \in K(E) \subseteq I$.
  Hence $I_i \cap \bigtimes_{i=1}^l A \ne \emptyset$ for all $i \in \bbN$ and so our conclusion follows.
\end{proof}

\begin{cor}
  Let $(S, \cdot)$ be a semigroup, then $c\ell\bigl( K(\beta S) \bigr) \subseteq J(S)$.
\end{cor}
\begin{proof}
  Let $p \in c\ell\bigl( K(\beta S) \bigr)$ and $A \in p$.
  By \cite[Corollary 4.41]{Hindman:1998fk}, $A$ is a piecewise syndetic set and so $A$ is a $J$-set.
  Therefore $p \in J(S)$.
\end{proof}

\begin{cor}
  If $A \subseteq \bbN$ be a piecewise syndetic set, then $A$ is also an $AP$-set. 
\end{cor}
\begin{proof}
  Since $A$ is also a $J$-set, it suffices to show that all $J$-sets in $\bbN$ are $AP$-sets. 
  Let $k$ be a positive integer and let $\la x_n \ra_{n=1}^\infty$ be a sequence in $\bbN$. 
  For each $i \in \{1, 2, \ldots, k\}$ put $f_i(t) = i \cdot x_t$.
  Since $A$ is a $J$-set, pick $a \in \bbN$ and $H \in \Pf(\bbN$ such that for all $i \in \{1, 2, \ldots, k\}$, $a + \sum_{t \in H} f_i(t) \in A$. 
  Put $d = \sum_{t \in H} x_t$ and observe that for all $i \in \{1, 2, \ldots, k\}$ we have
  \begin{align*}
    a+id &= \textstyle a+ i(\sum_{t \in H} x_t) = a + \sum_{t \in H} ix_t, \\
    &= \textstyle a + \sum_{t \in H} f_i(t) \in A. 
  \end{align*}
  Hence $A$ is an $AP$-set.
\end{proof}

We now provide another proof, that when combined with \cite[Theorem 3.11]{Hindman:1998fk}, proves that $J(S)$ is nonempty.

\begin{lem}
  \label{lem:pr-jsets}
  Let $(S, \cdot)$ be a semigroup with $A_1 \subseteq S$ and $A_2 \subseteq S$.
  If $A_1 \cup A_2$ is a $J$-set, then either $A_1$ is a $J$-set or $A_2$ is a $J$-set.
\end{lem}
\begin{proof}
  Suppose, to the contrary, that both $A_1$ and $A_2$ are \emph{not} $J$-sets.
  Pick $F_1$ and $F_2$ in $\Pf(\calT)$ such that for all $m \in \bbN$, every $a \in S^{m+1}$, and every $t \in \calJ_m$, there exist $f \in F_1$ and $g \in F_2$ such that $x(m, a, t, f) \not\in A_1$ and $x(m, a, t, g) \not\in A_2$.
  Let $F = F_1 \cup F_2$, put $k = |F|$, and enumerate $F$ as $\{f_1, f_2, \ldots, f_k\}$.

  By Theorem \ref{thm:finitary-shj} pick $n \in \bbN$ such that whenever $\{1, 2, \ldots, k\}^n$ is 2-colored there exists a variable word $w(\star)$ which begins and ends with a constant letter, with no two adjacent variable, and for which the combinatorial line $\bigl\{\, w(\ell) : \ell \{1, 2, \ldots, k\} \,\bigr\}$ is constant under the mapping. 

  Now for each $w = (x_1, x_2, \ldots, x_n) \in \{1, 2, \ldots, k\}^n$ define $g_w \in \calT$ by $g_w(t) = \prod_{i=1}^n f_{x_i}(nt + i)$.
  We have, by hypothesis, that $A_1 \cup A_2$ is a $J$-set and since $\{1, 2, \ldots, k\}^n$ is finite, we may pick $m \in \bbN$, $a \in S^{m+1}$, and $t \in \calJ_m$ such that for all $w \in \{1, 2, \ldots, k\}^n$, $x(m, a, t, g_w) \in A$.

  Define the function $\varphi \colon \{1, 2, \ldots, k\}^n \to \{1, 2\}$ as follows:
  \[
    \varphi(w) = 
    \begin{cases}
      1 & \mbox{if $x(m, a, t, g_w) \in A_1$, and} \\
      2 & \mbox{otherwise.}
    \end{cases}
  \]
  Since $\varphi$ is a 2-coloring of $\{1, 2, \ldots, k\}^n$ we may apply Theorem \ref{thm:finitary-shj} to pick a variable word $w(\star)$ which begins and ends with a constant letter, with no two adjacent variable letters, and for which the $varphi$ is constant on the combinatorial line $\bigl\{\, w(\ell) : \ell \in \{1, 2, \ldots, k\} \,\bigr\}$. 
  Without loss of generality we may suppose that $\varphi \bigl( w(\ell) \bigr) = 1$ for every $\ell \in \{1, 2, \ldots, k\}$.

  We claim that there exist $r \in \bbN$, $c \in S^{mr+1}$, and $s \in \calJ_{mr}$ such that for all $\ell \in \{1, 2, \ldots, k\}$ we have $x(mr, c, s, f_\ell) = x(m, a, t, g_{w(\ell)}) \in A_1$. 
  Observe that if we prove our claim, then we will have arrived at a contradiction. 
  Therefore for the rest of the proof we focus only on rewriting $x(m, a, t, g_{w(\ell)})$ into the appropriate form.

  Let $w(\star) = (x_1, x_2, \ldots, x_n)$ with each $x_i \in \{1, 2, \ldots, k\} \cup \{\star\}$. 
  Let $r$ be the number of variable letters in $w(\star)$, and if $r > 1$ for each $i \in \{1, 2, \ldots, r-1\}$, let $b_i$ be the position of the \mbox{$i$th} variable letter in $w(\star)$ and let $b_0$ be the position of the \mbox{$r$th} variable letter.
  (For instance, if $k = 4$ and $n = 10$, $w(\star) = (1, \star, 1, 4, \star, 2, \star, 3, 3, 1)$, then $r = 3$, $b_1 = 2$, $b_2 = 5$, and $b_0 = 7$.
  The reason we let $b_0$ represent the position of the \mbox{$r$th} variable letter is because we will be performing modulo arithmetic on the index of the $b$'s.)

  For each $i \in \{1, 2, \ldots, mr\}$ define $s_i = nt_{\lceil i/r \rceil} + b_{i \bmod r}$.
  Observe that $r \le n/2$, and since $t_1 < t_2 < \cdots < t_m$, it follows that $s = (s_1, s_2, \ldots, s_{mr}) \in \calJ_{mr}$.

  To define $c \in S^{mr+1}$ we will make use of some further helpful notation. 
  Let $L = \bigl( L(1), L(2), \ldots, L(r+1) \bigr)$ be a partition of $\{1, 2, \ldots, n\} \setminus \{b_1, b_2, \ldots, b_0\}$ such that each $L(i)$ is nonempty and $\max L(i) < \min L(i+1)$ for every $i \in \{1, 2, \ldots, r\}$. 
  For arithmetical convenience put $L(0) = L(r)$. 
  (For instance, if $k = 4$, $n = 10$, $w(\star) = (1, \star, 1, 4, \star, 2, \star, 3, 3, 1)$, then $L(1) = \{1\}$, $L(2) = \{3,4\}$, $L(3) = \{6\}$, $L(4) = \{8, 9, 10\}$, and $L(0) = \{6\}$.)

  Now we define $c \in S^{mr+1}$ as follows:
  \[
    c(i) = 
    \begin{cases}
      a(1)\prod_{j \in L(1)} f_j(nt_1 + j) & \mbox{if $i = 1$,} \\
      \prod_{j \in L(i \bmod r)}  f_j(nt_{\lceil i/(r+1) \rceil} + j) & \mbox{if $i \not\equiv 1 \pmod r$, }\\
      \prod_{j \in L(r+1)} \bigl( f_j(nt_{\lceil i/(r+1) \rceil} + j) \bigr) a\bigl( \lceil (i+1)/(r+1) \rceil \bigr) & \\
      \hspace{2em} \cdot\prod_{j \in L(1)} f_j(nt_{\lceil (i+1)/(r+1) \rceil} + j) & \mbox{if $i \bmod r \equiv 1$ and $i \ne mr +1$, }\\
      \prod_{j \in L(r+1)} \bigr( f_i(nt_m + j) \bigr) a(m+1) & \mbox{if $i = mr + 1$.}
    \end{cases}
  \]

  Then we have that $x(mr, c, s, f_\ell) = x(m, a, t, g_{w(\ell)})$ for all $\ell \in \{1, 2, \ldots, k\}$.
\end{proof}

With Lemma \ref{lem:pr-jsets} we can prove a stronger relationship between $J$-sets and $J(S)$.

\begin{thm}
  \label{thm:jsets-ideal}
  Let $(S, \cdot)$ be a semigroup and $A \subseteq S$.
  Then $J(S) \cap \overline{A} \ne \emptyset$ if and only if $A$ is a $J$-set.
\end{thm}
\begin{proof}
  ($\Rightarrow$)
  Let $p \in J(S) \cap \overline{A}$. 
  Since $p \in J(S)$, every member of $p$ is a $J$-set.
  Hence $A$ is a $J$-set.

  ($\Leftarrow$)
  Suppose $A$ is a $J$-set. 
  By Lemma \ref{lem:pr-jsets} and \cite[Theorem 3.11]{Hindman:1998fk}, there exists an ultrafilter $p \in \beta S$ such that $A \in p$ and every member of $p$ is a $J$-set. 
  Therefore $p \in J(S)$.
\end{proof}


\section{Central Sets Theorem}
The final proof that $J(S)$ is nonempty follows from the Central Sets Theorem.
\begin{defn}
  Let $(S, \cdot)$ be a semigroup and $A \subseteq S$.
  We call $A$ a \textsl{$C$-set} if and only if there exist $m \colon \Pf(\calT) \to \bbN$, $\alpha \in \bigtimes_{F \in \Pf(\calT)} S^{m(F)+1}$, and $\tau \in \bigtimes_{F \in \Pf(\calT)} \calJ_{m(F)}$ such that 
  \begin{itemize}
    \item[(1)] if $F$, $G \in \Pf(\calT)$ with $F \subsetneq G$, then $\tau(F)\bigl( m(F) \bigr) < \tau(G)(1)$, and
    \item[(2)] whenever $n \in \bbN$, $G_1$,$G_2$, \dots, $G_n \in \Pf(\calT)$  with $G_1 \subsetneq G_2 \subsetneq \cdots \subsetneq G_n$, and for each $i \in \{1, 2, \ldots, n\}$, $f_i \in G_i$, then we have $\prod_{i=1}^n x(m(G_i), \alpha(G_i), \tau(G_i), f_i) \in A$.
  \end{itemize}
\end{defn}

We now prove that $C$-sets and idempotents in $J(S)$ are closely related.
This proof uses the powerful \cite[Lemma 14.9]{Hindman:1998fk} which we now state.

\begin{lem}[{\cite[Lemma 14.9]{Hindman:1998fk}}]
  Let $F$ be a set, $(D, \le)$ a directed set, and let $(S, \cdot)$ be a semigroup.
  Let $\la T_i \ra_{i \in D}$ be a decreasing family of subsets of $S$ such that for each $i \in D$ and $x \in T_i$, there exists $j \in D$ with $xT_j \subseteq T_i$.
  Put $\mathbf{Q} = \bigcap_{i \in D} c\ell_{\beta S}(T_i)$.
  Then $\mathbf{Q}$ is a compact subsemigroup of $\beta S$.
  Let $\la E_i \ra_{i \in D}$ and $\la I_i \ra_{i \in D}$ be decreasing families of nonempty subsets of $\bigtimes_{f \in F} S$ with the following properties:
  \begin{itemize}
    \item[(a)]
      For each $i \in D$, $I_i \subseteq E_i$. 

    \item[(b)]
      For each $i \in D$ and every $\vec{x} \in I_i$, there exists $j \in D$ such that $\vec{x}E_j \subseteq I_i$.

    \item[(c)]
      For each $i \in D$ and every $\vec{x} \in E_i \setminus I_i$, there exists $j \in D$ such that $\vec{x} E_j \subseteq E_i$ and $\vec{x}I_j \subseteq I_i$.
  \end{itemize}

  Let $Y = \bigtimes_{f \in F} \beta S$, let $E = \bigcap_{i \in D} c\ell_{Y}(E_i)$, and let $I = \bigcap_{i \in D} c\ell_Y(I_i)$.
  Then $E$ is a subsemigroup of $\bigtimes_{f \in F} \mathbf{Q}$ and $I$ is an ideal of $E$. 
  Additionally, if either 
  \begin{itemize}
    \item[(d)]
      for each $i \in D$, $T_i = S$ and $\{\, a \in S : \overline{a} \not\in E_i \,\}$ is \emph{not} piecewise syndetic, or

    \item[(e)]
      for each $i \in D$ and each $a \in T_i$, $\overline{a} \in E_i$,
  \end{itemize}
then given any $p \in K(\mathbf{Q})$, we have $\overline{p} \in E \cap K(\bigtimes_{f \in F} \mathbf{Q}) = K(E) \subseteq I$. 
\end{lem}

\begin{thm}
  \label{thm:csets}
  Let $(S, \cdot)$ be a semigroup and let $A \subseteq S$. 
  Then $A$ is a $C$-set if and only if there exists an idempotent in $p \in \overline{A} \cap J(S)$.
\end{thm}
\begin{proof}
  ($\Rightarrow$)
  Pick $m \colon \Pf(\calT) \to \bbN$, $\alpha \in \bigtimes_{F \in \Pf(\calT)} S^{m(F)+1}$, and $\tau \in \bigtimes_{F \in \Pf(\calT)} \calJ_{m(F)}$ as guaranteed by the definition of a $C$-set. 
  For each $F \in \Pf(\calT)$ define 
  \begin{align*}
    T(F) &= \bigl\{\, \textstyle \prod_{i=1}^n x(m(F_i), \alpha(F_i), \tau(F_i), f_i) : \mbox{$n \in \bbN$, for each $i \in \{1, 2, \ldots, n\}$, $F_i \in \Pf(\calT)$,}\\
 &\hspace{12em}\mbox{$F \subsetneq F_1 \subsetneq F_2 \subsetneq \cdots \subsetneq F_n$, $\la f_i \ra_{i=1}^n \in \bigtimes_{i=1}^n F_i$} \,\bigr\}.
  \end{align*}

  Observe for each $F \in \Pf(\calT)$, $T(F)$ is a nonempty subset of $A$, and the collection $\{\, T(F) : F \in \Pf(\calT) \,\}$ has the finite intersection property since $T(F \cup G) \subseteq T(F) \cap T(G)$ for all $F$, $G \in \Pf(\calT)$.
  Therefore $\mathbf{Q} = \bigcap_{F \in \Pf(\calT)} c\ell_{\beta S}\bigl(T(F)\bigr)$ is a closed nonempty subset of $\beta S$. 

  We show that $\mathbf{Q}$ is in fact a subsemigroup of $\beta S$.
  To see that $\mathbf{Q}$ is a subsemigroup it suffices, by \cite[Theorem 4.20]{Hindman:1998fk}, to show that for all $F \in \Pf(\calT)$ and for every $y \in T(F)$, there exists $G \in \Pf(\calT)$ such that $yT(G) \subseteq T(F)$. 
  So let $F \in \Pf(\calT)$ and $y \in T(F)$.
  Pick $n \in \bbN$, for every $i \in \{1, 2, \ldots, n\}$ pick $F_i \in \Pf(\calT)$ with $F \subsetneq F_1 \subsetneq F_2 \subsetneq \cdots F_n$, and $\la f_i \ra_{i=1}^n \in \bigtimes_{i=1}^n F_i$ such that $y = \prod_{i=1}^n x(m(F_i), \alpha(F_i), \tau(F_i), f_i)$. 
  We show that $yT(F_n) \subseteq T(F)$.
  Let $z \in T(F_n)$ and pick $m \in \bbN$, for every $i \in \{1, 2, \ldots, m\}$ pick $G_i \in \Pf(\calT)$ with $F_n \subsetneq G_1 \subsetneq G_2 \subsetneq \cdots \subsetneq G_m$, and pick $\la g_i \ra_{i=1}^m \in \bigtimes_{i=1}^m G_i$ such that $z = \prod_{i=1}^m x(m(G_i), \alpha(G_i), \tau(G_i), g_i)$.

  For each $i \in \{1, 2, \ldots, n+m\}$ define 
  \[
    H_i = 
    \begin{cases}
      F_i & \mbox{if $i \in \{1, 2, \ldots, n\}$,} \\
      G_{i-n} & \mbox{if $i \in \{n+1, n+2, \ldots, n+m\}$,}
    \end{cases}
  \]
  and define the sequence
  \[
    h_i =
    \begin{cases}
      f_i & \mbox{if $i \in \{1, 2, \ldots, n\}$,} \\
      g_{i-n} & \mbox{if $i \in \{n+1, n+2, \ldots, n+m\}$.}
    \end{cases}
  \]
  Then $F \subsetneq H_1 \subsetneq H_2 \subsetneq \cdots \subsetneq H_{n+m}$, $\la h_i \ra_{i=1}^{n+m} \in \bigtimes_{i=1}^{n+m} H_i$, and 
  \[
    \textstyle
    yz = \prod_{i=1}^{n+m} x(m(H_i), \alpha(H_i), \tau(H_i), h_i) \in T(F)
  \]
  Hence $\mathbf{Q}$ is a subsemigroup of $\beta S$. 

  We now claim that $K(\mathbf{Q}) \subseteq \overline{A} \cap J(S)$. 
  If this claim is true, then we are done since any idempotent in $K(\mathbf{Q})$ will establish the theorem. 
  Since $\mathbf{Q} \subseteq c\ell_{\beta S} (A)$, it evident that $K(\mathbf{Q}) \subseteq c\ell_{\beta S} (A)$. 

  Now let $p \in K(\mathbf{Q})$ and $B \in p$.
  Using \cite[Lemma 14.9]{Hindman:1998fk} we shall show that $B$ is a $J$-set. 
  Let $F \in \Pf(\calT)$ and put $D = \{\, G \in \Pf(\calT) : F \subseteq G \,\}$. 
  Observe $\mathbf{Q} = \bigcap_{G \in D} c\ell_{\beta S} (T(G))$.

  For $G \in D$, define $I(G) \subseteq \bigtimes_{f \in F} S$ as follows: for $w \in \bigtimes_{f \in F} S$, $w \in I(G)$ if and only if there is some $n \in \bbN \setminus \{1\}$ such that 
  \begin{itemize}
    \item[(1)]
      there exist disjoint nonempty sets $C_1$ and $C_2$ with $\{1, 2, \ldots, n\} = C_1 \cup C_2$, 
      
    \item[(2)]
      there exists a strictly increasing sequence $\la G_i \ra_{i=1}^n$ in $\Pf(\calT)$ with $G \subsetneq G_1$, and

    \item[(3)]
      there exists $\sigma \in \bigtimes_{i \in C_1} G_i$, 
  \end{itemize}
such that for every $f \in F$, if $\gamma_f \in \bigtimes_{i=1}^n G_i$ is defined by 
  \[
    \gamma_f(i) = 
    \begin{cases}
      \sigma(i) & \mbox{if $i \in C_1$,} \\
      f & \mbox{if $i \in C_2$,} 
    \end{cases}
  \]
then $w(f) = \prod_{i=1}^n x(m(G_i), \alpha(G_i), \tau(G_i), \gamma_f(i))$.
  For each $G \in D$, define $E(G) = I(G) \cup \{\, \overline{b} : b \in T(G) \,\}$. 

  We claim that $\la E(G) \ra_{G \in D}$ and $\la I(G) \ra_{G \in D}$ satisfy statements (a), (b), (c), and (e) of \cite[Lemma 14.9]{Hindman:1998fk}.

  Assume, temporarily, that our claim is true. 
  Then by \cite[Lemma 14.9]{Hindman:1998fk} if $Y = \bigtimes_{f \in F} \beta S$, $E = \bigcap_{G \in D} c\ell_Y(E(G))$, and $I = \bigcap_{G \in D} c\ell_Y(I(G))$, then $E$ is a subsemigroup of $Y$, $I$ is an ideal of $E$, and for every $p \in K(\mathbf{Q})$, $\overline{p} = (p, p, \ldots, p) \in K(E) \subseteq I$. 
  Since $\bigtimes_{f \in F} c\ell_{\beta S} (B)$ is a neighborhood of $\overline{p}$ we can pick $w \in I(F) \cap \bigtimes_{f \in F} c\ell_{\beta S} (B)$. 
  Pick $n \in \bbN\setminus \{1\}$, $C_1$, $C_2$, $\la G_i \ra_{i=1}^n$, and $\sigma \in \bigtimes_{i \in C_1} G_i$ as guaranteed by the definition of $I(F)$. 
  Put $r = |C_2|$ and enumerate $C_2$ as a strictly increasing sequence $h_1$, $h_2$, \dots, $h_r$.
  Put $u = \sum_{i=1}^r m(G_i)$.
  We define $c \in S^{u+1}$ and $t \in \calJ_u$ such that for all $f \in F$, $x(u, c, t, f) \in B$.

  Define $c \in S^{u+1}$ as follows:
  \[
    c(1) = 
    \begin{cases}
      \alpha(G_1)(1) & \mbox{if $h_1 = 1$,} \\
      \bigl(\prod_{i=1}^{h_1-1} x(m(G_i), \alpha(G_i), \tau(G_i), \sigma(i))\bigr) \cdot \alpha(G_{h_1})(1) & \mbox{if $h_1 > 1$.}
    \end{cases}
  \]
  For each positive integer $j$ with $1 < j < m(G_{h_1})$ put $c(j) = \alpha(G_{h_2})(j)$, and for each positive integer $j$ with $1 \le j \le m(G_{h_1})$ define $t_j = \tau(G_{h_1})(1)$. 

  Now for each $s \in \{1, 2, \ldots, u-1\}$ put $v_s = \sum_{i=1}^s m(G_{h_i})$, and define
  \[
    c(v_s+1) = 
    \begin{cases}
      \alpha(G_{h_s})(m(G_{h_s}+1))\alpha(G_{h_{s+1}})(1) & \mbox{if $h_{s+1} = h_s + 1$,} \\

      \alpha(G_{h_s})(m(G_{h_s}+1))\cdot & \\
      \hspace{3.5em}\bigl(\prod_{i=h_{s}+1}^{h_{s+1}-1} x(m(G_i), \alpha(G_i), \tau(G_i), \sigma(i))\bigr)\alpha(G_{h_{s+1}})(1) & \mbox{if $h_{s+1} > h_s+1$.}
    \end{cases}
  \]
  For each $s \in \{1, 2, \ldots, u-1\}$ and every positive integer $j$ with $v_s < j \le \sum_{i=1}^{s+1} m(G_{h_i})$ put $t_j = \tau(G_{h_{s+1}})(j-u)$. 
  Finally, we define
  \[
    c(u+1) = 
    \begin{cases}
      \alpha(G_{h_r})(m(G_n) + 1) & \mbox{if $h_r = n$,} \\
      \alpha(G_{h_r})(m(G_{h_r}+1))\prod_{i=h_r+1}^n x(m(G_i), \alpha(G_i), \tau(G_i), \sigma(i)) & \mbox{if $h_r < n$.}
    \end{cases}
  \]
  Then for every $f \in F$, $x(u, c, t, f) \in B$, and so $B$ is a $J$-set.

  We now prove our claim that the families $\la E(G) \ra_{G \in D}$ and $\la I(G) \ra_{G \in D}$ satisfy statements (a), (b), (c), and (e) of \cite[Lemma 14.9]{Hindman:1998fk}.

  By definition of $\la E(G) \ra_{G \in D}$ and $\la I(G) \ra_{G \in D}$ it is immediate that statements (a) and (e) are satisfied. 

  We now show statement (b) which states that for every $G \in D$ and each $w \in I(G)$ there exists $H \in D$ such that $w E(H) \subseteq I(G)$.
  Let $G \in D$ and $w \in I(G)$. 
  Pick $n \in \bbN$, $C_1$, $C_2$, $\la G_i \ra_{i=1}^n$, and $\sigma \in \bigtimes_{i \in C_1} G_i$ as guaranteed by the definition of $I(G)$..
  We show that $w E(G_n) \subseteq I(G)$. 
  Let $z \in E(G_n)$. 
  First assume that $z = \overline{b}$ for some $b \in T(G_n)$. 
  Pick $m \in \bbN$, for each $i \in \{1, 2, \ldots, m\}$ pick $F_i \in \Pf(\calT)$ with $G_n \subsetneq F_1 \subsetneq F_2 \subsetneq \cdots \subsetneq F_n$, and pick $\la f_i \ra_{i=1}^n \in \bigtimes_{i=1}^n F_i$ such that $b = \prod_{i=1}^m x(m(F_i), \alpha(F_i), \tau(F_i), f_i)$. 
  Put $D_1 = C_1 \cup \{n+1, n+2, \ldots, n+m\}$ and for each $i \in \{1, 2, \ldots, n+m\}$ put 
  \[
    H_i = 
    \begin{cases}
      G_i & \mbox{if $i \le n$,} \\
      F_{i-n} & \mbox{if $i > n$.}
    \end{cases}
  \]
  Define $\rho \in \bigtimes_{i \in D_1} H_i$ by 
  \[
    \rho(i) =
    \begin{cases}
      \sigma(i) & \mbox{if $i \le n$,} \\
      f_{i-n} & \mbox{if $i > n$.}
    \end{cases}
  \]
  Then with $n+m$, $D_1$, $C_2$, $\la H_i \ra_{i=1}^{n+m}$, and $\rho$ we have that $w \cdot z \in I(G)$. 

  Now we assume that $z \in I(G_n)$. 
  Pick $m \in \bbN$, $D_1$, $D_2$, $\la F_i \ra_{i=1}^m$, and $\rho$ as guaranteed by the definition of $I(G_n)$. 
  Put $E_1 = C_1 \cup \{\,n + i : i \in D_1 \,\}$ and put $E_2 = C_2 \cup \{\, n + i : i \in D_2 \,\}$.
  For each $i \in \{1, 2, \ldots, n+m\}$ put
  \[
    H_i = 
    \begin{cases}
      G_i & \mbox{if $i \le n$,} \\
      F_{i-n} & \mbox{if $i > n$.}
    \end{cases}
  \]
  Define $\mu \in \bigtimes_{i \in E_i} H_i$ by
  \[
    \mu(i) = 
    \begin{cases}
      \sigma(i) & \mbox{if $i \le n$,} \\
      \rho(i) & \mbox{if $i > n$.}
    \end{cases}
  \]
  Then with $n+m$, $E_1$, $E_2$, $\la H_i \ra_{i=1}^{n+m}$, and $\mu$ we have that $w \cdot z \in I(G)$.

  We now verify statement (c) which states that for all $G \in D$ and every $w \in E(G) \setminus I(G)$, there exists $H \in D$ such that $wE(H) \subseteq E(G)$ and $wI(H) \subseteq I(G)$.
  So let $G \in D$ and $w \in E(G) \setminus I(G)$. 
  Pick $b \in T(G)$ such that $w = \overline{b}$. 
  Pick $n \in \bbN$, for each $i \in \{1, 2, \ldots, n\}$ pick $G_i \in \Pf(\calT)$ with $G \subsetneq G_1 \subsetneq G_2 \subsetneq \cdots \subsetneq G_n$, and pick $\la f_i \ra_{i=1}^n \in \bigtimes_{i=1}^n G_i$ such that $b = \prod_{i=1}^n x(m(G_i), \alpha(G_i), \tau(G_i), f_i)$. 
  Then similar to what we have done above, we have that $wE(G_n) \subseteq E(G)$ and $wI(G_n) \subseteq I(G)$.

  This finishes this direction of our proof. 

  ($\Leftarrow$)
  Let $p = p \cdot p \in \overline{A} \cap J(S)$.
  Since $A \in p$ and $p$ is an idempotent we have, by \cite[Lemma 4.14]{Hindman:1998fk}, that $x^{-1}A^\star \in p$ for all $x \in A^\star$ where $A^\star = \{\, x \in A : x^{-1}A \in p\}$. 
  We will recursively define our functions $m$, $\alpha$, and $\tau$ by the size of $F \in \Pf(\calT)$ such that we satisfy, for $F \in \Pf(\calT)$, the following hypothesis:
  \begin{itemize}
    \item[(1)] If $\emptyset \ne G \subsetneq F$, then $\tau(G)\bigl( m(G) \bigr) < \tau(F)(1)$.
    
    \item[(2)] If $n \in \bbN$, $\emptyset \ne G_1 \subsetneq G_2 \subsetneq \cdots \subsetneq G_n = F$, and $\la f_i \ra_{i=1}^n \in \bigtimes_{i=1}^n G_i$, then \[\textstyle \prod_{i=1}^n x(m(G_i), \alpha(G_i), \tau(G_i), f_i)) \in A^\star.\]
  \end{itemize}

  Let $F \in \Pf(\calT)$.
  First, assume that $|F| = 1$, that is $F = \{f\}$ for some sequence $f \in \calT$.
  Since $A^\star$ is a $J$-set, pick $m(F) \in \bbN$, $\alpha(F) \in S^{m(F)+1}$, and $\tau(F) \in \calJ_{m(F)}$ such that $x(m(F), \alpha(F), \tau(F), f) \in A^\star$. 

  Now assume that $|F| > 1$ and for all $\emptyset \ne G \subsetneq F$ we have defined $m(G)$, $\alpha(G)$, and $\tau(G)$ so that hypotheses (1) and (2) hold.
  Put 
  \begin{align*}
    M &= \bigl\{\, \textstyle \prod_{i=1}^n x(m(G_i), \alpha(G_i), \tau(G_i), f_i) : \mbox{$n \in \bbN$, $\emptyset \ne G_1 \subsetneq G_2 \subsetneq \cdots \subsetneq G_n \subsetneq F$} \\
    &\hspace{12em} \mbox{and $\la f_i \ra_{i=1}^n \in \bigtimes_{i=1}^n G_i$} \,\bigr\}.
  \end{align*}
  Observe that since $F$ is finite, $M$ is also finite.
  By hypothesis (2) we have that $M \subseteq A^\star$.
  Put $B = A^\star \cap \bigcap_{x \in M} x^{-1}A^\star$, then $B \in p$ and so $B$ is a $J$-set. 

  For each $\emptyset \ne G \subsetneq F$, put $l(G) = \tau(G)\bigl( m(G) \bigr)$ and put $k = \max\{\, l(G) : \emptyset \ne G \subsetneq F \,\}$.
  By Lemma \ref{lem:jset-start}, pick $m(F) \in \bbN$, $\alpha(F) \in S^{m(F)+1}$, and $\tau(F) \in \calJ_{m(F)}$ such that $\tau(F)(1) > k$ and for every $f \in F$, $x(m(F), \alpha(F), \tau(F), f) \in B$.
  Hypothesis (1) is satisfied, since $\tau(F)(1) > k \ge \max l(G)$ for all $\emptyset \ne G \subsetneq F$.
  We show that hypothesis (2) is also satisfied.
  Let $n \in \bbN$, $\emptyset \ne G_1 \subsetneq G_2 \subsetneq \cdots \subsetneq G_{n-1} \subsetneq G_n = F$, and $\la f_i \ra_{i=1}^n \bigtimes_{i=1}^n G_i$. 
  If $n = 1$, then $x(m(G_1), \alpha(G_1), \tau(G_1), f) \in B \subseteq A^\star$.
  Now assume that $n > 1$ and put $y = \prod_{i=1}^{n-1} x(m(G_i), \alpha(G_i), \tau(G_i), f_i)$. 
  By definition $y \in M$ and since $x(m(G_n), \alpha(G_n), \tau(G_n), f_n) = x(m(F), \alpha(F), \tau(F), f_n) \in B \subseteq y^{-1}A^\star$ we have
  \[
    \textstyle
    \prod_{i=1}^n x(m(G_i), \alpha(G_i), \tau(G_i), f_i) = y \cdot x(m(F), \alpha(F), \tau(F), f_n) \in A^\star.
  \]
  Hence hypotheses (1) and (2) are satisfied and this completes the proof for this direction.
\end{proof}

It's reasonably clear that any semigroup $S$ is a $C$-set in itself.
Then Theorem \ref{thm:csets} shows that there is an idempotent in $J(S)$ and hence $J(S)$ is nonempty.
The Central Sets Theorem is simply the assertion that central sets are $C$-sets. 

\begin{cor}[Central Sets Theorem]
  \label{cor:cst}
  Every central set in a semigroup is a $C$-set.
\end{cor}
\begin{proof}
  Let $(S, \cdot)$ be a semigroup with $A \subseteq S$ a central set. 
  Since $A$ is a central set, there exists an idempotent $p \in K(\beta S)$ with $A \in p$.
  Since $K(S) \subseteq J(S)$, we have that $A$ is also a $C$-set.
\end{proof}
\begin{rmk}
  Unfortunately, despite its name the Central Sets Theorem does not characterize central sets. 
  See \cite{Hindman:2007fk} for an example of a $C$-set in $\bbN$ which is not a central set.   
\end{rmk}

Similar to how the definition of a $J$-set is simpler when the underlying semigroup is commutative we have the following result which shows that $C$-sets also have a simpler characterization when the underlying semigroup is commutative. 

\begin{thm}
  Let $(S, +)$ be a commutative semigroup and $A \subseteq S$.
  Then $A$ is a $C$-set if and only if there exist functions $\alpha \colon \Pf(\calT) \to S$ and $H \colon \Pf(\calT) \to \Pf(\bbN)$ such that
  \begin{itemize}
    \item[(1)] if $F$, $G \in \Pf(\calT)$ with $F \subsetneq G$, then $\max H(F) < \min H(G)$, and
    
    \item[(2)] whenever $m \in \bbN$, $G_1$, $G_2$, \dots, $G_m \in \Pf(\calT)$ with $G_1 \subsetneq G_2 \subsetneq \cdots \subsetneq G_m$, and for each $i \in \{1, 2, \ldots, m\}$, $f_i \in G_i$, we have $\sum_{i=1}^m\bigl( \alpha(G_i) + \sum_{t \in H(G_i)} f_i(t)\bigr) \in A$.
  \end{itemize}
\end{thm}
\begin{proof}
  ($\Rightarrow$)
  Since $A$ is a $C$-set, we also have that $A$ is a $J$-set in a commutative semigroup.
  By Theorem \ref{thm:csets} we also know that $A \in p$ for some idempotent $p \in J(S)$. 
  We will recursively define our functions $\alpha$ and $H$ by the size of $F \in \Pf(\calT)$ such that we satisfy, for $F \in \Pf(\calT)$, the following hypotheses:
  \begin{itemize}
    \item[(i)] If $\emptyset \ne G \subsetneq F$, then $\max H(G) < \min H(F)$.
      
    \item[(ii)] If $n \in \bbN$, $\emptyset \ne G_1 \subsetneq G_2 \subsetneq \cdots \subsetneq G_n = F$, and $\la f_i \ra_{i=1}^n \in \bigtimes_{i=1}^n G_i$, then $\sum_{i=1}^n\bigl( \alpha(G_i) + \sum_{t \in H(G_i)} f_i(t)\bigr) \in A^\star$. 
  \end{itemize}
  
  Let $F \in \Pf(\calT)$.
  First, assume that $|F| = 1$, that is, $F = \{f\}$ for some sequence $f \in \calT$.
  Since $A^\star$ is a $J$-set (in a commutative semigroup), we may pick $\alpha(F) \in S$ and $H(F) \in \Pf(\bbN)$ such that $\alpha(F) + \sum_{t \in H(F)} f(t) \in A^\star$.

  Now assume that $|F| > 1$ and for all $\emptyset \ne G \subsetneq F$ we have defined $\alpha(G)$ and $H(G)$ so that hypotheses (i) and (ii) hold.
  Put
  \begin{align*}
    M &=\textstyle \bigl\{\, \sum_{i=1}^n \bigl( \alpha(G_i) + \sum_{t \in G_i} f_i(t) \bigr) : \mbox{$n \in \bbN$, $\emptyset \ne G_1 \subsetneq G_2 \subsetneq \cdots \subsetneq G_n \subsetneq F$,} \\
    &\hspace{12em} \mbox{and $\la f_i \ra_{i=1}^n \in \bigtimes_{i=1}^n G_i$} \,\bigr\}.
  \end{align*}
  Since $F$ is finite, $M$ is also finite and by hypothesis (ii) we have that $M \subseteq A^\star$.
  Put $B = A^\star \cap \bigcap_{x \in M} -x+A^\star$, then $B \in p$ and so $B$ is also a $J$-set.

  Put $k = \max \bigl( \bigcup\{\, H(G) : \emptyset \ne G \subsetneq F\,\} \bigr)$.
  Since $B$ is a $J$-set, pick $\alpha(F) \in S$ and $H(F) \in \Pf(\bbN)$ with $\min H(F) > k$ such that for all $f \in F$, $\alpha(F) + \sum_{t \in H(F)} f(t) \in B$. 

  Now hypothesis (i) is satisfied since $\min H(F) > k \ge \max H(G)$ for all $\emptyset \ne G \subsetneq F$. 
  We show that hypothesis (ii) is also satisfied.
  Let $n \in \bbN$, $\emptyset \ne G_1 \subsetneq G_2 \subsetneq \cdots \subsetneq G_n = F$, and $\la f_i \ra_{i=1}^n \in \bigtimes_{i=1}^n G_i$. 
  If $n = 1$, then $\sum_{i=1}^n \bigl( \alpha(G_i) + \sum_{t \in H(G_i)} f_i(t)\bigr) = \alpha(F) + \sum_{t \in H(F)} f_1(t) \in B \subseteq A^\star$. 
  Now assume that $n > 1$ and let $y = \sum_{i=1}^{n-1}\bigl( \alpha(G_i) + \sum_{t \in H(G_i)} f_i(t) \bigr)$.
  Then $y \in M$ and so $\alpha(G_n) + \sum_{t \in H(G_n)} f_n(t) = \alpha(F) + \sum_{t \in H(F)} f_n(t) \in B \subseteq -y + A^\star$. 
  Hence $\sum_{i=1}^n\bigl( \alpha(G_i) + \sum_{t \in H(G_i)} f_i(t) \bigr) \in A^\star$.

  ($\Leftarrow$)
  Pick $\alpha \colon \Pf(\calT) \to S$ and $H \colon \Pf(\calT) \to \Pf(\bbN)$ as guaranteed by assumption.
  For all $F \in \Pf(\calT)$ put
  \begin{align*}
    T(F) &= \bigl\{\, \textstyle \sum_{i=1}^n \bigl( \alpha(G_i) + \sum_{t \in H(G_i)} f_i(t)\bigr) : \mbox{$n \in \bbN$, for each $i \in \{1, 2, \ldots, n\}$, $F_i \in \Pf(\calT)$,}\\
 &\hspace{12em}\mbox{$F \subsetneq F_1 \subsetneq F_2 \subsetneq \cdots \subsetneq F_n$, $\la f_i \ra_{i=1}^n \in \bigtimes_{i=1}^n F_i$} \,\bigr\}.
  \end{align*}

  Observe that for every $F \in \Pf(\calT)$, $T(F)$ is a nonempty subset of $A$ and that the collection $\{\, T(F) : F \in \Pf(\calT) \,\}$ has the finite intersection property since $T(F \cup G) \subseteq T(F) \cap T(G)$ for all $F$, $G \in \Pf(\calT)$.
  Therefore $\mathbf{Q} = \bigcap_{F \in \Pf(\calT)} c\ell_{\beta S}\bigl(T(F)\bigr)$ is a closed nonempty subset of $\beta S$ contained in $c\ell_{\beta S}(A)$. 
  We show that $\mathbf{Q}$ is, in fact, a subsemigroup of $\beta S$. 
  To see that $\mathbf{Q}$ is a subsemigroup it suffices, by \cite[Theorem 4.20]{Hindman:1998fk}, to show that for all $F \in \Pf(\calT)$ and for every $y \in T(F)$, there exists $G \in \Pf(\calT)$ such that $yT(G) \subseteq T(F)$. 
  Let $F \in \Pf(\calT)$ and $y \in T(F)$. 
  Pick $n \in \bbN$, for every $i \in \{1, 2, \ldots, n\}$ pick $F_i \in \Pf(\calT)$ with $F \subsetneq F_1 \subsetneq F_2 \subsetneq \cdots \subsetneq F_n$, and pick $\la f_i \ra_{i=1}^n \in \bigtimes_{i=1}^n F_i$ such that $y = \sum_{i=1}^n \bigl( \alpha(F_i) + \sum_{t \in H(F_i)} f_i(t) \bigr)$.
  We show that $yT(F_n) \subseteq T_(F)$. 
  Let $z \in T(F_n)$ and pick $m \in \bbN$, for every $i \in \{1, 2, \ldots, m\}$ pick $G_i \in \Pf(\calT)$ with $F_n \subsetneq G_1 \subsetneq G_2 \subsetneq \cdots \subsetneq G_m$, and pick $\la g_i \ra_{i=1}^m \in \bigtimes_{i=1}^m G_i$ such that $z = \sum_{i=1}^m \bigl( \alpha(G_i) + \sum_{t \in H(G_i)} g_i(t) \bigr)$.

  For each $i \in \{1, 2, \ldots, n+m\}$ define
  \[
    K_i =
    \begin{cases}
      F_i & \mbox{if $i \in \{1, 2, \ldots, n\}$,} \\
      G_{i-n} & \mbox{if $i \in \{n+1, n+2, \ldots, n+m\}$,}
    \end{cases}
  \]
  and define the sequence
  \[
    k_i =
    \begin{cases}
      f_i & \mbox{if $i \in \{1, 2, \ldots, n\}$,} \\
      g_{i-n} & \mbox{if $i \in \{n+1, n+2, \ldots, n+m\}$,}
    \end{cases}
  \]
  Then $F \subsetneq K_1 \subsetneq K_2 \subsetneq \cdots \subsetneq K_{n+m}$, $\la k_i \ra_{i=1}^{n+m} \in \bigtimes_{i=1}^{n+m} K_i$, and so $yz = \sum_{i=1}^{n+m}\bigl( \alpha(K_i) + \sum_{t \in H(K_i)} k_i(t) \bigr) \in T(F)$.
  Hence $\mathbf{Q}$ is a subsemigroup of $\beta S$.

  We now claim that $K(\mathbf{Q}) \subseteq c\ell_{\beta S}(A) \cap J(S)$. 
  If this claim is true, then we can pick any idempotent in $K(\mathbf{Q})$ to satisfy the sufficiency condition of Theorem \ref{thm:csets} and conclude that $A$ is a $J$-set.
  Since $\mathbf{Q} \subseteq c\ell_{\beta S}(A)$, it's evident that $K(\mathbf{Q}) \subseteq c\ell_{\beta S}(A)$. 

  Now let $p \in K(\mathbf{Q})$ and $B \in p$.
  Using \cite[Lemma 14.9]{Hindman:1998fk} we shall show that $B$ is a $J$-set.
  Let $F \in \Pf(\calT)$ and put $D = \{\, G \in \Pf(\calT) : F \subseteq G \,\}$. 
  Observe that $\mathbf{Q} = \bigcap_{G \in D} c\ell_{\beta S} (T(G))$.

  For each $G \in D$ define $I(G) \subseteq \bigtimes_{f \in F} S$ as follows: for $w \in \bigtimes_{f \in F} S$, $w \in I(G)$ if and only if there is some $n \in \bbN \setminus \{1\}$ such that
  \begin{itemize}
    \item[(i)]
      there exist disjoint nonempty sets $C_1$ and $C_2$ with $\{1, 2, \ldots, n\} = C_1 \cup C_2$,

    \item[(ii)]
      there exists a strictly increasing sequence $\la G_i \ra_{i=1}^n$ in $\Pf(\calT)$ with $G \subsetneq G_1$, and

    \item[(iii)]
      there exists $\sigma \in \bigtimes_{i \in C_1} G_i$,
  \end{itemize}
  such that for every $f \in F$, if $\gamma_f \in \bigtimes_{i=1}^n G_i$ is defined by
  \[
    \gamma_f(i) =
    \begin{cases}
      \sigma(i) & \mbox{if $i \in C_1$,} \\
      f & \mbox{if $i \in C_2$.}
    \end{cases}
  \]
  then $w(f) = \sum_{i=1}^n\bigl( \alpha(G_i) + \sum_{t \in H(G_i)} \gamma_f(i) \bigr)$.
  For each $G \in D$, define $E(G) = I(G) \cup \{\, \overline{b} : b \in T(G) \,\}$.

  We claim that $\la E(G) \ra_{G \in D}$ and $\la I(G) \ra_{G \in D}$ satisfy statements (a), (b), (c), and (e) of \cite[Lemma 14.9]{Hindman:1998fk}.

  Assume, temporarily, that our claim is true. 
  Then by \cite[Lemma 14.9]{Hindman:1998fk} if $Y = \bigtimes_{f \in F} \beta S$, $E = \bigcap_{G \in D} c\ell_Y(E(G))$, and $I = \bigcap_{G \in D} c\ell_Y(I(G))$, then $E$ is a subsemigroup of $Y$, $I$ is an ideal of $E$, and for every $p \in K(\mathbf{Q})$, $\overline{p} = (p, p, \ldots, p) \in K(E) \subseteq I$. 
  Since $\bigtimes_{f \in F} c\ell_{\beta S} (B)$ is a neighborhood of $\overline{p}$ we can pick $w \in I(F) \cap \bigtimes_{f \in F} c\ell_{\beta S} (B)$. 
  Pick $n \in \bbN\setminus \{1\}$, $C_1$, $C_2$, $\la G_i \ra_{i=1}^n$, and $\sigma \in \bigtimes_{i \in C_1} G_i$ as guaranteed by the definition of $I(F)$. 
  By Lemma \ref{lem:comm-jsets} it suffices to define $a \in S$ and $H \in \Pf(\bbN)$ such that $a + \sum_{t \in H} f(t) \in B$ for every $f \in F$. 
  Put $a = \sum_{i=1}^n \alpha(G_i) + \sum_{i \in C_1} \sigma(i)$ and put $H = C_2$.
  Then for every $f \in F$, $a + \sum_{t \in H} f(t) \in B$.

  We now prove our claim that the families $\la E(G) \ra_{G \in D}$ and $\la I(G) \ra_{G \in D}$ satisfy statements (a), (b), (c), and (e) of \cite[Lemma 14.9]{Hindman:1998fk}.

  By definition of $\la E(G) \ra_{G \in D}$ and $\la I(G) \ra_{G \in D}$ it is immediate that statements (a) and (e) are satisfied. 

  We now show statement (b) which states that for every $G \in D$ and each $w \in I(G)$ there exists $H \in D$ such that $w E(H) \subseteq I(G)$.
  Let $G \in D$ and $w \in I(G)$. 
  Pick $n \in \bbN$, $C_1$, $C_2$, $\la G_i \ra_{i=1}^n$, and $\sigma \in \bigtimes_{i \in C_1} G_i$ as guaranteed by the definition of $I(G)$.
  We show that $w E(G_n) \subseteq I(G)$. 
  Let $z \in E(G_n)$. 
  First assume that $z = \overline{b}$ for some $b \in T(G_n)$. 
  Pick $m \in \bbN$, for each $i \in \{1, 2, \ldots, m\}$ pick $F_i \in \Pf(\calT)$ with $G_n \subsetneq F_1 \subsetneq F_2 \subsetneq \cdots \subsetneq F_n$, and pick $\la f_i \ra_{i=1}^n \in \bigtimes_{i=1}^n F_i$ such that $b = \sum_{i=1}^m \bigl( \alpha(F_i) + \sum_{t \in H(F_i)} f_i(t) \bigr)$. 
  Put $D_1 = C_1 \cup \{n+1, n+2, \ldots, n+m\}$ and for each $i \in \{1, 2, \ldots, n+m\}$ put 
  \[
    K_i = 
    \begin{cases}
      G_i & \mbox{if $i \le n$,} \\
      F_{i-n} & \mbox{if $i > n$.}
    \end{cases}
  \]
  Define $\rho \in \bigtimes_{i \in D_1} H_i$ by 
  \[
    \rho(i) =
    \begin{cases}
      \sigma(i) & \mbox{if $i \le n$,} \\
      f_{i-n} & \mbox{if $i > n$.}
    \end{cases}
  \]
  Then with $n+m$, $D_1$, $C_2$, $\la K_i \ra_{i=1}^{n+m}$, and $\rho$ we have that $w \cdot z \in I(G)$. 

  Now we assume that $z \in I(G_n)$. 
  Pick $m \in \bbN$, $D_1$, $D_2$, $\la F_i \ra_{i=1}^m$, and $\rho$ as guaranteed by the definition of $I(G_n)$. 
  Put $E_1 = C_1 \cup \{\,n + i : i \in D_1 \,\}$ and put $E_2 = C_2 \cup \{\, n + i : i \in D_2 \,\}$.
  For each $i \in \{1, 2, \ldots, n+m\}$ put
  \[
    K_i = 
    \begin{cases}
      G_i & \mbox{if $i \le n$,} \\
      F_{i-n} & \mbox{if $i > n$.}
    \end{cases}
  \]
  Define $\mu \in \bigtimes_{i \in E_i} H_i$ by
  \[
    \mu(i) = 
    \begin{cases}
      \sigma(i) & \mbox{if $i \le n$,} \\
      \rho(i) & \mbox{if $i > n$.}
    \end{cases}
  \]
  Then with $n+m$, $E_1$, $E_2$, $\la K_i \ra_{i=1}^{n+m}$, and $\mu$ we have that $w \cdot z \in I(G)$.

  We now verify statement (c) which states that for all $G \in D$ and every $w \in E(G) \setminus I(G)$, there exists $H \in D$ such that $wE(H) \subseteq E(G)$ and $wI(H) \subseteq I(G)$.
  So let $G \in D$ and $w \in E(G) \setminus I(G)$. 
  Pick $b \in T(G)$ such that $w = \overline{b}$. 
  Pick $n \in \bbN$, for each $i \in \{1, 2, \ldots, n\}$ pick $G_i \in \Pf(\calT)$ with $G \subsetneq G_1 \subsetneq G_2 \subsetneq \cdots \subsetneq G_n$, and pick $\la f_i \ra_{i=1}^n \in \bigtimes_{i=1}^n G_i$ such that $b = \sum_{i=1}^n \bigl( \alpha(G_i) + \sum_{t \in H(G_i)} f_i(t)\bigr)$. 
  Then similar to what we have done above, we have that $wE(G_n) \subseteq E(G)$ and $wI(G_n) \subseteq I(G)$.

  This finishes this direction of our proof. 
\end{proof}

 \theendnotes

% % Draft of my Dynamical Characterization of C-sets Results
% by: John H. Johnson
% email: john.j.jr@gmail.com

\newcommand{\ds}{(X, \la T_s \ra_{s \in S})}
\chapter{Dynamical Characterization of $C$-sets}
\section{Dynamical Systems and $\beta S$}
Furstenburg in \cite[Chapter 8]{Furstenberg:1981fk} originally defined central sets and proved the original Central Sets Theorem using the notions of topological dynamics. 
It is a natural to investigate whether $C$-sets admit a dynamical characterization, and in this chapter we shall prove that such a characterization exists.
Before we can begin we will need to define what we mean by a dynamical
system and provide a connection to $\beta S$.
  \begin{defn}
    \label{defn:semiact}
    Let $S$ be a set. 
    A triple $(X, S, \pi)$ is a \textsl{semigroup action of $S$ on
      $X$} (or more shortly, \textsl{$S$ acts on $X$}) if and only if 
      \begin{itemize}
        \item[(1)] $S$ is a semigroup; and
        
        \item[(2)] $\pi : S \times X \to X$ is a function such that
          for all $s$, $t \in S$ and for every $x \in X$,
          \[ \pi\bigl(s, \pi(t,x)\bigr) = \pi(st, x). \]
      \end{itemize}
  \end{defn}
  
  \begin{rmk}
    The usual (and sometimes confusing) convention is to write the value
    $\pi(s,x)$ as $s \cdot x$.
    Following this convention the condition on $\pi$ becomes $s \cdot (t
    \cdot x) = st \cdot x$.
    We will not follow this convention since we will soon introduce
    better notation that will suit our purposes.
  \end{rmk}

Eventually we will want to extend our action of $S$ on $X$ to an
action of $\beta S$ on $X$.
To produce this extension we will be using \cite[Lemma
3.30]{Hindman:1998fk} (and also \cite[Corollary
4.22]{Hindman:1998fk}). 
Before we can use these results directly we will need to characterize
semigroup actions into a more convenient form. 

  \begin{prop} 
    \label{prop:semiact}
    Let $X$ be a set, $(S,\cdot)$ a semigroup and $\pi : S \times X \to X$.
    The triple $(X, S, \pi)$ is a semigroup action if and only if
    there exists a semigroup homomorphism $T : S \to \setfunc{X}{X}$ such
    that for all $s \in S$ and every $x \in X$,
      \[ T(s)(x) = \pi(s,x). \]
  \end{prop}
  \begin{proof}
    First observe that $(\setfunc{X}{X}, \circ)$ is a semigroup.
    Now assume that the triple $(X, S, \pi)$ is a semigroup action. 
    Define the map $T : S \to \setfunc{X}{X}$ by $T(s)(x) =
    \pi(s,x)$.
    To see that $T$ is a semigroup homomorphism, let $s$, $t \in S$
    and $x \in X$. 
    Then 
      \begin{align*}
        T(st)(x) = \pi(st,x) &= \pi\bigl(s, \pi(t,x)\bigr), \\
        &= \pi\bigl(s, T(t)(x)\bigr), \\
        &= T(s)\bigl(T(t)(x)\bigr), \\
        &= \bigl(T(s) \circ T(t)\bigr) (x).
      \end{align*}
    Hence $T(st) = T(s) \circ T(t)$.

    Conversely, suppose $T$ is a semigroup homomorphism. 
    Let $s$, $t \in S$ and $x \in X$.
    Then 
      \begin{align*}
        \pi(st, x) &= T(st)(x), \\
        &= \bigl(T(s) \circ T(t)\bigr) (x), \\
        &= \pi\bigl(s, T(t)(x)\bigr), \\
        &= \pi\bigl(s, \pi(t,x)\bigr).
      \end{align*}
    Hence $(X, S, \pi)$ is a semigroup action.
  \end{proof}

With this Proposition we can now effectively forgot about our original Definition \ref{defn:semiact} and simply consider semigroup actions as homomorphisms of $S$ into $\setfunc{X}{X}$. 

However, in this chapter we are not just concerned with any type of semigroup action.
After all we are taking $\beta S$ to be a compact right-topological semigroup, and so we would like to apply this topological algebra to our action in some way. 
We shall soon see that the notion of a dynamical system is one way to accomplish this goal.

  \begin{defn}
    A pair $\ds$ is a \textsl{dynamical system} if and only if
      \begin{itemize}
        \item[(1)] $X$ is a compact Hausdorff space;
        \item[(2)] $S$ is a semigroup;
        \item[(3)] $T_s : X \to X$ is continuous for all $s \in S$;
          and
        \item[(4)] $T_s \circ T_t = T_{st}$ for all $s$, $t \in S$.%
% I'm sure that there are dynamical systems where the phase space is
% not compact Hausdorff. Would be nice to mention were the reader can
% go to read about such dynamical systems. 
      \end{itemize}
  \end{defn}

  \begin{rmk}
    Let $\ds$ be a dynamical system and define $T : S \to
    \setfunc{X}{X}$ by $T(s) = T_s$.
    Then by Proposition \ref{prop:semiact} we are justified in saying
    that \textsl{$S$ acts on $X$ via $\la T_s \ra_{s \in S}$}.
  \end{rmk}

Using this remark we can extend the action of the dynamical system to
$\beta S$ as follows.
Giving $\setfunc{X}{X}$ the product topology and taking $S$ to be 
discrete, we have that the function $T :
S \to \setfunc{X}{X}$ is a continuous semigroup
homomorphism into a compact space.
Therefore by \cite[Theorem 3.27]{Hindman:1998fk} we can produce a
continuous extension $\widetilde{T}$ of $T$.
(More directly, $\widetilde{T}$ is defined for each $p \in \beta S$ by
$\widetilde{T}(p) \in \bigcap \{\, c\ell\bigl( T[A] \bigr) : A \in p \,\}$, where
the closure is in $\setfunc{X}{X}$.)
Now by \cite[Theorem 2.22(a)]{Hindman:1998fk} the space $\setfunc{X}{X}$ is
a compact right-topological semigroup.
In order to show that $\widetilde{T}$ is a semigroup homomorphism it
suffices by, \cite[Corollary 4.22]{Hindman:1998fk}, to show that for all
$s \in S$, the
map $\lambda_{T(s)}$ is continuous.
However by \cite[Theorem 2.2(b)]{Hindman:1998fk} we have that
$\lambda_{T(s)}$ is continuous if and only if $T(s)$ is continuous. 
Since $\ds$ is a dynamical system and $T(s) = T_s$ we know by
definition that $T(s)$ is continuous. 
Hence $\widetilde{T} : \beta S \to \setfunc{X}{X}$ is a continuous semigroup
homomorphism.

  \begin{rmk}
    By using the map $\widetilde{T} : \beta S \to X$, we can define
    $T_p : X \to X$, for $p \in \beta S$, as $T_p =
    \widetilde{T}(p)$. 
    Since $\widetilde{T}$ is a semigroup homomorphism we immediately
    conclude that $T_p \circ T_q = T_{pq}$ for all $p$, $q \in \beta
    S$.
   \end{rmk}

It's important to note that in general $(\beta S, \la T_p \ra_{p
  \in \beta S})$ is not a dynamical system, and 
the next example shows that $T_p$ may not be continuous for $p \in S^*$.
 

  \begin{example}
    We have that $(\beta\bbN, \la \lambda_s \ra_{s\in\bbN})$ is a
    dynamical system, but if $p \in \bbN^*$, then $\lambda_p$ is not
    continuous.
    This is proved in \cite[Theorem 6.10 and Remark 6.11]{Hindman:1998fk}.
   \end{example}

This example is somewhat disappointing since we lose the ``dynamical
part'' when extending the dynamical system to $\beta S$.
However even with this lost we will be able to prove something
intelligible about certain dynamical systems. 
Intuitively, the points of $S^*$ can be thought of as points at
infinity. 
Since in dynamical systems we are often concerned with the
``long-run'' behavior of our maps $T_s$ we will often be able to
correspond any interesting long run behavior with a `point at infinity'
in $S^*$.

To make this idea precise, we will be using the notion of a limit
along an ultrafilter. 

  \begin{defn}
    \label{defn:plim}
    Let $S$ be a discrete space, $p \in \beta S$, $X$ a compact
    Hausdorff topological space, $\la x_s : s \in S \ra$ a family
    of points in $X$, and $y \in X$.
    Then \hbox{$p$-$\displaystyle\lim_{s \in S} x_s = y$} if and only
    if for every
    neighborhood $U$ of $y$ we have $\{\, s \in S : x_s \in U \,\} \in p$.
  \end{defn}

  \begin{rmk}
    Our definition of a \hbox{$p$-limit} is a bit more restrictive
    then usual.
    Normally the definition only takes $X$ be any topological space.
    We placed these extra restrictions on $X$ to ensure that every
    \hbox{$p$-limit} exists and is unique.
  \end{rmk}

  \begin{prop}
    \label{prop:dsplim}
    Let $\ds$ be a dynamical system.
    Then for every $p \in \beta S$ and each $x \in X$ we have $T_p(x)
    = \hbox{$p$--$\lim_{s \in S} T_s(x)$}$.
  \end{prop}
  \begin{proof}
    By definition, we need to show that for all neighborhoods $U$ of
    $T_p(x)$, we have $\{\, s \in S : T_s(x) \in U \,\} \in p$. 
    Define $\pi_x : \setfunc{X}{X} \to X$ by $\pi_x(f) = f(x)$ and let
    $U$ be a neighborhood of $T_p(x)$.
    Note that $\pi_x^{-1}[U]$ is a neighborhood of $T_p$.
    By definition of $T_p$, we have that for all $A \in p$,
    $\pi_x^{-1}[U] \cap \{\, T_s : s \in A \,\} \ne \emptyset$. 
    Hence for all $A \in p$, there exists $s \in A$ such that $T_s(x)
    \in U$. 

    Suppose there exists a neighborhood $U$ of $T_p(x)$ such that
    $\{\, s \in S : T_s(x) \in U \,\} \not\in p$. 
    Put $A = \{\, s \in S : T_s(x) \not\in U\,\}$.
    Then $A \in p$.
    Therefore there exists $s \in A$ such that $T_s(x) \in U$ (by our
    first paragraph) and $T_s(x) \not\in U$ (by our definition of
    $A$), a contradiction.
  \end{proof}

With these preliminaries out of the way, we can now provide a
dynamical characterization of $C$-sets.

\section{Dynamical Characterization of $C$-sets}
\label{sec:dyncsets}
 \begin{defn}
    \label{defn:JSUR}
    Let $\ds$ be a dynamical system, and let $x$, $y \in X$. 
    The pair $(x,y)$ is \textsl{jointly sparsely uniformly recurrent}
    (we'll abbreviate this to JSUR) if and only if $\{\, s \in S :
    \hbox{$T_s(x) \in U$ and $T_s(y) \in U$} \,\}$ is a $J$-set for every
    neighborhood $U$ of $y$.%
    \endnote{
      In this section Definition \ref{defn:JSUR}, Lemma \ref{lem:JSUR},
      and Theorem \ref{thm:dyncsets} and their proofs are all, essentially,
      minor modifications of \cite[Definition 3.1]{Burns:2007uq},
      \cite[Lemma 3.3]{Burns:2007uq}, \cite[Theorem 3.4]{Burns:2007uq}
      respectively. 
    }
  \end{defn}


  \begin{lem}
    \label{lem:JSUR}
    Let $\ds$ be a dynamical system, and let $x$, $y \in X$.
    The following statements are equivalent.
    \begin{itemize}
      \item[(a)] The pair $(x, y)$ is JSUR.
      \item[(b)] There exists $r \in J(S)$ such that $T_r(x) = y = T_r(y)$.
      \item[(c)] There exists an idempotent $r \in J(S)$ such that $T_r(x)
        = y = T_r(y)$. 
    \end{itemize}
  \end{lem}
  \begin{proof}
    (a) $\Rightarrow$ (b). 
    For each neighborhood $U$ of $y$, put 
      \[  
        B_U = \{\, s \in S : \hbox{$T_s(x) \in U$ and $T_s(y) \in U$}
        \,\}.
      \]
    By assumption each $B_U$ is a $J$-set. 
    We now show that the collection 
    \[
      \{\, B_U : \hbox{$U$ is a neighborhood of $y$} \,\}
    \] 
    is closed under finite intersection
    by showing that, for all neighborhoods $U$ and $V$ of $y$ we have $B_{U
      \cap V} = B_U \cap B_V$.
    Let $s \in S$, then 
      \begin{align*}
        s \in B_{U \cap V} &\iff \hbox{$T_s(x) \in U \cap V$ and $T_s(y) \in U
        \cap V$}, \\
      &\iff \hbox{$T_s(x) \in U$, $T_s(x) \in V$, $T_s(y) \in U$, and
        $T_s(y) \in V$}, \\
      &\iff s \in B_U \cap B_V.
      \end{align*}
    
    By Lemma \ref{lem:pr-jsets}, we know that every $J$-set
    of $S$ is partition regular. 
    Therefore by \cite[Theorem 3.11 (b)]{Hindman:1998fk},  we can pick 
    $r \in J(S)$ such that $\{\, B_U : \hbox{$U$ is a neighborhood of
      $y$}\,\} \subseteq r$. 
    
    Now for all neighborhoods $U$ of $y$, we have $B_U \subseteq \{\,
    s \in S : T_s(x) \in U \,\}$ and $B_U \subseteq \{\, s \in S :
    T_s(y) \in U \,\}$. 
    Therefore $\{\, s \in S : T_s(x) \in U\,\} \in r$ and $\{\, s \in
    S : T_s(y) \in U \,\} \in r$. 
    By Definition \ref{defn:plim}, we can conclude that
    $r$-$\displaystyle\lim_{s\in S} T_s(x) = y$ and
    $r$-$\displaystyle\lim_{s \in S} T_s(y) = y$. 
    Hence by Proposition \ref{prop:dsplim}, we have $T_r(x) =
    r$-$\displaystyle\lim_{s \in S} T_s(x) = y = r$-$\displaystyle\lim_{s \in S}
    T_s(y) = T_r(y)$.
  
    (b) $\Rightarrow$ (c).
    Put $M  = \{\, r \in J(S) : T_r(x) = y = T_r(y) \,\}$. 
    We'll show that $M$ is a nonempty compact subsemigroup of $J(S)$. 
    If we can show this, then we can pick an idempotent in $M$ and our
    result follows.
    The fact that $M \ne \emptyset$ follows from our assumption. 
    To see that $M$ is compact it suffices to show that $M$ is
    closed. 
    Let $r \not\in M$, then either $T_r(x) \ne y$ or $T_r(y) \ne y$. 
    First, assume that $T_r(x) \ne y$. 
    By Definition \ref{defn:plim} and Proposition \ref{prop:dsplim},
    pick a
    neighborhood $U$ of $y$ such that 
    $\{\, s \in S : T_s(x) \in U \,\} \not\in r$.
    Put $A = \{\, s \in S : T_s(x) \in U \,\}$ and note that $S
    \setminus A \in r$ and $\overline{S \setminus A} \cap M =
    \emptyset$.
    [If $p \in \overline{S \setminus A} \cap M$, then $A = \{\, s \in
    S : T_s(x) \in U \,\} \in p$ by Definition \ref{defn:plim} and
    Proposition \ref{prop:dsplim}.
    We have also have that $S \setminus A \in p$.
    Hence $\emptyset = A \cap (S \setminus A) \in p$, a
    contradiction.]
    Now assume that $T_r(y) \ne y$. 
    By Definition \ref{defn:plim} and \ref{prop:dsplim}, pick a
    neighborhood $U$ of $y$ such that
    $\{\, s \in S : T_s(y) \in U \,\} \not\in r$.
    Put $A = \{\, s \in S : T_s(y) \in U \,\}$ and note that $S
    \setminus A \in r$ and $\overline{S \setminus A} \cap M =
    \emptyset$.
    Hence $M$ is a nonempty closed subset of $J(S)$.

    To see that $M$ is a subsemigroup, let $q$, $r \in M$.
    Then $T_{qr}(x) = T_q\bigl(T_r(x)\bigr) = T_q(y) =
    T_q\bigl(T_r(y)\bigr) = T_{qr}(y)$ and $T_q(y) = y$. 
    Hence $qr \in M$. 
    
    (c) $\Rightarrow$ (a).
    Pick $r$ as guaranteed in (c). 
    Let $U$ be a neighborhood of $y$. 
    Then $\{\, s \in S : T_s(x) \in U \,\} \in r$ and $\{\,  s \in S :
    T_s(y) \in U \,\} \in r$.
    Hence $\{\, s \in S : \hbox{$T_s(x) \in U$ and $T_s(y) \in U$}
    \,\} \in r$.  
  \end{proof}

  \begin{thm}
    \label{thm:dyncsets}
    Let $(S,\cdot)$ be a semigroup and $A \subseteq S$. 
    Then $A$ is a $C$-set if and only if there exist a dynamical
    system $\ds$ with points $x$, $y \in X$ where $(x,y)$ is JSUR, and
    a neighborhood $U$ of $y$ such that $A = \{\, s \in S : T_s(x) \in
    U \,\}$.
  \end{thm}
  \begin{proof}
    ($\Rightarrow$) Let $A \subseteq S$ be our $C$-set, and by
    Theorem \ref{thm:csets} pick an idempotent $r \in J(S)$ such that
    $A \in r$. 
    Let $R = S \cup \{e\}$ be the semigroup with an identity $e$
    adjoined to S. 
    (For expository convenience, we still add this new identity even
    if $S$ already contains an identity.)
    Give $\{0,1\}$ the discrete topology, and take
    $\setfunc{R}{\{0,1\}}$ to have
    the product topology, and put $X = \setfunc{R}{\{0,1\}}$.
    Hence $X$ is a compact Hausdorff space.
    For each $s \in S$, define $T_s : X \to X$ by $T_s(f) = f \circ
    \rho_s$. 
    % Perhaps I should sketch the argument here?
    By \cite[Theorem 19.14]{Hindman:1998fk}, $\ds$ is a dynamical
    system. 
    
    Now let $x = \cchi_A$ be the characteristic function of $A$, and
    put $y = T_r(x)$.
    % Perhaps I should provide a proof since there is not one
    % explicitly given in the book?
    Then by \cite[Remark 19.13]{Hindman:1998fk}, we have that $T_r(y)
    = T_r\bigl(T_r(x)\bigr) = T_{rr}(x) = T_r(x) = y$.
    Therefore by (c) in Lemma \ref{lem:JSUR}, the pair $(x, y)$ is JSUR.

    Put $U = \{\, w \in X : w(e) = y(e) \,\}$, and note that $U =
    \pi^{-1}\bigl[\{y(e)\}\bigr]$ and so $y \in U$. 
    Hence $U$ is a (subbasic) open neighborhood of $y$.
    To help us show that $U$ is the neighborhood of $y$ we are looking
    for we will show that $y(e) = 1$.
    Since $y = T_r(x)$ we have that $\{\, s \in S : T_s(x) \in U \,\}
    \in r$ by Definition \ref{defn:plim} and Proposition \ref{prop:dsplim}.
    Since $A \in r$, we can pick $s \in A$ such that $T_s(x) \in U$. 
    Then by definition of $U$ and our choice of $T_s(x) \in U$, we
    have $y(e) = T_s(x)(e) = x\bigl(\rho_s(e)\bigr) = x(es) = x(s) =
    \chi_{A}(s) = 1$. 
    Finally, given $s \in S$, we have
      \begin{align*}
        s \in A &\iff \chi_{A}(s) = 1, \\
                &\iff x(s) = 1, \\
                &\iff x(es) = 1, \\
                &\iff (x \circ \rho_s)(e) = 1, \\
                &\iff T_s(x)(e) = 1 = y(e), \\
                &\iff T_s(x) \in U.
      \end{align*}
   Hence $A = \{\, s \in S : T_s(x) \in U \,\}$. 
   
   ($\Leftarrow$) Let $\ds$ and let the points $x$, $y \in X$ be given as
   guaranteed.
   By Theorem \ref{thm:csets}, pick an idempotent $r \in J(S)$
   such that $T_r(x) = y = T_r(y)$. 
   Since $U$ is a neighborhood of $y$ and $T_r(x) = y$, we have that
   $A \in r$ by Definition \ref{defn:plim} and Proposition \ref{prop:dsplim}. 
 \end{proof}

\theendnotes

% % This chapter provides an analogue statement about the ideal J(S) and
% subsemigroups of $\beta S$. [6/20/2011/ John]

\chapter{Subsemigroups of $S$ and $J(S)$}
\section{Introduction}
In this chapter we prove that Theorem \ref{thm:smallest-subsemigrp} has an analogue statement in $J(S)$. 
The reason why this analogue is interesting is that there are many nice algebraic and combinatorial results that utilize the smallest ideal.%
\endnote{
  It can be argued that the smallest ideal in the Stone-\v{C}ech compactification is the best understood ideal.
}
It is then natural to wonder how much of this `niceness' can be imported into other ideals.

More precisely, in this chapter we shall prove an analogue to the following theorem.

\begin{thm}
  Let $(S, \cdot)$ be a semigroup, and let $T \subseteq S$ be a subsemigroup.
  The following statements are equivalent.
  \begin{itemize}
    \item[(a)] 
      $K(\beta T) = \beta T \cap K(\beta S)$.
    
    \item[(b)]
      $\beta T \cap K(\beta S)$ is nonempty.

    \item[(c)]
      $T$ is piecewise syndetic in $S$.
  \end{itemize}
\end{thm}
\begin{proof}
  It follows from Theorem \ref{thm:smallest-subsemigrp} that statement \textsl{(a)} is equivalent to statement \textsl{(b)}.
  
 From \cite[Theorem 4.40]{Hindman:1998fk}, we have that statement \textsl{(b)} is equivalent to statement \textsl{(c)}.
\end{proof}
\begin{rmk}
  We can, and do, identify the set $c \ell_{\beta S}(T)$ with $\beta T$. 
\end{rmk}

We shall see that the form of the analogue depends on whether the underlying semigroup $S$ is commutative or not. 
In particular, we shall prove the following two theorems.

\begin{thm}
  Let $(S, +)$ be a commutative semigroup, and let $T \subseteq S$ be a subsemigroup.
  The following statements are equivalent.
  \begin{itemize}
    \item[(a)]
      $J(T) = \beta T \cap J(S)$.

    \item[(b)]
      $\beta T \cap J(S)$ is nonempty.

    \item[(c)]
      $T$ is a $J$-set in $S$. 
  \end{itemize}
\end{thm}

\begin{thm}
  \label{thm:ideal}
  Let $(S, \cdot)$ be a semigroup, and let $T \subseteq S$ be a subsemigroup.
  The following statements are equivalent.
  \begin{itemize}
    \item[(a)]
      $J(T) \subseteq \beta T \cap J(S)$.

    \item[(b)]
      $\beta T \cap J(S)$ is nonempty.

    \item[(c)]
      $T$ is a $J$-set in $S$. 
  \end{itemize}
\end{thm}

In Section \ref{sec:example} we prove that statement (a) in Theorem \ref{thm:ideal} cannot be strengthen to $J(T) = \beta T \cap J(S)$. 
We are in a position to already prove parts of these theorems.

\begin{thm}
  Let $(S, \cdot)$ be a semigroup, and let $T \subseteq S$ be a subsemigroup.
  In the following statement (a) implies statement (b), and statements (b) and (c) are equivalent.
  \begin{itemize}
    \item[(a)]
      $J(T) \subseteq \beta T \cap J(S)$.

    \item[(b)]
      $\beta T \cap J(S)$ is nonempty.

    \item[(c)]
      $T$ is a $J$-set in $S$. 
  \end{itemize}
\end{thm}
\begin{proof}
  \textsl{(a) $\Rightarrow$ (b).}
  It follows from Lemma \ref{lem:pr-jsets} and \cite[Theorem 3.11]{Hindman:1998fk} that $J(T)$ is nonempty. 
  Therefore by assumption we have that $\beta T \cap J(S)$ is nonempty. 

  \textsl{(b) $\iff$ (c).}
  By Theorem \ref{thm:jsets-ideal} we have that $T$ is a $J$-set in $S$ if and only if $c \ell_{\beta S} (T) \cap J(S) \ne \emptyset$. 
  Identifying $c \ell_{\beta S}(T)$ with $\beta T$ completes the proof.
\end{proof}

With this theorem in hand, in the next few sections we need only consider the case when statement (c) implies statement (a).

\section{Commutative Semigroup Case}
We first focus on the case when the underlying semigroup is commutative. 
In this case we obtain a stronger result, and via Lemma \ref{lem:comm-jsets} we have a simply characterization of a $J$-set in a commutative semigroup.
(Recall that Lemma \ref{lem:comm-jsets} states that a set $A$ is a $J$-set in a commutative semigroup if and only if for every $F \in \Pf(\calT)$ there exist $a \in S$ and $H \in \Pf(\bbN)$ such that $a + \sum_{t \in H} f(t) \in A$ for every $f \in F$.
Also recall that if $S$ is a semigroup, then $\calT(S)$ is the set of all sequences on $S$.)


The following lemma shows that, in a commutative semigroup, we may pick our `translate' $a$ in the $J$-set itself.
\begin{lem}
  \label{lem:comm-trans}
  Let $(S, +)$ be a commutative semigroup and $A \subseteq S$.
  Then $A$ is a $J$-set if and only if for every $F \in \Pf(\calT)$, there exist $a \in A$ and $H \in \Pf(\bbN)$ such that for all $f \in F$, $a + \sum_{t \in H} f(t) \in A$.
\end{lem}
\begin{proof}
  ($\Rightarrow$)
  Let $F \in \Pf(\calT)$ and let $\overline{c} \in \calT$ be some constant sequence.
  Put $G = (F + \{\overline{c}\}) \cup \{\overline{c}\}$. 
  Since $A$ is a $J$-set, by the necessity of Lemma \ref{lem:comm-jsets}, pick $b \in S$ and $H \in \Pf(\bbN)$ such that for all $g \in G$, $b + \sum_{t \in H} g(t) \in A$. 

  Put $a = b + \sum_{t \in H} \overline{c}(t)$, then for every $f \in F$, we have $a + \sum_{t \in H} f(t) = b + \sum_{t \in H}\bigl( \overline{c}(t) + f(t) \bigr) \in A$. 
  
  ($\Leftarrow$)
  By assumption, $A$ satisfies the sufficiency condition of Lemma \ref{lem:comm-jsets}, and hence $A$ is a $J$-set.
\end{proof}

Our next lemma show that a $J$-set contains `many' translates of sum of sequences. 
\begin{lem}
  \label{lem:comm-many}
  Let $(S, +)$ be a commutative semigroup and $A \subseteq S$.
  Then $A$ is a $J$-set if and only if for every $F \in \Pf(\calT)$ there exist a sequence $\la a_n \ra_{n=1}^\infty$ in $S$ and a sequence $\la H_n \ra_{n=1}^\infty$ in $\Pf(\bbN)$ with the following properties:
  \begin{itemize}
    \item[(1)]
      $\max H_n < \min H_{n+1}$ for every positive integer $n$.

    \item[(2)]
      For every positive integer $n$ and for all $f \in F$, $a_n + \sum_{t \in H_n} f(t) \in A$. 
  \end{itemize}
\end{lem}
\begin{proof}
  ($\Rightarrow$)
  Let $F \in \Pf(\calT)$.
  We shall recursively construct our sequences $\la a_n \ra_{n=1}^\infty$ and $\la H_n \ra_{n=1}^\infty$ such that the following hypotheses are satisfied for every positive integer $m$. 
  \begin{itemize}
    \item[(i)]
      $\max H_k < \max H_{k+1}$ for all $k \in \{1, 2, \ldots, m-1\}$.

    \item[(ii)]
      For every $k \in \{1, 2, \ldots, m\}$ and all $f \in F$, $a_k + \sum_{t \in H_k} f(t) \in A$.
  \end{itemize}

  Let $m$ be a positive integer.
  First assume that $m = 1$. 
  Since $A$ is a $J$-set by the necessity condition of Lemma \ref{lem:comm-jsets}, pick $a_1 \in S$ and $H_1 \in \Pf(\bbN)$ such that for all $f \in F$, $ a_1 + \sum_{t \in H_1} f(t)\in A$. 
  With these choices, hypothesis (i) is vacuously true, and hypothesis (ii) is true.

  Now let $m > 1$ and assume $\la a_n \ra_{n=1}^{m-1}$ and $\la H_n \ra_{n=1}^{m-1}$ have been chosen to satisfy hypotheses (i) and (ii).
  By the necessity condition of Lemma \ref{lem:comm-jsets} and by Lemma \ref{lem:jset-start} pick $a_m \in S$ and $H_m \in \Pf(\bbN)$ such that $\min H_m \ge L$ and for all $f \in F$, $a_m + \sum_{t \in H_m} f(t) \in A$. 
  Then hypothesis (i) is true since $\max H_{m-1} < L \le \min H_m$, and hypothesis (ii) is true by our choice of $a_m$ and $H_m$. 

  Hence the conclusion follows.

  ($\Leftarrow$)
  Since $A$ satisfies the sufficiency condition of Lemma \ref{lem:comm-jsets}, we have that $A$ is a $J$-set.
\end{proof}

Our final lemma in this section shows that $J$-sets in subsemigroups `lift' to be $J$-sets in the containing semigroup if the subsemigroup itself is a $J$-set in the containing semigroup.

\begin{lem}
  \label{lem:comm-lift}
  Let $(S, +)$ be a commutative semigroup, and let $T \subseteq S$ a subsemigroup which is also a $J$-set in $S$.
  If $A \subseteq T$ is a $J$-set in $T$, then $A$ is a $J$-set in $S$. 
\end{lem}
\begin{proof}
  Let $F \in \Pf\bigl( \calT(S) \bigr)$. 
  Since $T$ is a $J$-set in $S$, we may pick sequences $\la a_n \ra_{n=1}^\infty$ in $S$ and $\la H_n \ra_{n=1}^\infty$ in $\Pf(\bbN)$ as guaranteed by Lemma \ref{lem:comm-many}.

  For each $f \in F$ define $g_f \in \calT(T)$ by $g_f(n) = a_n + \sum_{t \in H_n} f(t)$. 
  Since $A$ is a $J$-set in $T$, by the necessity condition of Lemma \ref{lem:comm-jsets} we may pick $a \in T$ and $H \in \Pf(\bbN)$ such that $a + \sum_{t \in H} g_f(t) \in A$ for every $f \in F$. 
  Put $G = \bigcup_{t \in H} H_t$ and $b = a + \sum_{t \in H} a_t$. 
  Then for every $f \in F$, $b + \sum_{t \in G} f(t) \in A$. 
  Therefore by the sufficiency condition of Lemma \ref{lem:comm-jsets} $A$ is a $J$-set in $S$.
\end{proof}

\begin{thm}
  Let $(S,+)$ be a commutative semigroup, and let $T \subseteq S$ be a subsemigroup.
  If $T$ is a $J$-set in $S$, then $J(T) = \beta T \cap J(S)$.
\end{thm}
\begin{proof}
  Let $p \in J(T)$. 
  Then $T \in p$ since $T$ is a $J$-set in itself.
  If $A \in p$, then since $A$ is a $J$-set in $T$ and $T$ is a $J$-set in $S$ we have, by Lemma \ref{lem:comm-lift}, that $A$ is a $J$-set in $S$ too.
  Hence $p \in \beta T \cap J(S)$. 

  Now let $p \in \beta T \cap J(S)$ and $A \in p$. 
  Since $T \in p$ we have that $A \cap T \in p$.
  Therefore we may assume that $A \subseteq T$. 
  Let $F \in \Pf\bigl( \calT(T) \bigr)$, then $F \in \Pf\bigl( \calT(S) \bigr)$ also.
  Since $A$ is a $J$-set in $S$, by Lemma \ref{lem:comm-trans} we may pick $a \in A$ and $H \in \Pf(\bbN)$ such that for all $f \in F$, $a + \sum_{t \in H} f(t) \in A$.
  Since $a \in A \subseteq T$, it follows from the necessity condition of Lemma \ref{lem:comm-jsets} that $A$ is a $J$-set in $T$.
  Hence $p \in J(T)$.
\end{proof}

\section{General Semigroup Case}
To handle the general semigroup case we will prove versions of Lemma \ref{lem:comm-many} and \ref{lem:comm-lift} that use the usual definition of a $J$-set. 
Unfortunately, in the next section we shall see that Lemma \ref{lem:comm-trans} is not valid in every noncommutative semigroup. 
In order to prove a `general' version of Lemma \ref{lem:comm-many} we shall use the following technical `rewriting' type lemma.

Recall that, given a semigroup $S$, for $n \in \bbN$, $a \in S^{m+1}$, $t \in \calJ_m$, and $f \in \calT$, we put $x(n, a, t, f) = \Bigl( \prod_{i=1}^n \bigl( a(i) f(t_i) \bigr) \Bigr) a(n+1)$. 

\begin{lem}
  Let $(S, \cdot)$ be a semigroup, $F \in \Pf(\calT)$, $\la m_k \ra_{k=1}^\infty$ a sequence of positive integers, $\la a_k \ra_{k=1}^\infty \in \bigtimes_{k=1}^\infty S^{m_k + 1}$, and $\tau \in \bigtimes_{k=1}^\infty \calJ_{m_k}$ with $\tau(k)(m_k) < \tau(k+1)(1)$ for every positive integer $k$.
  For each $f \in F$ define $g_f \in \calT$ by $g_f(k) = x(m_k, a_k, \tau(k), f)$.

  If $G \in \Pf(\bbN)$ and $n = \sum_{k \in G} m_k$, then there exist $c \in S^{n+1}$ and $t \in \calJ_n$ such that $t(n) = \tau(\max G)(m_{\max G})$ and $\prod_{k \in G} g_f(k) = x(n, c, t, f)$ for every $f \in F$.
\end{lem}
\begin{proof}
  We prove this result by induction on the size of $G$.
  First suppose $|G| = 1$, then $G = \{k\}$ for some positive integer $k$.
  For $f \in F$, by definition of $g_f$ we directly have that $\prod_{k \in G} g_f(k) = x(m_k, a_k, \tau(k), f) = x(n, a, t, f)$ with $a = a_k$ and $t = \tau(k)$.

  Now assume that $|G| > 1$ and suppose that our implication is true for all nonempty finite subsets of $\bbN$ with size less than $|G|$.
  Put $K = G \setminus \{\max G\}$, and for notational convenience, put $s = \max G$.
  By our induction hypothesis, with $\ell = \sum_{k \in K} m_k$, pick $d \in S^{\ell+1}$ and $u \in \calJ_\ell$ as guaranteed.
  Then for $f \in F$
  \begin{align*}
    \textstyle
    \prod_{k \in G} g_f(k) &= \textstyle \prod_{k \in K} g_f(k) g_f(s), \\
                         &= x(\ell, d, u, f) g_f(s), \\
                         &= x(\ell, d, u, f) x(m_s, a_s, \tau(s), f).
  \end{align*}
  Put $n = \sum_{k \in G} m_k$ and define $c \in S^{n+1}$ by 
  \[
    c(j) =
    \begin{cases}
      d(j) & \mbox{if $j \in \{1, 2, \ldots, \ell\}$,} \\
      d(\ell+1)a_s(1) & \mbox{if $j = \ell+1$,} \\
      a_s(j - \ell) & \mbox{if $j \in \{\ell+2, \ell+3, \ldots, \ell+m_s+1\}$.}
    \end{cases}
  \]
  
  By assumption on $\tau$ and our induction hypothesis, $\tau(\max K)(m_{\max G}) < \tau(s)(1)$ and $u(\ell) = \tau(\max K)(m_{\max G})$
  Therefore we can define $t \in \calJ_n$ by
  \[
    t_j =
    \begin{cases}
      u(j) & \mbox{if $j \in \{1, 2, \ldots, \ell\}$,} \\
      \tau(s)(j-\ell) & \mbox{if $j \in \{\ell+1, \ell+2, \ldots, n\}$.}
    \end{cases}
  \]
  Hence it follows that $\prod_{k \in G} g_f(k) = x(n, c, t, f)$ for all $f \in F$.
\end{proof}

\begin{lem}
  \label{lem:many}
  Let $(S, \cdot)$ be a semigroup and let $A \subseteq S$.
  Then $A$ is a $J$-set in $S$ if and only if for every $F \in \Pf(\calT)$, there exist $\la m_k \ra_{k=1}^\infty$ a sequence of positive integers, $\la a_k \ra_{k=1}^\infty \in \bigtimes_{k=1}^\infty S^{m_k+1}$, and $\tau \in \bigtimes_{k=1}^\infty \calJ_{m_k}$ with $\tau(k)(m_k) < \tau(k+1)(1)$ for every positive integer $k$ such that for every positive integer $k$ and for all $f \in F$, $x(m_k, a_k, \tau(k), f) \in A$.
\end{lem}
\begin{proof}
  ($\Rightarrow$)
  Let $F \in \Pf(\calT)$.
  We shall recursively construct our sequences $\la a_k \ra_{k=1}^\infty$, $\la m_k \ra_{k=1}^\infty$, and $\tau$ such that the following hypotheses are satisfied for all positive integers $n$. 
  \begin{itemize}
    \item[(i)]
      $\tau(k)(m_k) < \tau(k+1)(1)$ for every $k \in \{1, 2, \ldots, n-1\}$.

    \item[(ii)]
      For every $k \in \{1, 2, \ldots, n\}$ and all $f \in F$, $x(m_k, a_k, \tau(k), f) \in A$.
  \end{itemize}
  
  Let $n$ be a positive integer.
  First suppose that $n = 1$. 
  Since $A$ is a $J$-set, pick $m_1 \in \bbN$, $a_1 \in S^{m_1+1}$, and $\tau(1) \in \calJ_{m_1}$ such that for all $f \in F$, $x(m_1, a_1, \tau(1), f) \in A$. 
  Hypothesis (i) is satisfied vacuously, and hypothesis (ii) is satisfied by our choices of $m_1$, $a_1$, and $\tau(1)$. 

  Now assume that $n > 1$ and that we have pick our sequences $\la m_k \ra_{k=1}^{n-1}$, $\la a_k \ra_{k=1}^{n-1}$, and $\tau \in \bigtimes_{i=1}^{n-1} \calJ_{m_k}$ to satisfy hypotheses (i) and (ii). 
  By Lemma \ref{lem:jset-start} pick $m_n \in \bbN$, $a_n \in S^{m_n+1}$, and $\tau(n) \in \calJ_{m_n}$ with $\tau(n)(1) > \tau(n-1)(m_{n-1})$ and $x(m_n, a_n, \tau(n), f) \in A$ for all $f \in F$. 
  Then by construction of $\tau(n)$ hypothesis (i) is satisfied, and by our choice of $m_n$, $a_n$, and $\tau(n)$, hypothesis (ii) is satisfied. 

  Hence the conclusion follows. 

  ($\Leftarrow$)
  $A$ satisfies the definition of a $J$-set.
\end{proof}

\begin{lem}
  \label{lem:lift}
  Let $(S, \cdot)$ be a semigroup, and let $T \subseteq S$ be a subsemigroup which is also a $J$-set in $S$.
  If $A \subseteq T$ is a $J$-set in $T$, then $A$ is a $J$-set in $S$.
\end{lem}
\begin{proof}
  Let $F \in \Pf\bigl( \calT(S) \bigr)$.
  Since $T$ is a $J$-set in $S$, we may pick sequences $\la m_k \ra_{k=1}^\infty$, $\la a_k \ra_{k=1}^\infty$, and $\tau$ as guaranteed by Lemma \ref{lem:many}.

  For each $f \in F$ define $g_f \in \calT(T)$ by $g_f(k) = x(m_k, a_k, \tau(k), f)$. 
  Since $A$ is a $J$-set in $T$, we may pick a positive integer $n$, $b \in T^{n+1}$, and $u \in \calJ_n$ such that for all $f \in F$, $x(n, b, u, g_f) \in A$.
  We claim that there exist $m \in \bbN$, $a \in S^{m+1}$, and $t \in \calJ_m$ such that for all $f \in F$, $x(m, a, t, f) \in A$.

  Put $m = \sum_{i=1}^n m_{u_i}$. 
  For each $s \in \{1, 2, \ldots, n\}$ put $h_s = \sum_{i=1}^s m_{u_i}$ and let $h_0 = 1$. 

  For each $s \in \{1, 2, \ldots, n\}$ put $h_s = \sum_{i=1}^s m_{u_i}$, put $h_0 = 0$, and put $m = h_n$ .
  We define $a \in S^{m+1}$ as follows:
  \[
    a(j) =
    \begin{cases}
      b(1)a_{u_1}(1) & \mbox{if $j=1$,} \\
      a_{u_s}(j-h_{s-1}) & \mbox{if $s \in \{1, 2, \ldots, n\}$ and $h_{s-1} + 1 < j \le h_s$,} \\
      a_{u_s}(m_{u_s}+1)b(s+1)a_{u_{s+1}}(1) & \mbox{if $s \in \{1, 2, \ldots, n-1\}$ and $j = h_s + 1$,} \\
      a_{u_n}(m_{u_n}+1)b(n+1) & \mbox{if $j = m+1$.}
    \end{cases}
  \]
  And, we define $t \in \calJ_m$ as follows:
  \[
    t_j = 
      \mbox{$\tau(u_s)(j- h_s)$ if $s \in \{1, 2, \ldots, s\}$ and $h_{s-1} < j \le h_s$.}
  \]
\end{proof}

\begin{thm}
  \label{thm:comb-ideal}
  Let $(S, +)$ be a semigroup, and let $T \subseteq S$ be a subsemigroup.
  If $T$ is a $J$-set in $S$, then $J(T) \subseteq \beta T \cap J(S)$.
\end{thm}
\begin{proof}
  Let $p \in J(T)$.
  Then $T \in p$ since $T$ is a $J$-set in itself.
  If $A \in p$, then since $A$ is a $J$-set in $T$ and $T$ is a $J$-set in $S$, we have, by Lemma \ref{lem:lift}, that $A$ is a $J$-set in $S$ too.
  Hence $p \in \beta T \cap J(S)$. 
\end{proof}

\section{A Free Semigroup Example}
\label{sec:example}

We now produce an example showing that the conclusion of Theorem \ref{thm:comb-ideal} cannot be improved to $J(T) = \beta T \cap J(S)$.
In this section we let $S$ be the free semigroup on two generators $a$ and $b$.

\begin{lem}
  \label{lem:seqs}
  Enumerate $\Pf(S)$ as $\la A_n \ra_{n=1}^\infty$.
  There exist two sequences $\la v_n \ra_{n=1}^\infty$ and $\la w_n \ra_{n=1}^\infty$ in $S$ satisfying the following three conditions.
  \begin{itemize}
    \item[(1)]
      $\{\, v_n : n \in \bbN \,\} \subseteq \{\, (ab)^t : t \in \bbN \,\}$ and $\{\, w_n : n \in \bbN \,\} \subseteq \{\, b^t : t \in \bbN \,\}$.

    \item[(2)]
      For each positive integer $n$, the length $v_n$ and $w_n$ is longer than every word in $A_n$.

    \item[(3)]
      For each positive integer $n > 1$, for every $k \in \{1, 2, \ldots, n-1\}$, and for all $x \in A_k$, the length of $v_n$ and $w_n$ is longer than the length of $v_kxw_k$.
  \end{itemize}
\end{lem}
\begin{proof}
  % INSERT PROOF
\end{proof}

\begin{defn}
  Pick $\la v_n \ra_{n=1}^\infty$ and $\la v_n \ra_{n=1}^\infty$ as guaranteed by Lemma \ref{lem:seqs}.
  Put $A = \bigcup_{n=1}^\infty v_nA_nw_n$.
\end{defn}

\begin{lem}
  $A$ is a $J$-set in $S$. 
\end{lem}
\begin{proof}
  Let $F \in \Pf(\calT)$.
  Pick $t \in \bbN$ such that $\{\, f(t) : f \in F\,\} \in \Pf(S)$.
  Pick $n \in \bbN$ such that $A_n = \{\, f(t) : f \in F\,\}$. 
  Then for every $f \in F$, $v_nf(t)w_n \in A$. 
\end{proof}

Now that we have our $J$-set, we will use it form our required subsemigroup of $S$.

\begin{defn}
  Put 
  \[
    T = \bigr\{\, u_1u_2\cdots u_s : \mbox{$s \in \bbN$ and for every $i \in \{1, 2, \ldots, s\}$, $u_i \in A \cup\{a\}$} \,\bigr\}.
  \]
\end{defn}

\begin{lem}
  $T$ is a proper subsemigroup of $S$ which is also a $J$-set in $S$ and with $A \subseteq T$.
\end{lem}
\begin{proof}
  The fact that $T$ is a subsemigroup of $S$ with $A \subseteq T$ follows directly from the definition of $T$.
  Since $A \subseteq T$ and $A$ is a $J$-set in $S$, we have that $T$ is a $J$-set in $S$ too.
  To see that $T \subsetneq S$ it suffices to show that $b \not\in T$.
  Let $\ell(x)$, for $x \in S$, denote the length of the word $x$.
  Observe that $\ell(x) \ge 3$ for every $x \in A$ and so $b \not\in A$.
  It follows that if $x \in T$ with $\ell(x) = 1$, then $x = a$.
  Therefore $b \not\in T$. 
\end{proof}

\begin{thm}
  $A$ is a $J$-set in $S$ with the property that there exists $F \in \Pf\bigl( \calT(T) \bigr)$ such that for all $m \in \bbN$, for each $c \in S^{m+1}$, and for any $t \in \calJ_m$, there exists $f \in F$ such that if $c(1) \in T$, then $x(m, c, t, f) \not\in A$.
\end{thm}
\begin{proof}
  Put $F = \{\overline{a}\}$, and let $m \in \bbN$, $c \in S^{m+1}$, and $t \in \calJ_m$. 
  Then
  \begin{align*}
    x(m, c, t, \overline{a}) &= c(1)\overline{a}(t_1)c(2) \overline{a}(t_2)c(3) \cdots c(m) \overline{a}(t_m)c(m+1), \\
                  &= c(1)ac(2)ac(3) \cdots c(m)ac(m+1).
  \end{align*}
  
  Suppose $c(1) \in T$ and $x(m, c, t, \overline{a}) \in A$. 
  Since $c(1) \in T$ there exists a positive integer $s$ such that $c(1) = u_1u_2 \cdots u_s$ with $u_i \in A \cup \{a\}$ for all $i \in \{1, 2, \ldots, s\}$.
  Since $x(m, c, t, \overline{a}) \in A$, pick a positive integer $k$ such that $x(m, c, t, \overline{a}) \in v_kA_kw_k$.
  Pick $y \in A_k$ such that $x(m, c, t, \overline{a}) = v_kyw_k$.
  We shall consider two cases and three sub-cases. 
  
  \textbf{Case 1 ($u_1 = a$).}
  Observe that by condition (1) of Lemma \ref{lem:seqs} the word $v_kyw_k$ starts with the substring $ab$.

  If $s = 1$, then the word $x(m, c, t, \overline{a})$ starts with the substring $aa$, a contradiction.

  If $s > 1$, then regardless of whether $u_2 \in A$ or $u_2 = a$, the word $x(m, c, t, \overline{a})$ stills starts with the string $aa$, a contradiction.
  (By construction, that is, by condition (1) of Lemma \ref{lem:seqs} words in $A$ start with the letter $a$.)

  \textbf{Case 2 ($u_1 \in A$).}
  For this case, we consider three sub-cases. 
  Pick a positive integer $n$ such that $u_1 \in v_n A_n w_n$ and pick $x \in A_n$ such that $u_1 = v_n x w_n$. 
  Let $\ell(w)$ denote the length of a word $w$ in $S$.

  \textbf{Sub-case I ($k < n$).}
  Since $k < n$, by condition (3) of Lemma \ref{lem:seqs} $v_n$ is longer than $v_kxw_k$.
  Then $\ell\bigl( c(1) \bigr) \ge \ell(u_1) > \ell(v_kyw_k)$, but this is a contradiction since $c(1)$ is a substring of $v_kyw_k$.

  \textbf{Sub-case II ($k = n$).}
  We have that
  \begin{align*}
    x(m, c, t, \overline{a}) &= c(1)ac(2)ac(3) \cdots c(m)ac(m+1), \\
                             &= v_kxw_ku_2 \cdots u_sac(2)ac(3) \cdots c(m)ac(m+1).
  \end{align*}
  Therefore $yw = xw_ku_2 \cdots u_sac(2)ac(3) \cdots c(m)ac(m+1)$. 
  The substring $ac(m+1)$ starts with the letter $a$ and so it follows from condition (1) of Lemma \ref{lem:seqs} that $w_k$ is a substring of $c(m+1)$.
  Pick $v \in S \cup \{\emptyset\}$ such that $c(m+1) = vw_k$. 
  Therefore 
  \[
    y = xw_ku_2 \cdots u_sac(2)ac(3) \cdots c(m)av.
  \]
  Since $w_k$ is a proper substring of $y$ we have that $\ell(y) > \ell(w_k)$.
  However, by condition (2) of Lemma \ref{lem:seqs} we have that $\ell(w_k) > \ell(y)$, a contradiction. 

  \textbf{Sub-case III ($k > n$).}
  Put $t = \ell(v_nx)$.
  By condition (3) of Lemma \ref{lem:seqs}, $\ell(v_k) > t + \ell(w_n) \ge t+1$.
  In fact, by condition (2) of Lemma \ref{lem:seqs} we can sharpen this last inequality to $t + \ell(w_n) \ge t+2$.
  (Every word in $A_n$ has length at least one and by condition (2) of Lemma \ref{lem:seqs}, $w_n$ is longer than every word in $A_n$.)
  Therefore the letter in position $t+1$ and $t+2$ of $v_nxw_n$ is $b$. 

  Now by condition (3) of Lemma \ref{lem:seqs}, $\ell(v_k) > \ell(v_nxw_n)$, and by condition (1) of Lemma \ref{lem:seqs} we have that the $t+1$th and $t+2$th letter of $v_k$ is $a$. 
  Since $x(m, c, t, \overline{a}) = v_kyw_k = v_nxw_nu_2\cdots u_sc(2)ac(3)\cdots c(m)ac(m+1)$, this is a contradiction.

Since we have exhausted all of our cases and sub-cases, the conclusion follows.
\end{proof}

It follows from this theorem that we have a $A$ $J$-set in $S$ contained $T$, a subsemigroup of $S$, that is also a $J$-set in $S$, such that $A$ is not a $J$-set in $T$.
% \include{backmatter}


Bartel Leendert van der Waerden in \cite{Van-der-Waerden:1927fk} proved the following remarkable assertion about arithmetic progressions in the positive integers.


  Let $r$ and $k$ be positive integers.
  Then there exists a positive integer $N$ such that if $\{1, 2, \ldots, N\} = \bigcup_{i=1}^r C_i$, then there exists $i \in \{1, 2, \ldots, r\}$ such that $C_i$ contains a $k$-term arithmetic progression, that is, there exist positive integers $a$ and $d$ with $\{a, a+d, \ldots, a + (k-1)d\} \subseteq C_i$.



\end{document}

