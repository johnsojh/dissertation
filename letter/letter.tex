% Mathematical Letter to Dr. Hindman on Dissertation Progress
% February 17, 2011

\documentclass[12pt]{letter}

\usepackage{amsthm, amssymb, amsmath}

\newtheoremstyle{plain}{3mm}{3mm}{\slshape}{}{\bfseries}{.}{.5em}{}
\theoremstyle{plain}
\newtheorem*{nvdw}{Noncommutative van der Waerden Theorem}
\newtheorem*{thm}{Theorem}
\newtheorem*{ques}{Question}

\theoremstyle{definition}
\newtheorem*{defn}{Definition}

\newcommand{\la}{\langle}
\newcommand{\ra}{\rangle}
\newcommand{\bbN}{\mathbb{N}}
\newcommand{\bbZ}{\mathbb{Z}}
\newcommand{\calB}{\mathcal{B}}
\newcommand{\calC}{\mathcal{C}}
\newcommand{\calX}{\mathcal{X}}
\newcommand{\calI}{\mathcal{I}}
\newcommand{\calJ}{\mathcal{J}}
\newcommand{\calP}{\mathcal{P}}
\newcommand{\calT}{\mathcal{T}}
\newcommand{\Pf}{\mathcal{P}_f}
\newcommand{\dom}{\mathrm{dom}}
\newcommand{\setfunc}[2]{\hbox{${}^{\hbox{$#1$}}\hskip -3 pt #2$}}

\font\bigmath=cmsy10 scaled \magstep 3
\newcommand{\bigtimes}{\hbox{\bigmath \char'2}}

\newcommand{\cchi}{\raise 2 pt \hbox{$\chi$}}


\makeatletter
\let\@texttop\relax
\makeatother

\date{February 18, 2011}
\signature{John H.~Johnson}

\begin{document}
\begin{letter}{}
\opening{Dr.~Hindman}

I'm still working on my Dissertation, but as you have probably already
deduced, my
writing is proceeding slower than I originally planned.  
(For instance, I have clearly missed my February 16 deadline!)
Recently I have been working on 2.05 (mathematical) projects on things
I would like to put into my Dissertation.
However, currently everything is stalled out and perhaps you can
provide some guidance. 

My first project I would like to motivate the definitions
surrounding the ``noncommutative version'' of the Central Sets Theorem
by showing that $J$-sets should satisfy the conclusion of the
Noncommutative van der Waerden Theorem.  
Recall that in your paper ``Ramsey Theory in Noncommutative
Semigroups'', you and Bergelson formulated and proved NVDW.

\begin{nvdw}
  Let $S$ be a semigroup, $l$, $r \in \bbN$, and let $\la d_n
  \ra_{n=1}^\infty$ be a sequence in $S$. 
  If $S = \bigcup_{i=1}^r C_i$, then there exist $i \in \{1, 2,
  \ldots, r\}$, $m \in \bbN$, $a \in S^{m+1}$, and $n(1) < n(2) <
  \cdots < n(m)$ in $\bbN$ such that 
  \[
    \biggl\{ \prod_{j=1}^{m+1}a(j) \biggr\} \cup \biggl\{\,
    \prod_{j=1}^m\Bigl(a(j)d_{n(j)}^k\Bigr)a(m+1) : k \in \{1, 2, \ldots, l\}
    \,\biggr\} \subseteq C_i.
  \]
\end{nvdw}

You and Bergelson prove this fact by showing that piecewise syndetic
sets satisfy the conclusion of NVDW via Theorem 2.6 (in the paper).
Now in analogy with how the strengthened version of VDW easily follows
the definition of $J$-sets, I would like to show only using the
definition, that $J$-sets satisfy the conclusion of NVDW.
Unfortunately, I'm not able to prove such a statement, and I'm not
even sure if it is a fact.

Using the definition of $J$-sets and another theorem of mine, I am able
to easily prove the following (possibly) weaker result.

\begin{thm}
  Let $A$ be a $J$-set of a semigroup $S$.
  Then there exist $m \in \bbN$, $a \in S^{m+1}$, and $d \in S^{m+1}$
  such that
  \[
    \biggl\{ \prod_{j=1}^{m+1}a(j) \biggr\} \cup \biggl\{\,
    \prod_{j=1}^m\Bigl(a(j)d(j)^k\Bigr)a(m+1) : k \in \{1, 2, \ldots, l\}
    \,\biggr\} \subseteq A.
  \]
\end{thm}

The reason for my pessimism expressed above is that Theorem 2.6 (in
the paper) made me realize that piecewise syndetic sets are not only
$J$-sets, but they are $J$-sets in which you can choose how many terms of
your set of sequences you multiply together.
So it appears we may need a similar level of control for $J$-sets too.
More precisely, we have the following theorem.

\begin{defn}
  For each $m$, $k \in \bbN$, put
  \begin{align*}
    \calI_{m,k} &= \Bigl\{\, \bigl(H(1), H(2), \ldots, H(m)\bigr) :
    \mbox{for each $i \in \{1, 2, \ldots, m\}$, $H(i) \in \Pf(\bbN)$}
    \\
    &\hspace{5em}\mbox{and $|H(i)| = k$ with $\max H(i) < \min
      H(i+1)$}\\
    &\hspace{5em}\mbox{for every $i \in \{1, 2, \ldots, m-1\}$} \,\Bigr\}.
  \end{align*}

  Given a semigroup $S$, $a \in S^{m+1}$, $H \in \calI_{m,k}$, and a
  sequence $f$ in $S$, define
  \[
    x(m, a, H, f) = \prod_{j=1}^m\Bigl( a(j)\prod_{t \in H(j)} f(t)
    \Bigr) a(m+1).
  \]
\end{defn}

\begin{thm}
  Let $A$ be a piecewise syndetic subset of a semigroup $S$. 
  Let $F \in \Pf(\setfunc{\bbN}{S})$.
  Then for all $k \in \bbN$, there exist $m \in \bbN$, $a \in
  S^{m+1}$, $H \in \calI_{m,k}$ such that for all $f \in F$, $x(m, a,
  H, f) \in A$.
\end{thm}
\begin{proof}
  Let $k \in \bbN$ and enumerate $F$ as $\{f_1, f_2, \ldots, f_l \}$.
  Put $Y = \bigtimes_{t=1}^l \beta S$.
  For each $i \in \bbN$, put
  \begin{align*}
    I_i &= \Bigl\{\, \bigl( x(m, a, H, f_1), x(m, a, H, f_2), \ldots,
    x(m, a, H, f_l) \bigr) : \mbox{$m \in \bbN$, $a \in S^{m+1}$,} \\
    &\hspace{5em} \mbox{$H \in \calI_{m,k}$, and $\min H(1) > i$}
    \,\Bigr\},
  \end{align*}
  and put $E_i = I_i \cup \{(a, a, \ldots, a) : a \in S\}$.
  Put $I = \bigcap_{i=1}^\infty \overline{I_i}$ and $E =
  \bigcap_{i=1}^\infty \overline{E_i}$.
  Observe $I$ and $E$ are nonempty closed subsets of $Y$.
  [Since each $I_i \ne \emptyset$ and $I_{i+1} \subseteq I_i$.]
  
  We claim that $E$ is a subsemigroup of $Y$ and $I$ is an ideal of
  $E$. 
  Let $p$, $q \in E$, $U$ be an open neighborhood of $pq$, and let $i
  \in \bbN$. 
  Since $\rho_q$ is continuous, pick $V$ a neighborhood of $p$ such
  that $Vq \subseteq U$. 
  If $p \in I$, then pick $\vec{x} \in I_i \cap V$ otherwise pick
  $\vec{x} \in E_i \cap V$.
  If $\vec{x} \in I_i$, then pick $m \in \bbN$, $a \in S^{m+1}$, and
  $H \in \calI_{m,k}$ with $\min H(1) > i$ such that
  \[
    \vec{x} = \bigl( x(m, a, H, f_1), x(m, a, H, f_2), \ldots, x(m,
    a, H, f_l) \bigr)
  \]
  In this case put $j = \max H(m)$, otherwise put $j=i$. 

  Since $\lambda_{\vec{x}}$ is continuous, pick $W$ a neighborhood of
  $q$ such that $xW \subseteq U$. 
  If $q \in I$, then pick $\vec{y} \in I_j \cap W$, otherwise pick
  $\vec{y} \in E_j \cap W$.
  Then $\vec{x} \vec{y} \in E_i \cap U$, and if $p \in I$ or $q \in
  I$, then $\vec{x} \vec{y} \in I_i \cap U$. 
  Hence, it follows that $E$ is a subsemigroup of $Y$ and $I$ is an
  ideal of $E$.

  Pick $p \in K(\beta S) \cap \overline{A}$ and put $\overline{p} =
  (p, p, \ldots, p)$. 
  Since $K(Y) = \bigtimes_{t=1}^l K(\beta S)$, we have that
  $\overline{p} \in K(Y)$. 
  We show that $\overline{p} \in E$.
  Let $U$ be a neighborhood of $\overline{p}$ and pick $B_1$, $B_2$,
  \dots, $B_l \in p$ such that $\bigtimes_{t=1}^l \overline{B_t}
  \subseteq U$. 
  Pick $a \in \bigcap_{t=1}^l B_t$, then $(a, a, \ldots, a) \in U \cap
  E_i$ for all $i \in \bbN$. 
  Hence $\overline{p} \in E$ and moreover $\overline{p} \in K(Y) \cap
  E$.

  Since $K(Y) \cap E \ne \emptyset$, we know that $K(E) = K(Y) \cap
  E$.
  Therefore $\overline{p} \in K(E) \subseteq I$.
  Hence $I_i \cap \bigtimes_{i=1}^l A \ne \emptyset$ for all $i \in
  \bbN$ and so our conclusion follows.
\end{proof}

If we call a set $A$ that satisfies the conclusion of this theorem a
\textsl{strong $J$-set}, we can formulate a natural question. 

\begin{ques}
  Is every $J$-set a strong $J$-set?
\end{ques}

At one point, I believed I had an approach, using Lemma 2.10 in
``Cartesian Products of Sets Satisfying the Central Sets Theorem''
paper, that would give a ``yes'' answer to this question. 
However it didn't stand up to close scrutiny, and I'm not sure how to
fix it. (Or, if it can even be fixed.)

My second project has been to find a combinatorial proof that central
sets are partition regular. 
Of course this would imply Hindman's Theorem, so it's no surprise that
I'm stuck here. 
Recall the definition ([HS, Def.~14.23]) for a \mbox{$*$-tree} in $A$,
where $A$ is a subset of a semigroup $S$.

\begin{defn}
  Let $S$ be a semigroup and $A \subseteq S$.
  We say a \mbox{$*$-tree} $T$ in $A$ has \textsl{f.i.p.}~if and only if
  $(\forall F \in \Pf(T))(\bigcap_{f \in F} B_f \ne \emptyset)$.
\end{defn}

It's easy to show that a set $A$ with a \mbox{$*$-tree} with f.i.p.~is
an IP-set. 
The converse is true by Lemma 2.6 in ``A simple characterization of
sets satisfying the Central Sets Theorem''. 
One of the nice things about working with \mbox{$*$-trees} with
f.i.p.~instead of IP-sets is that the fact that idempotent
ultrafilters exist follows ``naturally'' from an application of Zorn's
Lemma.

\begin{thm}
  Let $S$ be a semigroup and $A \subseteq S$.
  If there exists a \mbox{$*$-tree} in $A$ with f.i.p. and $A = C_1
  \cup C_2 \cup \cdots \cup C_r$, then there exists $i \in \{1, 2,
  \ldots, r\}$ such that $C_i$ has a \mbox{$*$-tree} with f.i.p.
\end{thm}
\begin{proof}
  Let $T$ be a \mbox{$*$-tree} in $A$ with f.i.p.
  Put $\calB = \{\, B_f(T) : f \in T \,\}$ and put 
  \begin{align*}
      \cal{X} &= \bigl\{\, \calC \subseteq \calP(A) : \mbox{$\calC$ has
      f.i.p., $\calB \subseteq \calC$,} \\
    &\hspace{4em}\mbox{and $(\forall C \in
      \calC)(\{\, x \in C: (\exists D \in \calC)(D \subseteq x^{-1}C)
      \,\} \in \calC)$} \,\bigr\}.
  \end{align*}
  Since $T$ has f.i.p. it follows that $\calB$ has f.i.p.
  Since $T$ is a \mbox{$*$-tree} it follows that for all $f \in T$, 
  $B_f = \{\, x \in B_f : \mbox{$(\exists g \in T)\bigr(B_g \subseteq
    x^{-1}B_f(T)\bigl)$} \,\} \in \calB$ since $B_{f^\frown x} \subseteq
  x^{-1}B_f$. 
  Hence $\calB \in \calX$ and $\calX \ne \emptyset$. 
  Now every chain in $\calX$ also has an upper bound, so we can apply
  Zorn's Lemma to pick a maximal member.
  This maximal member will be an idempotent ultrafilter.
\end{proof}

What I would like is a proof of this Theorem without the Axiom of
Choice, but using filters (and possibly ultrafilters).
I know that Hindman's Theorem is equivalent to the existence of
idempotent ultrafilters in $\beta S$, and idempotent ultrafilters in
$\beta \bbN$ are necessarily nonprincipal.
Hence, I also know that Hindman's Theorem implies the so-called Weak
Ultrafilter Principle, that is the statement that, a nonprincipal
ultrafilter exists on $\bbN$.
Since Hindman's Theorem implies this weak form of choice, any proof of
Hindman's Theorem can't be totally constructive.  

With this in mind, I have slowly been reading Yehuda Rav's paper,
``Variants of Rado's Selection Lemma and their Applications''.
The interesting thing about this paper is that Rav proves that there
are several combinatorial-type statements equivalent to the
Ultrafilter Principle.
(Recall the Ultrafilter Principle states that every filter can be
extended to an ultrafilter.
This Principle is stronger than the Weak Ultrafilter Principle.)
I've been trying to apply these combinatorial-type statements to
produce another proof of Hindman's Theorem, but so far no luck.

Another possible avenue I haven't investigated yet is to use a
combination of the Axiom of Choice for Finite Sets (ACF) and Rado's
Selection Lemma (RL).  
It's known that under ZF, \mbox{ACF+RL} implies the Ultrafilter Principle.
However, I recently found a paper, ``The Axiom of Choice for Finite
Sets'' by R.~L.~Blair and M.~L.~Tomber, that reformulates the ACF into a
restricted form of Zorn's Lemma. 
So now my plan is to see if I can use this restricted Zorn's Lemma
along with RL to produce another proof of Hindman's Theorem.
Once (or if!) this is done, I'll then try to translate this into the
combinatorial-type theorems in Rav's paper.

Finally, my 0.05 project is to prove, combinatorially, that piecewise
syndetic sets are $J$-sets.
This is only $5/100$ of a project since I don't really have any 
ideas here.
The only plan I have is to carefully read the original Hales-Jewett
paper and try to modify their methods. 

\closing{Sincerely, }

\end{letter}
\end{document}