% Mathematical Letter to Dr. Hindman on Dissertation Progress
% February 17, 2011

\documentclass[12pt]{letter}

\usepackage{amsthm, amssymb, amsmath}

\newtheoremstyle{plain}{3mm}{3mm}{\slshape}{}{\bfseries}{.}{.5em}{}
\theoremstyle{plain}
\newtheorem*{nvdw}{Noncommutative van der Waerden Theorem}
\newtheorem*{thm}{Theorem}
\newtheorem*{ques}{Question}

\theoremstyle{definition}
\newtheorem*{defn}{Definition}

\newcommand{\la}{\langle}
\newcommand{\ra}{\rangle}
\newcommand{\bbN}{\mathbb{N}}
\newcommand{\bbZ}{\mathbb{Z}}
\newcommand{\calI}{\mathcal{I}}
\newcommand{\calJ}{\mathcal{J}}
\newcommand{\calT}{\mathcal{T}}
\newcommand{\Pf}{\mathcal{P}_f}
\newcommand{\ds}{(X, \la T_s \ra_{s\in S})}
\newcommand{\setfunc}[2]{\hbox{${}^{\hbox{$#1$}}\hskip -3 pt #2$}}

\font\bigmath=cmsy10 scaled \magstep 3
\newcommand{\bigtimes}{\hbox{\bigmath \char'2}}

\newcommand{\cchi}{\raise 2 pt \hbox{$\chi$}}


\makeatletter
\let\@texttop\relax
\makeatother

\date{February 17, 2011}
\signature{John H.~Johnson}

\begin{document}
\begin{letter}{}
\opening{Dr.~Hindman}

I'm still working on my Dissertation, but as you can deduce, my
writing is proceeding slower than I originally planned.  
(For instance, I have clearly missed my February 16 deadline!)
In addition, I have also been working on several other (mathematical)
projects connected to my Dissertation.

In my Introduction I would like to motivate the definitions
surrounding the ``noncommutative version'' of the Central Sets Theorem
by showing that J-sets should satisfy the conclusion of the
Noncommutative van der Waerden Theorem.  
(I'll abbreviate this theorem by NVDW from now on.)
Recall that in your paper ``Ramsey Theory in Noncommutative
Semigroups'', you and Bergelson formulated and proved NVDW.

\begin{nvdw}
  Let $S$ be a semigroup, $l$, $r \in \bbN$, and let $\la d_n
  \ra_{n=1}^\infty$ be a sequence in $S$. 
  If $S = \bigcup_{i=1}^r C_i$, then there exist $i \in \{1, 2,
  \ldots, r\}$, $m \in \bbN$, $a \in S^{m+1}$, and $n(1) < n(2) <
  \cdots < n(m)$ in $\bbN$ such that 
  \[
    \biggl\{ \prod_{j=1}^{m+1}a(j) \biggr\} \cup \biggl\{\,
    \prod_{j=1}^m\Bigl(a(j)d_{n(j)}^k\Bigr)a(m+1) : k \in \{1, 2, \ldots, l\}
    \,\biggr\} \subseteq C_i.
  \]
\end{nvdw}

You and Bergelson prove this fact by showing that piecewise syndetic
sets satisfy the conclusion of NVDW via Theorem 2.6 (in the paper).
Now in analogy with how the strengthened version of VDW easily follows
the definition of J-sets, I would like to show, only using the
definition, that J-sets satisfy the conclusion of NVDW.
Unfortunately, I'm not able to prove such a statement, and I'm not
even sure if it is a fact.

Using the definition of J-sets and another theorem of mine, I am able
to easily prove the following (possibly) weaker result.

\begin{thm}
  Let $A$ be a $J$-set of a semigroup $S$.
  Then there exist $m \in \bbN$, $a \in S^{m+1}$, and $d \in S^{m+1}$
  such that
  \[
    \biggl\{ \prod_{j=1}^{m+1}a(j) \biggr\} \cup \biggl\{\,
    \prod_{j=1}^m\Bigl(a(j)d(j)^k\Bigr)a(m+1) : k \in \{1, 2, \ldots, l\}
    \,\biggr\} \subseteq A.
  \]
\end{thm}

The reason for my pessimism expressed above is that Theorem 2.6 (in
the paper) made me realize that piecewise syndetic sets are not only
J-sets, but they are J-sets in which you can choose how many terms of
your set of sequences you multiple together.
More precisely, we have the following theorem.

\begin{defn}
  For each $m$, $k \in \bbN$, put
  \begin{align*}
    \calI_{m,k} &= \Bigl\{\, \bigl(H(1), H(2), \ldots, H(m)\bigr) :
    \mbox{for each $i \in \{1, 2, \ldots, m\}$, $H(i) \in \Pf(\bbN)$}
    \\
    &\hspace{5em}\mbox{and $|H(i)| = k$ with $\max H(i) < \min
      H(i+1)$}\\
    &\hspace{5em}\mbox{for every $i \in \{1, 2, \ldots, m-1\}$} \,\Bigr\}.
  \end{align*}
\end{defn}

\begin{thm}
  Let $A$ be a piecewise syndetic subset of a semigroup $S$. 
  Let $F \in \Pf(\setfunc{\bbN}{S})$.
  Then for all $k \in \bbN$, there exist $m \in \bbN$, $a \in
  S^{m+1}$, $H \in \calI_{m,k}$ such that for all $f \in F$, $x(m, a,
  H, f) \in A$.
\end{thm}
\begin{proof}
  Let $k \in \bbN$ and enumerate $F$ as $\{f_1, f_2, \ldots, f_l \}$.
  Put $Y = \bigtimes_{t=1}^l \beta S$.
  For each $i \in \bbN$, put
  \begin{align*}
    I_i &= \Bigl\{\, \bigl( x(m, a, H, f_1), x(m, a, H, f_2), \ldots,
    x(m, a, H, f_l) \bigr) : \mbox{$m \in \bbN$, $a \in S^{m+1}$,} \\
    &\hspace{5em} \mbox{$H \in \calI_{m,k}$, and $\min H(1) > i$}
    \,\Bigr\},
  \end{align*}
  and put $E_i = I_i \cup \{(a, a, \ldots, a) : a \in S\}$.
  Put $I = \bigcap_{i=1}^\infty \overline{I_i}$ and $E =
  \bigcap_{i=1}^\infty \overline{E_i}$.
  Observe $I$ and $E$ are nonempty closed subsets of $Y$.
  [Since each $I_i \ne \emptyset$ and $I_{i+1} \subseteq I_i$.]
  
  We claim that $E$ is a subsemigroup of $Y$ and $I$ is an ideal of
  $E$. 
  Let $p$, $q \in E$, $U$ be an open neighborhood of $pq$, and let $i
  \in \bbN$. 
  Since $\rho_q$ is continuous, pick $V$ a neighborhood of $p$ such
  that $Vq \subseteq U$. 
  If $p \in I$, then pick $\vec{x} \in I_i \cap V$ otherwise pick
  $\vec{x} \in E_i \cap V$.
  If $\vec{x} \in I_i$, then pick $m \in \bbN$, $a \in S^{m+1}$, and
  $H \in \calI_{m,k}$ with $\min H(1) > i$ such that
  \[
    \vec{x} = \bigl( x(m, a, H, f_1), x(m, a, H, f_2), \ldots, x(m,
    a, H, f_l) \bigr)
  \]
  In this case put $j = \max H(m)$, otherwise put $j=i$. 

  Since $\lambda_{\vec{x}}$ is continuous, pick $W$ a neighborhood of
  $q$ such that $xW \subseteq U$. 
  If $q \in I$, then pick $\vec{y} \in I_j \cap W$, otherwise pick
  $\vec{y} \in E_j \cap W$.
  Then $\vec{x}+\vec{y} \in E_i \cap U$, and if $p \in I$ or $q \in
  I$, then $\vec{x} + \vec{y} \in I_i \cap U$. 
  Hence, it follows that $E$ is a subsemigroup of $Y$ and $I$ is an
  ideal of $E$.

  Pick $p \in K(\beta S) \cap \overline{A}$ and put $\overline{p} =
  (p, p, \ldots, p)$. 
  Since $K(Y) = \bigtimes_{t=1}^l K(\beta S)$, we have that
  $\overline{p} \in K(Y)$. 
  We show that $\overline{p} \in E$.
  Let $U$ be a neighborhood of $\overline{p}$ and pick $B_1$, $B_2$,
  \dots, $B_l \in p$ such that $\bigtimes_{t=1}^l \overline{B_t}
  \subseteq U$. 
  Pick $a \in \bigcap_{t=1}^l B_t$, then $(a, a, \ldots, a) \in U \cap
  E_i$ for all $i \in \bbN$. 
  Hence $\overline{p} \in E$ and moreover $\overline{p} \in K(Y) \cap
  E$.

  Since $K(Y) \cap E \ne \emptyset$, we know that $K(E) = K(Y) \cap
  E$.
  Therefore $\overline{p} \in K(E) \subseteq I$.
  Hence $I_i \cap \bigtimes_{i=1}^l A \ne \emptyset$ for all $i \in
  \bbN$ and so our conclusion follows.
\end{proof}

If we call a set $A$ that satisfies the conclusion of this theorem a
\textsl{strong $J$-set}, we can formulate a natural questions. 

\begin{ques}
  Is every $J$-set a strong $J$-set?
\end{ques}


\end{letter}
\end{document}