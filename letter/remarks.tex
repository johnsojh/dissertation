% Remarks on mathematical letter pointed out by Hindman.


\documentclass[12pt]{article}

\usepackage{amsthm, amssymb, amsmath}

\newtheoremstyle{plain}{3mm}{3mm}{\slshape}{}{\bfseries}{.}{.5em}{}
\theoremstyle{plain}
\newtheorem*{nvdw}{Noncommutative van der Waerden Theorem}
\newtheorem*{thm}{Theorem}
\newtheorem*{cor}{Corollary}
\newtheorem*{lem}{Lemma}
\newtheorem*{ques}{Question}

\theoremstyle{definition}
\newtheorem*{defn}{Definition}

\newcommand{\la}{\langle}
\newcommand{\ra}{\rangle}
\newcommand{\bbN}{\mathbb{N}}
\newcommand{\bbZ}{\mathbb{Z}}
\newcommand{\calB}{\mathcal{B}}
\newcommand{\calC}{\mathcal{C}}
\newcommand{\calX}{\mathcal{X}}
\newcommand{\calI}{\mathcal{I}}
\newcommand{\calJ}{\mathcal{J}}
\newcommand{\calP}{\mathcal{P}}
\newcommand{\calT}{\mathcal{T}}
\newcommand{\Pf}{\mathcal{P}_f}
\newcommand{\dom}{\mathrm{dom}}
\newcommand{\setfunc}[2]{\hbox{${}^{\hbox{$#1$}}\hskip -3 pt #2$}}

\font\bigmath=cmsy10 scaled \magstep 3
\newcommand{\bigtimes}{\hbox{\bigmath \char'2}}

\newcommand{\cchi}{\raise 2 pt \hbox{$\chi$}}


\begin{document}
Recall from the letter that we have the following definition and theorem.
\begin{defn}
  For each $m$, $k \in \bbN$, put
  \begin{align*}
    \calI_{m,k} &= \Bigl\{\, \bigl(H(1), H(2), \ldots, H(m)\bigr) :
    \mbox{for each $i \in \{1, 2, \ldots, m\}$, $H(i) \in \Pf(\bbN)$}
    \\
    &\hspace{5em}\mbox{and $|H(i)| = k$ with $\max H(i) < \min
      H(i+1)$}\\
    &\hspace{5em}\mbox{for every $i \in \{1, 2, \ldots, m-1\}$} \,\Bigr\}.
  \end{align*}

  Given a semigroup $S$, $a \in S^{m+1}$, $H \in \calI_{m,k}$, and a
  sequence $f$ in $S$, define
  \[
    x(m, a, H, f) = \prod_{j=1}^m\Bigl( a(j)\prod_{t \in H(j)} f(t)
    \Bigr) a(m+1).
  \]
\end{defn}

\begin{thm}
  Let $A$ be a piecewise syndetic subset of a semigroup $S$. 
  Let $F \in \Pf(\setfunc{\bbN}{S})$.
  Then for all $k \in \bbN$, there exist $m \in \bbN$, $a \in
  S^{m+1}$, $H \in \calI_{m,k}$ such that for all $f \in F$, $x(m, a,
  H, f) \in A$.
\end{thm}

Calling sets which satisfy the conclusion of this theorem
\textsl{strong $J$-sets} I asked the following natural question.

\begin{ques}
  Is every $J$-set a strong $J$-set?
\end{ques}

By a simple argument of Hindman, it turns out that the answer to this
question is ``yes''.
This argument is based on a few observations concerning piecewise
syndetic sets and $\calI_{m,k}$.

\begin{thm}
  Let $A$ be a piecewise syndetic subset of a semigroup $S$.
  For each $k \in \bbN$ and $F \in \Pf(\setfunc{\bbN}{S})$ let $S(k,
  F)$ denote the following statement:
  \[
    (\exists m \in \bbN) (\exists a \in S^{m+1}) (\exists H \in
    \calI_{m,k}) (\forall f \in F) \bigl( x(m, a, H, f) \in A \bigr).
  \]
  If for all $F \in \Pf(\setfunc{\bbN}{S})$, $S(1,F)$ is true,
  then $S(1, F) \implies S(k, F)$ for all $k \in \bbN$.
\end{thm}
\begin{proof}
  Let $k \in \bbN$.
  If $k = 1$, then the conclusion is trivially true.
  So we may assume that $k > 1$.
  For each $f \in F$, define $g_f \in \setfunc{\bbN}{S}$ by $g_f(t) =
  f(k t) f(k t+1) \cdots f(k t+(k-1))$. 
  
  Put $G = \{\, g_f : f \in F \,\}$.
  Then since $S(1, G)$ is true, pick $m \in \bbN$, $a \in S^{m+1}$,
  and a strictly increasing sequence $n(1) < n(2) < \cdots < n(m)$ in
  $\bbN$ such that for every $f \in F$,
  $\prod_{j=1}^m\Bigl(a(j)g_f \bigl(n(j)\bigr)\Bigr) a(m+1) \in A$. 

  For each $j \in \{1, 2, \ldots, m\}$, define $H(j) \in \Pf(\bbN)$ by
  $H(j) = \{\, k  n(j), k n(j) + 1, \ldots, k n(j) + (k-1) \,\}$. 
  I claim that $\max H(j) < min H(j+1)$ for all $j \in \{1, 2, \ldots,
  m\}$. 
  Now $\max H(j) = k n(j) + (k-1)$ and $\min H(j+1) = k n(j+1)$. 
  Observe that we have the following:
  \begin{align*}
    n(j) < n(j+1) &\implies n(j) < n(j) + 1 \le n(j+1), \\
    &\implies k n(j) < k n(j) + k \le k n(j+1), \\
    &\implies k n(j) < k n(j) + (k-1) < k n(j) + k \le k n(j+1).
  \end{align*}
  Hence our claim follows.
  Moreover, $H = \bigl(H(1), H(2), \ldots, H(m) \bigr) \in
  calI_{m,k}$ and it follows that for all $f \in F$, $x(m, a, H, f)
  \in A$.
\end{proof}

\begin{thm}
  Let $A$ be $J$-set in a semigroup $S$.
  If $F \in \Pf(\setfunc{\bbN}{S})$, then there exist $m \in \bbN$, $a
  \in S^{m+1}$, and an strictly increasing sequence $n(1) < n(2) <
  \cdots < n(m)$ in $\bbN$ such that for all $f \in F$, 
  \[
    \prod_{j=1}^m \Bigl( a(j) f(n(j) \Bigr)a(m+1) \in A.
  \]
\end{thm}
\begin{proof}
  True but requires a somewhat tedious computation of the appropriate 
  vector $a$. 
\end{proof}

\begin{defn}
  Let $S$ be a semigroup and $A \subseteq S$.
  We call $A$ a \textsl{$J$-set (in S)} if and only if for every $F
  \in \Pf(\setfunc{\bbN}{S})$ there exist $m \in \bbN$, $a \in
  S^{m+1}$, and a strictly increasing sequence $n(1) < n(2) < \cdots <
  n(m)$ in $\bbN$ such that for all $f \in F$,
  \[
    \prod_{j=1}^m \Bigr( a(j) f(n(j)) \Bigl) a(m+1) \in A.
  \]
\end{defn}

\begin{defn}
  Let $S$ be a semigroup.
  \begin{itemize}
    \item[(a)] Given $m \in \bbN$, $a \in S^{m+1}$, a strictly
      increasing sequence $n(1) < n(2) < \cdots < n(m)$ in $\bbN$, and
      $f \in \calT(S)$, define
      \[
        x(m, a, n, f) = \biggl( \prod_{j=1}^m \bigl( a(j) f(n(j))
        \bigr) \biggr) a(m+1).
      \]
      
    \item[(b)] A set $A \subseteq S$ is a \textsl{$J$-set (in S)} if
      and only if for every $F \in \Pf(\calT(S))$, there exist $m \in
      \bbN$, $a \in S^{m+1}$, and a strictly increasing sequence $n(1)
      < n(2) < \cdots < n(m)$ in $\bbN$, such that for all $f \in F$,
      $x(m, a, n, f) \in A$. 

    \item[(c)] A set $C \subseteq S$ is a \textsl{$C$-set (in S)} if
      and only if there exist functions $m \colon \Pf(\calT) \to
      \bbN$, $\alpha \in \bigtimes_{F \in \Pf(\calT)} S^{m(F) + 1}$,
      $n \in \bigtimes_{F \in \Pf(\calT)} \bbN^{m(F)}$ such that 
      \begin{itemize}
        \item[(1)] for each $F \in \Pf(\calT)$, $n(F)$ is a strictly
          increasing sequence;
        
        \item[(2)] if $F$, $G \in \Pf(\calT)$ with $F \subsetneq G$,
          then $n(F)\bigl(m(F)\bigr) < n(G)(1)$; and

        \item[(3)] whenever $p \in \bbN$, $G_1$, $G_2$, \ldots, $G_p
          \in \Pf(\calT)$ with $G_1 \subsetneq G_2 \subsetneq \cdots
          \subsetneq G_p$ and $\la f_i \ra_{i=1}^p \in
          \bigtimes_{i=1}^p G_i$, then $\prod_{i=1}^p x(m(G_i),
          \alpha(G_i), n(G_i), f_i) \in A$. 
      \end{itemize}
  \end{itemize}
\end{defn}

\begin{lem}
  Let $S$ be a semigroup, $A \subseteq S$ a $J$-set, and $A = A_1 \cup
  A_2$.
  Then either $A_1$ is a $J$-set in $S$ or $A_2$ is a $J$-set in $S$.
\end{lem}

\begin{thm}
  Let $S$ be a semigroup and $A \subseteq S$.
  Then $\overline{A} \cap J(S) \ne \emptyset$ if and only if $A$ is a
  $J$-set. 
\end{thm}

\begin{thm}
  Let $S$ be a semigroup and $C \subseteq S$.
  Then $C$ is a $C$-set if and only if there is an idempotent $p \in
  \overline{C} \cap J(S)$. 
\end{thm}

\begin{cor}
  Let $S$ be a semigroup and $C \subseteq S$.
  If $C$ is a central set, then $C$ is a $C$-set. 
\end{cor}
\end{document}


