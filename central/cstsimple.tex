% Author: John H. Johnson
% email: john.j.jr@gmail.com
%
% Brief description of file
% -------------------------
%
% This is a chapter in my dissertation tentatively entitled "A New and
% Simpler Central Sets Theorem".  The new and simpler part is due to a
% different definition of J-sets, which in turns leads to a different
% definition of C-sets.  Happily the new definition is equivalent to
% the old definition.  Therefore no proofs need to be rewritten; but
% there is value in rewriting the proofs, since as this chapter shows,
% the rewritten proofs are often simpler than the old proofs.
%
% N.B. The new definition of J-sets essentially first appears in an
% article, "Ramsey Theory in Noncommutative Semigroups", by Bergelson
% and Hindman in their Theorem 2.6.

\documentclass[12pt]{article}

\usepackage{amsthm, amssymb, amsmath}
\usepackage{color}
\usepackage{endnotes}
\usepackage{url}


\usepackage[margin=1in]{geometry}

\usepackage[doublespacing]{setspace}
\usepackage{url}

\newtheoremstyle{plain}{3mm}{3mm}{\slshape}{}{\bfseries}{.}{.5em}{}
\theoremstyle{plain}

% Numbered theorems 
\newtheorem{thm}{Theorem}[section]
\newtheorem{lem}[thm]{Lemma}
\newtheorem{prop}[thm]{Proposition}
\newtheorem{cor}[thm]{Corollary}
\newtheorem{up}[thm]{Ultrafilter Principle}
\newtheorem{radoSelect}[thm]{Rado's Selection Lemma}
\newtheorem{php}[thm]{Pigeonhole Principle}
\newtheorem{vdw}[thm]{Van der Waerden's Theorem}
\newtheorem{hj}[thm]{Hales-Jewett Theorem}
\newtheorem{shj}[thm]{Special Hales-Jewett Theorem}


\newtheorem{FST}[thm]{Hindman's Theorem}
\newtheorem{MBR}[thm]{Multiple Birkhoff Recurrence Theorem}
\newtheorem{recur}[thm]{Recurrence Theorem}
\newtheorem{OCST}[thm]{Furstenburg's Original Central Sets Theorem}
\newtheorem{cst}[thm]{Central Sets Theorem}



\newtheorem{claim}[thm]{Claim}
\newtheorem{ques}[thm]{Question}
\newtheorem{conj}[thm]{Conjecture}


\theoremstyle{definition}

% Numbered "definition" style theorem environments
\newtheorem{defn}[thm]{Definition}
\newtheorem{rmk}[thm]{Remark}
\newtheorem{example}[thm]{Example}

\newcommand{\la}{\langle}
\newcommand{\ra}{\rangle}
\newcommand{\bbN}{\mathbb{N}}
\newcommand{\bbZ}{\mathbb{Z}}
\newcommand{\bbR}{\mathbb{R}}
\newcommand{\AP}{\mathcal{AP}}
\newcommand{\AL}{\mathcal{AL}}

% Short names for calligraphic math letters.
\newcommand{\calA}{\mathcal{A}}
\newcommand{\calB}{\mathcal{B}}
\newcommand{\calC}{\mathcal{C}}
\newcommand{\calE}{\mathcal{E}}
\newcommand{\calF}{\mathcal{F}}
\newcommand{\calG}{\mathcal{G}}
\newcommand{\calH}{\mathcal{H}}
\newcommand{\calI}{\mathcal{I}}
\newcommand{\calJ}{\mathcal{J}}
\newcommand{\calP}{\mathcal{P}}
\newcommand{\calR}{\mathcal{R}}
\newcommand{\calS}{\mathcal{S}}
\newcommand{\calT}{\mathcal{T}}
\newcommand{\calU}{\mathcal{U}}

\newcommand{\Pf}{\mathcal{P}_f}


\newcommand{\setfunc}[2]{\hbox{${}^{\hbox{$#1$}}\hskip -1 pt #2$}}

\font\bigmath=cmsy10 scaled \magstep 3
\newcommand{\bigtimes}{\hbox{\bigmath \char'2}}

\newcommand{\cchi}{\raise 2 pt \hbox{$\chi$}}

\begin{document}
\section{$J$-sets}
\begin{defn}
  Let $(S, \cdot)$ be a semigroup.
  \begin{itemize}
    \item[(b)] For each $m \in \bbN$, define
      \[
        \calJ_m = \{\, (t_1, t_2, \ldots, t_m) \in \bbN^m : t_1 < t_2 < \cdots < t_m \,\}.
      \]

    \item[(b)] Put $\calT(S) = \setfunc{\bbN}{S}$.
      If the semigroup is clear from context, we write $\calT$ instead of $\calT(S)$.

    \item[(c)] For each $m \in \bbN$, $a \in S^{m+1}$, $t \in \calJ_m$, and $f \in \calT$, put
      \[
        \textstyle
        x(m, a, t, f) = \Bigl( \prod_{i=1}^m \bigl( a(i)f(t_i) \bigr) \Bigr)a(m+1).
      \]

    \item[(d)] We call a subset $A \subseteq S$ a \textsl{$J$-set (in $S$)} if and only if for every $F \in \Pf(\calT)$ there exist $m \in \bbN$, $a \in S^{m+1}$, and $t \in \calJ_m$ such that for each $f \in F$ we have $x(m, a, t, f) \in A$.

    \item[(d)] Put
      \[
        J(S) = \{\, p \in \beta S : \mbox{$A$ is a $J$-set for every $A \in p$} \,\}.
      \]
  \end{itemize}
\end{defn}
\begin{rmk}
  Despite appearances to the contrary, I must point out that $J$-sets are \textsl{not} named after the author!
\end{rmk}

\begin{thm}
  Let $(S, \cdot)$ be a semigroup.
  If $J(S)$ is nonempty, then $J(S)$ is a compact two-sided ideal of $\beta S$.
\end{thm}
\begin{proof}
  To show that $J(S)$ is compact it suffices to show that $J(S)$ is topologically closed in $\beta S$.
  Let $p \not\in J(S)$ and pick $A \in p$ such that $A$ is not a $J$-set.
  By definition of $J(S)$ we must have $\overline{A} \cap J(S) = \emptyset$.
  Since $\overline{A}$ is a (basic) open neighborhood of $p$, it follows that $J(S)$ is topologically closed in $\beta S$.

  Now let $p \in J(S)$ and $q \in \beta S$.
  To see that $J(S)$ is an ideal, we show that $pq \in J(S)$ and $qp \in J(S)$. 

  We first show that $pq \in J(S)$.
  Let $F \in \Pf(\calT)$, let $A \in pq$, and put $B = \{\, x \in S : x^{-1}A \in q \,\}$.
  Then $B \in p$ and so $B$ is a $J$-set.
  Pick $m \in \bbN$, $a \in S^{m+1}$, and $t \in \calJ_m$ such that for all $f \in F$ we have $x(m, a, t, f) \in B$.
  By definition of $B$ this means that for all $f \in F$, $x(m, a, t, f)^{-1}A \in q$. 
  Since $q$ is an ultrafilter and $F$ is finite, we have $\bigcap_{f \in F} x(m, a, t, f)^{-1}A \in q$, and moreover, $\bigcap_{f \in F} x(m, a, t, f)^{-1}A \ne \emptyset$.
  Pick $b \in \bigcap_{f \in F} x(m, a, t, f)^{-1}A$ and define $c \in S^{m+1}$ by
  \[
    c =
    \begin{cases}
      a(j) & \mbox{if $j \in \{1, 2, \ldots, m\}$,} \\
      a(m+1)b & \mbox{if $j = m+1$.}
    \end{cases}
  \]
  Therefore, for each $f \in F$, $x(m, c, t, f) \in A$, that is, $A$ is a $J$-set and so $pq \in J(S)$.

  Now we show that $qp \in J(S)$.
  Again let $F \in \Pf(\calT)$, let $A \in qp$, and put $B = \{\, x \in S : x^{-1}A \in p \,\}$.
  Then $B \in q$, and moreover $B \ne \emptyset$.
  Pick $b \in B$, then $b^{-1}A \in p$ implies $b^{-1}A$ is a $J$-set.
  Pick $m \in \bbN$, $a \in S^{m+1}$, and $t \in \calJ_m$ such that for all $f \in F$, we have $x(m, a, t, f) \in b^{-1}A$.
  Define $c \in S^{m+1}$ by
  \[
    c =
    \begin{cases}
      ba(1) & \mbox{if $j =1$,} \\
      a(j) & \mbox{if $j \in \{2, \ldots, m, m+1\}$.}
    \end{cases}
  \]
  Therefore, for each $f \in F$, $x(m, c, t, f) \in A$, that is, $A$ is a $J$-set and so $qp \in J(S)$.
\end{proof}

Even though it is clear that every semigroup $S$ is a $J$-set, this easy fact doesn't imply that $J(S) \ne \emptyset$. 
By our work in the Preliminaries Chapter, to show that $J(S)$ is nonempty, it suffices to show that $J$-sets are partition regular.

\begin{lem}
  Let $(S, \cdot)$ be a semigroup, $m$,$r \in \bbN$, $a \in S^{m+1}$, $t \in \calJ_m$, and for each $y \in \bbN$ 

\end{lem}

\begin{lem}
  Let $(S, \cdot)$ be a semigroup with $A_1 \subseteq S$ and $A_2 \subseteq S$.
  If $A_1 \cup A_2$ is a $J$-set, then either $A_1$ is a $J$-set or $A_2$ is a $J$-set.
\end{lem}
\begin{proof}
  Put $A = A_1 \cup A_2$ and suppose that both $A_1$ and $A_2$ are \textsl{not} $J$-sets. 
  Then pick $F_1 \in \Pf(\calT)$ such that for all $m \in \bbN$, $a \in S^{m+1}$, and $t \in \calJ_m$, there exists $f \in F_1$ such that $x(m, a, t, f) \not\in A_1$; and also pick $F_2 \in \Pf(\calT)$ such that for all $m \in \bbN$, $a \in S^{m+1}$, and $t \in \calJ_m$, there exists $f \in F_1$ such that $x(m, a, t, f) \not\in A_2$.
  With $F_1$ and $F_2$ we will arrive at a contradiction that proves the lemma.

  Let $F = F_1 \cup F_2$, put $k = |F|$, and enumerate $F = \{f_1, f_2, \ldots, f_k\}$.
  By the Hales-Jewett Theorem, pick $n \in \bbN$, such that whenever $\{1, 2, \ldots, k\}^n$ is 2-colored there exists a variable word $w(\star)$ with no two adjacent variable letters such that $w(\star)$ begins and ends with a constant and the combinatorial line $\bigl\{\, w(\ell) : \ell \in \{1, 2, \ldots, k\} \,\bigr\}$ is monochromatic. 
  (For instance, if $k = 4$ and $n = 10$, then $w(\star) = (1, \star, 1, 4, \star, 2, \star, 3, 3, 1)$ is a variable word with no two adjacent variable letters such that it begins and ends with a constant letter.)
  
  For each $w = (x_1, x_2, \ldots, x_n) \in \{1, 2, \ldots, k\}^n$ define $g_w \in \calT$ by $g_w(t) = \prod_{i=1}^n f_{x_i}(nt + i)$.
  Of course $\{1, 2, \ldots, k\}^n$ is finite and $A$ is a $J$-set, so we may pick $m \in \bbN$, $a \in S^{m+1}$, and $t \in \calJ_m$ such that for all $w \in \{1, 2, \ldots, k\}^n$ we have $x(m, a, t, g_w) \in A$.

  Define the function $\varphi \colon \{1, 2, \ldots, k\} \to \{1, 2\}$ as follows:
  \[
    \varphi(w) = 
    \begin{cases}
      1 & \mbox{if $x(m, a, t, g_w) \in A_1$, and} \\
      2 & \mbox{otherwise.}
    \end{cases}
  \]
  Since $\varphi$ is a 2-coloring of $\{1, 2, \ldots, k\}^n$ we may apply the Hales-Jewett Theorem to pick a variable word $w(\star)$ with no two adjacent variable letters such that $w(\star)$ begins and ends with a constant letter and the combinatorial line $\bigl\{\, w(\ell) : \ell \in \{1, 2, \ldots, k\} \,\bigr\}$ is monochromatic. 
  Without loss of generality we may suppose that $\varphi \bigl( w(\ell) \bigr) = 1$ for every $\ell \in \{1, 2, \ldots, k\}$.

  In order to arrive at our contradiction, we show that there exist $p \in \bbN$, $c \in S^{p+1}$, and $s \in \calJ_p$ such that for all $\ell \in \{1, 2, \ldots, k\}$, $ x(p, c, s, f_\ell) = x(m, a, t, g_{w(\ell)}) \in A_1$. 
  To this end, for the rest of the proof, we focus on rewriting $x(m, a, t, g_{w(\star)})$ into the appropriate form. 

  Let $w(\star) = (x_1, x_2, \ldots, x_n)$, where each $x_i \in \{1, 2, \ldots, k\} \cup \{\star\}$. 
  Let $r$ be the number of variable letters in $w(\star)$, and for each $i \in \{1, 2, \ldots, r-1\}$, let $b_i$ be the position of the \mbox{$i$th} variable letter in $w(\star)$ and let $b_0$ be the position of the \mbox{$r$th} variable letter.
  (For instance, if $k = 4$ and $n = 10$, $w(\star) = (1, \star, 1, 4, \star, 2, \star, 3, 3, 1)$, then $r = 3$, $b_1 = 2$, $b_2 = 5$, and $b_0 = 7$.)

  Put $p = mr$ and for each $i \in \{1, 2, \ldots, p\}$ define $s_i$ by $s_i = nt_{\lceil i/r \rceil} + b_{i \bmod r}$.
  We show that $s = (s_1, s_2, \ldots, s_p) \in \calJ_p$.
  Let $i \in \{1, 2, \ldots, p-1\}$.
  If $\lceil i/r \rceil = \lceil (i+1)/r \rceil$, then $b_{i \bmod r} < b_{(i+1) \bmod r}$ and so $s_i < s_{i+1}$.
  If $\lceil i/r \rceil < \lceil (i+1)/r \rceil$, then it follows that $s_i < s_{i+1}$ also.

  Let $\bigl\{\, L(i) \in \Pf(\{1, 2, \ldots, n\} \setminus \{b_1, b_2, \ldots, b_r\}) : i \in \{1, 2, \ldots, r+1\} \,\bigr\}$ be a partition of $\{1, 2, \ldots, n\} \setminus \{b_1, b_2, \ldots, b_r\}$ such that $\max L(i) < \min L(i+1)$ for all $i \in \{1, 2, \ldots, r\}$.
  (For instance, if $k = 4$, $n = 10$, $w(\star) = (1, \star, 1, 4, \star, 2, \star, 3, 3, 1)$, then $L(1) = \{1\}$, $L(2) = \{2\}$, $L(3) = \{6\}$, and $L(4) = \{8, 9, 10\}$.)
  Define $c \in S^{p+1}$ as follows:
  \[
    c(i) = 
    \begin{cases}
      
    \end{cases}
  \]
  
  Since $\varphi$ is a 2-coloring of $[k]^n$, by the Hales-Jewett Theorem, there exists a variable word $w(\star)$ which begins and ends with a constant letter and there are no adjacent variable letters such that the combinatorial line $\bigl\{\, w(s) : s \in \{1, 2, \ldots, k\} \,\bigr\}$ is monochromatic. 
  Without loss of generality we may suppose that $\varphi\bigl( w(s) \bigr) = 1$ for all $s \in \{1, 2, \ldots, k\}$.
  Let $w(\star) = (x_1, x_2, \ldots, x_n)$ where each $x_i \in \{1, 2, \ldots, k\} \cup \{\star\}$ and some $x_i = \star$.
  Let $r$ be the number of variable letters in $w(\star)$.
  (For instance, if $k = 3$, $n = 12$, and $w(\star) = (1, \star, 1, 4, \star, 2, \star, 3, 3, 1, \star, 2)$, then $r = 4$.)
  Pick $L \in \calI_{r+1}$ and $M \in \calI_r$ such that for each $i \in \{1, 2, \ldots, r\}$ we have $\max L(i) < \min M(i)$, $\max M(i) < \min L(i+1)$, and 
  \begin{align*}
    \bigcup_{i=1}^{r+1} L(i) &= \bigl\{\, i \in \{1, 2, \ldots, n\} : x_i \in \{1, 2, \ldots, k\} \,\bigr\}, \\
    \bigcup_{i=1}^r M(i) &= \bigl\{\, i \in \{1, 2, \ldots, n\} : x_i = \star \,\}.
  \end{align*}
  (For instance, if $k = 3$, $n = 12$, and $w(\star) = (1, \star, 1, 4, \star, 2, \star, 3, 3, 1, \star, 2)$, then $L(1) = \{1\}$, $M(1) = \{2\}$, $L(2) = \{3, 4\}$, $M(2) = \{5\}$, $L(3) = \{6\}$, $M(3) = \{7\}$, $L(4) = \{8, 9, 10\}$, $M(4) = \{11\}$, and $L(12) = \{12\}$.)
  We now define $c \in S^{r+1}$ as follows:
  \[
    c(j) =
    \begin{cases}
      \prod_{t \in L(1)} \bigl( f_{x_i}(nt+i)b \bigr) & \mbox{if $j = 1$,} \\
      b\prod_{t \in L(j)} \bigl( f_{x_i}(nt+i)b \bigr) & \mbox{if $j \in \{2, 3, \ldots, r+1\}$.}
    \end{cases}
  \]
  And, define $u \in \calJ_r$ by $u_i = $
  
  Since for each $s \in \{1, 2, \ldots, k\}$ have $\varphi\bigl( w(s) \bigr) = 1$, we have that $x(m, a, t, g_{w(s)}) \in A_1$ for each $s \in \{1, 2, \ldots, k\}$.
\end{proof}


\section{Central Sets Theorem}

% Endnotes 
%\theendnotes

% Things referenced in the preliminaries chapter. Eventually this will
% placed in a separate file so the References appear at the end.
\bibliographystyle{amsplain}
\bibliography{../references}

\end{document}