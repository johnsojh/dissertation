\documentclass[12pt]{article}

\usepackage{amsthm, amssymb, amsmath}
\usepackage{color}
\usepackage{endnotes}
\usepackage{url}


\usepackage[margin=1in]{geometry}

\usepackage[doublespacing]{setspace}
\usepackage{url}

\newtheoremstyle{plain}{3mm}{3mm}{\slshape}{}{\bfseries}{.}{.5em}{}
\theoremstyle{plain}

% Numbered theorems 
\newtheorem{thm}{Theorem}[section]
\newtheorem{lem}[thm]{Lemma}
\newtheorem{prop}[thm]{Proposition}
\newtheorem{cor}[thm]{Corollary}
\newtheorem{up}[thm]{Ultrafilter Principle}


% Unnumbered named theorems or results
\newtheorem*{fact}{Fact}
\newtheorem*{hj}{Hales-Jewett Theorem}
\newtheorem*{php}{Pigeonhole Principle}
\newtheorem*{ramsey}{Ramsey's Theorem}
\newtheorem*{vdw}{Van der Waerden's Theorem}
\newtheorem*{schur}{Schur's Theorem}

\newtheorem{FST}[thm]{Hindman's Theorem}
\newtheorem{MBR}[thm]{Multiple Birkhoff Recurrence Theorem}
\newtheorem{recur}[thm]{Recurrence Theorem}
\newtheorem{OCST}[thm]{Furstenburg's Original Central Sets Theorem}
\newtheorem{cst}[thm]{Central Sets Theorem}



\newtheorem{claim}[thm]{Claim}
\newtheorem{ques}[thm]{Question}
\newtheorem{conj}[thm]{Conjecture}


\theoremstyle{definition}

% Numbered "definition" style theorem environments
\newtheorem{defn}[thm]{Definition}
\newtheorem{rmk}[thm]{Remark}
\newtheorem{example}[thm]{Example}

\newcommand{\la}{\langle}
\newcommand{\ra}{\rangle}
\newcommand{\bbN}{\mathbb{N}}
\newcommand{\bbZ}{\mathbb{Z}}
\newcommand{\bbR}{\mathbb{R}}
\newcommand{\AP}{\mathcal{AP}}
\newcommand{\AL}{\mathcal{AL}}

% Short names for calligraphic math letters.
\newcommand{\calA}{\mathcal{A}}
\newcommand{\calB}{\mathcal{B}}
\newcommand{\calC}{\mathcal{C}}
\newcommand{\calE}{\mathcal{E}}
\newcommand{\calF}{\mathcal{F}}
\newcommand{\calG}{\mathcal{G}}
\newcommand{\calH}{\mathcal{H}}
\newcommand{\calI}{\mathcal{I}}
\newcommand{\calJ}{\mathcal{J}}
\newcommand{\calP}{\mathcal{P}}
\newcommand{\calR}{\mathcal{R}}
\newcommand{\calS}{\mathcal{S}}
\newcommand{\calT}{\mathcal{T}}
\newcommand{\calU}{\mathcal{U}}

\newcommand{\Pf}{\mathcal{P}_f}

\newcommand{\setfunc}[2]{\hbox{${}^{\hbox{$#1$}}\hskip -1 pt #2$}}

\font\bigmath=cmsy10 scaled \magstep 3
\newcommand{\bigtimes}{\hbox{\bigmath \char'2}}

\newcommand{\cchi}{\raise 2 pt \hbox{$\chi$}}

\begin{document}
\section{Ultrafilters and Partition Regularity}
Cardinality is commonly thought of as a ``measure'' on the size of sets.
Intuitively this ``notion of largeness'' provides a good mathematical formalization of size since every set has a unique cardinal number; and under ZFC (the usual Zermelo-Fraenkel axioms of set theory along with the axiom of choice) there is a fixed nontrivial order relation such that every set of cardinals is wellordered.
However the concept of cardinality has its own mathematical peculiarities as a notion of largeness.
One peculiarity is starkly illustrated by the formal independence of the Continuum Hypothesis from ZFC (provided ZFC is a consistent theory).
Recall that the Continuum Hypothesis is the assertion that $|\bbR|$ is the first cardinal after $|\bbN|$.
The independence of this statement means that, without adding extra set-theoretical axioms, we cannot determine the precise location of the cardinal $|\bbR|$ in the class of all cardinals.

Of course by using the well-known diagonal argument of Cantor%
\endnote{
  Interestingly Grattan-Guinness's observation in \cite[page 134, footnote 1]{Grattan-Guinness:1978kx} implies that Paul de Bois-Reymond was the first to publish a diagonal-type argument. 
}
we can easily prove the weaker assertion that $|\bbR|$ is strictly greater than $|\bbN|$.
The point we wish to emphasize is that questions about relative sizes are often easier to study than questions about absolute sizes.
Unfortunately, another peculiarity of cardinality is that the concept is a somewhat blunt tool to use in the study of relative sizes.
Therefore in this section we introduce two different (but ultimately related) notions of largeness that are more amendable to answering certain interesting questions on relative sizes.

\subsection{Filters, Filter bases, Filter subbases, and Ultrafilters}
Since the material in this section is mostly standard and can be found in several places we choose to leave most of the proofs of our assertions to the reader.
% Add some references to the results we state.
\begin{defn}
  Let $X$ be a nonempty set.
  \begin{itemize}
    \item[(a)] We call $\calF \subseteq \calP(X)$ a \textsl{filter on $X$} if and only if $\calF$ satisfies the following three conditions:
    \begin{itemize}
      \item[(1)] $\emptyset \ne \calF$ and $\emptyset \not\in\calF$.
      \item[(2)] If $A$ and $B$ are elements of $\calF$, then $A
        \cap B \in \calF$.
      \item[(3)] If $A \in \calF$ and $A \subseteq B \subseteq X$,
        then $B \in \calF$.
    \end{itemize}
  \end{itemize}
\end{defn}

% Endnotes 
\theendnotes

% Things referenced in the preliminaries chapter. Eventually this will
% placed in a separate file so the References appear at the end.
\bibliographystyle{amsplain}
\bibliography{../references}

\end{document}