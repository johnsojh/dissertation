\documentclass[12pt]{article}

\usepackage{amsthm, amssymb, amsmath}
\usepackage{endnotes}
\usepackage{todonotes}
% \usepackage[doublespacing]{setspace}
\usepackage{url}

\newtheoremstyle{plain}{3mm}{3mm}{\slshape}{}{\bfseries}{.}{.5em}{}
\theoremstyle{plain}

% Unnumbered, named theorems.
\newtheorem*{php}{Pigeonhole Principle}
\newtheorem*{schur}{Schur's Theorem}
\newtheorem*{vdw}{Van der Waerden's Theorem}
\newtheorem*{ramsey}{Ramsey's Theorem}
\newtheorem*{ffst}{Rado-Sanders-Folkman Theorem}
\newtheorem*{fst}{Hindman's Theorem}

\newtheorem{thm}{Theorem}[section]


\begin{document}
\section{Partition Regularity and Ultrafilters}

% Angle brackets for sequences and families
\newcommand{\la}{\langle}
\newcommand{\ra}{\rangle}

% Blackboard bold letter
\newcommand{\bbN}{\mathbb{N}}

\subsection{Partition Regularity}
Suppose we are given an infinite set and two colors.
A moments thought shows that no matter how we color members of our
infinite set, we can always pick an infinite subset all of whose
members share the same color.
This is nothing more than a version of the pigeonhole principle, and a
bit more thought easily proves more.

\begin{php}
  Let $X$ be an infinite set and $r \in \bbN$.
  If $X = \bigcup_{i=1}^r C_i$, then there exist an infinite subset $Y
  \subseteq X$ and $i \in \{1, 2, \ldots, r\}$ such that $Y \subseteq
  C_i$.
\end{php}

Now suppose that instead of coloring our infinite set with two colors,
we color all of the two element subsets of our infinite set. 
Can we find an infinite subset, of our original infinite set, such
that all of the two element subsets of this infinite subset are
monochromatic?
The answer to this question turns out to be yes, but this time the
proof is a bit harder.

\begin{thm}
  Let $X$ be an infinite set and $r \in \bbN$.
  If $\{\, A \subseteq X : |A| = 2 \,\} = C_1 \cup C_2$, then there
  exist an infinite subset $Y \subseteq X$ and $i \in \{1, 2\}$ such
  that $\{\, B \subseteq Y : |B| = 2 \,\} \subseteq C_i$.
\end{thm}
\begin{proof}
  We will inductively pick sequences $\la X_i \ra_{i=1}^\infty$, $\la
  x_i \ra_{i=1}^\infty$, $\la C_1^{(i)} \ra_{i=1}^\infty$, $\la
  C_2^{(i)} \ra_{i=1}^\infty$, and $\la r_i \ra_{i=1}^\infty$
  satisfying the following hypotheses:
  \begin{itemize}
    \item[(1)] For all $i \in \bbN \setminus \{1\}$, $x_i \in X_{i-1}$.
      
    \item[(2)] For all $i \in \bbN \setminus \{1\}$, $X_i \subseteq
      X_{i-1} \setminus \{x_i\}$ is infinite, and $X_1 \subseteq X$ is
      infinite.
     
    \item[(3)] For all $j \in \{1, 2\}$ and $j \in \bbN \setminus
      \{1\}$, $C_j^{(i)} = \{\, x \in X_{i-1} \setminus \{x_i\} : \{x,
      x_i\} \in C_i \,\}$.
     
    \item[(4)] For all $i \in \bbN$, $X_i \subseteq C_{r_i}^{(i)}$.
  \end{itemize}
  Let $x_1 \in X$ and for each $j \in \{1, 2\}$ define $C^{(1)}_j =
  \bigl\{\, x \in X \setminus \{x_1\} : \{x, x_1\} \in C_j \,\bigr\}$.
  By the pigeonhole principle, pick an infinite subset $X_1 \subseteq
  X \setminus \{x_1\}$ and $r_1 \in \{1, 2\}$ such that $X_1 \subseteq
  C_{r_1}^{(1)}$. 
  Let $k \in \bbN \setminus \{1\}$ and assume we have chosen sequences
  $\la X_i \ra_{i=1}^{k-1}$, $\la x_i \ra_{i=1}^{k-1}$, $\la C_1^{(i)}
  \ra_{i=1}^{k-1}$, $\la C_2^{(i)} \ra_{i=1}^{k-1}$, and $\la r_i \ra_{i=1}^{k-1}$
  satisfying hypotheses (1)--(4). 
  Let $x_k \in X_{k-1}$ and for each $j \in \{1, 2\}$ define
  $C_j^{(k)} = \{\, x \in X_{k-1} \setminus \{x_k\} : \{x, x_k\} \in
  C_j \,\}$. 
  Since $X_{k-1}\setminus \{x_k\}$ is infinite and $X_{k-1}\setminus
  \{x_k\} = C_1^{(k)} \cup C_2^{(k)}$, by the pigeonhole principle, we
  can pick an infinite subset $X_k \subseteq X_{k-1} \setminus
  \{x_k\}$ and $r_k \in \{1, 2\}$ such that $X_k \subseteq
  C_{r_k}^{(k)}$.
  
  Our hypothesis (1) -- (4) are satisfied. 
  Pick an infinite subsequence $\la i_k \ra_{k=1}^\infty$ such that
  $r_{i_1} = r_{i=2} = r_{i=3} = \cdots$. 
  (This can be done by the pigeonhole principle.)
  Put $Y = \{\, x_{i_k} : k \in \bbN$ and $i = r_{i_1}$, we claim that
  $\{\, B \subseteq Y : |B| = 2 \,\} \subseteq C_i$.
  Let $k$, $\ell \in \bbN$ and suppose $k < \ell$. 
  Then $X_{\ell} \subseteq X_k$, by hypothesis (2).
  Now $X$ ...
\end{proof}
\end{document}