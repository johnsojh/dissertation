\documentclass[12pt]{article}

\usepackage{amsthm, amssymb, amsmath}
\usepackage{endnotes}
\usepackage{todonotes}
% \usepackage[doublespacing]{setspace}
\usepackage{url}

\newtheoremstyle{plain}{3mm}{3mm}{\slshape}{}{\bfseries}{.}{.5em}{}
\theoremstyle{plain}

% Unnumbered, named theorems.
\newtheorem*{php}{Pigeonhole Principle}
\newtheorem*{schur}{Schur's Theorem}
\newtheorem*{vdw}{Van der Waerden's Theorem}
\newtheorem*{ramsey}{Ramsey's Theorem}
\newtheorem*{ffst}{Rado-Sanders-Folkman Theorem}
\newtheorem*{fst}{Hindman's Theorem}

% Numbered theorems
\newtheorem{thm}{Theorem}[section]
\newtheorem{cor}[thm]{Corollary}


\begin{document}
\section{Partition Regularity and Ultrafilters}

% Angle brackets for sequences and families
\newcommand{\la}{\langle}
\newcommand{\ra}{\rangle}

% Blackboard bold letter
\newcommand{\bbN}{\mathbb{N}}

\subsection{Partition Regularity}
Suppose we are given an infinite set and two colors.
A moments thought shows that no matter how we color members of our
infinite set, we can always pick an infinite subset all of whose
members share the same color.
This is nothing more than a version of the pigeonhole principle, and a
bit more thought easily proves more.

\begin{php}
  Let $X$ be an infinite set and $r \in \bbN$.
  If $X = \bigcup_{i=1}^r C_i$, then there exist an infinite subset $Y
  \subseteq X$ and $i \in \{1, 2, \ldots, r\}$ such that $Y \subseteq
  C_i$.
\end{php}

Now suppose that instead of coloring our infinite set with two colors,
we color all of the two element subsets of our infinite set. 
Can we find an infinite subset, of our original infinite set, such
that all of the two element subsets of this infinite subset are
monochromatic?
The answer to this question turns out to be yes, but the proof of this
result will use several applications of the pigeonhole principle and a
diagonal type argument.

\begin{thm}
  \label{thm:ramsey2Col}
  Let $X$ be an infinite set and $r \in \bbN$.
  If $\{\, A \subseteq X : |A| = 2 \,\} = C_1 \cup C_2$, then there
  exist an infinite subset $Y \subseteq X$ and $i \in \{1, 2\}$ such
  that $\{\, B \subseteq Y : |B| = 2 \,\} \subseteq C_i$.
\end{thm}
\begin{proof}
  We will recursively construct sequences $\la X_i \ra_{i=0}^\infty$,
  $\la x_i \ra_{i=1}^\infty$, $\la (C_1^{(i)}, C_2^{(i)})
  \ra_{i=1}^\infty$ and $\la r_i \ra_{i=1}^\infty$ satisfying the
  following hypotheses:
  \begin{itemize}
    \item[(1)] $x_i \in X_{i-1}$ for all $i \in \bbN$.
      
    \item[(2)] For $i \in \bbN$, $X_i$ is an infinite subset of
      $X_{i-1} \setminus \{x_i\}$.

    \item[(3)] For all $j \in \{1, 2\}$ and $i \in \bbN$, $C_j^{(i)} =
      \bigl\{\, x \in X_{i-1} \setminus \{x_i\} : \{x, x_i\} \in C_j
      \,\bigr\}$.

    \item[(4)] For all and $i \in \bbN$, $X_i
      \subseteq C_{r_i}^{(i)}$.
  \end{itemize}
  
  Put $X_0 = X$ and $x_1 \in X_0$.
  For each $j \in \{1,2\}$, define
  \[
    C_j^{(1)} = \bigl\{\, x \in X_0 \setminus \{x_1\} : \{x, x_1\} \in
    C_j \,\bigr\}.
  \]
  Since $X_0 \setminus \{x_1\} = C_1^{(1)} \cup C_2^{(2)}$ and $X_0
  \setminus \{x_1\}$ is infinite, then by the pigeonhole principle we
  can pick an infinite subset $X_1 \subseteq X_0 \setminus \{x_1\}$
  and $r_1 \in \{1, 2\}$ such that $X_1 \subseteq C_{r_1}^{(1)}$.
  Our sequences $\la X_i \ra_{i=0}^1$, $\la x_i \ra_{i=1}^1$, $\la
  (C_1^{(i)}, C_2^{(i)}) \ra_{i=1}^1$, and $\la r_i \ra_{i=1}^1$
  satisfy hypotheses (1)-(4).

  Let $k > 1$ and assume we have chosen sequences $\la X_i
  \ra_{i=0}^{k-1}$, $\la x_i \ra_{i=1}^{k-1}$, $\la (C_1^{(i)}, C_2^{(i)})
  \ra_{i=1}^{k-1}$, and $\la r_i \ra_{i=1}^{k-1}$ satisfying
  hypotheses (1)-(4).
  By hypothesis (2) we have that $X_{k-1}$ is infinite, so pick $x_k
  \in X_{k-1}$.
  For each $j \in \{1, 2\}$, define
  \[
    C_j^{(k)} = \bigl\{\, x \in X_{k-1} \setminus \{x_k\} : \{x, x_k\} \in
    C_j \,\bigr\}.
  \]
  By the pigeonhole principle, we can pick an infinite subset $X_k
  \subseteq X_{k-1} \setminus \{x_k\}$ and $r_k \in \{1, 2\}$ such
  that $X_k \subseteq C_{r_k}^{(k)}$.
  Our hypotheses (1)-(4) are satisfied by $\la X_i
  \ra_{i=0}^{k}$, $\la x_i \ra_{i=1}^{k}$, $\la (C_1^{(i)}, C_2^{(i)})
  \ra_{i=1}^{k}$, and $\la r_i \ra_{i=1}^{k}$ by construction. 
  Hence our sequence $\la X_i \ra_{i=0}^{\infty}$, $\la x_i
  \ra_{i=1}^{\infty}$, $\la (C_1^{(i)}, C_2^{(i)}) \ra_{i=1}^{\infty}$, and
  $\la r_i \ra_{i=1}^{\infty}$ is recursively defined.

  By the pigeonhole principle, either $\{\, i \in \bbN : r_i = 1 \,\}$
  is infinite or $\{\, i \in \bbN : r_i = 2 \,\}$ is infinite.
  Without loss of generality suppose that $\{\, i \in \bbN : r_i =
  1\,\}$ is infinite.
  Put $Y = \{\, x_i : r_i = 1 \,\}$.
  We show that $Y$ is the infinite subset of $X$ we are looking for,
  that is, we show that $\{\, B \subseteq Y : |B| = 2 \,\} \subseteq
  C_1$.  
  Let $i$, $j \in \{\, i \in \bbN : r_i = 1 \,\}$ and assume that $i <
  j$.
  Since $i \le j-1$ and by hypothesis (2), it follows that $X_j
  \subseteq X_{i} \setminus \{x_{i+1}\}$. 
  In particular, $x_j \in X_i \setminus \{x_{i+1}\}$.
  By hypothesis (4), $X_i \subseteq C_{r_i}^{(i)}$, that is, for all
  $x \in X_i$, $\{x, x_i\} \in C_{r_i}$. 
  (This follows from hypothesis (3).)
  Moreover, $\{x_j, x_i\} \in C_{r_i}$. 
  However $r_i = r_j = 1$ by our choice of $i$ and $j$.
  This completes the proof of this theorem.
\end{proof}
\begin{cor}
  \label{cor:ramseyrCol}
  Let $X$ be an infinite set and $r \in \bbN$.
  If $\{\, A \subseteq X : |A| = 2 \,\} = \bigcup_{i=1}^r C_i$, then
  there exist an infinite subset $Y \subseteq X$ and $i \in \{1, 2,
  \ldots, r\}$ such that $\{\, B \subseteq Y : |B| = 2 \,\} \subseteq
  C_i$.
\end{cor}
\begin{proof}
  We proceed by induction on $r$.
  For $r = 1$, the result is obvious.
  Let $r > 1$ and assume that our result is true for $r-1$.
  Let $\{\, A \subseteq X : |A| = 2 \,\} = \bigcup_{i=1}^r C_i$.
  For $i \in \{1, 2, \ldots, r-2\}$ put $D_i = C_i$, and put $D_{r-1}
  = C_{r-1} \cup C_r$.
  Then $\{\, A \subseteq X : |A| = 2 \,\} = \bigcup_{i=1}^{r-1} D_i$. 
  
  By our induction hypothesis, pick an infinite subset $Y \subseteq
  X$ and  $i \in \{1, 2, \ldots, r-1\}$ such that $\{\, B \subseteq Y
  : |B| = 2 \,\} \subseteq D_i$.
  If $i \in \{1, 2, \ldots, r-2\}$, then we're done.
  If $i = r-1$, then our result follows from Theorem
  \ref{thm:ramsey2Col}.
\end{proof}

What we have given is only a special case of a much more general
theorem due to Ramsey in \cite[Theorem A]{Ramsey:1930uq}.
To state his theorem concisely we introduce the following notation: If
$X$ is a set and $k \in \bbN$, define $[X]^k = \{\, A \subseteq X :
|A| = k \,\}$.

\begin{ramsey}
  Let $X$ be an infinite set and $r$, $k \in \bbN$.
  If $[X]^k = \bigcup_{i=1}^r C_i$, then there exist an infinite subset
  $Y \subseteq X$ and $i \in \{1, 2, \ldots, r\}$ such that $[Y]^k
  \subseteq C_i$.
\end{ramsey}
\begin{proof}[Proof sketch]
  We have essentially done most of the hard-work in proving Theorem
  \ref{thm:ramsey2Col}. 
  First prove the theorem for $r = 2$ and by induction on $k$.
  Once this is done, proceed as in Corollary \ref{cor:ramseyrCol} to
  prove the result for all $r \in \bbN$.

  A full combinatorial proof can be found in Ramsey's original paper
  \cite[Theorem A]{Ramsey:1930uq}. Another combinatorial proof is
  given in Hindman's survey paper \cite[Theorem ??]{Hindman:1979fk}.
  A proof using a nonprincipal ultrafilter can be found in
  \cite[Theorem 18.2]{Hindman:1998fk}.

\end{proof}

% Things referenced in the preliminaries chapter. Eventually this will
% placed in a separate file so the References appear at the end.
\bibliographystyle{amsplain}
\bibliography{../references}

\end{document}