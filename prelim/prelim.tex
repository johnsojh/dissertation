\documentclass[12pt]{article}

\usepackage{amsthm, amssymb, amsmath}
\usepackage{color}
\usepackage{endnotes}
\usepackage{url}
% Uncomment the following line if you want to actually print this
% document. 
% \usepackage[margin=1in]{geometry}

% The following bit of code is taken from
% http://jcl.posterous.com/latex-todonotes-and-margins to enable todo
% notes to take up the large right margins.
\usepackage[paperwidth=275.9mm, paperheight=279.4mm]{geometry}
% regular letter size is 215.9 wide by 279.44 long  
\setlength{\oddsidemargin}{35mm}  
\setlength{\evensidemargin}{35mm}  
\setlength{\voffset}{-1in}  
\setlength{\hoffset}{-1in}  
\setlength{\textwidth}{156mm}  
\setlength{\topmargin}{4mm}  
\setlength{\headheight}{10mm}  
\setlength{\headsep}{12mm}  
\setlength{\topskip}{0mm}  
\setlength{\textheight}{228mm}  


\setlength{\evensidemargin}{95mm}

\usepackage[colorinlistoftodos, textwidth=65mm, shadow]{todonotes}  
% End block taken from site

\usepackage[doublespacing]{setspace}
\usepackage{url}

\newtheoremstyle{plain}{3mm}{3mm}{\slshape}{}{\bfseries}{.}{.5em}{}
\theoremstyle{plain}
\newtheorem{thm}{Theorem}[section]
\newtheorem{CSTv1}{Original Central Sets Theorem}

% Unnumbered named theorems or results
\newtheorem*{fact}{Fact}
\newtheorem*{hj}{Hales-Jewett Theorem}
\newtheorem*{php}{Pigeonhole Principle}
\newtheorem*{prop}{Proposition}
\newtheorem*{ramsey}{Ramsey's Theorem}
\newtheorem*{vdw}{Van der Waerden's Theorem}
\newtheorem*{schur}{Schur's Theorem}

\newtheorem{FST}[thm]{Hindman's Theorem}
\newtheorem{MBR}[thm]{Multiple Birkhoff Recurrence Theorem}
\newtheorem{recur}[thm]{Recurrence Theorem}
\newtheorem{OCST}[thm]{Furstenburg's Original Central Sets Theorem}
\newtheorem{cst}[thm]{Central Sets Theorem}
\newtheorem{cor}[thm]{Corollary}

\newtheorem{lem}[thm]{Lemma}
\newtheorem{claim}[thm]{Claim}
\newtheorem{ques}[thm]{Question}
\newtheorem{conj}[thm]{Conjecture}


\theoremstyle{definition}
% Numbered "definition" style theorem environments
\newtheorem{defn}[thm]{Definition}
\newtheorem{rmk}[thm]{Remark}
\newtheorem{example}[thm]{Example}

\newcommand{\la}{\langle}
\newcommand{\ra}{\rangle}
\newcommand{\bbN}{\mathbb{N}}
\newcommand{\bbZ}{\mathbb{Z}}
\newcommand{\bbR}{\mathbb{R}}
\newcommand{\AP}{\mathcal{AP}}
\newcommand{\AL}{\mathcal{AL}}

% Short names for calligraphic math letters.
\newcommand{\calB}{\mathcal{B}}
\newcommand{\calF}{\mathcal{F}}
\newcommand{\calG}{\mathcal{G}}
\newcommand{\calI}{\mathcal{I}}
\newcommand{\calJ}{\mathcal{J}}
\newcommand{\calP}{\mathcal{P}}
\newcommand{\calR}{\mathcal{R}}
\newcommand{\calT}{\mathcal{T}}
\newcommand{\calU}{\mathcal{U}}

\newcommand{\Pf}{\mathcal{P}_f}

\newcommand{\setfunc}[2]{\hbox{${}^{\hbox{$#1$}}\hskip -1 pt #2$}}

\font\bigmath=cmsy10 scaled \magstep 3
\newcommand{\bigtimes}{\hbox{\bigmath \char'2}}

\newcommand{\cchi}{\raise 2 pt \hbox{$\chi$}}


\begin{document}
\section{Partition Regularity and Ultrafilters}

\subsection{Ultrafilters}
Cardinality is commonly thought of as a ``measure'' on 
the size a set.
Intuitively this ``notion of largeness'' provides a good mathematical
formalization of size since every set has exactly one cardinality;
and under ZFC (that is, the usual Zermelo-Frankel axioms of set
theory along with the axiom of choice), every set of cardinals is
wellordered.
However the concept of cardinality has its own mathematical
peculiarities starkly illustrated by the formal independence of the
Continuum Hypothesis from ZFC (provided ZFC is consistent). 
Recall that the Continuum Hypothesis is the affirmative answer to the
question: ``Is $|\bbR|$ the first cardinal after $|\bbN|$?''
Its independence means that, without adding extra set-theoretical axioms,
we cannot determine the precise size of $|\bbR|$.%
\endnote{
  For the reader interested in more information about the Continuum
  Hypothesis can consult the articles  \cite{Woodin:2001uq} and
  \cite{Woodin:2001kx} by Woodin. 
  Also check out the interesting answers to several questions
  \cite{Gindi:2010vn} ask on \url{http://mathoverflow.net/}.%
  \todo{
    Figure out a proper way to include bibliographic entries for
    websites.
  }
} 

\todo{Add endnote reference to Grattan-Guinness paper ``How Betrand
  Russell Discovered His Paradox'' for note on how Cantor is not the
  first documented case of the use of a diagonal argument. 
  Also mention Woodin's papers ``The Continuum Hypothesis'' Parts I
  and II and several Mathoverflow questions on the current status of
  CH.
}%
Of course, if we weaken the Continuum Hypothesis to the question, ``Is
$|\bbR|$ bigger than $|\bbN|$?'', then the answer is, under ZFC, yes and
follows from the famous diagonal argument of Cantor. 
To point we wish to emphasize is that questions about relative sizes are
often easier to work with than questions about absolute sizes. 
Hence we with to focus on studying collection of sets that are
relatively large with respect to some base set.
\todo{
  Insert brief history about filters and references to Cartan's paper,
  Borubaki, and the \textsl{Handbook on the Foundations of Analysis}
  book.
}%
The first mathematical formulation of largeness that we will look at is
the notion of a filter. 

\begin{defn}
  Let $X$ be a nonempty set and $\calF \subseteq \calP(X)$. 
  We call $\calF$ a \textsl{filter on $X$} if and only if $\calF$
  satisfies the following three conditions:
  \begin{itemize}
    \item[(1)] $\emptyset \not\in \calF$ and $\emptyset \ne \calF$.

    \item[(2)] If $A$ and $B$ are elements of $\calF$, then $A \cap B
      \in \calF$.

    \item[(3)] If $A \in \calF$ and $A \subseteq B \subseteq X$, then
      $B \in \calF$. 
  \end{itemize}
\end{defn}

Intuitively, we may think of elements of a filter $\calF$ on a set $X$
as a collection of ``large'' subsets of $X$. 
From this point-of-view, conditions (1) and (3) line up nicely with
our intuition and implies that $X \in \calF$. 
(We normally won't consider the empty set as large and surely any set
which contains a large set must also be considered large.)
However from this intuitive point-of-view of filters condition (2)
doesn't necessary align cleaning with our intuition. 
This particular condition may be thought of as saying that we further
require our large sets to interlock in a somewhat rigid fashion in
that the intersection of two large sets is also large. 
We shall see that this condition (2) is the reason for much of the
interesting and useful mathematical uses of filters. 
\missingfigure{
Make a figure of some filters on the set $\{1, 2, 3, 4\}$.
This figure should look similar to the one on wikipedia. 
}


\subsection{Partition Regularity}
The cardinality of a set is commonly thought of as a ``measure'' on
how large a particular set is.
Intuitively this ``notion of largeness'' provides a good mathematical
formalization of the size of sets since every set has exactly one
cardinality, and under ZFC, every cardinality is comparable.%
\endnote{
  However despite the success of the notion of cardinality, one
  possible disturbing aspect is the formal independence of the
  Continuum Hypothesis from the axioms of ZFC.
  This particular independence roughly asserts that, without adding
  additional axioms to ZFC, we are unable to determine the exact
  size of $|\bbR|$.
  (The well-known diagonal argument enables us to determine the
  relative size of $|\bbR|$ with respect to $|\bbN|$.)
}
In this section we will introduce another ``notion of largeness'' on a
set that is of a different character from cardinality. 

\begin{defn}
  Let $X$ be a nonempty set and $\calR$ a collection of sets with
  $\emptyset\not\in \calR$. 
  \begin{itemize}
    \item[(a)] Let $r$ be a natural number. 
      The pair $(X, \calR)$ is called \textsl{\mbox{$r$-partition}
        regular} if and only if whenever $X = \bigcup_{i=1}^r
      C_i$, there exist $i \in \{1, 2, \ldots, r\}$ and $A \in \calR$
      such that $A \subseteq C_i$.

    \item[(b)] We call the pair $(X, \calR)$ \textsl{partition
        regular} if and only if $(X, \calR)$ is \mbox{$r$-partition}
      regular for every $r \in \bbN$.
  \end{itemize}
\end{defn}

% Say that we will only focus on partition regular statements and that
% for κ>=ω is the domain for "infinite combinatorics.  Also mention
% how this is used in modern set theory to investigate several notions
% of large cardinals.

Partition regularity roughly asserts that some property of $X$, here
represented by members of $\calR$, occurs a ``large'' number of times
in $X$.
In fact so large, that no matter how we finitely divide up $X$, at
least one cell in the division has our specified property. 
One of the easiest partition regular property to observe is the
infinite form of the pigeonhole principle.

\begin{php}
  Let $X$ be an infinite put $\calR = \{\, A \subseteq X : \mbox{$A$
    is infinite} \,\}$.
  Then $(X, \calR)$ is partition regular.
\end{php}
\begin{proof}
  Let $r \in \bbN$ and $X = \bigcup_{i=1}^r C_i$.
  Then $|X| \le \sum_{i=1}^r |C_i|$.
  If $C_i$ is finite for each $i \in \{1, 2, \ldots, r\}$,
  then $|X|$ is finite too, a contradiction.
\end{proof}

The Pigeonhole Principle is an easy result to prove. 
However, a powerful generalization of this principle was discovered by
Ramsey in \cite[Theorem A]{Ramsey:1930uq}. 

\begin{ramsey}
  Let $X$ be an infinite set, and let $k \in \bbN$.
  Put $\calR = \bigl\{\, \{\, A \subseteq Y : |A| = k \,\} : \mbox{$Y
    \subseteq X$ is infinite} \,\bigr\}$.
  Then $\bigl(\{\, A \subseteq X : |A| = k \,\}, \calR\bigr)$ is
  partition regular. 
\end{ramsey}
\begin{proof}[Proof Sketch]
  Observe that when $k = 1$ the result follows from the Pigeonhole
  Principle. 
  We now suppose that $k=2$ and we show that $\bigl(\{\, A
  \subseteq X : |A| = 2 \,\}, \calR\bigr)$ is \mbox{2-partition}
  regular: that is, we show that if $\{\, A \subseteq X : |A| = 2 \,\}
  = C_1 \cup C_2$, then there exist an infinite subset $Y \subseteq
  X$ and $i \in \{1, 2\}$ such that $\{\, B \subseteq Y : |B| = 2 \,\}
  \subseteq C_i$. 

  To this end, we will recursively construct sequences $\la X_n
  \ra_{n=0}^\infty$, $\la x_n \ra_{n=1}^\infty$, $\bigl\la (C_1^{(n)},
  C_2^{(n)}) \bigr\ra_{n=1}^\infty$, and $\la r_n \ra_{n=1}^\infty$
  satisfying the following hypotheses:
  \begin{itemize}
    \item[(0)] $X_0 = X$.
    \item[(1)] $x_n \in X_{n-1}$ for all $n \in \bbN$.
    \item[(2)] $X_n$ is an infinite subset of $X_{n-1} \setminus
      \{x_n\}$ for each $n \in \bbN$.
    \item[(3)] For all $i \in \{1, 2\}$ and $n \in \bbN$, 
      \[
        C_i^{(n)} = \{\, x \in X_{n-1} \setminus \{x_n\} : \{x, x_n\}
        \in C_i \,\}.
      \]
    \item[(4)] $X_n \subseteq C_{r_n}^{(n)}$ for all $n \in \bbN$. 
  \end{itemize}
  We first show how these sequences prove our special case of Ramsey's
  Theorem. 
  By hypotheses (0), (1), and (2), the set $\la x_n \ra_{n=1}^\infty$
  is a one-to-one sequence in $X$; and by the Pigeonhole Principle,
  either $\{\, n \in \bbN : r_n = 1 \,\}$ is infinite or $\{\, n \in
  \bbN : r_n = 2 \,\}$ is infinite. 
  Without loss of generality suppose $\{\, n \in \bbN : r_n = 1 \,\}$
  is infinite and put $Y = \{\, x_n : \mbox{$n \in \bbN$ and $r_n =
    1$} \,\}$.
  Then $Y \subseteq X$ is infinite and we claim that $\{\, A \subseteq
  Y : |A| = 2\,\} \subseteq C_1$.
  To see this claim, let $m$ and $n$ be natural numbers with $m < n$
  and $r_m = r_n = 1$. 
  By hypothesis (2), it follows that $X_{n-1} \subseteq X_m$. 
  Hence $x_n \in X_m$ since $x_n \in X_{n-1}$. 
  By hypotheses (3) and (4), it follows that $\{x_n , x_m \} \in
  C_1$. 
  This finishes the proof of our claim.

  We now recursively construct our sequences to satisfy hypotheses
  (0)-(4). 
  Put $X_0 = X$ let $k > 1$ and assume we have constructed sequences
  $\la X_n \ra_{n=0}^{k-1}$, $\la x_n \ra_{n=1}^{k-1}$, $\bigl\la
  (C_1^{(n)}, C_2^{(n)}) \bigr\ra_{n=1}^{k-1}$, and $\la r_n
  \ra_{n=1}^{k-1}$ satisfying hypotheses (0)-(4).
  By hypothesis (2), $X_{k-1}$ is infinite so pick $x_k \in X_{k-1}$.  
  For each $i \in \{1, 2\}$ define
  \[
    C_i^{(k)} = \bigl\{\, x \in X_{k-1} \setminus \{x_k\} : \{x, x_k\} \in
    C_i \,\bigr\}.
  \]
  Then $X_{k-1} \setminus \{x_k\}= C_1^{(k)} \cup C_2^{(k)}$ and so by the Pigeonhole
  Principle, we can pick an infinite subset $X_k \subseteq X_{k-1}
  \setminus \{x_k\}$ and $r_k \in \{1, 2\}$ such that $X_k \subseteq
  C_{r_k}^{(k)}$. 
  By construction the sequences $\la X_n \ra_{n=0}^k$, $\la x_n
  \ra_{n=1}^k$, $\bigl\la (C_1^{(n)}, C_2^{(n)})
  \bigr\ra_{n=1}^k$, and $\la r_n \ra_{n=1}^k$ satisfying
  hypotheses (0)-(4).
  This completes our recursive construction.

  To show that $\bigl(\{\, A \subseteq X : |A| = 2 \,\}, \calR\bigr)$
  is \mbox{$r$-partition} regular follows by induction on $r$. 
  Finally to show that for all $k \in \bbN$, $\bigl(\{\, A \subseteq X
  : |A| = k \,\}, \calR\bigr)$ is partition regular follows by
  induction on $k$.
\end{proof}
\begin{rmk}
  The current proof sketch is essentially Ramsey's original
  combinatorial proof is \cite[Theorem A]{Ramsey:1930uq}. 
  A complete combinatorial proof, which is where the above is modeled
  from, is in Hindman's survey paper \cite[Theorem
  \textcolor{red}{??}]{Hindman:1979fk}.
\end{rmk}

Notice that the above argument is simliar to a diagonal
type-argument. 
We will show that in fact that there is a close connection between a
partition regular set $(X, \calR)$ an a ultrafilter on $X$. 

% While Ramsey's Theorem is important and provides a good example of a
% partition regular pair $(X, \calR)$, in this dissertation we will
% mainly be considering partition regular sets where the underlying set
% $X$ has an algebraic structure of a semigroup and elements in $\calR$
% are defined in terms of this structure. 

\begin{schur}
  Let $r \in \bbN$ and $\bbN = \bigcup_{i=1}^r C_i$.
  Then there exist $i \in \{1, 2, \ldots, r\}$ and $x$, $y$, and $z
  \in \bbN$ such that $\{\, x, y, x+y \,\} \subseteq C_i$.
\end{schur}

\begin{vdw}
  Let $r \in \bbN$ and $\bbN = \bigcup_{i=1}^r C_i$.
  Then for every $\ell \in \bbN$, there exist $i \in \{1, 2, \ldots,
  r\}$ and $a$, $d \in \bbN$ such that $\{\, a, a+d, \ldots, a+\ell d
  \,\} \subseteq C_i$.
\end{vdw}

\subsection{Ultrafilters}
\begin{defn}
  Let $X$ be a nonempty set and $\calF \subseteq \calP(X)$.
  We call $\calF$ a \textsl{filter (on X)} if and only if it satisfies
  the following three conditions:
  \begin{itemize}
    \item[(1)] $\emptyset \ne \calF$ and $\emptyset \not\in \calF$.
    \item[(2)] If $A$, $B \in \calF$, then $A \cap B \in \calF$.
    \item[(3)] If $A \in \calF$ and $A \subseteq B \subseteq X$, then
      $B \in \calF$.
  \end{itemize}
\end{defn}

% Endnotes 
\theendnotes

% Things referenced in the preliminaries chapter. Eventually this will
% placed in a separate file so the References appear at the end.
\bibliographystyle{amsplain}
\bibliography{../references}

\end{document}