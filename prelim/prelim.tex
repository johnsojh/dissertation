\documentclass[12pt]{article}

\usepackage{amsthm, amssymb, amsmath}
\usepackage{color}
\usepackage{endnotes}
\usepackage{url}
% Uncomment the following line if you want to actually print this
% document. 
% \usepackage[margin=1in]{geometry}

% The following bit of code is taken from
% http://jcl.posterous.com/latex-todonotes-and-margins to enable todo
% notes to take up the large right margins.
\usepackage[paperwidth=275.9mm, paperheight=279.4mm]{geometry}
% regular letter size is 215.9 wide by 279.44 long  
\setlength{\oddsidemargin}{35mm}  
\setlength{\evensidemargin}{35mm}  
\setlength{\voffset}{-1in}  
\setlength{\hoffset}{-1in}  
\setlength{\textwidth}{156mm}  
\setlength{\topmargin}{4mm}  
\setlength{\headheight}{10mm}  
\setlength{\headsep}{12mm}  
\setlength{\topskip}{0mm}  
\setlength{\textheight}{228mm}  


\setlength{\evensidemargin}{95mm}

\usepackage[colorinlistoftodos, textwidth=65mm, shadow]{todonotes}  
% End block taken from site

\usepackage[doublespacing]{setspace}
\usepackage{url}

\newtheoremstyle{plain}{3mm}{3mm}{\slshape}{}{\bfseries}{.}{.5em}{}
\theoremstyle{plain}

% Numbered theorems 
\newtheorem{thm}{Theorem}[section]
\newtheorem{lem}[thm]{Lemma}
\newtheorem{prop}[thm]{Proposition}
\newtheorem{cor}[thm]{Corollary}
\newtheorem{up}[thm]{Ultrafilter Principle}


% Unnumbered named theorems or results
\newtheorem*{fact}{Fact}
\newtheorem*{hj}{Hales-Jewett Theorem}
\newtheorem*{php}{Pigeonhole Principle}
\newtheorem*{ramsey}{Ramsey's Theorem}
\newtheorem*{vdw}{Van der Waerden's Theorem}
\newtheorem*{schur}{Schur's Theorem}

\newtheorem{FST}[thm]{Hindman's Theorem}
\newtheorem{MBR}[thm]{Multiple Birkhoff Recurrence Theorem}
\newtheorem{recur}[thm]{Recurrence Theorem}
\newtheorem{OCST}[thm]{Furstenburg's Original Central Sets Theorem}
\newtheorem{cst}[thm]{Central Sets Theorem}



\newtheorem{claim}[thm]{Claim}
\newtheorem{ques}[thm]{Question}
\newtheorem{conj}[thm]{Conjecture}


\theoremstyle{definition}
% Numbered "definition" style theorem environments
\newtheorem{defn}[thm]{Definition}
\newtheorem{rmk}[thm]{Remark}
\newtheorem{example}[thm]{Example}

\newcommand{\la}{\langle}
\newcommand{\ra}{\rangle}
\newcommand{\bbN}{\mathbb{N}}
\newcommand{\bbZ}{\mathbb{Z}}
\newcommand{\bbR}{\mathbb{R}}
\newcommand{\AP}{\mathcal{AP}}
\newcommand{\AL}{\mathcal{AL}}

% Short names for calligraphic math letters.
\newcommand{\calA}{\mathcal{A}}
\newcommand{\calB}{\mathcal{B}}
\newcommand{\calC}{\mathcal{C}}
\newcommand{\calF}{\mathcal{F}}
\newcommand{\calG}{\mathcal{G}}
\newcommand{\calI}{\mathcal{I}}
\newcommand{\calJ}{\mathcal{J}}
\newcommand{\calP}{\mathcal{P}}
\newcommand{\calR}{\mathcal{R}}
\newcommand{\calS}{\mathcal{S}}
\newcommand{\calT}{\mathcal{T}}
\newcommand{\calU}{\mathcal{U}}

\newcommand{\Pf}{\mathcal{P}_f}

\newcommand{\setfunc}[2]{\hbox{${}^{\hbox{$#1$}}\hskip -1 pt #2$}}

\font\bigmath=cmsy10 scaled \magstep 3
\newcommand{\bigtimes}{\hbox{\bigmath \char'2}}

\newcommand{\cchi}{\raise 2 pt \hbox{$\chi$}}


\begin{document}
\section{Partition Regularity and Ultrafilters}
Cardinality is commonly thought of as a ``measure'' on the size of sets.
Intuitively this ``notion of largeness'' provides a good mathematical
formalization of size since every set has a unique cardinal number;
and under ZFC (the usual Zermelo-Fraenkel axioms of set theory along
with the axiom of choice) there is a fixed nontrivial order relation
such that every set of cardinals is wellordered.
However the concept of cardinality has its own mathematical
peculiarities as a notion of largeness.
This peculiarity is starkly illustrated by the formal independence of
the Continuum Hypothesis from ZFC (provided ZFC is a consistent
theory).
Recall that the Continuum Hypothesis is the assertion that $|\bbR|$ is
the first cardinal after $|\bbN|$.
The independence of this assertion means that, without adding extra
set-theoretical axioms, we cannot determine the precise location of
the cardinal $|\bbR|$ in the class of all cardinals.

Of course by using the well known diagonal argument of Cantor%
\endnote{
  Interestingly Grattan-Guinness's observation in \cite[page 134,
  footnote 1]{Grattan-Guinness:1978kx} implies that Paul de
  Bois-Reymond was the first to publish a diagonal-type argument. 
}
we can easily prove the weaker assertion that $|\bbR|$ is strictly
greater than $|\bbN|$.
The point we wish to emphasize is that questions about relative sizes
are often easier to study than questions about absolute sizes.
Unfortunately, cardinality is a somewhat blunt tool to use in this
regard.
Therefore in this section we introduce two different (but ultimately
related) types of notion of largeness that are more amendable to
answering certain questions on relative sizes.

\subsection{Filters, Filter bases, Filter subbases and Ultrafilters}
\begin{defn}
  Let $X$ be a nonempty set.
  \begin{itemize}
    \item[(a)] We call $\calF \subseteq \calP(X)$ a \textsl{filter on
        $X$} if and only if $\calF$ satisfies the following three
      conditions:
      \begin{itemize}
        \item[(1)] $\emptyset \ne \calF$ and $\emptyset \not\in
          \calF$.
        
        \item[(2)] If $A$ and $B$ are elements of $\calF$, then $A
          \cap B \in \calF$.

        \item[(3)] If $A \in \calF$ and $A \subseteq B \subseteq X$,
          then $B \in \calF$.
      \end{itemize}

    \item[(b)] We call $\calB \subseteq \calP(X)$ a \textsl{filter
        base on $X$} if and only if $\calB$ satisfies the following two
      conditions:
      \begin{itemize}
        \item[(1)] $\emptyset \ne \calB$ and $\emptyset \not\in
          \calB$.
        
        \item[(2)] If $A$ and $B$ are elements of $\calB$, then there
          exists $C \in \calB$ such that $C \subseteq A \cap B$.
      \end{itemize}

    \item[(c)] We call $\calS \subseteq \calP(X)$ a \textsl{filter
        subbase on $X$} or we will commonly write that $\calS$ has the
      \textsl{finite intersection property} (abbreviated f.i.p.%
      \endnote{
        I will often write something of the form ``let $\calS$ have
        f.i.p.'' or ``let $\calS$ be f.i.p.''.
        Strictly speaking this should be written as ``let
        $\calS$ have the f.i.p.''.
        However since this last form makes it sound like our set
        $\calS$ is stricken with some horrible disease I prefer the
        first two forms.
      }%
      ) if and
      only for every nonempty finite subset $\calA \subseteq \calS$ we
      have $\bigcap \calA \ne \emptyset$. 
  \end{itemize}
\end{defn}
\begin{rmk}
  Intuitively we may think of elements of a filter as simply a
  collection of relatively large subsets.
  Conditions (1) and (3) of a filter nicely align with this intuition,
  but condition (2) may at first look a litte strange.
  This condition may be thought of as saying that we require our large
  sets to interlock in a highly nontrivial way.%
  \endnote{
    Filters, filter bases, and ultrafilters were introduced in two
    notes of Cartan, \cite{Cartan:1937vn} and \cite{Cartan:1937ys}, as
    one way to generalize the use of arguments based on sequence in
    metric spaces to topological spaces. 
    However, see Sundstr\"{o}m's paper \cite[Section
    4.2]{Sundstrom:2010zr} for some history and references to others
    that discovered the concepts of filters and ultrafilters
    independently. 
  }
\end{rmk}

\begin{example}
  Let $X$ be a nonempty set.
  If $\emptyset \ne A \subseteq X$, then the set $\{\, B \subseteq X :
  A \subseteq B \,\}$ is a filter on $X$.
  We call such filters \textsl{principal filters}.
\end{example}

Using condition (2) of a filter, a simple argument shows that every
filter on a nonempty finite set is necessarily a principal filter. 
In this dissertation we will adopt the bias that principal filters are
essentially trivial or well understood objects. 
The next example shows that nonprincipal filters exist on every
infinite set.

\begin{example}
  Let $X$ be an infinite set.
  Then $\{\, A \subseteq X : \mbox{$X \setminus A$ is finite} \,\}$ is
  a nonprincipal filter on $X$.
  We call this filter the \textsl{cofinite filter} or
  \textsl{Fr\'{e}chet filter}.
\end{example}

\todo{
  Try to say more here about filter bases and filter subbases.
  Perhaps draw the analogy with a topological base and subbase?
}
Filter bases and subbases can be used to naturally generate a filter.


\begin{prop}
  Let $\calB$ be a filter base on $X$.
  The set $\{\, A \subseteq X : \mbox{$B \subseteq A$ for some $B \in
    \calB$} \,\}$ is a filter on $X$.
  We call this filter the \textsl{filter generated by the base $\calB$}.
\end{prop}

\begin{prop}
  Let $\calS$ be a filter subbase on $X$.
  Then $\calB = \{\, \bigcap \calA : \mbox{$\emptyset \ne \calA
    \subseteq \calS$ is finite} \,\}$ is a filter base on $X$.
\end{prop}

\begin{cor}
  The set $\{\, A \subseteq X : \mbox{$\bigcap\calA \subseteq A$ for
    some finite $\emptyset \ne \calA \subseteq \calS$} \,\}$ is a
  filter on $X$.
  We call this filter the \textsl{filter generated by the subbase $\calS$}.
\end{cor}

\begin{defn}
  Let $\calF_1$ and $\calF_2$ be filters on $X$.
  We say that $\calF_1$ is \textsl{coarser than} $\calF_2$ or
  $\calF_2$ is \textsl{finer than} $\calF_1$ if and only if $\calF_1
  \subseteq \calF_2$. 
\end{defn}

The relation $\subseteq$ is a partially ordering on the collection of
all filters on a set. 
And with this partial order we have the following definition and
result to show that our definition is not vacuous. 

\begin{defn}
  A filter $\calU$ on $X$ is called an \textsl{ultrafilter} if and
  only if $\calU$ is a \mbox{$\subseteq$-maximal} filter.
\end{defn}

\begin{lem}
  Let $\calC$ be a collection of filters on a set $X$.
  Then $\bigcap\calC$ is a filter on $X$.
  Moreover, if $\calC$ is a \mbox{$\subseteq$-chain}, then
  $\bigcup\calC$ is a filter on $X$.
\end{lem}

\begin{up}
  Every filter is contained in an ultrafilter. 
\end{up}

\begin{thm}
  Let $\calU$ be a filter on $X$.
  The following statements are equivalent.
  \begin{itemize}
    \item[(a)] $\calU$ is an ultrafilter on $X$.
    \item[(b)] For every $A \subseteq X$, either $A \in \calU$ or $X
      \setminus A \in \calU$.
    \item[(c)] For every $A$, $B \subseteq X$, if $A \cup B \in
      \calU$, then either $A \in \calU$ or $B \in \calU$.
  \end{itemize}
\end{thm}


\subsection{Partition Regularity}

\begin{defn}
  Let $X$ be a nonempty set and $\calR$ a collection of sets with
  $\emptyset\not\in \calR$. 
  \begin{itemize}
    \item[(a)] Let $r$ be a natural number. 
      The pair $(X, \calR)$ is called \textsl{\mbox{$r$-partition}
        regular} if and only if whenever $X = \bigcup_{i=1}^r
      C_i$, there exist $i \in \{1, 2, \ldots, r\}$ and $A \in \calR$
      such that $A \subseteq C_i$.

    \item[(b)] We call the pair $(X, \calR)$ \textsl{partition
        regular} if and only if $(X, \calR)$ is \mbox{$r$-partition}
      regular for every $r \in \bbN$.
  \end{itemize}
\end{defn}

% Say that we will only focus on partition regular statements and that
% for κ>=ω is the domain for "infinite combinatorics.  Also mention
% how this is used in modern set theory to investigate several notions
% of large cardinals.

Partition regularity roughly asserts that some property of $X$, here
represented by members of $\calR$, occurs a ``large'' number of times
in $X$.
In fact so large, that no matter how we finitely divide up $X$, at
least one cell in the division has our specified property. 
One of the easiest partition regular property to observe is the
infinite form of the pigeonhole principle.

\begin{php}
  Let $X$ be an infinite put $\calR = \{\, A \subseteq X : \mbox{$A$
    is infinite} \,\}$.
  Then $(X, \calR)$ is partition regular.
\end{php}
\begin{proof}
  Let $r \in \bbN$ and $X = \bigcup_{i=1}^r C_i$.
  Then $|X| \le \sum_{i=1}^r |C_i|$.
  If $C_i$ is finite for each $i \in \{1, 2, \ldots, r\}$,
  then $|X|$ is finite too, a contradiction.
\end{proof}

The Pigeonhole Principle is an easy result to prove. 
However, a powerful generalization of this principle was discovered by
Ramsey in \cite[Theorem A]{Ramsey:1930uq}. 

\begin{ramsey}
  Let $X$ be an infinite set, and let $k \in \bbN$.
  Put $\calR = \bigl\{\, \{\, A \subseteq Y : |A| = k \,\} : \mbox{$Y
    \subseteq X$ is infinite} \,\bigr\}$.
  Then $\bigl(\{\, A \subseteq X : |A| = k \,\}, \calR\bigr)$ is
  partition regular. 
\end{ramsey}
\begin{proof}[Proof Sketch]
  Observe that when $k = 1$ the result follows from the Pigeonhole
  Principle. 
  We now suppose that $k=2$ and we show that $\bigl(\{\, A
  \subseteq X : |A| = 2 \,\}, \calR\bigr)$ is \mbox{2-partition}
  regular: that is, we show that if $\{\, A \subseteq X : |A| = 2 \,\}
  = C_1 \cup C_2$, then there exist an infinite subset $Y \subseteq
  X$ and $i \in \{1, 2\}$ such that $\{\, B \subseteq Y : |B| = 2 \,\}
  \subseteq C_i$. 

  To this end, we will recursively construct sequences $\la X_n
  \ra_{n=0}^\infty$, $\la x_n \ra_{n=1}^\infty$, $\bigl\la (C_1^{(n)},
  C_2^{(n)}) \bigr\ra_{n=1}^\infty$, and $\la r_n \ra_{n=1}^\infty$
  satisfying the following hypotheses:
  \begin{itemize}
    \item[(0)] $X_0 = X$.
    \item[(1)] $x_n \in X_{n-1}$ for all $n \in \bbN$.
    \item[(2)] $X_n$ is an infinite subset of $X_{n-1} \setminus
      \{x_n\}$ for each $n \in \bbN$.
    \item[(3)] For all $i \in \{1, 2\}$ and $n \in \bbN$, 
      \[
        C_i^{(n)} = \{\, x \in X_{n-1} \setminus \{x_n\} : \{x, x_n\}
        \in C_i \,\}.
      \]
    \item[(4)] $X_n \subseteq C_{r_n}^{(n)}$ for all $n \in \bbN$. 
  \end{itemize}
  We first show how these sequences prove our special case of Ramsey's
  Theorem. 
  By hypotheses (0), (1), and (2), the set $\la x_n \ra_{n=1}^\infty$
  is a one-to-one sequence in $X$; and by the Pigeonhole Principle,
  either $\{\, n \in \bbN : r_n = 1 \,\}$ is infinite or $\{\, n \in
  \bbN : r_n = 2 \,\}$ is infinite. 
  Without loss of generality suppose $\{\, n \in \bbN : r_n = 1 \,\}$
  is infinite and put $Y = \{\, x_n : \mbox{$n \in \bbN$ and $r_n =
    1$} \,\}$.
  Then $Y \subseteq X$ is infinite and we claim that $\{\, A \subseteq
  Y : |A| = 2\,\} \subseteq C_1$.
  To see this claim, let $m$ and $n$ be natural numbers with $m < n$
  and $r_m = r_n = 1$. 
  By hypothesis (2), it follows that $X_{n-1} \subseteq X_m$. 
  Hence $x_n \in X_m$ since $x_n \in X_{n-1}$. 
  By hypotheses (3) and (4), it follows that $\{x_n , x_m \} \in
  C_1$. 
  This finishes the proof of our claim.

  We now recursively construct our sequences to satisfy hypotheses
  (0)-(4). 
  Put $X_0 = X$ let $k > 1$ and assume we have constructed sequences
  $\la X_n \ra_{n=0}^{k-1}$, $\la x_n \ra_{n=1}^{k-1}$, $\bigl\la
  (C_1^{(n)}, C_2^{(n)}) \bigr\ra_{n=1}^{k-1}$, and $\la r_n
  \ra_{n=1}^{k-1}$ satisfying hypotheses (0)-(4).
  By hypothesis (2), $X_{k-1}$ is infinite so pick $x_k \in X_{k-1}$.  
  For each $i \in \{1, 2\}$ define
  \[
    C_i^{(k)} = \bigl\{\, x \in X_{k-1} \setminus \{x_k\} : \{x, x_k\} \in
    C_i \,\bigr\}.
  \]
  Then $X_{k-1} \setminus \{x_k\}= C_1^{(k)} \cup C_2^{(k)}$ and so by the Pigeonhole
  Principle, we can pick an infinite subset $X_k \subseteq X_{k-1}
  \setminus \{x_k\}$ and $r_k \in \{1, 2\}$ such that $X_k \subseteq
  C_{r_k}^{(k)}$. 
  By construction the sequences $\la X_n \ra_{n=0}^k$, $\la x_n
  \ra_{n=1}^k$, $\bigl\la (C_1^{(n)}, C_2^{(n)})
  \bigr\ra_{n=1}^k$, and $\la r_n \ra_{n=1}^k$ satisfying
  hypotheses (0)-(4).
  This completes our recursive construction.

  To show that $\bigl(\{\, A \subseteq X : |A| = 2 \,\}, \calR\bigr)$
  is \mbox{$r$-partition} regular follows by induction on $r$. 
  Finally to show that for all $k \in \bbN$, $\bigl(\{\, A \subseteq X
  : |A| = k \,\}, \calR\bigr)$ is partition regular follows by
  induction on $k$.
\end{proof}
\begin{rmk}
  The current proof sketch is essentially Ramsey's original
  combinatorial proof is \cite[Theorem A]{Ramsey:1930uq}. 
  A complete combinatorial proof, which is where the above is modeled
  from, is in Hindman's survey paper \cite[Theorem
  \textcolor{red}{??}]{Hindman:1979fk}.
\end{rmk}

Notice that the above argument is simliar to a diagonal
type-argument. 
We will show that in fact that there is a close connection between a
partition regular set $(X, \calR)$ an a ultrafilter on $X$. 

% While Ramsey's Theorem is important and provides a good example of a
% partition regular pair $(X, \calR)$, in this dissertation we will
% mainly be considering partition regular sets where the underlying set
% $X$ has an algebraic structure of a semigroup and elements in $\calR$
% are defined in terms of this structure. 

\begin{schur}
  Let $r \in \bbN$ and $\bbN = \bigcup_{i=1}^r C_i$.
  Then there exist $i \in \{1, 2, \ldots, r\}$ and $x$, $y$, and $z
  \in \bbN$ such that $\{\, x, y, x+y \,\} \subseteq C_i$.
\end{schur}

\begin{vdw}
  Let $r \in \bbN$ and $\bbN = \bigcup_{i=1}^r C_i$.
  Then for every $\ell \in \bbN$, there exist $i \in \{1, 2, \ldots,
  r\}$ and $a$, $d \in \bbN$ such that $\{\, a, a+d, \ldots, a+\ell d
  \,\} \subseteq C_i$.
\end{vdw}

\subsection{Ultrafilters}
\begin{defn}
  Let $X$ be a nonempty set and $\calF \subseteq \calP(X)$.
  We call $\calF$ a \textsl{filter (on X)} if and only if it satisfies
  the following three conditions:
  \begin{itemize}
    \item[(1)] $\emptyset \ne \calF$ and $\emptyset \not\in \calF$.
    \item[(2)] If $A$, $B \in \calF$, then $A \cap B \in \calF$.
    \item[(3)] If $A \in \calF$ and $A \subseteq B \subseteq X$, then
      $B \in \calF$.
  \end{itemize}
\end{defn}

% Endnotes 
\theendnotes

% Things referenced in the preliminaries chapter. Eventually this will
% placed in a separate file so the References appear at the end.
\bibliographystyle{amsplain}
\bibliography{../references}

\end{document}