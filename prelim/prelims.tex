\documentclass[12pt,showtrims]{memoir}

\usepackage{amsthm, amssymb, amsmath}
\usepackage{color}
\usepackage{endnotes}
\usepackage{url}


\usepackage[margin=1in]{geometry}

% \usepackage[doublespacing]{setspace}
% \usepackage{url}

\newtheoremstyle{plain}{3mm}{3mm}{\slshape}{}{\bfseries}{.}{.5em}{}
\theoremstyle{plain}

% Numbered theorems 
\newtheorem{thm}{Theorem}[section]
\newtheorem{lem}[thm]{Lemma}
\newtheorem{prop}[thm]{Proposition}
\newtheorem{cor}[thm]{Corollary}
\newtheorem{up}[thm]{Ultrafilter Principle}
\newtheorem{radoSelect}[thm]{Rado's Selection Lemma}
\newtheorem{php}[thm]{Pigeonhole Principle}
\newtheorem{vdw}[thm]{Van der Waerden's Theorem}
\newtheorem{hj}[thm]{Hales-Jewett Theorem}
\newtheorem{shj}[thm]{Special Hales-Jewett Theorem}


\newtheorem{FST}[thm]{Hindman's Theorem}
\newtheorem{MBR}[thm]{Multiple Birkhoff Recurrence Theorem}
\newtheorem{recur}[thm]{Recurrence Theorem}
\newtheorem{OCST}[thm]{Furstenburg's Original Central Sets Theorem}
\newtheorem{cst}[thm]{Central Sets Theorem}



\newtheorem{claim}[thm]{Claim}
\newtheorem{ques}[thm]{Question}
\newtheorem{conj}[thm]{Conjecture}


\theoremstyle{definition}

% Numbered "definition" style theorem environments
\newtheorem{defn}[thm]{Definition}
\newtheorem{rmk}[thm]{Remark}
\newtheorem{example}[thm]{Example}

\newcommand{\la}{\langle}
\newcommand{\ra}{\rangle}
\newcommand{\bbN}{\mathbb{N}}
\newcommand{\bbZ}{\mathbb{Z}}
\newcommand{\bbR}{\mathbb{R}}
\newcommand{\AP}{\mathcal{AP}}
\newcommand{\AL}{\mathcal{AL}}

% Short names for calligraphic math letters.
\newcommand{\calA}{\mathcal{A}}
\newcommand{\calB}{\mathcal{B}}
\newcommand{\calC}{\mathcal{C}}
\newcommand{\calE}{\mathcal{E}}
\newcommand{\calF}{\mathcal{F}}
\newcommand{\calG}{\mathcal{G}}
\newcommand{\calH}{\mathcal{H}}
\newcommand{\calI}{\mathcal{I}}
\newcommand{\calJ}{\mathcal{J}}
\newcommand{\calP}{\mathcal{P}}
\newcommand{\calR}{\mathcal{R}}
\newcommand{\calS}{\mathcal{S}}
\newcommand{\calT}{\mathcal{T}}
\newcommand{\calU}{\mathcal{U}}

\newcommand{\Pf}{\mathcal{P}_f}


\newcommand{\setfunc}[2]{\hbox{${}^{\hbox{$#1$}}\hskip -1 pt #2$}}

\font\bigmath=cmsy10 scaled \magstep 3
\newcommand{\bigtimes}{\hbox{\bigmath \char'2}}

\newcommand{\cchi}{\raise 2 pt \hbox{$\chi$}}

\begin{document}

\addtolength{\baselineskip}{1.7pt}

\section{Van der Waerden's Theorem and Semigroup Theory}
Bartel Leendert van der Waerden in \cite{Van-der-Waerden:1927fk} proved the following remarkable assertion about arithmetic progressions in the positive integers.

\begin{vdw}
  Let $r$ and $k$ be positive integers.
  Then there exists a positive integer $N$ such that if $\{1, 2, \ldots, N\} = \bigcup_{i=1}^r C_i$, then there exists $i \in \{1, 2, \ldots, r\}$ such that $C_i$ contains a $k$-term arithmetic progression, that is, there exist positive integers $a$ and $d$ with $\{a, a+d, \ldots, a + (k-1)d\} \subseteq C_i$.
\end{vdw}
 
Van der Waerden's proof of this theorem uses a complicated, but elementary, combinatorial argument based on the pigeonhole principle and `double induction'.% 
\endnote{
  I'm not sure of the exact form of van der Waerden's argument since I have never read the paper \cite{Van-der-Waerden:1927fk}. 
  This statement is based on my opinion that all known combinatorial proofs of van der Waerden's Theorem are not simple, Khinchin's opinion on the difficulty of the original proof \cite[expressed in Section 1, pages 11--12]{Khinchin:1998fk}, and van der Waerden's description on how he found his proof in \cite{Van-der-Waerden:1971fk}. 
  (Van der Waerden's description in \cite{Van-der-Waerden:1971fk} seems to imply that his proof is similar to the one in \cite[Chapter 1]{Khinchin:1998fk}.)
  The interested reader can find a combinatorial proof of van der Waerden's Theorem in Section 2 of \cite{Hindman:1979fk} or \cite[Chapter 1]{Khinchin:1998fk}.
}
Despite being such an old result, van der Waerden's Theorem still forms the basis for much of the contemporary research in the field of Ramsey Theory%
\endnote{
  Almost a decade after van der Waerden published his proof, Paul Erd\H{o}s and Paul Tur\'{a}n in \cite{Erdos:1936fk} started investigating when a finite set of natural numbers would contain a 3-term arithmetic progression.
  One of their conjectures contained in this paper is the following: For every $\delta > 0$ there exists a natural number $N$ such that if $A \subseteq \{1, 2, \ldots, N\}$ with $|A| \ge \delta N$, then $A$ contains a 3-term arithmetic progression. 
  This conjecture turned out to be correct and hard to prove.

  The first proof of the conjecture came from Klaus Roth in \cite{Roth:1952uq} and \cite{Roth:1953kx} using ``an adaptation of the Hardy-Littlewood [circle] method'' \cite[page 106]{Roth:1953kx}.  
  Over a decade later, Endre Szemer\'{e}di was able to extend this result by using a combinatorial argument in \cite{Szemeredi:1969vn} by replacing `3-term arithmetic progression' by `4-term arithmetic progression'. 
  A few years later in \cite{Szemeredi:1975ys}, again using a difficult combinatorial argument, Szemer\'{e}di was able to further extend the result by replacing `3-term arithmetic progression' by `$k$-term arithmetic progression' for every natural number $k$.

  William Timothy Gowers in \cite{Gowers:2001zr} discovered a new proof of Szemer\'{e}di's Theorem using an important technical adaption of Roth's original argument.
  Gowers's method was one of the original motivations behind Ben Green's and Terence Tao's remarkable result that the primes contain a $k$-term arithmetic progression for every natural number $k$ \cite{Green:2008ly}. 
},
and in this dissertation it will play a similar foundational role. 

This theorem can be rewritten to apply to any semigroup, where we replace an arithmetic progression $\{a, a+d, ..., a+(k-1)d\}$ by an `algebraic line' $\{a, ad, ..., ad^{k-1}\}$. As observed in the introduction of \cite{Bergelson:1992fk}, this `generalization' follows directly from van der Waerden's Theorem itself. 

\begin{thm}
  \label{thm:semigrp-vdw}
  Let $(S, \cdot)$ be a semigroup, and let $r$ and $k$ be positive integers.
  If $b$ and $c$ are elements of $S$, then there exists a positive integer $N$ such that whenever $\{bc, bc^2, \ldots, bc^N\} = \bigcup_{i=1}^r C_i$, there exist $i \in \{1, 2, \ldots, r\}$ and $a$, $d \in S$ with $\{a, ad, \ldots, ad^{k-1}\} \subseteq C_i$.
\end{thm}
\begin{proof}
  Pick a positive integer $N$ as guaranteed by van der Waerden's Theorem for $r$ and $k$.
  For each $i \in \{1, 2, \ldots, r\}$ define $B_i = \bigl\{\, j \in \{1, 2, \ldots, N\} : bc^j \in C_i \,\bigr\}$.
  Then $\{1, 2, \ldots, N\} = \bigcup_{i=1}^r B_i$, and by van der Waerden's Theorem we may pick $i \in \{1, 2, \ldots, r\}$ and $e$, $f \in \bbN$ with $\{e, e+f, \ldots, e+(k-1)f\} \subseteq B_i$.
  
  By definition of $B_i$ we have that $bc^{e+jf} \in C_i$ for all $j \in \{0, 1, \ldots, k-1\}$.
  To finish the proof simply put $a = bc^e$ and $d = c$.
\end{proof}

While we get Theorem \ref{thm:semigrp-vdw} for free from van der Waerden's Theorem there are three potential problems with such a straightforward approach:
Our increment $d$ may be a right identity in our underlying semigroup, and, even if $d$ is not a right identity there is no guarantee that all the elements in $\{a, ad, \ldots, ad^{k-1}\}$ are distinct.
(The introduction of \cite{Bergelson:1992fk} contains an example for both of these problems.)
The last potential problem is the most subtle and can only occur when the underlying semigroup is noncommutative.
To precisely formulate this obscure problem we make a brief detour to quickly review some needed algebraic definitions and results.

\subsection{Semigroup Theory}
We give a laconic review of the basic definitions and results of semigroup theory.
The interested reader can consult the article \cite{Hollings:2007uq} for a more expansive, but still brief, introduction to algebraic semigroup theory or the monographs \cite{Clifford:1961fk}, \cite{Clifford:1967fk}, and \cite[Chapter 1]{Hindman:1998fk} for more extensive information. 

Throughout this subsection we let $(S, \cdot)$ be some fixed semigroup. 

For each $x \in S$ we define the functions $\lambda_x \colon S \to S$ and $\rho_x \colon S \to S$ by $\lambda_x(y) = xy$ and $\rho_x(y) = yx$. 
For $A \subseteq S$ and $x \in S$, we use the special notations $x^{-1}A$ and $Ax^{-1}$ to refer to $\lambda_x^{-1}[A]$ and $\rho_x^{-1}[A]$, respectively. 
Equivalently, $x^{-1}A = \{\, y \in S : xy \in A \,\}$ and $Ax^{-1} = \{\, y \in S : yx \in A \,\}$.
(If $S$ is commutative, we usually denote its binary operation by $+$ instead of $\cdot$ and write $x^{-1}A$ and $Ax^{-1}$ as $-x+A$ and $A-x$, respectively.)
We call an element $e \in S$ an \emph{idempotent} if and only if $ee = e$.

We call $L \subseteq S$ a \emph{left ideal of $S$} if and only if $L$ is nonempty and for all $x \in S$, $\lambda_x[L] \subseteq L$. 
We call $R \subseteq S$ a \emph{right ideal of $S$} if and only if $R$ is nonempty and for all $x \in S$, $\rho_x[R] \subseteq R$. 
Finally, we call $\emptyset \ne I \subseteq S$ a \emph{(two-sided) ideal of $S$} if and only if $I$ is both a left and right ideal.

We call a left ideal a \emph{minimal left ideal} if and only if it does not properly contain a left ideal.
We define a \emph{minimal right ideal} and \emph{minimal two-sided ideal} similarly.
There can be at most one minimal two-sided ideal in a semigroup, and if this minimal ideal exists, then it is contained in every ideal of our semigroup.
We denote this \emph{smallest ideal}, if it exists, by $K(S)$.%
\endnote{
  The notation of $K(S)$ denoting the smallest ideal comes from the Russian mathematician Suschkewitsch.
  He called the smallest ideal the \emph{kernel} of the semigroup.
  Suschkewitsch,according to the description given by \cite{Hollings:2009uq}, in \cite{Suschkewitsch:1928kx} proved that all finite semigroups have a smallest ideal and using this fact was able to produce a structure theorem for finite semigroups. 
}

The existence of the smallest ideal essentially depends on what may be regarded as certain `boundedness' conditions on the semigroup.
For instance, it is a classical result in algebraic semigroup theory that every finite semigroup contains a smallest ideal.
Another example of a boundedness condition that implies the existence of a smallest ideal is the existence of a minimal left ideal with an idempotent element \cite[Theorem 1.59]{Hindman:1998fk}. 
Using this boundedness condition we have the following important result.

\begin{thm}
  Let $(S, \cdot)$ be a semigroup with a minimal left ideal of $S$ that has an idempotent. 
  Let $T$ be a subsemigroup of $S$ and assume that $T$ has a minimal left ideal of $T$ with an idempotent.
  Then $K(T) = K(S) \cap T$.
\end{thm}
\begin{proof}
  A proof of this fact is in \cite[Theorem 1.65]{Hindman:1998fk}.
\end{proof}

The final boundedness condition that we shall be interested in occurs when $S$ has a compact Hausdorff topology that interacts appropriately with the algebraic structure of $S$.

We call $S$ a \emph{compact right-topological semigroup} if and only if $S$ is a compact Hausdorff topological space and a semigroup such that for every $x \in S$, $\rho_x$ is continuous. 
Every compact right-topological semigroup contains an idempotent and the smallest ideal.
(The existence of idempotents is proved in \cite[Theorem 2.5]{Hindman:1998fk}, while the existence of the smallest ideal is proved in \cite[Theorem 2.8]{Hindman:1998fk}.)

\subsection{Algebra in the Stone-\v{C}ech Compactification}
Let $(S, \cdot)$ be a fixed discrete semigroup, that is, $S$ is a semigroup with the discrete topology. 
In this subsection we describe how we can densely embed $S$, and extend its semigroup operation, into a compact right-topological semigroup.

Put $\beta S = \{\, p : \mbox{$p$ is an ultrafilter on $S$} \,\}$.
We topologize $\beta S$ as follows: For each $A \subseteq S$ define $\overline{A} = \{\, p \in \beta S : p \in A \,\}$.
Then $\{\, \overline{A} : A \subseteq S \,\}$ forms a basis for a compact Hausdorff topology on $\beta S$ which is the Stone-\v{C}ech Compactification \cite[Theorem 3.27]{Hindman:1998fk} of the discrete space $S$.

We can extend the semigroup operation of $S$ to $\beta S$ \cite[Theorem 4.1]{Hindman:1998fk} such that for all $p$, $q \in \beta S$, we have $A \in p \cdot q$ if and only if $\{\, x \in S : x^{-1}A \in q \,\} \in p$ \cite[Theorem 4.12(b)]{Hindman:1998fk}.
Moreover, this extension turns $\beta S$ into a compact right-topological semigroup with the property that for all $s \in S$, $\lambda_s$ is continuous. 
Therefore $\beta S$ has at least one idempotent (it typically has many more) and the smallest ideal $K(\beta S)$ exists. 

\subsection{Arithmetic Progression and Algebraic Lines}
The arithmetic progressions of positive integers interact nicely with the algebraic structure of $(\beta \bbN, +)$ (and also, $(\beta \bbN, \cdot)$). 
To see this interaction we define a special subset of $\bbN$ and prove, via Theorem \ref{thm:semigrp-vdw}, that this special subset is partition regular.%
\endnote{
 

  The close connection between partition regularity (although not using this terminology) and ultrafilters was first noticed by Choquet in \cite{Choquet:1947uq}. 
}

\begin{defn}
  Let $\calR$ be a nonempty family of sets.
  We say $\calR$ is \emph{partition regular} if and only if whenever $\calF$ is a finite family of sets and $\bigcup \calF \in \calR$, there exists $A \in \calF$ and $B \in \calR$ with $B \subseteq A$.
\end{defn}
\begin{rmk}
  Given some `property' $\Phi$ of a nonempty set $S$, we call $\Phi$ \emph{partition regular} if and only if whenever $A \subseteq S$ has property $\Phi$, $r$ is a positive integer, and $A = \bigcup_{i=1}^r C_i$, then there exists $i \in \{1, 2, \ldots, r\}$ such that $C_i$ has property $\Phi$.
\end{rmk}

\begin{defn}
  We call $A \subseteq \bbN$ an \emph{$AP$-set} if and only if for every $k \in \bbN$, $A$ contains a $k$-term arithmetic progression.
\end{defn}

\begin{thm}
  \label{thm:ap-partition-reg}
  Let $A \subseteq \bbN$ be an $AP$-set and $r \in \bbN$.
  If $A = \bigcup_{i=1}^r C_i$, then there exists $i \in \{1, 2, \ldots, r\}$ such that $C_i$ is an $AP$-set.
\end{thm}
\begin{proof}
  Observe that it suffices to show that for each positive integer $k$, there exists $i \in \{1, 2, \ldots, r\}$ such that $C_i$ contains a $k$-term arithmetic progression.

  So let $k \in \bbN$ and pick a positive integer $N$ as guaranteed by Theorem \ref{thm:semigrp-vdw} for $k$ and $r$.
  Since $A$ is an $AP$-set, pick positive integers $b$ and $c$ with $\{b+c, b+2c, \ldots, b+Nc\} \subseteq A$. 
  Since $\{b+c, b+2c, \ldots, b+Nc \} \subseteq A = \bigcup_{i=1}^r C_i$ we may pick $i \in \{1, 2, \ldots, r\}$ and $a$, $d \in \bbN$ such that $\{a, a+d, \ldots, a+(k-1)d\} \subseteq C_i$.
\end{proof}

By Theorem \ref{thm:ap-partition-reg} and \cite[Theorem 3.11]{Hindman:1998fk} we have that the following subset of $\beta \bbN$ is nonempty.

\begin{defn}
  $\AP = \{\, p \in \beta \bbN : \mbox{for every $A \in p$, $A$ is an $AP$-set} \,\}.$
\end{defn}

\begin{thm}
  $\AP$ is a closed ideal of $(\beta \bbN, +)$ and $(\beta \bbN, \cdot)$.
\end{thm}
\begin{proof}
  The fact that $\AP$ is an ideal of $(\beta \bbN, +)$ and $(\beta \bbN, \cdot)$ is proved in \cite[Theorem 14.5]{Hindman:1998fk}. 
  The fact that $\AP$ is closed is easy to see using the following common technique.
  Let $p \not\in \AP$ and pick $A \in p$ such that $A$ is not an $AP$-set.
  Then $\overline{A} \cap \AP = \emptyset$, that is, a (basic open) neighborhood of $p$ is contained in the complement of $\AP$.
\end{proof}

We are now in a position to formulate our obscure problem mentioned earlier.
Therefore we start by simply modifying the definition of an $AP$-set, Theorem \ref{thm:ap-partition-reg}, and the definition of $\AP$ to be applicable to a semigroup.

\begin{defn}
  Let $(S, \cdot)$ be a semigroup.
  \begin{itemize}
    \item[(a)] Given $a$, $d \in S$ and a positive integer $k$ we call the set $\{a, ad, \ldots, ad^{k-1}\}$ a \emph{$k$-term algebraic line}. 

    \item[(b)] We call $A \subseteq S$ an \emph{$AL$-set} if and only if for every $k \in \bbN$, $A$ contains a $k$-term algebraic line.

    \item[(c)] $\AL(S) = \{\, p \in \beta S : \mbox{for every $A \in p$, $A$ is an $AL$-set} \,\}$.
  \end{itemize}
\end{defn}

\begin{thm}
  \label{thm:al-partition-reg}
  Let $(S, \cdot)$ be a semigroup, $A \subseteq S$ an $AL$-set, and $r \in \bbN$.
  If $A = \bigcup_{i=1}^r C_i$, then there exists $i \in \{1, 2, \ldots, r\}$ such that $C_i$ is an $AL$-set.
\end{thm}
\begin{proof}
  Simliar to Theorem \ref{thm:ap-partition-reg} it suffices to prove the superficially weaker assertion that for each positive integer $k$, there exists $i \in \{1, 2, \ldots, r\}$ such that $C_i$ contains a $k$-term algebraic line.
  
  To this end pick a positive integer $N$ for $r$ and $k$ as guaranteed by Theorem \ref{thm:semigrp-vdw}. 
  Pick elements $b$ and $c$ in our semigroup such that $\{b, bc, \ldots, bc^{N+1}\} \subseteq A$. 
  Since $\{b, bc, \ldots, bc^{N+1}\} \subseteq A = \bigcup_{i=1}^r C_i$, it follows from Theorem \ref{thm:semigrp-vdw} that there exists $i \in \{1, 2, \ldots, r\}$ such that $C_i$ contains a $k$-term algebraic line.
\end{proof}

\begin{thm}
  Let $(S, \cdot)$ be a semigroup.
  Then $\AL(S)$ is a closed left ideal of $\beta S$.
\end{thm}
\begin{proof}
  The fact that $\AL(S)$ is nonempty follows from Theorem \ref{thm:al-partition-reg} and \cite[Theorem 3.11]{Hindman:1998fk}. 

  To see that $\AL(S)$ is closed let $p \not\in AL(S)$ and pick $A \in p$ such that $A$ is not an $AL$-set.
  Then $\overline{A} \cap \AL(S) = \emptyset$, that is, a (basic open) neighborhood of $p$ is contained in the complement of $\AL(S)$.

  Finally to see that $\AL(S)$ is a left ideal, let $p \in \beta S$ and $q \in \AL(S)$. 
  Let $A \in pq$, then $\{\, x \in S : x^{-1}A \in q \,\} \in p$ and so we may pick $x \in S$ such that $x^{-1}A \in q$. 
  We show that $A$ is an $AL$-set.
  Let $k$ be a positive integer.
  Pick elements $b$ and $d$ in $S$ such that $\{b, bd, \ldots, bd^{k-1}\} \in x^{-1}A$.
  Put $a = xb$ and observe that $\{a, ad, \ldots, ad^{k-1}\} \subseteq A$.
\end{proof}

While $\AL(S)$ and $\AP$ are both closed left ideals of $\beta S$ and $\beta \bbN$, respectively, the next result shows that $\AL(S)$ is not a right ideal in every semigroup. 

\begin{thm}
  \label{thm:AL-not-left-ideal}
  Let $S$ be the free semigroup on two generators $a$ and $b$.
  Then $\AL(S)$ is not a right ideal of $\beta S$.
\end{thm}
\begin{proof}
  To see that $\AL(S)$ is not a right ideal of $\beta S$ we show that there exists $p \in \AL(S)$ such that $pa \not\in \AL(S)$. 
  Put $B = \{\, b^n : n \in \bbN \,\}$. and pick $p \in \AL(S)$ with $B \in p$. 
  (Such a $p$ exists since $B$ is a $AL$-set and by \cite[Theorem 3.11]{Hindman:1998fk}.)
  By \cite[Exercise 4.1.8(b)]{Hindman:1998fk}, $Ba \in pa$. 

  We show that $Ba$ is not an $AL$-set by showing that $Ba$ doesn't contain even a 2-term algebraic line. 
  Suppose there exists $c$, $d \in S$ with $\{c, cd\} \subseteq Ba$. 
  Pick positive integers $i$ and $j$ such that $c = b^ia$ and $cd = b^ja$.
  Then $b^ja = cd = b^iad$.
  If $i < j$, then $b^{j-i}a = ad$.
  If $i > j$, then $b^{i-j}ad = a$.
  However, both of these implications produces a contradiction since $b$ is the leftmost letter of $b^{j-i}a$ and $b^{i-j}a$.
  If $i = j$, then $a = ad$, that is, $d$ must be the empty word.
  This is also a contradiction since the empty word is not in the free semigroup.
  Hence $Ba$ is not an $AL$-set and we have $\overline{Ba} \cap \AL(S) = \emptyset$, that is, $pa \not\in \AL(S)$. 
\end{proof}

\section{Piecewise Syndetic and Central Sets}
The result embodied in Theorem \ref{thm:AL-not-left-ideal} is aesthetically disturbing since $\AL(S)$ may not be a right ideal when $S$ is noncommutative. 
To recover this property in $\beta S$ we shall specialize and drastically modify the definition of an $AP$-set when we have a (noncommutative) semigroup.
The two modifications we mention in this section are piecewise syndetic sets and central sets.
We start by giving the definition of a piecewise syndetic set. 
However since the definition of a piecewise syndetic set is, initially, somewhat difficult to understand, we choose to give some motivation on how this concept can be `naturally' derived before explicitly defining it.


\subsection{Syndetic Sets}
We start by proving that a certain statement is equivalent to van der Waerden's Theorem.

\begin{thm}
  \label{thm:vdwEqSyn}
  The following statements are equivalent.
  \begin{itemize}
    \item[(a)] Let $r$ be a positive integer.
      If $\bbN = \bigcup_{i=1}^r C_i$, then there exists $i \in \{1, 2, \ldots, r\}$ such that for all $k \in \bbN$, there exist $a$, $d \in \bbN$ with $\{\, a, a+d, \ldots, a+k d \,\} \subseteq C_i$. 

    \item[(b)] Let $\la x_n \ra_{n=1}^\infty$ be a strictly increasing sequence in $\bbN$ with $\max\{\, x_{n+1} - x_n : n \in \bbN \,\}$ finite.
      Then for all $k \in \bbN$, the set $\{\, x_n : n \in \bbN \,\}$ contains a $k$-term arithmetic progression.
      
    \item[(c)] Let $r$ be a positive integers.
      If $\bigcup_{i=1}^r C_i \subseteq \bbN$ is an $AP$-set, then there exists $i \in \{1, 2, \ldots, r\}$ such that $C_i$ is an $AP$-set.
  \end{itemize}
\end{thm}
\begin{proof}
  \textsl{(a) $\implies$ (b).}
  Let $\la x_n \ra_{n=1}^\infty$ be a strictly increasing sequence in $\bbN$ with $\max\{\, x_{n+1} - x_n : n \in \bbN \,\}$ finite. 
  Put $b =\max\bigl(\{\, x_{n+1} - x_n : n \in \bbN \,\} \cup \{x_1\}\bigr)$ and $A = \{\, x_n : n \in \bbN \,\}$. 
  We claim that $\bbN = \bigcup_{t=1}^b -t+A$. 
  Suppose temporarily that this claim is true.
  We can then apply \textsl{(a)} to pick $i \in \{1, 2, \ldots, b\}$ such that the set $-i + A$ is an $AP$-set. 
  From this situation it follows that $A$ contains a $k$-term arithmetic progression for every $k \in \bbN$. 
  We now prove our claim.
  
  Clearly, $\bigcup_{t=1}^b -t+A \subseteq \bbN$.
  Let $x \in \bbN$.
  First, assume that $x < x_1$, then since $x_1 \le b$, we can pick $t \in \{1, 2, \ldots, b\}$ such that $t + x = x_1$, that is, $x \in -t+A$.
  Now assume that $x_1 \le x$.
  Since $\la x_n \ra_{n=1}^\infty$ is a strictly increasing sequence, we can pick $n \in \bbN$ such that $x_n \le x < x_{n+1}$.
  Moreover, since $x_{n+1} - x_n \le b$, we can pick $t \in \{1, 2, \ldots, b\}$ such that $t + x = x_{n+1}$, that is, $x \in -t + A$. 
  This completes the proof for this direction. 

  \textsl{(b) $\implies$ (c).}
  We proceed by induction on $r$. 
  If $r = 1$, then \textsl{(c)} is trivially true.
  Now let $r > 1$ and assume that \textsl{(c)} is true for $r-1$.

  Let $\bigcup_{i=1}^r C_i$ be an $AP$-set and observe that $\bigcup_{i=1}^r C_i$ is infinite. 
  Without loss of generality suppose that $C_r$ is infinite and enumerate $C_r$ as a strictly increasing sequence $\la x_n \ra_{n=1}^\infty$.
  If $\max\{x_{n+1} - x_n : n \in \bbN\}$ is finite, then by \textsl{(b)} we are done. 
  On the other hand suppose instead that $\max\{x_{n+1} - x_n : n \in \bbN\}$ is infinite, that is, for every $k \in \bbN$, there exists $n \in \bbN$ with $x_{n+1} - x_n \ge k+2$. 
  Therefore for every $k \in \bbN$, there exists $n \in \bbN$ with $\{x_n + 1, x_n + 2, \ldots, x_n +(k+1) \} \subseteq \bbN \setminus C_r$, that is, $\bbN \setminus C_r$ is an $AP$-set. 

  \textsl{(b) $\implies$ (a).}
  Let $r \in \bbN$, and $\bbN = \bigcup_{i=1}^r C_i$. 
  By the pigeonhole principle, we can pick $i \in \{1, 2, \ldots, r\}$ such that $C_i$ is infinite.
  Without loss of generality suppose that $C_1$ is infinite. 
  Enumerate $C_1$ as a strictly increasing sequence $\la x_n \ra_{n=1}^\infty$.
  If $\max\{\, x_{n+1} - x_n : n \in \bbN \,\}$ is finite, then we are done. 
  Hence suppose instead that $\max\{\, x_{n+1} - x_n : n \in \bbN \,\}$ is infinite, that is, for all $k \in \bbN$, there exists $n \in \bbN$ with $x_{n+1} - x_n \ge k + 2$. 
  Therefore, for all $k \in \bbN$, there exists $n \in \bbN$ such that $\{\, x_n + 1, x_n + 2, \ldots, x_n + (k + 1) \,\} \subseteq \bigcup_{i=2}^r C_i$. 
  This means that $\bigcup_{i=2}^r C_i$ contains a \mbox{$k$-term} arithmetic progression for every $k \in \bbN$.
  In particular, $\bigcup_{i=2}^r C_i$ is infinite and so by the pigeonhole principle, we may pick $i \in \{2, 3, \ldots, r\}$ such that $C_i$ is infinite.
  Without loss of generality, suppose that $C_2$ is infinite.

  Enumerate $C_2$ as a strictly increasing sequence $\la x_n \ra_{n=1}^\infty$.
  If $\max\{\, x_{n+1} - x_n : n \in \bbN \,\}$ is finite, then again we are done.
  Suppose that $\max\{\, x_{n+1} - x_n : n \in \bbN \,\}$ is infinite, then, as above, for every $k \in \bbN$, there exists $n \in \bbN$, such that $\{\, x_n +1, x_n + 2, \ldots, x_n + (k +1) \,\}  \subseteq \bigcup_{i=3}^r C_i$. 
  Since $\bigcup_{i=3}^r C_i$ is an $AP$-set, we may continue as before.
  Eventually, we will stop at a set with bounded gaps or a set that contains arbitrarily long arithmetic progressions.

  \textsl{(c) $\implies$ (a).}
  This implication is trivial.
\end{proof}
\begin{rmk}
  Theorem \ref{thm:vdwEqSyn}(a) is equivalent to van der Waerden's Theorem.
  One direction in this equivalence is easy, while the other direction essentially depends on what have been termed `compactness arguments'. 
  \cite[Theorem 2.5]{Landman:2004qf} contains a more complete list of statements equivalent to van der Waerden's Theorem.
\end{rmk}

We wish to translate statement Theorem \ref{thm:vdwEqSyn} (b) into the language of semigroups.
Of course, given an arbitrary semigroup, there is no guarantee that the terms ``strictly increasing'' and ``distance between two elements'' makes sense.
It is the goal of the next result to show that we can forgo these terms.

In what follows, if $X$ is a set, we let $\Pf(X)$ denote the set of all nonempty finite subsets of $X$. 

\begin{prop}
  \label{prop:syn}
  Let $A \subseteq \bbN$ be infinite.
  The following statements are equivalent.
  \begin{itemize}
    \item[(a)] Enumerate $A$ as a strictly increasing sequence $\la x_n \ra_{n=1}^\infty$.
      Then $\max\{\, x_{n+1} - x_n : n \in \bbN \,\}$ is finite.

    \item[(b)] There exists $G \in \Pf(\bbN)$ such that $\bbN = \bigcup_{t \in G} -t + A$. 
  \end{itemize}
\end{prop}
\begin{proof}
  \textsl{(a) $\implies$ (b).}
  Let $b = \max\bigl(\{\, x_{n+1} - x_n : n \in \bbN \,\} \cup \{x_1\}\bigr)$ and put $G = \{1, 2, \ldots, b\}$. 
  Clearly, $\bigcup_{t \in G} -t + A \subseteq \bbN$ so let $x \in \bbN$. 
  First, assume that $x < x_1$, then since $x_1 \le b$, we can pick $t \in G$ such that $t + x = x_1$, that is $x \in -t + A$.
  Now assume that $x_1 \le x$.
  Since $\la x_n \ra_{n=1}^\infty$ is strictly increasing, there exists $n \in \bbN$ such that $x_n \le x < x_{n+1}$.  
  Then $1 \le x_{n+1} - x \le x_{n+1} - x_n \le b$, that is, $x_{n+1} - x \in G$.
  Hence $(x_{n+1} - x) + x = x_{n+1} \in A$, and this completes the forward direction of the proof.

  \textsl{(b) $\implies$ (a).}
  Enumerate $A$ a strictly increasing sequence $\la x_n \ra_{n=1}^\infty$. 
  Pick $b \in \bbN$ such that $G \subseteq \{1, 2, \ldots, b\}$.
  We will show that $\max\{\, x_{n+1} - x_n : n \in \bbN \,\} \le b$. 

  Let $n \in \bbN$.
  Since $\bbN = \bigcup_{t \in G} -t+A$ and $x_n \in \bbN$, pick $t \in G$ such that $t + x_n \in A$.
  Hence pick $m \in \bbN$ such that $x_m = t + x_n$, that is, $x_m - x_n = t \le b$. 
  Observe that $x_{n+1} \le x_m$.
  Therefore it follows that $x_{n+1} - x_n \le x_m - x_n \le b$, and this completes the proof of the reverse direction.
\end{proof}

\begin{defn}
  Let $(S, \cdot)$ be a semigroup and $A \subseteq S$.
  We call $A$ a \emph{syndetic (set in $S$)} if and only if there exist $G \in \Pf(S)$ such that $S = \bigcup_{t \in G} t^{-1}A$. 
\end{defn}

Even though we have successfully translated Theorem \ref{thm:vdwEqSyn}(b) into algebraic terms it's an unfortunate fact that the notion of a syndetic set is \emph{not} partition regular. 
See the Introduction in \cite{Bergelson:2001ve} for an example in $\bbN$.
(Although syndetic sets do intereact nicely with smallest ideal of $\beta S$; see \cite[Theorems 4.39 and 4.43]{Hindman:1998fk}.)
In order to recover partition regularity, we weaken the definition of a syndetic set to obtain a piecewise syndetic set.

\subsection{Piecewise Syndetic Sets}
For a first definition of a piecewise syndetic set we introduce some temporary terminology and notation.
Given $F \in \Pf(\bbN)$ with $|F| = n$, enumerate $F$ as a strictly increasing sequence $x_1 < x_2 < \cdots < x_n$. 
We define the \emph{gap size of $F$} as $g(F) = \max\bigr\{\, x_{j+1} - x_j : j \in \{1, 2, \ldots, n-1\} \,\bigl\}$.
We call a set $A \subseteq \bbN$ \emph{piecewise syndetic} if and only if there exists $b \in \bbN$ such that for all $N \in \bbN$, there exists $B \in \Pf(A)$ with $|B| \ge N$ and $g(B) \le b$.

It's clear that a syndetic set in $\bbN$ is a piecewise syndetic set in $\bbN$.
It's also clear that  parts of this definition may not make sense for arbitrarily semigroups.
It's is the purpose of this next result to show that we can dispense with the potentially troublesome parts.

\begin{prop}
  Let $A \subseteq \bbN$.
  The following statements are equivalent.
  \begin{itemize}
    \item[(a)] $A$ is piecewise syndetic.
    
    \item[(b)] There exists $G \in \Pf(\bbN)$ such that for all $F \in \Pf(\bbN)$, there exists $x \in \bbN$ such that $F+x \subseteq \bigcup_{t \in G} -t+A$. 
  \end{itemize}
\end{prop}
\begin{proof}
  \textsl{(a) $\implies$ (b).}
  Pick $b \in \bbN$ as guaranteed by the definition. 
  Put $G = \{1, 2, \ldots, b\}$.
  We show that for all $F \in \Pf(\bbN)$, there exists $x \in \bbN$ such that $F+x \subseteq \bigcup_{t \in G} -t + A$.

  Let $F \in \Pf(\bbN)$ and put $N =  \max F + 2$. 
  Pick $B \in \Pf(A)$ with $|B| \ge N$ and $g(B) \le b$. 
  Put $n = |B|$ and enumerate $B$ as a strictly increasing sequence $x_1 < x_2 < \cdots  < x_n$.
  It suffices to show that $F + x_1 \subseteq \bigcup_{t \in G} -t + B$.
  (Observe that $B \subseteq A$ implies $\bigcup_{t \in G} -t + B \subseteq \bigcup_{t \in G} -t + A$.)
  Let $f \in F$, then $f + x_1 \le \max F + x_1$. 
  Now $n \ge \max F + 2$ and so $f + x_1 \le (n-2) + x_1 < x_n$. 
  So pick $m \in \{1, 2, \ldots, n-1\}$ such that $x_m \le f+x_1  < x_{m+1}$. 
  Put $t = x_{m+1} - x_1 - f_1$, then $t + f + x_1 = x_{m+1} \in B$ and $t \in G$ since $1 \le t = x_{m+1}- (f_1 + x_1) < x_{m+1} - x_m \le b$. 
  This completes this direction of the proof.  
  
  \textsl{(b) $\implies$ (a).}
  Pick $b \in \bbN$ such that $G \subseteq \{1, 2, \ldots, b\}$. 
  Let $N \in \bbN$ and put $F = \{1, 2, \ldots, 2bN\}$.
  Pick $x \in \bbN$ such that $F + x \subseteq \bigcup_{t \in G} -t + A$. 
  For each $i \in \{1, 2, \ldots, 2bN\}$, pick $t_i \in G$ such that $t_i + i + x \in A$. 
  Put $B = \bigl\{\, t_i + i + x : i \in \{1, 2, \ldots, 2bN\} \,\bigr\}$. 
  We first claim that $|B| \ge N$. 

  % An induction or pigeonhole argument should be useful here.
  Consider the collection of ordered pairs $H = \bigl\{\, (i,j) : \mbox{$i \in \{1, 2, \ldots, b\}$ and $j \in \{1, 2, \ldots, 2bN\}$} \,\bigr\}$.
  To show that $|B| \ge N$ observe that it suffices to show that for any $(i,j) \in H$ we have $|\{\, (s,t) \in H : s+t = i+j \,\}| \le b$. 

  % Not sure how to fix this yet.
  We show that $g(B) \le b$. 
  For $i \in \{1, 2, \ldots, 2bN-1\}$ we have that $t_{i+1} + (i+1) + x - (t_i + i + x) = t_{i+1} - t_i + 1 \le b - 1 + 1 = b$.
  This completes the reverse direction of the proof.
\end{proof}

With this equivalence we now can define the notion of a piecewise
syndetic set in a semigroup. 
Moreover, unlike syndetic sets, we shall soon see that piecewise syndetic sets are partition regular.%
\endnote{
  In \cite[Lemma 1]{Brown:1971bh}, proves that if the natural numbers are finitely partitioned, then at least one cell of the partition is piecewise syndetic.
  It's an interesting question as to whether, Brown's Lemma can be extended to prove Proposition \ref{prop:psReg}.
}

\begin{defn}
  Let $(S, \cdot)$ be a semigroup and $A \subseteq S$.
  We say $A$ is \emph{piecewise syndetic (set in S)} if and only if there exists $G \in \Pf(S)$ such that for all $F \in \Pf(S)$ there exists $x \in S$ with $Fx \subseteq \bigcup_{t \in G} t^{-1}A$.
\end{defn}

\begin{thm}
  \label{prop:psReg}
  Let $S$ be a semigroup and $A \subseteq S$ piecewise syndetic.
  If $A = C_1 \cup C_2$, then either $C_1$ is piecewise syndetic or $C_2$ is piecewise syndetic.
\end{thm}
\begin{proof}
  Pick $G \in \Pf(S)$ such that for all $F \in \Pf(S)$, there exists $x \in S$ such that $Fx \subseteq \bigcup_{t \in G} t^{-1}A$. 
  Suppose, to the contrary, that $C_1$ is not piecewise syndetic and $C_2$ is not piecewise syndetic. 
  Since $C_1$ is not piecewise syndetic, we can pick $F \in \Pf(S)$ such that for all $x \in S$, $Fx \setminus (\bigcup_{t \in G} t^{-1}C_1) \ne \emptyset$. 
  Since $C_2$ is not piecewise syndetic, we can pick $H \in \Pf(S)$ such that for all $x \in S$, $Hx \setminus (\bigcup_{t \in GF} t^{-1}C_1) \ne \emptyset$.
  Since $FH \in \Pf(S)$ pick $x \in S$ such that 
  \[
    FHx \subseteq \bigcup_{t \in G} t^{-1}(C_1 \cup C_2). 
  \]
  
  Pick $y \in H$ such that $yx \not\in \bigcup_{t \in GF} t^{-1}C_1$. 
  Now $Fyx \subseteq \bigcup_{t \in G} t^{-1}(C_1 \cup C_2)$, $yx \in S$, and $C_1$ is not piecewise syndetic, so we have that 
  \[
    Fyx \setminus (\bigcup_{t \in G} t^{-1}C_1) \ne \emptyset. 
  \]
  Pick $z \in F$ such that $zyx \not\in \bigcup_{t \in G} t^{-1}C_1$. 
  Now $zyx \in \bigcup_{t \in G} t^{-1}(C_1 \cup C_2)$, so pick $t \in G$ such that $tzyx \in C_1 \cup C_2$. 
  If $tzyx \in C_1$, then $zyx \in t^{-1}C_1$, a contradiction.
  If $tzyx \in C_2$, then $yx \in (tz)^{-1}C_2$, a contradiction.
  Hence we conclude that either $C_1$ is piecewise syndetic or $C_2$ is piecewise syndetic.
\end{proof}

  Piecewise syndetic sets interact nicely with the smallest ideal of $\beta S$ \cite[Theorem 4.40]{Hindman:1998fk} and moreover $c\ell\bigl(K(\beta S)\bigr) = \{\, p \in \beta S : \mbox{for every $A \in p$, $A$ is piecewise syndetic} \,\}$ is a closed ideal of $\beta S$ \cite[Corollary 4.41 and Theorem 4.44]{Hindman:1998fk}. 

\subsection{Central Sets}

In contrast to our definition of a piecewise syndetic set trying to describe a central subset of a semigroup using only the algebraic structure of our semigroup is shockingly complicated. 
(The curious reader can consult \cite[Section 14.5]{Hindman:1998fk} for the details.)
By using the algebraic structure of the Stone-\v{C}ech compactification we can give a very simple definition of central sets.

\begin{defn}
  Let $(S, \cdot)$ be a semigroup and $A \subseteq S$.
  We call $A$ \emph{central} if and only if there exists an idempotent $p = pp \in K(\beta S)$ such that $A \in p$.
\end{defn}

In the next chapter we shall prove that central sets do enjoy a powerful combinatorial structure embodied by the Central Sets Theorem.


% Endnotes 
\theendnotes

% Things referenced in the preliminaries chapter. Eventually this will
% placed in a separate file so the References appear at the end.
\bibliographystyle{amsplain}
\bibliography{../references}

\end{document}