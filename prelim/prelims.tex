\documentclass[12pt]{article}

\usepackage{amsthm, amssymb, amsmath}
\usepackage{color}
\usepackage{endnotes}
\usepackage{url}


\usepackage[margin=1in]{geometry}

\usepackage[doublespacing]{setspace}
\usepackage{url}

\newtheoremstyle{plain}{3mm}{3mm}{\slshape}{}{\bfseries}{.}{.5em}{}
\theoremstyle{plain}

% Numbered theorems 
\newtheorem{thm}{Theorem}[section]
\newtheorem{lem}[thm]{Lemma}
\newtheorem{prop}[thm]{Proposition}
\newtheorem{cor}[thm]{Corollary}
\newtheorem{up}[thm]{Ultrafilter Principle}
\newtheorem{radoSelect}[thm]{Rado's Selection Lemma}


% Unnumbered named theorems or results
\newtheorem*{fact}{Fact}
\newtheorem*{hj}{Hales-Jewett Theorem}
\newtheorem*{php}{Pigeonhole Principle}
\newtheorem*{ramsey}{Ramsey's Theorem}
\newtheorem*{vdw}{Van der Waerden's Theorem}
\newtheorem*{schur}{Schur's Theorem}

\newtheorem{FST}[thm]{Hindman's Theorem}
\newtheorem{MBR}[thm]{Multiple Birkhoff Recurrence Theorem}
\newtheorem{recur}[thm]{Recurrence Theorem}
\newtheorem{OCST}[thm]{Furstenburg's Original Central Sets Theorem}
\newtheorem{cst}[thm]{Central Sets Theorem}



\newtheorem{claim}[thm]{Claim}
\newtheorem{ques}[thm]{Question}
\newtheorem{conj}[thm]{Conjecture}


\theoremstyle{definition}

% Numbered "definition" style theorem environments
\newtheorem{defn}[thm]{Definition}
\newtheorem{rmk}[thm]{Remark}
\newtheorem{example}[thm]{Example}

\newcommand{\la}{\langle}
\newcommand{\ra}{\rangle}
\newcommand{\bbN}{\mathbb{N}}
\newcommand{\bbZ}{\mathbb{Z}}
\newcommand{\bbR}{\mathbb{R}}
\newcommand{\AP}{\mathcal{AP}}
\newcommand{\AL}{\mathcal{AL}}

% Short names for calligraphic math letters.
\newcommand{\calA}{\mathcal{A}}
\newcommand{\calB}{\mathcal{B}}
\newcommand{\calC}{\mathcal{C}}
\newcommand{\calE}{\mathcal{E}}
\newcommand{\calF}{\mathcal{F}}
\newcommand{\calG}{\mathcal{G}}
\newcommand{\calH}{\mathcal{H}}
\newcommand{\calI}{\mathcal{I}}
\newcommand{\calJ}{\mathcal{J}}
\newcommand{\calP}{\mathcal{P}}
\newcommand{\calR}{\mathcal{R}}
\newcommand{\calS}{\mathcal{S}}
\newcommand{\calT}{\mathcal{T}}
\newcommand{\calU}{\mathcal{U}}

\newcommand{\Pf}{\mathcal{P}_f}


\newcommand{\setfunc}[2]{\hbox{${}^{\hbox{$#1$}}\hskip -1 pt #2$}}

\font\bigmath=cmsy10 scaled \magstep 3
\newcommand{\bigtimes}{\hbox{\bigmath \char'2}}

\newcommand{\cchi}{\raise 2 pt \hbox{$\chi$}}

\begin{document}
Cardinality is commonly thought of as a ``measure'' on the size of sets.
Intuitively this ``notion of largeness'' provides a good mathematical formalization of size since every set has a unique cardinal number; and under ZFC (the usual Zermelo-Fraenkel axioms of set theory along with the axiom of choice) there is a fixed nontrivial order relation such that every set of cardinals is wellordered.
However the concept of cardinality has its own mathematical peculiarities as a notion of largeness.
One peculiarity is starkly illustrated by the formal independence of the Continuum Hypothesis from ZFC (provided ZFC is a consistent theory).
Recall that the Continuum Hypothesis is the assertion that $|\bbR|$ is the first cardinal after $|\bbN|$.
The independence of this statement means that, without adding extra set-theoretical axioms, we cannot determine the precise location of the cardinal $|\bbR|$ in the class of all cardinals.

Of course by using the well-known diagonal argument of Cantor%
\endnote{
  Interestingly Grattan-Guinness's observation in \cite[page 134, footnote 1]{Grattan-Guinness:1978kx} implies that Paul de Bois-Reymond was the first to publish a diagonal-type argument. 
}
we can easily prove the weaker assertion that $|\bbR|$ is strictly greater than $|\bbN|$.
The point we wish to emphasize is that questions about relative sizes are often easier to study than questions about absolute sizes.
Unfortunately, another peculiarity of cardinality is that the concept is a somewhat blunt tool to use in the study of relative sizes.
Therefore in this chapter we start by introducing two different (but ultimately related) notions of largeness that are more amendable to answering certain interesting questions on relative sizes.

To motivate the first two sections, let's start by considering a fixed set $X$ and a collection $\calF$ of ``large'' subsets of $X$. 
We will perform a bit of intuitive reasoning to determine some reasonable properties of our collection $\calF$.
Naturally we would like to consider $X$ be a large subset of itself.
Therefore it seems reasonable to require that $X \in \calF$.
If $A \in \calF$ and $A \subseteq B \subseteq X$, then again it appears reasonable to require that $B \in \calF$ also.
(For this simply says that if $B$ contains a large subset, then $B$ must be large itself.)
The requirement that $\calF$ be closed under supersets places some restrictions on any reasonable properties $\calF$ may have.
For instance, if $\emptyset \in \calF$, then $\calF = \calP(X)$. 
To avoid this undesirable situation we require that $\emptyset \not\in \calF$.
Furthermore, since $X \in \calF$ and $\emptyset \not\in \calF$, it we must also require that $X$ be nonempty.

\section{Filters and Ultrafilters}
Since the material in this section is mostly standard and can be found in several places we choose to leave most of the proofs of our assertions to the reader.
% TODO: Add some references to the results we state.
\begin{defn}
  \label{defn:filters}
  Let $X$ be a nonempty set.
  We call $\calF \subseteq \calP(X)$ a \textsl{filter on $X$} if and only if $\calF$ satisfies the following three conditions:
  \begin{itemize}
    \item[(1)] $\emptyset \ne \calF$ and $\emptyset \not\in\calF$.
    \item[(2)] If $A$ and $B$ are elements of $\calF$, then $A \cap B \in \calF$.
    \item[(3)] If $A \in \calF$ and $A \subseteq B \subseteq X$, then $B \in \calF$.
  \end{itemize}
\end{defn}
\begin{rmk}
  Intuitively we may think of elements of a filter as simply a collection of relatively large subsets of some fixed set.
  Conditions (1) and (3) of a filter nicely align with this intuition, but condition (2) may at first look a litte strange.
  This condition may be thought of as saying that we require our large sets to interlock in a highly nontrivial way.%
  \endnote{
    Filters, filter bases, and ultrafilters were introduced in two  notes of Cartan, \cite{Cartan:1937vn} and \cite{Cartan:1937ys}, as one way to generalize the use of arguments based on sequences in metric spaces to topological spaces. 
    However, see Sundstr\"{o}m's article \cite[Section 4.2]{Sundstrom:2010zr} for some history and references to others that discovered the concepts of filters and ultrafilters independently. 
  }
\end{rmk}

\begin{example}
  \label{ex:prinFilt}
  Let $X$ be a nonempty set.
  If $\emptyset \ne A \subseteq X$, then the set $\{\, B \subseteq X : A \subseteq B \,\}$ is a filter on $X$.
  We call such filters \textsl{principal filters}.
\end{example}

Using condition (2) in Definition \ref{defn:filters}, a simple argument shows that every filter on a nonempty finite set is necessarily a principal filter.
In this dissertation we wil adopt the bias that principal filters are essentially trivial or well-known objects.%
\endnote{
  The mathematical truth of the matter is that in fact principal filters are highly \textsl{non-trivial} objects.
  The easiest way to see this is to observe that if $\calF$ is a filter on a discrete space $X$ and $\overline{\calF} = \{\, p \in \beta X : \calF \subseteq p \,\}$, then the set $\{\, \overline{\calF} : \mbox{$\calF$ is a principal filter} \,\}$ is a clopen basis for $\beta X$.
}
Given this bias we shall primarily be concerned with filters on infinite sets.
Accordingly, our next example shows that there is at least one nonprincipal filter on every infinite set.

\begin{example}
  Let $X$ be an infinite set, then $\calC =\{\, A \subseteq X : \mbox{$X \setminus A$ is finite} \,\}$ is a nonprincipal filter on $X$.
  (To see that this filter is nonprincipal simply observe that for all $x \in X$, $X \setminus \{x\} \in \calC$ and hence $\bigcap\calC = \emptyset$.) 
  We call this filter the \textsl{cofinite filter} or \textsl{Fr\'{e}chet filter}.
\end{example}

The relation $\subseteq$ is a natural partial ordering on the collection on the set of all filters on a nonempty set. 
In analogy with coarser and finer topologies on a set, we make the following definition about coarser and finer filters.


\begin{defn}
  Let $\calF_1$ and $\calF_2$ be filters on $X$.
  We say that $\calF_1$ is \textsl{coarser than} $\calF_1$ or $\calF_2$ is \textsl{finer than} $\calF_1$ if and only if $\calF_1 \subseteq \calF_2$.
\end{defn}

Since the set of all filters on a set $X$ is partially ordered, it is natural to wonder about minimal and maximal elements with respect to this partial ordering.
There is only one minimal filter, the smallest filter $\{X\}$ (which is contained in every filter).
There is no largest filter (that is, there is no filter that contains every filter)%
\endnote{
  Some mathematicians define a filter as follows:
  \begin{defn}
    Let $X$ be a nonempty set and $\calF \subseteq \calP(X)$.
    We call $\calF$ a \textsl{filter} if and only if $\calF$ satisfies the following three conditions:
    \begin{itemize}
      \item[(1)] $\emptyset \ne \calF$.
      \item[(2)] If $A$ and $B$ are elements of $\calF$, then $A cap B \in \calF$.
      \item[(3)] If $A \in \calF$ and $A \subseteq B \subseteq X$, then $B \in \calF$.
    \end{itemize}
    We call a filter $\calF$ \textsl{proper} if $\emptyset \in \calF$.
  \end{defn}
  With this definition the only improper filter is $\calP(X)$ and this filter is also the largest filter in the poset of all filters on $X$.
  Of course if we consider the poset of all proper filters on $X$, then there is no largest proper filter.
}%
, but the situation for the existence of maximal elements is mathematically interesting (or mathematically worrying depending on your point-of-view).
To befit this added complication we give maximal filters a special name.

\begin{defn}
  A filter $\calU$ on $X$ is called an \textsl{ultrafilter on $X$} if and only if $\calU$ is a \mbox{$\subseteq$-maximal} filter.
\end{defn}

To even show that any ultrafilters exists we take a small, but important, detour to show how various collections of sets can be used to generate filters.

\begin{defn}
  Let $X$ be a nonempty set.
  \begin{itemize}
    \item[(a)] We call $\calB \subseteq \calP(X)$ a \textsl{filter base on $X$} if and only if $\calB$ satisfies the following two conditions:
    \begin{itemize}
      \item[(1)] $\emptyset \ne \calB$ and $\emptyset \not\in \calB$.
        
      \item[(2)] If $A$ and $B$ are elements of $\calB$, then there exists $C \in \calB$ such that $C \subseteq A \cap B$.
    \end{itemize}

    \item[(b)] We call $\calS \subseteq \calP(X)$ a \textsl{filter subbase on $X$} or $\calS$ has the \textsl{finite intersection property} (abbreviated f.i.p.) if and only if for every nonempty finite subset $\calA \subseteq \calS$ we have $\bigcap \calA \ne \emptyset$.
  \end{itemize}
\end{defn}

\begin{prop}
  \label{prop:fltBase}
  Let $\calB$ be a filter base on $X$, then the set
  \[
    \{\, A \subseteq X : \mbox{$B \subseteq A$ for some $B \in \calB$} \,\}
  \]
  is a filter on $X$.
  Moreover, this is the coarsest filter that contains $\calB$.
  We call this filter the \textsl{filter generated by the base $\calB$}.
\end{prop}

\begin{prop}
  Let $\calS$ be a filter subbase on $X$, then the set
  \[
     \calB = \bigl\{\, \bigcap \calA : \mbox{$\emptyset \ne \calA \subseteq \calS$ is finite} \,\bigr\}
  \]
  is a filter base on $X$. 
\end{prop}
\begin{cor}
  \label{cor:fltSubbase}
  Let $\calS$ be a filter subbase on $X$, then the set
  \[
    \{\, A \subseteq X : \mbox{$\bigcap\calA \subseteq A$ for some finite $\emptyset \ne \calA \subseteq \calS$} \,\}
  \]
  is a filter on $X$.
  Moreover, this is the coarsest filter that contains $\calS$.
  We call this filter the \textsl{filter generated by the subbase $\calS$}.
\end{cor}
\begin{rmk}
  Since a filter subbbase is often easier to describe than a filter base, we often only use Corollary \ref{cor:fltSubbase} directly when constructing new filters.
\end{rmk}

With this little detour done, we now resume our study of ultrafilters.

\begin{thm}
  \label{thm:equivUf}
  Let $\calU$ be a filter on $X$.
  The following statements are equivalent.
  \begin{itemize}
    \item[(a)] $\calU$ is an ultrafilter on $X$.
    \item[(b)] For every $A \subseteq X$, either $A \in \calU$ or $X \setminus A \in \calU$.
    \item[(c)] For every $A$, $B \subseteq X$, if $A \cup B \in \calU$, then either $A \in \calU$ or $B \in \calU$.
  \end{itemize}
\end{thm}
\begin{proof}
  (a) $\Rightarrow$ (b)
  Suppose that $A \not\in \calU$.
  If $\calU \cup \{A\}$ is a filter subbase, then by Corollary
  \ref{cor:fltSubbase} we can generate a filter that contains $\calU$
  and $A$.
  However since $\calU$ is maximal, this would imply that $A \in
  \calU$, a contradiction.
  Therefore $\calU \cup \{A\}$ is not a filter subbase and so there
  must exists $B \in \calU$ such that $B \cap A = \emptyset$, that is,
  $B \subseteq X \setminus A$.
  It follows that $X \setminus A \in \calU$.
  
  (b) $\Rightarrow$ (c)
  If $A \not\in \calU$ and $B \not\in \calU$, then by assumption $X
  \setminus A \in \calU$ and $X \setminus B \in \calU$.
  Then $X \setminus (A \cup B) = (X \setminus A) \cap (X \setminus B)$
  would be an element of $\calU$.
  However this implies that $\emptyset = \bigl(X \setminus (A \cup
  B)\bigr) \cap (A \cup B) \in \calU$, a contradiction.

  (c) $\Rightarrow$ (a)
  Let $\calF$ be a filter on $X$ with $\calU \subseteq \calF$.
  Let $A \in \calF$.
  Since $\calU \subseteq \calF$, it follows that $X \setminus A
  \not\in \calU$.
  However, $A \cup (X \setminus A) = X \in \calU$ and by assumption it
  follows that $A \in \calU$.
  Therefore $\calU = \calF$ and hence $\calU$ is a maximal filter.
\end{proof}
\begin{rmk}
  A more complete list of statements equivalent to the definition of an ultrafilter can be found in \cite[Theorem 3.6]{Hindman:1998fk}.
\end{rmk}

\begin{example}
  Let $X$ be a nonempty set and $a \in X$.
  Put $\calE(a) = \{\, A \subseteq X : a \in A \,\}$.
  Then $\calE(a)$ is an ultrafilter on $X$.
  The fact that $\calE(a)$ is a filter was first mentioned in Example  \ref{ex:prinFilt}, and the fact that $\calE(a)$ is an ultrafilter
  follows easily from Theorem \ref{thm:equivUf}(b).
  We call such ultrafilters \textsl{principal ultrafilters}.
\end{example}

We will also adopt the bias that principal ultrafilters are essentially trivial or well-known objects.
In contrast to the situation with nonprincipal filters the existence of nonprincipal ultrafilters is sensitive to the underlying axioms of our particular set theory.
(In particular to guarantee the existence of nonprincipal ultrafilters we need a weak form of the axiom of choice.)
It is a fact that there are models of ZF where no nonprincipal ultrafilters exists \cite{Blass:1977fk}.
The stronger statement that every filter can be extended to an ultrafilter follows from our next result and Zorn's Lemma.%
\endnote{
  More precisely the Ultrafilter Principle is stronger than the statement that every infinite set has a nonprincipal ultrafilter. 
  Rav's paper \cite[Section 2]{Rav:1977ys} contains several interesting collection of statements equivalent to the Ultrafilter Principle which are essentially combinatorial in character. 
  All of these statements are variants of Rado's Selection Lemma.
  (Rado's Selection Lemma is a combinatorial result used by Rado in \cite[Lemma 1]{Rado:1949fk}. 
  See \cite{Gottschalk:1951uq} for an easy proof of this selection lemma via Tychonoff's Theorem.)
}


\section{Grills}
\section{Partition Regularity}
\section{Algebraic Semigroup Theory}
\section{Stone-\v{C}ech Compactification of a Discrete Space}
\section{Algebra in the Stone-\v{C}ech Compactification of a Discrete Semigroup}
\section{Ramsey Theoretic Motivation}
\section*{Open Problems}
\section{Ultrafilters and Partition Regularity}

\subsection{Filters, Filter Bases, Filter Subbases, and Ultrafilters}


Our 

% TODO: Enter in more examples of nonprincipal filters.




\begin{lem}
  \label{lem:chainFlt}
  Let $\Phi$ be a collection of filters on a set $X$.
  \begin{itemize}
    \item[(a)] $\bigcap\Phi$ is a filter on $X$.

    \item[(b)] If $\Phi$ is a \mbox{$\subseteq$-chain}, then $\bigcup\Phi$ is a filter on $X$.
  \end{itemize}
\end{lem}

\begin{up}
  Every filter is contained in an ultrafilter.
\end{up}
\begin{proof}[Proof Sketch]
  Let $X$ be a set and $\calF$ a filter on $X$.
  Let $\Phi$ be the collection of all filters on $X$ that are finer than $\calF$.
  By Lemma \ref{lem:chainFlt} and Zorn's Lemma, $\Phi$ contains a maximal element.
\end{proof}
\begin{rmk}
  To produce a nonprincipal ultrafilters we simply apply the Ultrafilter Principle to the cofinite filter.
  It is a fact that every ultrafilter is nonprincipal if and only if it contains the cofinite filter.
\end{rmk}

Sometimes when working with topological spaces it is easier to describe a basis or subbasis for a topology instead of the full topology.
Similarly we can define a filter in terms of a filter base or filter subbase.


In order to produce a filter it is necessary and sufficient to produce a collection with f.i.p.
However as a practical matter it may be inconvenient or difficult to verify that a collection of sets has f.i.p.
Therefore in the next subsection we give some of the basic definitions and results around a concept ``dual'' to the notion of filters that will essentially make producing filters trivial.

\subsection{Grills, Grill Bases, and Grill Subbases}
The dual concept to filters that we will introduce in this section is due to a note of Choquet in \cite{Choquet:1947uq}. 
Since Choquet's note doesn't contain any proofs, and since the concepts he lays out appear to be slightly less well-known than the concept of a filter, in this subsection we choose to give complete proofs for (most of) Choquet's assertions.

\begin{defn}
  \label{defn:grill}
  Let $X$ be a nonempty set.
  \begin{itemize}
    \item[(a)] We call $\calG \subseteq \calP(X)$ a \textsl{grill on $X$} if and only if $\calG$ satisfies the following three conditions:
      \begin{itemize}
        \item[(1)] $\emptyset \ne \calG$ and $\emptyset \not\in \calG$.

        \item[(2)] If $A \in \calG$ and $A \subseteq B \subseteq X$, then $B \in \calG$.

        \item[(3)] If $A \in \calG$ and $B \not\in \calG$, then $A \setminus B \in \calG$. 
     \end{itemize}

    \item[(b)] We call $\calB \subseteq \calP(X)$ a \textsl{grill base on $X$} if and only if $\calB$     satisfies the following two conditions:
      \begin{itemize}
        \item[(1)] $\emptyset \ne \calB$ and $\emptyset \not\in \calB$.

        \item[(2)] If $A \in \calB$ and $A \subseteq B \subseteq X$, then $B \in \calB$.
      \end{itemize}

    \item[(c)] We call $\calS \subseteq \calP(X)$ a \textsl{grill subbase} if and only if $\emptyset \ne \calS$ and $\emptyset \not\in \calS$. 
  \end{itemize}
\end{defn}

Before connecting the concept of a grill with that of a filter, we first prove a convenient result that allows us to use an alternative third condition in Definition \ref{defn:grill}(a).

\begin{prop}
  \label{prop:alt3}
  Let $X$ be a nonempty set and $\calG \subseteq \calP(X)$.
  Then $\calG$ is a grill on $X$ if and only if $\calG$ satisfies the following three conditions:
  \begin{itemize}
    \item[(1)] $\emptyset \ne \calG$ and $\emptyset \not\in \calG$.

    \item[(2)] If $A \in \calG$ and $A \subseteq B \subseteq X$, then $B \in \calG$.

    \item[(3)] For all $A$, $B \subseteq X$, if $A \cup B \in \calG$, then either $A \in \calG$ or $B \in \calG$. 
  \end{itemize}
\end{prop}
\begin{proof}
  We only focus on proving that condition (3) in our Proposition is equivalent to condition (3) in Definition \ref{defn:grill}(a).

  ($\Rightarrow$)
  Let $A$ and $B$ be subsets of $X$ with $A \cup B \in \calG$.
  Suppose that $A \not\in \calG$, then by Definition \ref{defn:grill}(a) $(A \cup B) \setminus A \in \calG$.
  Since $(A \cup B) \setminus A \subseteq B$, condition (2) of Definition \ref{defn:grill}(a) implies that $B \in \calG$.

  ($\Leftarrow$)
  Let $A$ and $B$ be subsets of $X$ with $A \in \calG$ and $B \not\in \calG$.
  Observe that since $B \not\in \calG$ we have $A \cap B \not\in G$.
  Now $(A \setminus B) \cup (A \cap B) = A \in \calG$ and so by assumption we must have $A \setminus B \in \calG$.
\end{proof}

\begin{prop}
  \label{prop:FltGrl}
  Let $\calF$ be a filter on a set $X$ and put $G(\calF) = \{\, A \subseteq X : X \setminus A \not\in \calF \,\}$.
  \begin{itemize}
    \item[(a)] $G(\calF)$ is a grill on $X$.
      We call the grill $G(\calF)$ the grill \textsl{associated with the filter $\calF$}.
    \item[(b)] $G(\calF) = \{\, A \subseteq X : A \cap B \ne \emptyset \mbox{ for all $B \in \calF$} \,\}$.
    \item[(c)] $G(\calF) = \bigcup\{\, \calF' : \mbox{$\calF'$ is a finer filter than $\calF$} \,\}$.
  \end{itemize}
\end{prop}
\begin{proof}
  For notational convenience put $\calG = G(\calF)$.
 
  (a)
  To see that $\emptyset \ne \calG$ observe that $\calF \subseteq \calG$.
  Also $\emptyset \ne \calG$ since $X \setminus \emptyset = X \in \calF$.

  Now let $A \in \calG$ and $A \subseteq B \subseteq X$.
  If $X \setminus B \in \calF$, then we would have $X \setminus A \in \calF$ too.
  ($A \subseteq B$ implies $X \setminus B \subseteq X \setminus A$.)
  Therefore we have that $B \in \calG$.

  Instead of showing condition (3) of Definition \ref{defn:grill}(a) we prove that $\calG$ satisfies condition (3) of Proposition \ref{prop:alt3}.
  Hence let $A \cup B \in \calG$, then $X \setminus (A \cup B) \not\in \calF$.
  If $X \setminus A \in \calF$ and $X \setminus B \in \calF$, then $(X \setminus A) \cap (X \setminus B) \in \calF$.
  Therefore either $X \setminus A \not\in \calF$ or $X \setminus B \not\in \calF$, that is, either $A \in \calG$ or $B \in \calG$.

  (b)
  Let $A \in \calG$ and $B \in \calF$.
  If $A \cap B = \emptyset$, then $B \subseteq X \setminus A$. 
  Therefore $A \cap B \ne \emptyset$.

  Now let $A \subseteq X$ such that for all $B \in \calF$, $A \cap B \ne \emptyset$.
  Then it follows immediately that $X \setminus A \not\in \calF$, that is, $A \in \calG$.

  (c)
  Let $A \in \calG$, then by (b) $\calF \cup \{A\}$ has f.i.p. and so there exists a filter $\calF'$ that contains $\calF \cup \{A\}$.
  
  Now let $\calF'$ be a finer filter than $\calF$ and let $A \in \calF'$. 
  Since $\calF \subseteq \calF'$ and $\emptyset \ne \calF'$ we have that for all $B \in \calF$, $A \cap B \ne \emptyset$.
  Hence by applying (b) we are done.
\end{proof}

\begin{prop}
  \label{prop:GrlFlt}
  Let $\calG$ be grill on $X$ and put $F(\calG) = \{\, A \in \calG : X \setminus A \not\in \calG \,\}$.
  \begin{itemize}
    \item[(a)] Then $F(\calG)$ is a filter on $X$.
      We call the filter $(F\calG)$ the \textsl{filter associated with the grill $\calG$}.
    \item[(b)] $F(\calG) = \{\, A \in \calG : \mbox{$A \cap B \in \calG$ for all $B \in \calG$} \,\}$.
  \end{itemize}
\end{prop}
\begin{proof}
  For notational convenience put $\calF = F(\calG)$.
  (a)
  Since $X \setminus X = \emptyset \not\in \calG$ we have that $X \in \calF$ and so $\emptyset \ne \calF$.
  Also since $X \setminus \emptyset = X \in \calG$, we have that $\emptyset \not\in \calF$.

  Now let $A$ and $B$ be elements of $\calF$, that is, $X \setminus A$ and $X \setminus B$ are not in $\calG$.
  From Proposition \ref{prop:alt3} it follows that $X \setminus (A \cap B) \not\in \calG$, that is, $A \cap B \in \calF$. 

  Let $A \in \calF$ and $A \subseteq B \subseteq X$.
  Since $X \setminus B \subseteq X \setminus A$ and $X \setminus A \not\in \calG$, it follows from Definition \ref{defn:grill}(a) that $X \setminus B \not\in \calG$, that is, $B \in \calF$.

  (b)
  Let $A \in \calF$ and $B \in \calG$.
  Suppose that $A \cap B \not\in \calG$, then $X \setminus (A \cap B) \in \calF$.
  Hence $X \setminus A \cup X \setminus B) \cap A \in \calF$, that is, $\emptyset \cup (X \setminus B \cap A) \in \calF$.
  However $(X \setminus B) \cap A \in \calF$ if and only if $X \setminus (X \setminus B \cap A) \not\in \calG$, that is, $B \cup (X \setminus A) \not\in \calG$.
  But this is a contradiction since $B \in \calG$ and $B \subseteq B \cup (X \setminus A)$.
  
  Now let $A \in \calG$ such that $A \cap B \in \calG$ for all $B \in \calG$.
  Since $\emptyset \ne \calG$ it follows that $X \setminus A \not\in \calG$, that is, $A \in \calF$.
\end{proof}

We now give an easy result that asserts that the functions $G$ and $F$ implicitly defined in Propositions \ref{prop:FltGrl} and \ref{prop:GrlFlt} are one-to-one from the set of all filters onto the set of grills.

\begin{prop}
  Let $X$ be a nonempty set.
  \begin{itemize}
    \item[(a)] If $\calG$ is a grill on $X$, then $\calG = G\bigl(F(\calG)\bigr)$.
    
    \item[(b)] If $\calF$ be a filter on $X$, then $\calF = F\bigl(G(\calF)\bigr)$.
  \end{itemize}
\end{prop}

\begin{defn}
  Let $\calF$ be a filter on $X$ and let $\calG$ be a grill on $X$.
  We say $\calF$ and $\calG$ are \textsl{associated} if and only if $\calF = F(\calG)$ or $\calG = G(\calF)$. 
\end{defn}

With the relation of association we have the following duality type result for filters and grills.

\begin{prop}
  Let the filter $\calF_1$ be associated with the grill $\calG_1$, and let the filter $\calF_2$ be associated with the grill $\calG_2$.
  Then $\calF_1 \subseteq \calF_2$ if and only if $\calG_2 \subseteq \calG_1$.
\end{prop}
\begin{proof}
  ($\Rightarrow$)
  Let $A \in \calG_2$, that is, $X \setminus A \not\in \calF_2$.
  Since $\calF_1 \subseteq \calF_2$ we have $X \setminus A \not\in \calF_1$ too.
  Therefore $A \in G_1$.

  ($\Leftarrow$)
  Let $A \in \calF_1$.
  If $A \not\in \calF_2$, then $X \setminus A \in \calG_2$ and (by assumption) $X \setminus A \in \calG_1$, that is, $A \not\in \calF_1$.
  Therefore $A \in \calF_2$.
\end{proof}

\begin{prop}
  A grill $\calG$ is a filter if and only if $F(\calG)$ is an ultrafilter.
\end{prop}
\begin{proof}
  ($\Rightarrow$)
  Let $A \subseteq X$.
  We show that either $A \in F(\calG)$ or $X \setminus A \in F(\calG)$.
  Suppose that $A \not\in F(\calG)$, that is, assume that either $A \not\in \calG$ or $A \in \calG$ and $X \setminus A \in \calG$.
  By hypothesis $\calG$ is a filter and so we cannot have both $A \in \calG$ and $X \setminus A \in \calG$.
  Since $A \not\in \calG$ and $X \in \calG$ we have that $X \setminus A \in \calG$ and so $X \setminus A \in F(\calG)$.

  ($\Leftarrow$)
  It suffices to show that $\calG$ is closed under finite intersections.
  To this end let $A$ and $B$ be elements of $\calG$.
  However suppose that $A \cap B \not\in \calG$.
  If $A \cap B \in F(\calG)$, then $A \cap B \in \calG$.
  Therefore since $F(\calG)$ is an ultrafilter we have that $(X \setminus A) \cup (X \setminus B) = X \setminus (A \cap B) \in F(\calG)$.
  Also since $F(\calG)$ is an ultrafilter either $X \setminus A \in F(\calG)$ or $X \setminus B \in F(\calG)$. 
  This last sentence implies that either $A \not\in \calG$ or $B \not\in \calG$, a contradiction.
\end{proof}
\begin{cor}
  Let $\calU$ be a filter.
  $\calU$ is an ultrafilter if and only if $\calU = G(\calU)$.
\end{cor}
\begin{proof}
  Let $\calU$ be an ultrafilter.
  It's immediate that $\calU \subseteq G(\calU)$.
  Let $A \in G(\calU)$, that is, $X \setminus A \not\in \calU$. 
  Since $X \setminus A \not\in \calU$ we must have $A \in \calU$.

  Now suppose that $\calU$ is a filter with $\calU = G(\calU)$.
  Then $\calU = F\bigl(G(\calU)\bigr)$ is an ultrafilter.
\end{proof}

We now shift our attention to grill bases.
While grill bases are easier to describe then grills we shall see that in Proposition % TODO: Put in proposition number
that there is generally \textsl{no} coarsest grill that contains a given base.
(This negative result is in contrast to the fact that there is a coarsest filter that contains a filter base).

\begin{prop}
  \label{prop:grlBasePhi}
  Let $\calB$ be a grill base on $X$ and put 
  \[
    \Phi = \{\, \calF : \mbox{$\calF \subseteq \calB$ is a filter} \,\}.
  \]
  Then $\Phi$ is a nonempty, partially ordered by $\subseteq$, and every \mbox{$\subseteq$-chain} in $\Phi$ has an upper bound in $\Phi$.
\end{prop}
\begin{proof}
  Since $X \in \calB$ and $\{X\}$ is a filter contained in $\calB$ we have $\{X\} \in \Phi$.
  $\Phi$ is obviously partially ordered by $\subseteq$.
  Let $\calC$ be a chain in $\Phi$, then $\bigcup\calC$ is a filter by Lemma \ref{lem:chainFlt} and $\bigcup\calC \subseteq \calB$.
\end{proof}

\begin{prop}
  Let $\calB$ be a grill base on $X$ and let $\Phi$ be as in Proposition \ref{prop:grlBasePhi}.
  Put 
  \[
    F(\calB) = \bigcap\{\, \calF : \mbox{$\calF$ is a maximal filter in $\Phi$} \,\}.
  \]
  Then $F(\calB)$ is a filter on $X$.
  We call the filter $F(\calB)$ the \textsl{filter associated with the base $\calB$}.
\end{prop}
\begin{proof}[Proof Sketch]
  By Zorn's Lemma the set $\Phi$ has maximal elements, and by Lemma \ref{lem:chainFlt} $F(\calB)$ is a filter on $X$.
\end{proof}

\begin{cor}
  $G\bigl(F(\calB)\bigr)$ is a grill on $X$.
  We call the grill $G\bigl(F(\calB)\bigr)$ the \textsl{grill associated with the base $\calB$}.
\end{cor}

\begin{prop}
  Let $\calB$ be a grill base on $X$.
  \begin{itemize}
    \item[(a)] $F(\calB) = \{\, A \in \calB : \mbox{$A \cap B \in \calB$ for all $B \in \calB$} \,\}$.

    \item[(b)] $\calB \subseteq G\bigl(F(\calB)\bigr)$.

    \item[(c)] $\calB  \subseteq \bigcap\{\, \calG : \mbox{$\calB \subseteq \calG$ and $\calG$ is a grill} \,\}$.
  \end{itemize}
\end{prop}
\begin{proof}
  (a) 
  Let $A \in F(\calB$, then for every maximal filter $\calF \subseteq \calB$ we have $A \in \calF$. 
  Let $B \in \calB$, then the set $\{\, C \in \calB : B \subseteq C \,\}$ is a (principal) filter contained in $\calB$.
  By applying Zorn's Lemma, it follows that this filter is contained in a maximal filter $\calF \subseteq \calB$.
  In particular, $A \cap B \in \calF$ and so $A \cap B \in \calB$.

  Now let $A \in \calB$ such that for all $B \in \calB$, $A \cap B \in \calB$.
  Suppose that there exists a maximal filter $\calF \subseteq \calB$ with $A \not\in \calF$. 
  If $A \cap C \ne \emptyset$ for all $C \in \calF$, then $\calF \cup \{A\}$ is a filter subbase contained in $\calB$ that strictly contains the maximal filter $\calF$ in $\calB$.
  Therefore there exists $C \in \calF$ such that $A \cap C = \emptyset$.
  However in this case, $C \in \calB$, $A \cap C = \emptyset$, and $\emptyset \not\in \calB$.
  Therefore we must have that $A$ is a member of every maximal filter contained in $\calB$.

  (b)
  Observe that $A \in G\bigl(F(\calB)\bigr)$ if and only if $X \setminus A \not\in F(\calB)$ if and only if $X \setminus A \not\in \calB$ or $X \setminus A \in \calB$ and there exists $B \in \calB$ such that $(X \setminus A) \cap B \not\in \calB$.
  Let $A \in \calB$. 
  If $X \setminus A \not\in \calB$, then we're done.
  If $X \setminus A \in \calB$, then $(X \setminus A) \cap A = \emptyset \not\in \calB$.

  (c)
  Put
  \[
    \Gamma = \{\, \calG : \mbox{$\calB \subseteq \calG$ and $\calG$ is a grill} \,\}.
  \]
  First note that $\Gamma \ne \emptyset$ since by (b) $G\bigl(F(\calB)\bigr) \in \Gamma$.
  The fact that $\calB \subseteq \bigcap\Gamma$ is clear.
\end{proof}
\begin{prop}
  The intersection of two grills may not be a grill.
\end{prop}
\begin{proof}
  Let $\calC$ be the cofinite filter on $\bbN$, let $E$ represent the even numbers and $O$ the odd numbers.
  Then $E$ and $O$ are members of $G(\calC)$.
  Let $\calF_1$ represent the filter generated by $\calC \cup \{O\}$ and $\calF_2$ represent the filter generated by $\calC \cup \{E\}$.
  A straightforward check shows that $G(\calF_1) \cap G(\calF_2)$ is a grill base. 
  However $E \cup O = \bbN \in G(\calF_1) \cap G(\calF_2)$ but $E$ and $O$ are not members of $G(\calF_1) \cap G(\calF_2)$.
\end{proof}


\begin{prop}
  Let $\calS$ be a grill subbase on a set $X$.
  Then
  \[
    B(\calS) = \bigcap\{\, \calB : \mbox{$\calS \subseteq \calB$ and $\calB$ is a grill base} \,\}
  \]
  is a grill base on $X$.
  We call $B(\calS)$ the \textsl{grill base associated with the grill subbase $\calS$}.
\end{prop}
\begin{proof}
  Put $\Psi = \{\, \calB : \mbox{$\calS \subseteq \calB$ and $\calB$ is a grill base} \,\}$ and observe that $\Psi$ is nonempty since it has$\{\, B \subseteq X : \mbox{$A \subseteq B$ for some $A \in \calS$} \,\}$ as a member.
  Hence let $B(\calS) = \bigcap\Psi$.

  Now $B(\calS)$ is nonempty since $X \in \calB$ for every grill base and so in particular $X \in B(\calS)$.
  Also since the empty set is not an element of any grill base we have that $\emptyset \not\in \calB$ for every grill base we have $\emptyset \not\in B(\calS)$.

  Let $A \in B(\calS)$ and $A \subseteq B \subseteq X$. 
  For every grill base $\calB$ with $\calS \subseteq \calB$ we have $A \in \calB$ and hence $B \in \calB$.
  Therefore $B \in B(\calS)$.
\end{proof}

\begin{cor}
  Let $\calS$ be a grill subbase on a set $X$.
  Then $F\bigl(B(\calS)\bigr)$ is a filter on $X$ which we call the \textsl{filter associated with the grill subbase $\calS$}; and $G\Bigl(F\bigl(B(\calS)\bigr)\Bigr)$ with $\calS \subseteq G\Bigl(F\bigl(B(\calS)\bigr)\Bigr)$ is a grill on $X$ which we call the \textsl{grill associated with the grill subbase $\calS$}. 
\end{cor}
\begin{proof}
  The assertions that $F\bigl(B(\calS)\bigr)$ is a filter and $G\Bigl(F\bigl(B(\calS)\bigr)\Bigr)$ is a grill is follows immediately.
  To see that $\calS \subseteq G\Bigl(F\bigl(B(\calS)\bigr)\Bigr)$ observe that $\calS \subseteq B(\calS)$.
\end{proof}

With a grill subbase we have a simply and trivial way to generate filters.
Our general produce shall be this: if we are trying to study a nonempty collection of sets that doesn't contain the empty set, it may be easier to study the associated filter.
In the next section we give a definition for a special type of grill subbase for which this program has been successful.

\subsection{Partition Regularity}
\begin{defn}
  Let $\calS$ be a grill subbase on some nonempty set.
  We say that the $\calS$ is \textsl{partition regular} if and only if whenever $\calA$ is a finite collection of sets with $\bigcup\calA \in \calS$, then there exist $A \in \calA$ and $B \in \calS$ such that $B \subseteq A$.
\end{defn}

\begin{prop}
  Let $\calS$ be a grill subbase on some nonempty set.
  Put $\calB = \{\, B : \mbox{$A \subseteq B$ for some $A \in \calS$} \,\}$. 
  The following statements are equivalent.
  \begin{itemize}
    \item[(a)] $\calS$ is partition regular.
    \item[(b)] $\calB$ is a grill.
  \end{itemize}
\end{prop}
\begin{proof}
  ($\Rightarrow$)
  Clearly, $\calB$ is a grill base. 
  If $B \cup C \in \calB$, then there exists $A \in \calS$ such that $A \subseteq B \cup C$. 
  Therefore $(B \cap A) \cup (C \cap A) = A \in \calS$. 
  Since $\calS$ is partition regular, we have that there exists $D \in \calS$ such that either $D \subseteq B \cap A$ or $D \subseteq C \cap A$.
  
  Without loss of generality suppose $D \subseteq B \cap A \in \calS$, that is, $B \cap A \in \calB$.
  This shows that $\calB$ is a grill.

  ($\Leftarrow$)
  Let $\calA$ be finite collection of sets with $\bigcup\calA \in \calS$, then $\calA \in \calB$.
  Since $\calB$ is a grill pick $A \in \calA$ such that $A \in \calB$.
  Therefore there exists $B \in \calS$ such that $D \subseteq A$, that is, $\calS$ is partition regular.
\end{proof}

\begin{defn}
  Let $X$ be a nonempty set and $\calR$ a collection of sets with
  $\emptyset\not\in \calR$. 
  \begin{itemize}
    \item[(a)] Let $r$ be a natural number. 
      The pair $(X, \calR)$ is called \textsl{\mbox{$r$-partition}
        regular} if and only if whenever $X = \bigcup_{i=1}^r
      C_i$, there exist $i \in \{1, 2, \ldots, r\}$ and $A \in \calR$
      such that $A \subseteq C_i$.

    \item[(b)] We call the pair $(X, \calR)$ \textsl{partition
        regular} if and only if $(X, \calR)$ is \mbox{$r$-partition}
      regular for every $r \in \bbN$.
  \end{itemize}
\end{defn}

% Say that we will only focus on partition regular statements and that
% for κ>=ω is the domain for "infinite combinatorics.  Also mention
% how this is used in modern set theory to investigate several notions
% of large cardinals.

Partition regularity roughly asserts that some property of $X$, here
represented by members of $\calR$, occurs a ``large'' number of times
in $X$.
In fact so large, that no matter how we finitely divide up $X$, at
least one cell in the division has our specified property. 
One of the easiest partition regular property to observe is the
infinite form of the pigeonhole principle.

\begin{php}
  Let $X$ be an infinite put $\calR = \{\, A \subseteq X : \mbox{$A$
    is infinite} \,\}$.
  Then $(X, \calR)$ is partition regular.
\end{php}
\begin{proof}
  Let $r \in \bbN$ and $X = \bigcup_{i=1}^r C_i$.
  Then $|X| \le \sum_{i=1}^r |C_i|$.
  If $C_i$ is finite for each $i \in \{1, 2, \ldots, r\}$,
  then $|X|$ is finite too, a contradiction.
\end{proof}

The Pigeonhole Principle is an easy result to prove. 
However, a powerful generalization of this principle was discovered by
Ramsey in \cite[Theorem A]{Ramsey:1930uq}. 

\begin{ramsey}
  Let $X$ be an infinite set, and let $k \in \bbN$.
  Put $\calR = \bigl\{\, \{\, A \subseteq Y : |A| = k \,\} : \mbox{$Y
    \subseteq X$ is infinite} \,\bigr\}$.
  Then $\bigl(\{\, A \subseteq X : |A| = k \,\}, \calR\bigr)$ is
  partition regular. 
\end{ramsey}
\begin{proof}[Proof Sketch]
  Observe that when $k = 1$ the result follows from the Pigeonhole
  Principle. 
  We now suppose that $k=2$ and we show that $\bigl(\{\, A
  \subseteq X : |A| = 2 \,\}, \calR\bigr)$ is \mbox{2-partition}
  regular: that is, we show that if $\{\, A \subseteq X : |A| = 2 \,\}
  = C_1 \cup C_2$, then there exist an infinite subset $Y \subseteq
  X$ and $i \in \{1, 2\}$ such that $\{\, B \subseteq Y : |B| = 2 \,\}
  \subseteq C_i$. 

  To this end, we will recursively construct sequences $\la X_n
  \ra_{n=0}^\infty$, $\la x_n \ra_{n=1}^\infty$, $\bigl\la (C_1^{(n)},
  C_2^{(n)}) \bigr\ra_{n=1}^\infty$, and $\la r_n \ra_{n=1}^\infty$
  satisfying the following hypotheses:
  \begin{itemize}
    \item[(0)] $X_0 = X$.
    \item[(1)] $x_n \in X_{n-1}$ for all $n \in \bbN$.
    \item[(2)] $X_n$ is an infinite subset of $X_{n-1} \setminus
      \{x_n\}$ for each $n \in \bbN$.
    \item[(3)] For all $i \in \{1, 2\}$ and $n \in \bbN$, 
      \[
        C_i^{(n)} = \{\, x \in X_{n-1} \setminus \{x_n\} : \{x, x_n\}
        \in C_i \,\}.
      \]
    \item[(4)] $X_n \subseteq C_{r_n}^{(n)}$ for all $n \in \bbN$. 
  \end{itemize}
  We first show how these sequences prove our special case of Ramsey's
  Theorem. 
  By hypotheses (0), (1), and (2), the set $\la x_n \ra_{n=1}^\infty$
  is a one-to-one sequence in $X$; and by the Pigeonhole Principle,
  either $\{\, n \in \bbN : r_n = 1 \,\}$ is infinite or $\{\, n \in
  \bbN : r_n = 2 \,\}$ is infinite. 
  Without loss of generality suppose $\{\, n \in \bbN : r_n = 1 \,\}$
  is infinite and put $Y = \{\, x_n : \mbox{$n \in \bbN$ and $r_n =
    1$} \,\}$.
  Then $Y \subseteq X$ is infinite and we claim that $\{\, A \subseteq
  Y : |A| = 2\,\} \subseteq C_1$.
  To see this claim, let $m$ and $n$ be natural numbers with $m < n$
  and $r_m = r_n = 1$. 
  By hypothesis (2), it follows that $X_{n-1} \subseteq X_m$. 
  Hence $x_n \in X_m$ since $x_n \in X_{n-1}$. 
  By hypotheses (3) and (4), it follows that $\{x_n , x_m \} \in
  C_1$. 
  This finishes the proof of our claim.

  We now recursively construct our sequences to satisfy hypotheses
  (0)-(4). 
  Put $X_0 = X$ let $k > 1$ and assume we have constructed sequences
  $\la X_n \ra_{n=0}^{k-1}$, $\la x_n \ra_{n=1}^{k-1}$, $\bigl\la
  (C_1^{(n)}, C_2^{(n)}) \bigr\ra_{n=1}^{k-1}$, and $\la r_n
  \ra_{n=1}^{k-1}$ satisfying hypotheses (0)-(4).
  By hypothesis (2), $X_{k-1}$ is infinite so pick $x_k \in X_{k-1}$.  
  For each $i \in \{1, 2\}$ define
  \[
    C_i^{(k)} = \bigl\{\, x \in X_{k-1} \setminus \{x_k\} : \{x, x_k\} \in
    C_i \,\bigr\}.
  \]
  Then $X_{k-1} \setminus \{x_k\}= C_1^{(k)} \cup C_2^{(k)}$ and so by the Pigeonhole
  Principle, we can pick an infinite subset $X_k \subseteq X_{k-1}
  \setminus \{x_k\}$ and $r_k \in \{1, 2\}$ such that $X_k \subseteq
  C_{r_k}^{(k)}$. 
  By construction the sequences $\la X_n \ra_{n=0}^k$, $\la x_n
  \ra_{n=1}^k$, $\bigl\la (C_1^{(n)}, C_2^{(n)})
  \bigr\ra_{n=1}^k$, and $\la r_n \ra_{n=1}^k$ satisfying
  hypotheses (0)-(4).
  This completes our recursive construction.

  To show that $\bigl(\{\, A \subseteq X : |A| = 2 \,\}, \calR\bigr)$
  is \mbox{$r$-partition} regular follows by induction on $r$. 
  Finally to show that for all $k \in \bbN$, $\bigl(\{\, A \subseteq X
  : |A| = k \,\}, \calR\bigr)$ is partition regular follows by
  induction on $k$.
\end{proof}
\begin{rmk}
  The current proof sketch is essentially Ramsey's original
  combinatorial proof is \cite[Theorem A]{Ramsey:1930uq}. 
  A complete combinatorial proof, which is where the above is modeled
  from, is in Hindman's survey paper \cite[Theorem
  \textcolor{red}{??}]{Hindman:1979fk}.
\end{rmk}

Notice that the above argument is simliar to a diagonal
type-argument. 
We will show that in fact that there is a close connection between a
partition regular set $(X, \calR)$ an a ultrafilter on $X$. 

% While Ramsey's Theorem is important and provides a good example of a
% partition regular pair $(X, \calR)$, in this dissertation we will
% mainly be considering partition regular sets where the underlying set
% $X$ has an algebraic structure of a semigroup and elements in $\calR$
% are defined in terms of this structure. 

\begin{schur}
  Let $r \in \bbN$ and $\bbN = \bigcup_{i=1}^r C_i$.
  Then there exist $i \in \{1, 2, \ldots, r\}$ and $x$, $y$, and $z
  \in \bbN$ such that $\{\, x, y, x+y \,\} \subseteq C_i$.
\end{schur}

\begin{vdw}
  Let $r \in \bbN$ and $\bbN = \bigcup_{i=1}^r C_i$.
  Then for every $\ell \in \bbN$, there exist $i \in \{1, 2, \ldots,
  r\}$ and $a$, $d \in \bbN$ such that $\{\, a, a+d, \ldots, a+\ell d
  \,\} \subseteq C_i$.
\end{vdw}

% Endnotes 
\theendnotes

% Things referenced in the preliminaries chapter. Eventually this will
% placed in a separate file so the References appear at the end.
\bibliographystyle{amsplain}
\bibliography{../references}

\end{document}