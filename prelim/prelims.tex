\documentclass[12pt]{article}

\usepackage{amsthm, amssymb, amsmath}
\usepackage{color}
\usepackage{endnotes}
\usepackage{url}


\usepackage[margin=1in]{geometry}

\usepackage[doublespacing]{setspace}
\usepackage{url}

\newtheoremstyle{plain}{3mm}{3mm}{\slshape}{}{\bfseries}{.}{.5em}{}
\theoremstyle{plain}

% Numbered theorems 
\newtheorem{thm}{Theorem}[section]
\newtheorem{lem}[thm]{Lemma}
\newtheorem{prop}[thm]{Proposition}
\newtheorem{cor}[thm]{Corollary}
\newtheorem{up}[thm]{Ultrafilter Principle}
\newtheorem{radoSelect}[thm]{Rado's Selection Lemma}
\newtheorem{php}[thm]{Pigeonhole Principle}

% Unnumbered named theorems or results
\newtheorem*{fact}{Fact}
\newtheorem*{hj}{Hales-Jewett Theorem}

\newtheorem*{ramsey}{Ramsey's Theorem}
\newtheorem*{vdw}{Van der Waerden's Theorem}
\newtheorem*{schur}{Schur's Theorem}

\newtheorem{FST}[thm]{Hindman's Theorem}
\newtheorem{MBR}[thm]{Multiple Birkhoff Recurrence Theorem}
\newtheorem{recur}[thm]{Recurrence Theorem}
\newtheorem{OCST}[thm]{Furstenburg's Original Central Sets Theorem}
\newtheorem{cst}[thm]{Central Sets Theorem}



\newtheorem{claim}[thm]{Claim}
\newtheorem{ques}[thm]{Question}
\newtheorem{conj}[thm]{Conjecture}


\theoremstyle{definition}

% Numbered "definition" style theorem environments
\newtheorem{defn}[thm]{Definition}
\newtheorem{rmk}[thm]{Remark}
\newtheorem{example}[thm]{Example}

\newcommand{\la}{\langle}
\newcommand{\ra}{\rangle}
\newcommand{\bbN}{\mathbb{N}}
\newcommand{\bbZ}{\mathbb{Z}}
\newcommand{\bbR}{\mathbb{R}}
\newcommand{\AP}{\mathcal{AP}}
\newcommand{\AL}{\mathcal{AL}}

% Short names for calligraphic math letters.
\newcommand{\calA}{\mathcal{A}}
\newcommand{\calB}{\mathcal{B}}
\newcommand{\calC}{\mathcal{C}}
\newcommand{\calE}{\mathcal{E}}
\newcommand{\calF}{\mathcal{F}}
\newcommand{\calG}{\mathcal{G}}
\newcommand{\calH}{\mathcal{H}}
\newcommand{\calI}{\mathcal{I}}
\newcommand{\calJ}{\mathcal{J}}
\newcommand{\calP}{\mathcal{P}}
\newcommand{\calR}{\mathcal{R}}
\newcommand{\calS}{\mathcal{S}}
\newcommand{\calT}{\mathcal{T}}
\newcommand{\calU}{\mathcal{U}}

\newcommand{\Pf}{\mathcal{P}_f}


\newcommand{\setfunc}[2]{\hbox{${}^{\hbox{$#1$}}\hskip -1 pt #2$}}

\font\bigmath=cmsy10 scaled \magstep 3
\newcommand{\bigtimes}{\hbox{\bigmath \char'2}}

\newcommand{\cchi}{\raise 2 pt \hbox{$\chi$}}

\begin{document}
Cardinality is commonly thought of as a ``measure'' on the size of sets.
Intuitively this ``notion of largeness'' provides a good mathematical formalization of size since every set has a unique cardinal number; and under ZFC (the usual Zermelo-Fraenkel axioms of set theory along with the axiom of choice) there is a fixed nontrivial order relation such that every set of cardinals is wellordered.
However the concept of cardinality has its own mathematical peculiarities as a notion of largeness.
One peculiarity is starkly illustrated by the formal independence of the Continuum Hypothesis from ZFC (provided ZFC is a consistent theory).
Recall that the Continuum Hypothesis is the assertion that $|\bbR|$ is the first cardinal after $|\bbN|$.
The independence of this statement means that, without adding extra set-theoretical axioms, we cannot determine the precise location of the cardinal $|\bbR|$ in the class of all cardinals.

Of course by using the well-known diagonal argument of Cantor%
\endnote{
  Interestingly Grattan-Guinness's observation in \cite[page 134, footnote 1]{Grattan-Guinness:1978kx} implies that Paul de Bois-Reymond was the first to publish a diagonal-type argument. 
}
we can easily prove the weaker assertion that $|\bbR|$ is strictly greater than $|\bbN|$.
The point we wish to emphasize is that questions about relative sizes are often easier to study than questions about absolute sizes.
(We can summarize this observation into an elliptic misleading proverb: qualitative questions are easier to answer than quantitive questions.)
Unfortunately, another peculiarity of cardinality is that the concept is a somewhat blunt tool to use in the study of relative sizes.
Therefore in this chapter we start by introducing two different (but ultimately related) notions of largeness that are more amendable to answering certain interesting questions on relative sizes.

To get an idea of what we will be studying in this chapter, let's start by considering some fixed set $X$.
We want to produce a reasonable definition for a collection $\calF$ of subsets of $X$ such that members of $\calF$ are considered as ``relatively large'' subsets of $X$.
We ask what properties should $\calF$ possess?
To answer this question we perform a bit of intuitive reasoning.

Naturally, we would like to consider $X$ to be a large subset of itself, and therefore it seems reasonable to require that $X \in \calF$.
If a subset of $X$ contains a large subset, then we should also consider the containing subset as large itself.
In terms of $\calF$ this says that if $A \in \calF$ and $A \subseteq B \subseteq X$, then $B \in \calF$. 
The requirement that $\calF$ be closed under supersets places some restrictions on any reasonable properties $\calF$ may have.
For instance, if $\emptyset \in \calF$, then $\calF = \calP(X)$. 
($\calP(X)$ is the usual power set of $X$, that is, the collection of all subsets of $X$.)
Since it seems slightly perverse to have a collection of large subsets of $X$ to potentially be equal to $\calP(X)$, we further require that $\emptyset \not\in \calF$. 
Furthermore, since $X \in \calF$ and $\emptyset \not\in \calF$, we also require that $X$ be nonempty.

The requirements that $\calF$ be nonempty, doesn't contain the empty set, and is closed under supersets forms the core of several definitions we shall consider throughout this chapter. 

\section{Filters and Ultrafilters}
Since the material in this section is mostly standard and can be found in several places (for instance see \cite[Chapter 5]{Schechter:1997fk} or \cite[Chapter 3]{Hindman:1998fk}) we choose to leave most of the proofs of our assertions to the reader.
% TODO: Add some references to the results we state.
\begin{defn}
  \label{defn:filters}
  Let $X$ be a nonempty set.
  We call $\calF \subseteq \calP(X)$ a \textsl{filter on $X$} if and only if $\calF$ satisfies the following three conditions:
  \begin{itemize}
    \item[(1)] $\emptyset \ne \calF$ and $\emptyset \not\in\calF$.
    \item[(2)] If $A \in \calF$ and $A \subseteq B \subseteq X$, then $B \in \calF$.
    \item[(3)] If $A$ and $B$ are elements of $\calF$, then $A \cap B \in \calF$.
  \end{itemize}
\end{defn}
\begin{rmk}
  Intuitively we may think of elements of a filter as simply a collection of relatively large subsets of some fixed set.
  Conditions (1) and (2) of a filter nicely align with this intuition, but condition (3) may at first look a litte strange.
  This condition may be thought of as saying that we require our large sets to interlock in a highly nontrivial way.%
  \endnote{
    Filters, filter bases, and ultrafilters were introduced in two notes of Cartan, \cite{Cartan:1937vn} and \cite{Cartan:1937ys}, as one way to generalize the use of arguments based on sequences in metric spaces to topological spaces. 
    However, see Sundstr\"{o}m's article \cite[Section 4.2]{Sundstrom:2010zr} for some history and references to others that discovered the concepts of filters and ultrafilters independently. 
  }
\end{rmk}

\begin{example}
  \label{ex:prinFilt}
  Let $X$ be a nonempty set.
  If $\emptyset \ne A \subseteq X$, then the set $\calE(A) = \{\, B \subseteq X : A \subseteq B \,\}$ is a filter on $X$.
  If $A$ is a singleton set, say $A = \{a\}$, then we write $\calE(A)$ as $\calE(a)$. 
  We call such filters \textsl{principal filters}.
\end{example}

Using condition (3) in Definition \ref{defn:filters}, a simple argument shows that every filter on a nonempty finite set is necessarily a principal filter.
In this dissertation we wil adopt the bias that principal filters are essentially trivial or well-known objects.%
\endnote{
  The mathematical truth of the matter is that in fact principal filters are highly \textsl{non-trivial} objects.
  The easiest way to see this is to observe that if $\calF$ is a filter on a discrete space $X$ and $\overline{\calF} = \{\, p \in \beta X : \calF \subseteq p \,\}$, then the set $\{\, \overline{\calF} : \mbox{$\calF$ is a principal filter} \,\}$ is a clopen basis for $\beta X$.
}
Given this bias we shall primarily be concerned with filters on infinite sets.
Accordingly, our next example shows that there is at least one nonprincipal filter%
\endnote{
  To avoid the negative adjective `nonprincipal' to filters some mathematicians call nonprincipal filters \textsl{free filters}.
}
on every infinite set.

\begin{example}
  Let $X$ be an infinite set, then $\calC =\{\, A \subseteq X : \mbox{$X \setminus A$ is finite} \,\}$ is a nonprincipal filter on $X$.
  (To see that this filter is nonprincipal simply observe that for all $x \in X$, $X \setminus \{x\} \in \calC$ and hence $\bigcap\calC = \emptyset$.) 
  We call this filter the \textsl{cofinite filter} or \textsl{Fr\'{e}chet filter}.
\end{example}
\begin{rmk}
  It is a fact that all nonprincipal filters contain the cofinite filter. 
  See \cite[Section 5.5(e), page 103]{Schechter:1997fk} for a collection of equivalent statements to this fact. 
  (In the referenced statements it's probably easier to prove $(B) \implies (A)$, $(A) \implies (C)$, $(C) \implies (D)$, and $(D) \implies (B)$.)
\end{rmk}

The relation $\subseteq$ is a natural partial ordering on the collection of all filters on a nonempty set. 
In analogy with coarser and finer topologies on a set, we make the following definition about coarser and finer filters.

\begin{defn}
  Let $\calF_1$ and $\calF_2$ be filters on $X$.
  We say that $\calF_1$ is \textsl{coarser than} $\calF_1$ or $\calF_2$ is \textsl{finer than} $\calF_1$ if and only if $\calF_1 \subseteq \calF_2$.
\end{defn}

Since the set of all filters on a set $X$ is partially ordered, it is natural to wonder about minimal and maximal elements with respect to this partial ordering.
There is only one minimal filter, and happily it is contained in every filter, $\{X\}$.
There is no largest filter, that is, there is no filter that contains every filter%
\endnote{
  Some mathematicians define a filter as follows:
  \begin{defn}
    Let $X$ be a nonempty set and $\calF \subseteq \calP(X)$.
    We call $\calF$ a \textsl{filter} if and only if $\calF$ satisfies the following three conditions:
    \begin{itemize}
      \item[(1)] $\emptyset \ne \calF$.
      \item[(2)] If $A$ and $B$ are elements of $\calF$, then $A cap B \in \calF$.
      \item[(3)] If $A \in \calF$ and $A \subseteq B \subseteq X$, then $B \in \calF$.
    \end{itemize}
    We call a filter $\calF$ \textsl{proper} if $\emptyset \in \calF$.
  \end{defn}
  With this definition the only improper filter is $\calP(X)$ and this filter is also the largest filter in the poset of all filters on $X$.
  Of course if we consider the poset of all proper filters on $X$, then there is no largest proper filter.
}%
, but the situation for the existence of maximal filters is mathematically interesting (or mathematically worrying depending on your point-of-view).
To befit this added complication we give maximal filters a special name.

\begin{defn}
  A filter $\calU$ on $X$ is called an \textsl{ultrafilter on $X$} if and only if $\calU$ is a \mbox{$\subseteq$-maximal} filter.
\end{defn}

To even show that any ultrafilters exist we take a small, but important, detour to show how various collections of sets can be used to generate filters.

\begin{defn}
  Let $X$ be a nonempty set.
  \begin{itemize}
    \item[(a)] We call $\calB \subseteq \calP(X)$ a \textsl{filter base on $X$} if and only if $\calB$ satisfies the following two conditions:
    \begin{itemize}
      \item[(1)] $\emptyset \ne \calB$ and $\emptyset \not\in \calB$.
        
      \item[(2)] If $A$ and $B$ are elements of $\calB$, then there exists $C \in \calB$ such that $C \subseteq A \cap B$.
    \end{itemize}

    \item[(b)] We call $\calS \subseteq \calP(X)$ a \textsl{filter subbase on $X$} or $\calS$ has the \textsl{finite intersection property} (abbreviated f.i.p.) if and only if for every nonempty finite subset $\calA \subseteq \calS$ we have $\bigcap \calA \ne \emptyset$.

    \item[(c)] Given a collection $\calA \subseteq \calP(X)$, put 
      \[
         \calA^\uparrow = \{\, B \subseteq X : \mbox{$A \subseteq B$ for some $A \in \calA$} \,\}.
      \]
  \end{itemize}
\end{defn}

\begin{prop}
  \label{prop:fltBase}
  Let $\calB$ be a filter base on $X$, then the set
  \[
    \calB^\uparrow = \{\, A \subseteq X : \mbox{$B \subseteq A$ for some $B \in \calB$} \,\}
  \]
  is a filter on $X$.
  Moreover, this is the coarsest filter that contains $\calB$.
  We call this filter the \textsl{filter generated by the base $\calB$}.
\end{prop}

\begin{prop}
  Let $\calS$ be a filter subbase on $X$, then the set
  \[
     \calB = \bigl\{\, \bigcap \calA : \mbox{$\emptyset \ne \calA \subseteq \calS$ is finite} \,\bigr\}
  \]
  is a filter base on $X$. 
\end{prop}
\begin{cor}
  \label{cor:fltSubbase}
  Let $\calS$ be a filter subbase on $X$, then the set
  \[
    \{\, A \subseteq X : \mbox{$\bigcap\calA \subseteq A$ for some finite $\emptyset \ne \calA \subseteq \calS$} \,\}
  \]
  is a filter on $X$.
  Moreover, this is the coarsest filter that contains $\calS$.
  We call this filter the \textsl{filter generated by the subbase $\calS$}.
\end{cor}
\begin{rmk}
  Since a filter subbbase is often easier to describe than a filter base, we often only use Corollary \ref{cor:fltSubbase} directly when constructing new filters.
\end{rmk}

With this little detour done, we now resume our study of ultrafilters.

\begin{thm}
  \label{thm:equivUf}
  Let $\calU$ be a filter on $X$.
  The following statements are equivalent.
  \begin{itemize}
    \item[(a)] $\calU$ is an ultrafilter on $X$.
    \item[(b)] For every $A \subseteq X$, either $A \in \calU$ or $X \setminus A \in \calU$.
    \item[(c)] For every $A$, $B \subseteq X$, if $A \cup B \in \calU$, then either $A \in \calU$ or $B \in \calU$.
  \end{itemize}
\end{thm}
\begin{proof}
  (a) $\Rightarrow$ (b)
  Suppose that $A \not\in \calU$.
  If $\calU \cup \{A\}$ is a filter subbase, then by Corollary
  \ref{cor:fltSubbase} we can generate a filter that contains $\calU$
  and $A$.
  However since $\calU$ is maximal, this would imply that $A \in
  \calU$, a contradiction.
  Therefore $\calU \cup \{A\}$ is not a filter subbase and so there
  must exists $B \in \calU$ such that $B \cap A = \emptyset$, that is,
  $B \subseteq X \setminus A$.
  It follows that $X \setminus A \in \calU$.
  
  (b) $\Rightarrow$ (c)
  If $A \not\in \calU$ and $B \not\in \calU$, then by assumption $X
  \setminus A \in \calU$ and $X \setminus B \in \calU$.
  Then $X \setminus (A \cup B) = (X \setminus A) \cap (X \setminus B)$
  would be an element of $\calU$.
  However this implies that $\emptyset = \bigl(X \setminus (A \cup
  B)\bigr) \cap (A \cup B) \in \calU$, a contradiction.

  (c) $\Rightarrow$ (a)
  Let $\calF$ be a filter on $X$ with $\calU \subseteq \calF$.
  Let $A \in \calF$.
  Since $\calU \subseteq \calF$, it follows that $X \setminus A
  \not\in \calU$.
  However, $A \cup (X \setminus A) = X \in \calU$ and by assumption it
  follows that $A \in \calU$.
  Therefore $\calU = \calF$ and hence $\calU$ is a maximal filter.
\end{proof}
\begin{rmk}
  A more complete list of statements equivalent to the definition of an ultrafilter can be found in \cite[Theorem 3.6]{Hindman:1998fk}.
\end{rmk}

\begin{example}
  Let $X$ be a nonempty set and $a \in X$.
  Recall that $\calE(a) = \{\, A \subseteq X : a \in A \,\}$.
  Then $\calE(a)$ is an ultrafilter on $X$.
  The fact that $\calE(a)$ is a filter was first mentioned in Example  \ref{ex:prinFilt}, and the fact that $\calE(a)$ is an ultrafilter
  follows easily from Theorem \ref{thm:equivUf}(b).
  We call such ultrafilters \textsl{principal ultrafilters}.
\end{example}

We will also adopt the bias that principal ultrafilters are essentially trivial or well-known objects.
In contrast to the situation with nonprincipal filters the existence of nonprincipal ultrafilters is sensitive to the underlying axioms of our particular set theory.
Without some reasonable strong version of the axiom of choice there may be no nontrivial ultrafilters at all.
It is a fact that there are models of ZF where no nonprincipal ultrafilters exists \cite{Blass:1977fk}.
The stronger statement that every filter can be extended to an ultrafilter follows from our next result and Zorn's Lemma.%
\endnote{
  More precisely the Ultrafilter Principle is stronger than the statement that every infinite set has a nonprincipal ultrafilter. 
  Rav's paper \cite[Section 2]{Rav:1977ys} contains several interesting collection of statements equivalent to the Ultrafilter Principle which are essentially combinatorial in character. 
  All of these statements are variants of Rado's Selection Lemma.
  (Rado's Selection Lemma is a combinatorial result used by Rado in \cite[Lemma 1]{Rado:1949fk}. 
  See \cite{Gottschalk:1951uq} for an easy proof of this selection lemma via Tychonoff's Theorem.)
}

\begin{lem}
  \label{lem:chainFlt}
  Let $\Phi$ be a collection of filters on a set $X$.
  \begin{itemize}
    \item[(a)] $\bigcap\Phi$ is a filter on $X$.

    \item[(b)] If $\Phi$ is a \mbox{$\subseteq$-chain}, then $\bigcup\Phi$ is a filter on $X$.
  \end{itemize}
\end{lem}

\begin{up}
  Every filter is contained in an ultrafilter.
\end{up}
\begin{proof}[Proof Sketch]
  Let $X$ be a set and $\calF$ a filter on $X$.
  Let $\Phi$ be the collection of all filters on $X$ that are finer than $\calF$.
  By Lemma \ref{lem:chainFlt} and Zorn's Lemma, $\Phi$ contains a maximal element.
\end{proof}
\begin{rmk}
  To produce a nonprincipal ultrafilters we simply apply the Ultrafilter Principle to the cofinite filter.
  It is a fact that every ultrafilter is nonprincipal if and only if it contains the cofinite filter.
\end{rmk}

In order to produce a filter it is necessary and sufficient to produce a collection with f.i.p.
However as a practical matter it may be inconvenient or difficult to verify that a collection of sets has f.i.p.
Therefore in the next section we give some of the basic definitions and results around a concept ``dual'' to the notion of filters that will essentially make producing filters trivial.

\section{Grills}
The dual concept to filters that we will introduce in this section is due to a note of Choquet in \cite{Choquet:1947uq}. 
Since Choquet's note doesn't contain any proofs, and since the concepts he lays out appear to be slightly less well-known than the concept of a filter, in this subsection we choose to give complete proofs for (most of)%
\endnote{
  We have been unable to verify (or falsify!) two of Choquet's assertions.
  One assertion is in the last paragraph of section II, and the other assertion is in section IV of \cite{Choquet:1947uq}.
}
Choquet's assertions.

\begin{defn}
  \label{defn:grill}
  Let $X$ be a nonempty set.
  We call $\calG \subseteq \calP(X)$ a \textsl{grill on $X$} if and only if $\calG$ satisfies the following three conditions:
      \begin{itemize}
        \item[(1)] $\emptyset \ne \calG$ and $\emptyset \not\in \calG$.

        \item[(2)] If $A \in \calG$ and $A \subseteq B \subseteq X$, then $B \in \calG$.

        \item[(3)] If $A \in \calG$ and $B \not\in \calG$, then $A \setminus B \in \calG$. 
      \end{itemize}
\end{defn}
\begin{rmk}
  Again we can regard a grill as a collection of relatively large subsets of $X$. 
  Conditions (1) and (2) align with our intuition, but condition (3) can be thought of as stating that we can throw away a portion of a large set provided that the part we throw away is not large itself.
  Proposition \ref{prop:alt3} shows that condition (3) is equivalent to saying that large sets cannot be finitely decomposed into a union of non-large sets.
\end{rmk}

\begin{prop}
  \label{prop:alt3}
  Let $X$ be a nonempty set and $\calG \subseteq \calP(X)$.
  Then $\calG$ is a grill on $X$ if and only if $\calG$ satisfies the following three conditions:
  \begin{itemize}
    \item[(1)] $\emptyset \ne \calG$ and $\emptyset \not\in \calG$.

    \item[(2)] If $A \in \calG$ and $A \subseteq B \subseteq X$, then $B \in \calG$.

    \item[(3)] For all $A$, $B \subseteq X$, if $A \cup B \in \calG$, then either $A \in \calG$ or $B \in \calG$. 
  \end{itemize}
\end{prop}
\begin{proof}
  We only focus on proving that condition (3) in our Proposition is equivalent to condition (3) in Definition \ref{defn:grill}(a).

  ($\Rightarrow$)
  Let $A$ and $B$ be subsets of $X$ with $A \cup B \in \calG$.
  Suppose that $A \not\in \calG$, then by Definition \ref{defn:grill}(a) $(A \cup B) \setminus A \in \calG$.
  Since $(A \cup B) \setminus A \subseteq B$, condition (2) of Definition \ref{defn:grill}(a) implies that $B \in \calG$.

  ($\Leftarrow$)
  Let $A$ and $B$ be subsets of $X$ with $A \in \calG$ and $B \not\in \calG$.
  Observe that since $B \not\in \calG$ we have $A \cap B \not\in G$.
  Now $(A \setminus B) \cup (A \cap B) = A \in \calG$ and so by assumption we must have $A \setminus B \in \calG$.
\end{proof}

\begin{prop}
  \label{prop:FltGrl}
  Let $\calF$ be a filter on a set $X$ and put $G(\calF) = \{\, A \subseteq X : X \setminus A \not\in \calF \,\}$.
  \begin{itemize}
    \item[(a)] $G(\calF)$ is a grill on $X$.
      We call the grill $G(\calF)$ the grill \textsl{associated with the filter $\calF$}.
    \item[(b)] $G(\calF) = \{\, A \subseteq X : A \cap B \ne \emptyset \mbox{ for all $B \in \calF$} \,\}$.
    \item[(c)] $G(\calF) = \bigcup\{\, \calF' : \mbox{$\calF'$ is a finer filter than $\calF$} \,\}$.
  \end{itemize}
\end{prop}
\begin{proof}
  For notational convenience put $\calG = G(\calF)$.
 
  (a)
  To see that $\emptyset \ne \calG$ observe that $\calF \subseteq \calG$.
  Also $\emptyset \not\in \calG$ since $X \setminus \emptyset = X \in \calF$.

  Now let $A \in \calG$ and $A \subseteq B \subseteq X$.
  If $X \setminus B \in \calF$, then we would have $X \setminus A \in \calF$ too.
  Therefore we have that $B \in \calG$.

  Instead of showing condition (3) of Definition \ref{defn:grill}(a) we prove that $\calG$ satisfies condition (3) of Proposition \ref{prop:alt3}.
  Let $A \cup B \in \calG$, then $X \setminus (A \cup B) \not\in \calF$.
  If $X \setminus A \in \calF$ and $X \setminus B \in \calF$, then $(X \setminus A) \cap (X \setminus B) \in \calF$.
  Therefore either $X \setminus A \not\in \calF$ or $X \setminus B \not\in \calF$, that is, either $A \in \calG$ or $B \in \calG$.

  (b)
  Let $A \in \calG$ and $B \in \calF$.
  If $A \cap B = \emptyset$, then $B \subseteq X \setminus A$ and so $X \setminus A \in \calF$, a contradiction. 
  Therefore $A \cap B \ne \emptyset$.

  Now let $A \subseteq X$ such that for all $B \in \calF$, $A \cap B \ne \emptyset$.
  Then it follows immediately that $X \setminus A \not\in \calF$, that is, $A \in \calG$.

  (c)
  Let $A \in \calG$, then by (b) $\calF \cup \{A\}$ has f.i.p. and so there exists a filter $\calF'$ that contains $\calF \cup \{A\}$.
  
  Now let $\calF'$ be a finer filter than $\calF$ and let $A \in \calF'$. 
  Since $\calF \subseteq \calF'$ and $\emptyset \ne \calF'$ we have that for all $B \in \calF$, $A \cap B \ne \emptyset$.
  Hence by applying (b) we are done.
\end{proof}
\begin{rmk}
  From Proposition \ref{prop:FltGrl} (b) or (c) we see that the grill associated with a given filter is simply the collection of all sets we can use to form a filter finer than our given filter.
  However it follows from Proposition \ref{prop:duality}, and working with simply examples shows, that as we generate finer and finer filters, the corresponding grills become coarser.
\end{rmk}

\begin{prop}
  \label{prop:GrlFlt}
  Let $\calG$ be grill on $X$ and put $F(\calG) = \{\, A \in \calG : X \setminus A \not\in \calG \,\}$.
  \begin{itemize}
    \item[(a)] Then $F(\calG)$ is a filter on $X$.
      We call the filter $F(\calG)$ the \textsl{filter associated with the grill $\calG$}.
    \item[(b)] $F(\calG) = \{\, A \in \calG : \mbox{$A \cap B \in \calG$ for all $B \in \calG$} \,\}$.
  \end{itemize}
\end{prop}
\begin{proof}
  For notational convenience put $\calF = F(\calG)$.

  (a)
  Since $X \setminus X = \emptyset \not\in \calG$ we have that $X \in \calF$ and so $\emptyset \ne \calF$.
  Also since $X \setminus \emptyset = X \in \calG$, we have that $\emptyset \not\in \calF$.

  Now let $A$ and $B$ be elements of $\calF$, that is, $X \setminus A$ and $X \setminus B$ are not in $\calG$.
  From Proposition \ref{prop:alt3} it follows that $X \setminus (A \cap B) \not\in \calG$, that is, $A \cap B \in \calF$. 

  Let $A \in \calF$ and $A \subseteq B \subseteq X$.
  Since $X \setminus B \subseteq X \setminus A$ and $X \setminus A \not\in \calG$, it follows from Definition \ref{defn:grill}(a) that $X \setminus B \not\in \calG$, that is, $B \in \calF$.

  (b)
  Let $A \in \calF$ and $B \in \calG$.
  Suppose that $A \cap B \not\in \calG$, then $X \setminus (A \cap B) \in \calF$.
  Hence $\bigl( (X \setminus A) \cup (X \setminus B) \bigr) \cap A \in \calF$, that is, $\emptyset \cup \bigl( (X \setminus B) \cap A) \bigr) \in \calF$.
  However $(X \setminus B) \cap A \in \calF$ if and only if $X \setminus \bigl( X \setminus (B \cap A) \bigr) \not\in \calG$, that is, $B \cup (X \setminus A) \not\in \calG$.
  But this is a contradiction since $B \in \calG$ and $B \subseteq B \cup (X \setminus A)$.
  
  Now let $A \in \calG$ such that $A \cap B \in \calG$ for all $B \in \calG$.
  Since $\emptyset \not\in \calG$ it follows that $X \setminus A \not\in \calG$, that is, $A \in \calF$.
\end{proof}

We now give an easy result that asserts that the functions $G$ and $F$ implicitly defined in Propositions \ref{prop:FltGrl} and \ref{prop:GrlFlt} are one-to-one from the set of all filters onto the set of grills.

\begin{prop}
  Let $X$ be a nonempty set.
  \begin{itemize}
    \item[(a)] If $\calG$ is a grill on $X$, then $\calG = G\bigl(F(\calG)\bigr)$.
    
    \item[(b)] If $\calF$ be a filter on $X$, then $\calF = F\bigl(G(\calF)\bigr)$.
  \end{itemize}
\end{prop}

\begin{defn}
  Let $\calF$ be a filter on $X$ and let $\calG$ be a grill on $X$.
  We say $\calF$ and $\calG$ are \textsl{associated} if and only if $\calF = F(\calG)$ or $\calG = G(\calF)$. 
\end{defn}

With the relation of association we have the following duality type result for filters and grills.

\begin{prop}
  \label{prop:duality}
  Let the filter $\calF_1$ be associated with the grill $\calG_1$, and let the filter $\calF_2$ be associated with the grill $\calG_2$.
  Then $\calF_1 \subseteq \calF_2$ if and only if $\calG_2 \subseteq \calG_1$.
\end{prop}
\begin{proof}
  ($\Rightarrow$)
  Let $A \in \calG_2$, that is, $X \setminus A \not\in \calF_2$.
  Since $\calF_1 \subseteq \calF_2$ we have $X \setminus A \not\in \calF_1$ too.
  Therefore $A \in \calG_1$.

  ($\Leftarrow$)
  Let $A \in \calF_1$.
  If $A \not\in \calF_2$, then $X \setminus A \in \calG_2$ and (by assumption) $X \setminus A \in \calG_1$, that is, $A \not\in \calF_1$, a contradiction.
  Therefore $A \in \calF_2$.
\end{proof}

\begin{prop}
  A grill $\calG$ is a filter if and only if $F(\calG)$ is an ultrafilter.
\end{prop}
\begin{proof}
  ($\Rightarrow$)
  Let $A \subseteq X$.
  We show that either $A \in F(\calG)$ or $X \setminus A \in F(\calG)$.
  Suppose that $A \not\in F(\calG)$, that is, assume that either $A \not\in \calG$ or $A \in \calG$ and $X \setminus A \in \calG$.
  However by hypothesis, $\calG$ is a filter and so we cannot have both $A \in \calG$ and $X \setminus A \in \calG$.
  Since $A \not\in \calG$ and $X \in \calG$ we have that $X \setminus A \in \calG$ and so $X \setminus A \in F(\calG)$.

  ($\Leftarrow$)
  It suffices to show that $\calG$ is closed under finite intersections.
  To this end let $A$ and $B$ be elements of $\calG$.
  However suppose that $A \cap B \not\in \calG$.
  If $A \cap B \in F(\calG)$, then $A \cap B \in \calG$.
  Hence we can assume that $A \cap B \not\in F(\calG)$.
  Therefore since $F(\calG)$ is an ultrafilter we have that $(X \setminus A) \cup (X \setminus B) = X \setminus (A \cap B) \in F(\calG)$.
  Also since $F(\calG)$ is an ultrafilter either $X \setminus A \in F(\calG)$ or $X \setminus B \in F(\calG)$. 
  This last sentence implies that either $A \not\in \calG$ or $B \not\in \calG$, a contradiction.
\end{proof}
\begin{cor}
  Let $\calU$ be a filter.
  $\calU$ is an ultrafilter if and only if $\calU = G(\calU)$.
\end{cor}
\begin{proof}
  Let $\calU$ be an ultrafilter.
  It's immediate that $\calU \subseteq G(\calU)$.
  Let $A \in G(\calU)$, that is, $X \setminus A \not\in \calU$. 
  Since $X \setminus A \not\in \calU$ we must have $A \in \calU$.

  Now suppose that $\calU$ is a filter with $\calU = G(\calU)$.
  Then $\calU = F\bigl(G(\calU)\bigr)$ is an ultrafilter.
\end{proof}

We now shift our attention to grill bases.
While grill bases are easier to describe then grills we shall see that in Proposition \ref{prop:GrillB}(c) that there is generally no coarsest grill that contains a grill base.
(This negative result is in contrast to the fact that there is a coarsest filter that contains a filter base).

\begin{defn}
  Let $X$ be a nonempty set.
  \begin{itemize}
    \item[(b)] We call $\calB \subseteq \calP(X)$ a \textsl{grill base on $X$} if and only if $\calB$     satisfies the following two conditions:
      \begin{itemize}
        \item[(1)] $\emptyset \ne \calB$ and $\emptyset \not\in \calB$.

        \item[(2)] If $A \in \calB$ and $A \subseteq B \subseteq X$, then $B \in \calB$.
      \end{itemize}

    \item[(c)] We call $\calS \subseteq \calP(X)$ a \textsl{grill subbase} if and only if $\emptyset \ne \calS$ and $\emptyset \not\in \calS$. 
  \end{itemize}
\end{defn}

\begin{prop}
  \label{prop:grlBasePhi}
  Let $\calB$ be a grill base on $X$ and put $\Phi = \{\, \calF : \mbox{$\calF \subseteq \calB$ is a filter} \,\}$.
  Then $\Phi$ is nonempty, partially ordered by $\subseteq$, and every \mbox{$\subseteq$-chain} in $\Phi$ has an upper bound in $\Phi$.
\end{prop}
\begin{proof}
  Since $X \in \calB$ and $\{X\}$ is a filter contained in $\calB$ we have $\{X\} \in \Phi$.
  $\Phi$ is obviously partially ordered by $\subseteq$.
  Let $\calC$ be a chain in $\Phi$, then $\bigcup\calC$ is a filter by Lemma \ref{lem:chainFlt} and $\bigcup\calC \subseteq \calB$.
\end{proof}

\begin{prop}
  Let $\calB$ be a grill base on $X$ and let $\Phi$ be as in Proposition \ref{prop:grlBasePhi}.
  Put 
  \[
    F(\calB) = \bigcap\{\, \calF : \mbox{$\calF$ is a maximal filter in $\Phi$} \,\}.
  \]
  Then $F(\calB)$ is a filter on $X$.
  We call the filter $F(\calB)$ the \textsl{filter associated with the base $\calB$}.
\end{prop}
\begin{proof}[Proof Sketch]
  By Zorn's Lemma the set $\Phi$ has maximal elements, and by Lemma \ref{lem:chainFlt} $F(\calB)$ is a filter on $X$.
\end{proof}

\begin{cor}
  $G\bigl(F(\calB)\bigr)$ is a grill on $X$.
  We call the grill $G\bigl(F(\calB)\bigr)$ the \textsl{grill associated with the base $\calB$}.
\end{cor}

\begin{prop}
  \label{prop:GrillB}
  Let $\calB$ be a grill base on $X$.
  \begin{itemize}
    \item[(a)] $F(\calB) = \{\, A \in \calB : \mbox{$A \cap B \in \calB$ for all $B \in \calB$} \,\}$.

    \item[(b)] $\calB \subseteq G\bigl(F(\calB)\bigr)$.

    \item[(c)] $\calB = \bigcap\{\, \calG : \mbox{$\calB \subseteq \calG$ and $\calG$ is a grill} \,\}$.
  \end{itemize}
\end{prop}
\begin{proof}
  (a) 
  Let $A \in F(\calB)$, then for every maximal filter $\calF \subseteq \calB$ we have $A \in \calF$. 
  Let $B \in \calB$, then the set $\{\, C \in \calB : B \subseteq C \,\}$ is a (principal) filter contained in $\calB$.
  By applying Zorn's Lemma, it follows that this filter is contained in a maximal filter $\calF \subseteq \calB$.
  In particular, $A \cap B \in \calF$ and so $A \cap B \in \calB$.

  Now let $A \in \calB$ such that for all $B \in \calB$, $A \cap B \in \calB$.
  Suppose that there exists a maximal filter $\calF \subseteq \calB$ with $A \not\in \calF$. 
  If $A \cap C \ne \emptyset$ for all $C \in \calF$, then $\calF \cup \{A\}$ is a filter subbase contained in $\calB$ that strictly contains the maximal filter $\calF$ in $\calB$.
  Therefore there exists $C \in \calF$ such that $A \cap C = \emptyset$.
  However in this case, $C \in \calB$, $\emptyset = A \cap C \in \calB$, a contradiction since $\emptyset \not\in \calB$.
  Therefore we must have that $A$ is a member of every maximal filter contained in $\calB$.

  (b)
  Observe that $A \in G\bigl(F(\calB)\bigr)$ if and only if $X \setminus A \not\in F(\calB)$ if and only if $X \setminus A \not\in \calB$ or $X \setminus A \in \calB$ and there exists $B \in \calB$ such that $(X \setminus A) \cap B \not\in \calB$.
  Let $A \in \calB$. 
  If $X \setminus A \not\in \calB$, then we're done.
  If $X \setminus A \in \calB$, then $(X \setminus A) \cap A = \emptyset \not\in \calB$.

  (c)
  Put
  \[
    \Gamma = \{\, \calG : \mbox{$\calB \subseteq \calG$ and $\calG$ is a grill} \,\}.
  \]
  First note that $\Gamma \ne \emptyset$ since by (b) $G\bigl(F(\calB)\bigr) \in \Gamma$.
  The fact that $\calB \subseteq \bigcap\Gamma$ is clear.

  Now suppose that $(\bigcap\Gamma) \setminus \calB \ne \emptyset$ and let $A \in (\bigcap\Gamma) \setminus \calB$.
  Since $A \not\in \calB$, we have that $A \not\in F(\calB)$, and moreover $X \setminus A \in G\bigl(F(\calB)\bigr)$.
  Let $\calF$ be the filter generated by $F(\calB) \cup \{X \setminus A\}$. 
  Now $X \setminus A \in \calF$ and so $A \not\in G(\calF)$. 
  To arrive at a contradiction we prove that $\calB \subseteq G(\calF)$.
  To show that $\calB \subseteq G(\calF)$ it suffices to show that for every $B \in \calB$ and $C \in \calF$ we have $B \cap C \ne \emptyset$.
  In order to show to this it suffices to show that for every $B \in \calB$ and every finite nonempty set $\calA \subseteq F(\calB) \cup \{X \setminus A\}$ we have $B \cap \bigcap\calA \ne \emptyset$.
  So let $\calA \subseteq F(\calB) \cup \{X \setminus A\}$ be a finite nonempty subset with $|\calA| = n$.
  Enumerate $\calA = \{D_1, D_2, \ldots, D_n\}$.
  If $D_i \in F(\calB)$ for each $i \in \{1, 2, \ldots, n\}$, then we have that $B \cap \bigcap_{i=1}^n D_i \ne \emptyset$. 
  Now without loss of generality suppose $D_n = X \setminus A$ and $D_i \in F(\calB)$ for each $i \in \{1, 2, \ldots, n-1\}$. 
  Put $D = \bigcap_{i=1}^{n-1} D_i \in F(\calB)$.
  Suppose that $B \cap D \cap (X \setminus A) = \emptyset$, then $B \cap D \subseteq A$.
  Since $D \in F(\calB)$ we have that $B \cap D \in \calB$ and hence $A \in \calB$, a contradiction.
  Therefore $B \cap D \cap (X \setminus A) \ne \emptyset$ and so $\calB \subseteq G(\calF)$.
  However $G(\calF)$ is grill that contains $\calB$ with $A \not\in G(\calF)$, a contradiction. 
\end{proof}

Recall that given a collection of sets $\calA$ we let $\calA^\uparrow$ represent the set $\{\, B : \mbox{$A \subseteq B$ for some $A \in \calA$} \,\}$.

\begin{prop}
  \label{prop:GrillSb}
  Let $\calS$ be a grill subbase on a set $X$.
  Then $\calS^\uparrow$ and
  \[
    B(\calS) = \bigcap\{\, \calB : \mbox{$\calS \subseteq \calB$ and $\calB$ is a grill base} \,\}
  \]
  are both grill bases on $X$ and $\calS^\uparrow = B(\calS)$.
  We call or $B(\calS)$ the \textsl{grill base associated with the grill subbase $\calS$}.
\end{prop}
\begin{proof}
  The fact that $\calS^\uparrow$ is a grill base is clear.
  Put $\Psi = \{\, \calB : \mbox{$\calS \subseteq \calB$ and $\calB$ is a grill base} \,\}$ and observe that $\Psi$ is nonempty since $\calS^\uparrow \in \Psi$. 
  Put $B(\calS) = \bigcap\Psi$.

  Now $B(\calS)$ is nonempty since $X \in \calB$ for every grill base and so in particular $X \in B(\calS)$.
  Also since the empty set is not an element of any grill base we have that $\emptyset \not\in B(\calS)$.

  Let $A \in B(\calS)$ and $A \subseteq B \subseteq X$. 
  For every grill base $\calB$ with $\calS \subseteq \calB$ we have $A \in \calB$ and hence $B \in \calB$.
  Therefore $B \in B(\calS)$.

  Finally, we show that $B(\calS) = \calS^\uparrow$.
  By construction, $B(\calS) \subseteq \calS^\uparrow$.
  Let $A \in \calS^\uparrow$ and pick $B \in \calS$ such that $B \subseteq A$. 
  Let $\calB$ be a grill base with $\calS \subseteq \calB$.
  Since $B \in \calB$, we have that $A \in \calB$.
  Therefore $A \in B(\calS)$.
\end{proof}

\begin{cor}
  Let $\calS$ be a grill subbase on a set $X$.
  Then $F(S^\uparrow)$ is a filter on $X$ which we call the \textsl{filter associated with the grill subbase $\calS$}; and $G\bigl(F(\calS^\uparrow)\bigr)$ is a grill on $X$ with $\calS \subseteq G\bigl(F(\calS^\uparrow)\bigr)$ which we call the \textsl{grill associated with the grill subbase $\calS$}. 
\end{cor}
\begin{proof}
  The assertions that $F(\calS^\uparrow)$ is a filter and $G\bigl(F(\calS^\uparrow)\bigr)$ is a grill is follows immediately.
  To see that $\calS \subseteq G\bigl(F(\calS^\uparrow)\bigr)$ observe that $\calS \subseteq \calS^\uparrow$.
\end{proof}

With a grill subbase we have a simple and trivial way to generate filters.
Therefore if we are trying to study a nonempty collection of sets that doesn't contain the empty set, then it may be easier to study the associated filter.
In the next section we give a definition for a special type of grill subbase for which studying the corresponding filters has been successful.


\section{Partition Regularity}

\begin{defn}
  Let $\calS$ be a grill subbase on some nonempty set.
  We say that the $\calS$ is \textsl{partition regular} if and only if whenever $\calA$ is a finite collection of sets with $\bigcup\calA \in \calS$, then there exist $A \in \calA$ and $B \in \calS$ such that $B \subseteq A$.
\end{defn}
\begin{rmk}
Partition regularity roughly asserts that some property of some set $X$, here represented by members of $\calS$, occurs a ``large'' number of times in $X$.
In fact so large, that no matter how we finitely divide up $X$, at least one cell in the division has our specified property. 
\end{rmk}

The following important theorem, taken from \cite[Theorem 3.11]{Hindman:1998fk}, shows the intimate connection between partition regular sets and ultrafilters.

\begin{thm}
  Let $X$ be a set and $\calS$ a grill subbase on $X$.
  The following statements are equivalent.
  \begin{itemize}
    \item[(a)] $\calS$ is partition regular.
    \item[(b)] $\calS^\uparrow$ is partition regular.
    \item[(c)] $\calS^\uparrow$ is a grill.
    \item[(d)] If $\emptyset \ne\calA \subseteq \calP(X)$ has the property that every finite nonempty subfamily of $\calA$ has an intersection in $\calS^\uparrow$, then there is an ultrafilter $\calU$ on $X$ with $\calA \subseteq \calU \subseteq \calS^\uparrow$.
    \item[(e)] If $A \in \calS$, then there is some ultrafilter $\calU$ on $X$ such that $A \in \calU \subseteq \calS^\uparrow$.
  \end{itemize}
\end{thm}
\begin{proof}
  $(a) \Rightarrow (b)$
  Clearly $\calS^\uparrow$ is a grill subbase since $\calS$ is one.
  Let $\calA$ be a finite collection of sets with $\bigcup\calA \in \calS^\uparrow$.
  Pick $B \in \calS$ such that $B \subseteq \bigcup\calA$. 
  Since $\bigcup_{A \in \calA} (A \cap B) = B \in \calS$ and $\calS$ is partition regular, we may pick $C \in \calS$ and $A \in \calA$ such that $C \subseteq A \cap B$. 
  By definition of $\calS^\uparrow$ we have that $A \cap B \in \calS^\uparrow$ and $A \in \calS^\uparrow$.

  $(b) \Rightarrow (c)$
  By Proposition \ref{prop:GrillSb}, $\calS^\uparrow$ is a grill base.
  To see that $\calS^\uparrow$ is a grill let $A$ and $B$ be subsets of $X$ with $A \cup B \in \calS^\uparrow$.
  Since $\calS^\uparrow$ is partition regular, we may, without loss of generality, assume that there exists $C \in \calS^\uparrow$ with $C \subseteq A$.
  Hence $A \in \calS^\uparrow$.
  Therefore $\calS^\uparrow$ is a grill.

  $(c) \Rightarrow (d)$
  By hypothesis $\calS^\uparrow$ is a grill and so $F(\calS^\uparrow)$ is a filter with $F(\calS^\uparrow) \subseteq G\bigl(F(\calS^\uparrow)\bigr) = \calS^\uparrow$.
  By assumption it follows that $F(\calS^\uparrow) \cup \calA$ has f.i.p. so let $\calF$ be the filter generated by $F(\calS^\uparrow) \cup \calA$. 
  Since $\calF$ is a finer filter than $F(\calS^\uparrow)$ we have that $G(\calF) \subseteq \calS^\uparrow$. 
  Therefore there exists an ultrafilter $\calU$ with $\calA \subseteq \calU \subseteq \calS^\uparrow$.

  $(d) \Rightarrow (e)$
  Put $\calA = \{A\}$.

  $(e) \Rightarrow (a)$
  Let $\calA$ be a finite nonempty family of sets with $\bigcup\calA \in \calS$. 
  By hypothesis, there exists some ultrafilter $\calU$ on $X$ with $\bigcup\calA \in \calU \subseteq \calS^\uparrow$.
  Since $\calU$ is an ultrafilter, we may pick $A \in \calA$ such that $A \in \calU$. 
  Then since $\calU \subseteq \calS^\uparrow$, pick $B \in \calS$ such that $B \subseteq A$.
  Therefore $\calS$ is partition regular.
\end{proof}

In this dissertation, we shall mainly be concerned with a pair $(S, \calS)$, where $S$ is a semigroup, $\calS$ is partition regular, and elements of $\calS$ are defined in terms of the algebraic structure of $S$.

\section{Algebraic and Topological Semigroup Theory}
Given our algebraic point-of-view we now give a introduction to algebraic semigroup theory.
Since most of the material in this section is standard, we choose to omit the proofs.
The interested reader can consult the article \cite{Hollings:2007uq} for a brief introduction or the monographs \cite{Clifford:1961fk} and \cite{Clifford:1967fk} for more extensive information. 
For our purposes the semigroup theory contained in \cite[Chapters 1 and 2]{Hindman:1998fk} will suffice. 

\begin{defn}
  Let $S$ be a nonempty set, let $\cdot \colon S \times S \to S$ be
  a binary operation on $S$, and let $J \subseteq S$ be a nonempty
  subset of $S$.
  \begin{itemize}
    \item[(a)] We call the pair $(S, \cdot)$ a \textsl{semigroup} if
      and only if for all $x$, $y$, and $z \in S$, $(x \cdot y) \cdot
      z = x \cdot (y \cdot z)$.
    \item[(b)] For each $x \in S$ define the functions $\lambda_x
      \colon S \to S$ and $\rho_x \colon S \to S$ by $\lambda_x(y) = x
      \cdot y$ and $\rho_x(y) = y \cdot x$, respectively.
    \item[(c)] We call an element $x \in S$ \textsl{idempotent} if and
      only if $x\cdot x = x$. 
    \item[(d)] We say $J$ is a \textsl{left ideal (of S)} if and only
      if $S \cdot J \subseteq J$.
      A left ideal is a \textsl{minimal left ideal} if and only if the
      only left ideal it contains is itself.
    \item[(e)] We say $J$ is a \textsl{right ideal (of S)} if and only
      if $J \cdot S \subseteq J$.
      A right ideal is a \textsl{minimal right ideal} if and only if the
      only right ideal it contains is itself.
    \item[(f)] We say $J$ is a \textsl{(two-sided) ideal (of S)} if and only
      if $J$ is both a left ideal and right ideal.
      An ideal is a \textsl{minimal ideal} if and only if the only
      ideal it contains is itself.
  \end{itemize}
\end{defn}


The following two easy results show that a semigroup contains at most one minimal ideal, and that if a semigroup those has a minimal ideal, then this ideal is contained in every ideal of our semigroup.

\begin{thm}
  Let $S$ be a semigroup.
  If $J_1$ and $J_2$ are minimal ideals of $S$, then $J_1 = J_2$.
\end{thm}
\begin{proof}
  Follows from \cite[Lemma 1.29]{Hindman:1998fk}.
\end{proof}
\begin{thm}
  Let $S$ be a semigroup with the minimal ideal $J$. 
  If $I$ is an ideal of $S$, then $J \subseteq I$.
\end{thm}
\begin{proof}
  \cite[Lemma 1.49]{Hindman:1998fk}
\end{proof}

With these results we are justified in making the following important definition.
\begin{defn}
  Let $(S,\cdot)$ be a semigroup. 
  We call the minimal ideal of $S$ the \textsl{smallest ideal} and denote it by $K(S)$.%
  \endnote{
    The notation of $K(S)$ denoting the smallest ideal comes from the Russian mathematician Suschkewitsch.
    He called the smallest ideal the \textsl{kernel} of the semigroup.
    Suschkewitsch, based off of the description given in \cite{Hollings:2009uq}, in \cite{Suschkewitsch:1928kx} proved that all finite semigroups have a smallest ideal and using this fact was able to produce a structure theorem for finite semigroups. 
  }
\end{defn}
Unfortunately not all semigroups have a smallest ideal or idempotents. 
However, we will only be concerned with a certain class of topological semigroups that has both of these objects.

\begin{defn}
  A \textsl{compact right-topological semigroup} is a triple $(S,\calT, \cdot)$ such that $(S, \calT)$ is a compact Hausdorff space, $(S, \cdot)$ is a semigroup, and for every $x \in S$, $\rho_x \colon S \to S$ is continuous. 
\end{defn}

\begin{thm}
  Compact right-topological semigroups contains idempotents and the smallest ideal.
\end{thm}
\begin{proof}
  The existence of idempotents and the smallest ideal is proved in \cite[Theorem 2.5]{Hindman:1998fk} and \cite[Theorem 2.8]{Hindman:1998fk}, respectively.
\end{proof}





\section{Stone-\v{C}ech Compactification of a Discrete Space}
\section{Algebra in the Stone-\v{C}ech Compactification of a Discrete Semigroup}
\section{Ramsey Theoretic Motivation}
\section*{Open Problems}
\section{Ultrafilters and Partition Regularity}

% While Ramsey's Theorem is important and provides a good example of a
% partition regular pair $(X, \calR)$, in this dissertation we will
% mainly be considering partition regular sets where the underlying set
% $X$ has an algebraic structure of a semigroup and elements in $\calR$
% are defined in terms of this structure. 

\begin{schur}
  Let $r \in \bbN$ and $\bbN = \bigcup_{i=1}^r C_i$.
  Then there exist $i \in \{1, 2, \ldots, r\}$ and $x$, $y$, and $z
  \in \bbN$ such that $\{\, x, y, x+y \,\} \subseteq C_i$.
\end{schur}

\begin{vdw}
  Let $r \in \bbN$ and $\bbN = \bigcup_{i=1}^r C_i$.
  Then for every $\ell \in \bbN$, there exist $i \in \{1, 2, \ldots,
  r\}$ and $a$, $d \in \bbN$ such that $\{\, a, a+d, \ldots, a+\ell d
  \,\} \subseteq C_i$.
\end{vdw}

% Endnotes 
\theendnotes

% Things referenced in the preliminaries chapter. Eventually this will
% placed in a separate file so the References appear at the end.
\bibliographystyle{amsplain}
\bibliography{../references}

\end{document}