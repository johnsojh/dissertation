% Draft of my Dynamical Characterization of C-sets Results
% by: John H. Johnson
% email: john.j.jr@gmail.com

\documentclass[12pt]{article}

% \usepackage[doublespacing]{setspace}
\usepackage{amsthm, amssymb, amsmath}
\usepackage[margin=1.5in]{geometry}
\usepackage{endnotes}

\newtheoremstyle{plain}{3mm}{3mm}{\slshape}{}{\bfseries}{.}{.5em}{}
\theoremstyle{plain}
\newtheorem{thm}{Theorem}[section]
\newtheorem{vdW}[thm]{Van der Waerden's Theorem}
\newtheorem{FS}[thm]{Hindman's Theorem}
\newtheorem{MBR}[thm]{Multiple Birkhoff Recurrence Theorem}
\newtheorem{recur}[thm]{Recurrence Theorem}
\newtheorem{OCST}[thm]{Furstenburg's Original Central Sets Theorem}
\newtheorem{cor}[thm]{Corollary}
\newtheorem{prop}[thm]{Proposition}
\newtheorem{lem}[thm]{Lemma}
\newtheorem{claim}[thm]{Claim}
\newtheorem{ques}[thm]{Question}
\newtheorem{conj}[thm]{Conjecture}
\newtheorem{fact}[thm]{Fact}

\theoremstyle{definition}
\newtheorem{defn}[thm]{Definition}
\newtheorem{example}[thm]{Example}
\newtheorem{rmk}[thm]{Remark}

\newcommand{\la}{\langle}
\newcommand{\ra}{\rangle}
\newcommand{\bbN}{\mathbb{N}}
\newcommand{\ds}{(X, \la T_s \ra_{s\in S})}
\newcommand{\setfunc}[2]{\hbox{${}^{\hbox{$#1$}}\hskip -3 pt #2$}}

\font\bigmath=cmsy10 scaled \magstep 3
\newcommand{\bigtimes}{\hbox{\bigmath \char'2}}

\newcommand{\cchi}{\raise 2 pt \hbox{$\chi$}}

\begin{document}
% Much of this Introduction section will probably eventually be folded
% into the Introduction chapter.
\section{Introduction}
One of the early important theorems in Ramsey Theory
is van der Waerden's Theorem.%
\endnote{The field Ramsey Theory is named in honor of Frank P.~Ramsey,
  who contributed the important Ramsey's Theorem.
  This theorem can be thought of as a powerful generalization of the
  Pigeonhole Principle.
  Curiously, van der Waerden's Theorem was published, in 1927
  \cite{Van-der-Waerden:1927fk}, 
  three years early than the publication of Ramsey's Theorem in
  \cite{Ramsey:1930uq}.
  In fact, there are several important Ramsey Theoretic results
  published before Ramsey's Theorem.
}
This theorem can be stated in several equivalent ways, but for our
purposes it is useful to state it as follows.

  \begin{vdW}
    If $k \in \bbN$ and $\bbN = \bigcup_{i=1}^k A_i$, then for each $l \in
    \bbN$, there exist $i \in \{1, 2, \ldots, k\}$ and $a$, $d \in
    \mathbb{N}$ such that 
      \[
        \{\, a, a+d, \ldots, a+ld \,\} \subseteq A_i.
      \]
  \end{vdW}

Van der Waerden's proof of this theorem uses a complicated,
but elementary, combinatorial argument.%
\endnote{Actually I'm not sure of the exact form of van der Waerden's
  proof of his theorem. 
  This statement is based on \cite[Section 1, pages
  11--12]{Khinchin:1998fk}.
  In this book Khinchin doesn't give the original proof, but he states
  that van der Waerden explained the original proof to him. 
  Khinchin says, ``It was elementary, but not simple by any means''
  \cite[Section 1, page 11]{Khinchin:1998fk}.
  The combinatorial proof in this book is due to M.A.~Lukomskaya, and
  again according to Khinchin it is ``considerably simpler and more
  transparent'' \cite[Section 1, page 12]{Khinchin:1998fk}.
  For a very short combinatorial proof of van der Waerden's Theorem
  see \cite{Graham:1974uq}
}
Since the original argument, there have been several proofs of van der
Waerden's Theorem. 
Combinatorial proofs, using the pigeonhole principle and ``double
induction'', can be found in \cite[Chapter 1, pages
11--17]{Khinchin:1998fk} and \cite{Graham:1974uq}.
In my (necessarily biased) opinion, the easiest proof of van der
Waerden's Theorem is via the algebraic structure of the Stone-\v{C}ech
compactification of $\bbN$ \cite[Chapter 14, Section 1, pages
279--281]{Hindman:1998fk}.
Somewhat related to this last method is a proof of van der
Waerden's Theorem via topological dynamics due originally to
Furstenberg and Weiss in \cite{Furstenberg:1978kx}.

As motivation and warmup for this chapter, we shall sketch a
proof of van der Waerden's Theorem using topological dynamics.  
To proceed with this proof we will use the following ``recurrence''
type theorem.
Truthfully, all we have done is to hide the real difficulties in the
proof of van der Waerden's Theorem into the proof of this Recurrence
Theorem. 

In what follows, if $X$ is a set, $T : X \to X$ a function, and $n \in
\bbN$, then the notation $T^n$ means the $n$-fold composition of $T$
with itself.
So for example, $T^1 = T$, $T^2 = T \circ T$, $T^3 = T \circ T \circ
T$, and so on. 

  \begin{recur}
    Let $X$ be a compact Hausdorff space, $l \in \bbN$, and let $T_i
    : X \to X$ be a continuous map for all $i \in \{1, 2, \ldots, l\}$
    such that for all $i$, $j \in \{1, 2, \ldots, l\}$
    we have $T_i \circ T_j = T_j \circ T_i$.
    Then there exists a point $x \in X$ such that for all
    neighborhoods $U$ of $x$, there exists $n \in \mathbb{N}$ such
    that $U \cap \bigcap_{i=1}^l (T_i^n)^{-1}[U] \ne \emptyset$.%
      \endnote{
        This Recurrence Theorem follows easily from the following stronger
        result. 
% I should at least have a reference for the proof of my version of
% the Multiple Birkhoff Recurrence Theorem. Usually the Multiple
% Birkhoff Recurrence Theorem is stated as being true for compact
% metric spaces. However using the principle that anything true for
% compact metric spaces is true for compact Hausdorff spaces, then it
% becomes "intuitively obvious" that my version is true also. Maybe I
% should try to give a proof (perhaps in an Appendix) of the Multiple
% Birkhoff Recurrence Theorem using [HS, Exercise 19.3.1].
        \begin{MBR}
          Let $X$ be a compact Hausdorff space, $l \in \bbN$, and let
          $\bigl\la T_i : X \to X : i \in \{1, 2, \ldots, l\} \bigr\ra$ be
          a finite sequence of continuous maps such that for all $i$, $j
          \in \{1, 2, \ldots, l\}$ we have $T_i \circ T_j = T_j \circ
          T_i$. 
          Then there exist $x \in X$ and a strictly increasing sequence
          $\la n_k \ra_{k=1}^\infty$ in $\bbN$ such that $\lim_{k \to
            \infty} T_i^{n_k}(x) = x$ for all $i \in \{1, 2, \ldots, l\}$.
        \end{MBR}
        \begin{proof}[Pointers to proof]
          \textsc{Citation needed for proof of this statement.}
        \end{proof}
    
        \begin{recur}
          Let $X$ be a compact Hausdorff space, $l \in \bbN$, and let
          $\bigl\la T
          : X \to X : i \in \{1, 2, \ldots, l\} \bigr\ra$ be a finite
          sequence of
          continuous maps such that for all $i$, $j \in \{1, 2, \ldots, l\}$
          we have $T_i \circ T_j = T_j \circ T_i$.
          Then there exists a point $x \in X$ such that for all
          neighborhoods $U$ of $x$, there exists $n \in \mathbb{N}$ such
          that $U \cap \bigcap_{i=1}^l (T_i^n)^{-1}[U] \ne \emptyset$.
        \end{recur}
        \begin{proof}
          Pick $x \in X$ and a strictly increasing sequence $\la n_k
          \ra_{k=1}^\infty$ as guaranteed by the Multiple Birkhoff
          Recurrence Theorem.
          Let $U$ be a neighborhood of $x$.
          Then for all $i \in \{1, 2, \ldots, l\}$, there exists $K_i \in
          \bbN$ such that if $k \ge K_i$, then $T_i^{n_k}(x) \in U$, that
          is $x \in (T_i^{n_k})^{-1}[U]$. 
          Put $K = \max\{K_1, K_2, \ldots, K_l\}$.
          If $k \ge K$, then $x \in U \cap \bigcap_{i=1}^l (T_i^{n_k})^{-1}[U]$.
        \end{proof}
      }
  \end{recur}

This result roughly states that when given any finite collection of
commutating continuous functions there is a point in our space that is
not moved too far by our iterated functions. 

Using this theorem we can now provide a relatively straightforward
proof of van der Waerden's Theorem. 

  \begin{vdW}
    If $k \in \bbN$ and $\bbN = \bigcup_{i=1}^k A_i$, then for each $l \in
    \bbN$, there exist $i \in \{1, 2, \ldots, k\}$ and $a$, $d \in
    \mathbb{N}$ such that 
      \[
        \{\, a, a+d, \ldots, a+ld \,\} \subseteq A_i.
      \]
  \end{vdW}
  \begin{proof}
    The proof in outline runs as follows.
    First, we construct a compact Hausdorff space such that partitions
    of $\bbN$ with at most $k$ cells correspond with points in our
    topological space. 
    Once this correspondence is given, we can recast the conclusion of
    van der Waerden's Theorem into a statement about points in our
    space. 
    This construction is the topological part in our topological
    dynamics proof.
    Secondly, for the dynamical part, we construct a continuous
    function from our space to itself. 
    With this map we can apply the Recurrence Theorem to produce a
    point that corresponds to the conclusion of van der
    Waerden's Theorem. 

    We'll start with the construction of our compact Hausdorff space. 
    Put $X = \setfunc{\omega}{\{1, 2, \ldots, k \}}$, where $\omega =
    \bbN \cup \{0\}$. 
    Give each set $\{1, 2, \ldots, k\}$ the discrete topology, and give $X$
    the product topology. 
    Then $X$ is a compact Hausdorff space. 
    We can now translate van der Waerden's Theorem into a statement
    about certain points of $X$. 
    Given $\bbN = \bigcup_{i=1}^k A_i$, we can define a point $y \in
    X$ by, for $n \in \bbN$, $y(n) = i$ if and only if $i = \min\bigl\{\, j
    \in \{1, 2, \ldots, k\} : n \in A_j \,\bigr\}$.
    We define $y(0)$ arbitrarily. 
    (We can also note that any point in $X$ determines a finite
    partition of $\bbN$ into $k$ cells, but this fact will not be used 
    in the proof.)
    Observe that it is now necessary and sufficient to show that
    for every $l \in \bbN$, there exists $a$, $d \in \bbN$ such 
    that $y(a) = y(a+d) = \cdots = y(a+ld)$.

    With our topological space constructed, we now focus on the
    ``dynamical part'' of our proof.
    Define the function $T : X \to X$ by $T(x)(n) = x(n+1)$. 
    We will show that $T$ is continuous. 
    
    For $i \in \omega$ let $\pi_i : X \to \{1, 2, \ldots, k\}$ be the
    usual projection map given by $\pi_i(x) = x(i)$. 
    Recall that $\bigl\{\, \pi_i^{-1}[\{j\}] : \hbox{$i \in \omega$ and $j
      \in \{1, 2, \ldots, k\}$} \,\bigr\}$ is a subbasis for $X$. 
    Let $i \in \omega$ and $j \in \{1, 2, \ldots, k\}$, then
      \begin{align*}
        x \in T^{-1}\bigl[\pi_i^{-1}[\{j\}]\bigr] &\iff (\pi_i \circ
        T)(x) = j, \\
        &\iff T(x)(i) = j, \\
        &\iff x(i+1) = j, \\
        &\iff x \in \pi_{i+1}^{-1}[\{j\}].
      \end{align*}
    Hence $T^{-1}\bigl[\pi_i^{-1}[\{j\}]\bigr] =
    \pi_{i+1}^{-1}[\{j\}]$, and so $T$ is continuous. 

    We will now construct a compact subspace of $X$ that is invariant
    under our map $T$.
    We will then apply our Recurrence Theorem to this particular subspace.
    To this end, put $Y = c\ell_X\bigl(\{\, T^a(x) : a \in \bbN
    \,\}\bigr)$, and for each $i \in \{1, 2, \ldots, l\}$ put $T_i =
    T^i$. 
    By definition, it's evident that $Y$ is a compact Hausdorff
    subspace of $X$. 
    We will show that $T[Y] \subseteq Y$. 
    If we can show this claim, then it follows that $T_i[Y] \subseteq
    Y$ for all $i \in \{1, 2, \ldots, l\}$. 
    
    Let $y \in Y$, and let $U$ be a neighborhood of $T(y)$. 
    Since $T$ is continuous we can pick a neighborhood $V$ of $y$ such
    that $T[V] \subseteq U$. 
    By definition of $Y$, we have that $V \cap \{\, T^a(x) : a \in
    \bbN \,\} \ne \emptyset$. 
    Pick $a \in \bbN$ such that $T^a(x) \in V$. 
    Then $ T^{a+1}(x) = (T \circ T^a)(x) \in T[V] \subseteq U$.
    Hence $U \cap \{\, T^a(x) : a \in \bbN \,\} \ne \emptyset$, and it
    follows that $T(y) \in Y$. 

    We can now say that the finite sequence $\la T_1 \upharpoonright_Y
    , T_2 \upharpoonright_Y, \ldots, T_l \upharpoonright_Y
    \ra$ of functions are commutating continuous functions of $Y$ into
    itself.
    Hence we can apply the Recurrence Theorem to pick a point $y \in
    Y$ such that for all neighborhoods $U$ of $y$, there exists $d \in
    \bbN$ such that $(U \cap Y) \cap \bigcap_{i=1}^l (T_i^d)^{-1}[U
    \cap Y] \ne \emptyset$.
    Put $U = \pi_0^{-1}[\{y(0)\}]$. 
    Then $U$ is an open neighborhood of $y$ and 
    $U \cap \bigcap_{i=1}^l(T_i^d)^{-1}[U] \cap Y = (U \cap Y) \cap
    \bigcap_{i=1}^l (T_i^d)^{-1}[U \cap Y] \ne \emptyset$. 
    Hence we can pick $a \in \mathbb{N}$ such that $T^a(x) \in U \cap
    \bigcap_{i=1}^l(T_i^d)^{-1}[U]$. 
    
    Note that for all $i \in \{1, 2, \ldots, l\}$, we have
    $(T_i^d)^{-1}[U] = (T^{id})^{-1}[U]$. 
    By our choice of $a \in \bbN$, we have that $T^{a+id}(x) \in U$
    for all $i \in \{0, 1, \ldots, l\}$. 
    Hence $y(0) = \pi_0\bigl(T^{a+id}(x)\bigr)$ for all $i \in \{0, 1,
    \ldots, l\}$. 
    Since $x(a+id) = T^{a+id}(x)(0)$ for all $i \in \{0, 1, \ldots,
    l\}$, it follows that $x(a) = x(a + d) = \cdots = x(a+ ld)$. 
    This completes our proof.% 
    \endnote{
      This method of proving van der Waerden's Theorem via topological
      dynamics is due to Furstenberg and Weiss in
      \cite{Furstenberg:1978kx}. 
      For a nice and more leisurely exposition of the proof of van der
      Waerden's Theorem via topological dynamics see the article by
      Mulvey \cite{Mulvey:1997vn}.
    }
  \end{proof}

It turns out that our use of the Recurrence Theorem has been rather
weak.
It's possible to strengthen the conclusion, and derive several
different van der Waerden type
theorems from the Recurrence Theorem.
Just to name drop one extension it's possible to derive the
so called Multidimensional van der Waerden Theorem from the
Recurrence Theorem.%
\endnote{
  Despite the name, the Multidimensional van der Waerden Theorem is
  \textsl{not} due to van der Waerden himself. 
  A proof of the Multidimensional van der Waerden Theorem, using the
  Multiple Birkhoff Theorem, can be found in \cite[Chapter 2, Section
  3, pages 46--47]{Furstenberg:1981fk}.
}
We won't be concerned with all of the possible extensions of van
der Waerden's Theorem save one: Furstenberg's Central Sets Theorem.

Furstenburg in \cite[Chapter 8]{Furstenberg:1981fk}, using notions of
topological dynamics, defined the concept of a central subset of
$\bbN$.
He was able to prove three important facts about central sets.
  \begin{itemize}
    \item[(1)] If a central set is finitely partitioned, then at least
      one cell in the partition is a central set 
      \cite[Theorem 8.8, page 161]{Furstenberg:1981fk}.
    \item[(2)] Central sets satisfy a version of the Central Sets
      Theorem \cite[Theorem 8.21]{Furstenberg:1981fk}.
      (We shall give a statement of this theorem shortly.)
    \item[(3)] Central sets contain solutions to finite kernel
      partition regular matrices \cite[Theorem 8.22, page
      172]{Furstenberg:1981fk}. 
      % (Since this last property takes a bit of time to define properly,
      % we will skip over giving a proper definition of finite kernel
      % partition regular matrices.)
  \end{itemize}

The concept of a central set can be generalized to all semigroups and
admits a simple definition using the algebraic structure of the
Stone-\v{C}ech compactification of our semigroup. 
We chose to give the algebraic definition of central sets since it is
through this algebra that the Central Sets Theorem has recently been
extended.
We give a brief review of the algebraic structure of the
Stone-\v{C}ech compactification of a discrete semigroup.

% These next three paragraphs will probably eventually be removed and
% put into the Intro.

Given a discrete semigroup $(S, \cdot)$, we let $\beta S$ be the
collection of all ultrafilters on $S$, and identify the principal
ultrafilters with the points of $S$. 
For all $A \subseteq S$, we let $\overline{A} = \{\, p \in \beta S : A
\in p \,\}$. 
Then $\{\, \overline{A} : A \subseteq S \,\}$ is a basis of open sets
of $\beta S$. 
We can extend the semigroup operation on $S$ to $\beta S$ in the
following manner.
For $p$, $q \in \beta S$ and for all $A \subseteq S$, $A \in p \cdot
q$ if and only if $\{\, x \in S: x^{-1}A \in q \,\} \in p$, where
$x^{-1}A = \{\, y \in S : xy \in A \,\}$. 

With our definitions, $(\beta S, \cdot)$ becomes a compact Hausdorff
right-topological semigroup with $S$ contained in the topological
center of $\beta S$.
What makes $\beta S$ a right-topological semigroup is that for all $q
\in \beta S$, the map $\rho_q : \beta S \to \beta S$ given by
$\rho_q(p) = pq$ is continuous. 
A point $x \in \beta S$ is in the topological center of $\beta S$ if
and only if the map $\lambda_x : \beta S \to \beta S$, defined by
$\lambda_x(p) = xp$ is continuous.

It's a fact that any compact Hausdorff right-topological semigroup $T$ has
a smallest ideal, denoted by $K(T)$. 
We can now give the definition of a central subset of a discrete
semigroup $S$.

  \begin{defn}
    Let $(S, \cdot)$ be a discrete semigroup, and let $C \subseteq S$.
    We say $C$ is a \textsl{central} set if and only if there exists
    $p \in K(\beta S)$ such that $p = p \cdot p$ and $C \in p$.
  \end{defn}

% I'm definitely cheating some people out of credit here! Need to go
% back and insert the full origin on the algebraic definition of
% central sets and their equivalence to the definition via topological
% dynamics. 
The fact that this definition is equivalent to Furstenberg's original
definition is proved in \cite[Chapter 19, Section 3,
pages 404--407]{Hindman:1998fk}.

We can now formulate the Central Sets Theorem for (commutative)
semigroups.
(The reason we give the theorem only for commutative semigroups now,
is that the general version of the Central Sets Theorem is quite
complicated to state. 
We will eventually give the ``full'' version of the Central Sets
Theorem however.)

  \begin{thm}
    \label{thm:cst2}
    Let $(S,+)$ be a commutative semigroup, put $\mathcal{T} =
    \setfunc{\bbN}{S}$, and let $C \subseteq S$ be central.
    Let $F \in \mathcal{P}_f(\mathcal{T})$. 
    Then there exist sequences $\la a_n \ra_{n=1}^\infty$ in $S$ and
    $\la H_n \ra_{n=1}^\infty$ in $\mathcal{P}_f(\bbN)$ such that
      \begin{itemize}
        \item[(1)] $\max H_n < \min H_{n+1}$ for all $n \in \bbN$, and
        \item[(2)] for all $G \in \mathcal{P}_f(\bbN)$ and every $f
          \in F$, we have
          \[
            \sum_{n \in G}\Bigl(a_n + \sum_{t \in H_n} f(t)\Bigr) \in C.
          \]
      \end{itemize}
  \end{thm}

There are currently four different versions of the Central Sets
Theorem. 
(Theorem \ref{thm:cst2} is the second version.)
Currently the most general version of the Central Sets Theorem for
commutative semigroups is the following. 

  \begin{thm}
    \label{thm:newcst}
    Let $(S,+)$ be a commutative semigroup, put $\mathcal{T} =
    \setfunc{\bbN}{S}$, and let $C \subseteq S$ be central. 
    Then there exist functions $\alpha : \mathcal{P}_f(\mathcal{T})
    \to S$ and $H : \mathcal{P}_f(\mathcal{T}) \to \mathcal{P}_f(\bbN)$ such
    that
      \begin{itemize}
        \item[(1)] if $F$, $G \in \mathcal{P}_f(\mathcal{T})$ and $F
          \subsetneq G$, then $\max H(F) < \min H(G)$, and
        \item[(2)] whenever $m \in \bbN$, $\la G_1, G_2, \ldots, G_m
          \ra$ is a finite sequence in $\mathcal{P}_f(\mathcal{T})$
          with $G_1 \subsetneq G_2 \subsetneq \cdots \subsetneq G_m$,
          and for each $i \in \{1, 2, \ldots, m\}$, $f_i \in G_i$ we
          have
          \[
            \sum_{i=1}^m\Bigl(\alpha(G_i)+\sum_{t \in H(G_i)}
            f_i(t)\Bigr) \in C.
          \]
      \end{itemize}
  \end{thm}


Given that the Central Sets Theorem guarantees that central sets have
strong combinatorial properties, it's natural to wonder if the Central
Sets Theorem characterizes centrals sets. 
Unfortunately, there is evidence that the Central Sets Theorem (even
the newest one) misses some of the combinatorial richness of central
sets.%
\endnote{
  See \cite[Section 14.5, pages 288--294]{Hindman:1998fk} for a
  combinatorial characterization of central sets. 
  This characterization uses a complicated notion of a
  \textsl{collectionwise piecewise syndetic} set.
  In my opinion, since $C$-sets have a much simpler combinatorial
  characterization, this suggest that the Central Sets Theorem misses
  much of the combinatorial richness of central sets.
}
Since most of the applications of central sets involved the use of the
Central Sets Theorem, recently the focus has shifted to the study of
those sets which satisfy the conclusion of Theorem \ref{thm:newcst}. 
We will now give a definition of these sets.

  \begin{defn}
    Let $(S, \cdot)$ be a semigroup.
      \begin{itemize}
        \item[(a)] For each $m \in \bbN$, put 
          \begin{align*}
            \mathcal{I}_m &= \Bigl\{\, \bigl( H(1), H(2), \ldots, H(m)
            \bigr) : \hbox{$H(i) \in \mathcal{P}_f(\bbN)$ for all $i
              \in \{1, 2, \ldots, m\}$} \\
            &\hbox{ and $\max H(i) < \min H(i+1)$
            for all $i \in \{1, 2, \ldots, m-1\}$} \Bigr\}
         \end{align*}
        \item[(b)] $\mathcal{T} = \setfunc{\bbN}{S}$. 
        \item[(c)] Given $m \in \bbN$, $a \in S^{m+1}$, $H \in
          \mathcal{I}_m$, and $f \in \mathcal{T}$, put 
            \[
            x(m, a, H, f) = \prod_{i=1}^m\Bigl(a(i)\prod_{t \in H(i)}
            f(t) \Bigr)a(m+1).
            \]
        \item[(d)] A subset $A \subseteq S$ is a \textsl{$J$-set} if
          and only if for all $F \in \mathcal{P}_f(\mathcal{T})$ there
          exist $m \in \bbN$, $a \in S^{m+1}$, and $H \in \mathcal{I}_m$
          such that for all $f \in F$ we have $x(m,a,H,f) \in A$.
        \item[(e)] $J(S) = \{\, p \in \beta S : \hbox{$A$ is a $J$-set
            for all $A \in p$} \,\}$.
        % How do you create a big times symbol again?
        \item[(f)] A subset $C \subseteq S$ is a \textsl{$C$-set} if
          and only if there exist functions $m :
          \mathcal{P}_f(\mathcal{T}) \to \bbN$, $\alpha \in
          \bigtimes_{F \in \mathcal{P}_f(\mathcal{T})} S^{m(F)+1}$,
          and $H \in \bigtimes_{F \in \mathcal{P}_f(\mathcal{T})}
          \mathcal{I}_{m(F)}$ such that

          \begin{itemize}
            \item[(1)] if $F$, $G \in \mathcal{P}_f(\mathcal{T})$ and $F
              \subsetneq G$, then $\max H(F)\bigl(m(F)\bigr) < \min
              H(G)(1)$, and
           \item[(2)] whenever $n \in \bbN$, $\la G_1, G_2, \ldots, G_n
              \ra$ is a finite sequence in $\mathcal{P}_f(\mathcal{T})$
              with $G_1 \subsetneq G_2 \subsetneq \cdots \subsetneq G_n$,
              and for each $i \in \{1, 2, \ldots, n\}$, $f_i \in G_i$ we
              have
              \[
              \prod_{i=1}^n\Bigl(x\bigl(m(G_i), \alpha(G_i), H(G_i),
              f_i\bigr)\Bigr) \in C
             \]
      \end{itemize}
    \end{itemize}
  \end{defn}

  \begin{thm}
    $J(S)$ is a closed ideal of $\beta S$.
  \end{thm}
  \begin{proof}
   This is \cite[Theorem 3.5]{De:2008uq}.
  \end{proof}
  
  \begin{thm}
    \label{thm:csetid}
    Let $C \subseteq S$. 
    Then $C$ is a $C$-set if and only if there exists $p\cdot p = p
    \in J(S)$ such that $C \in p$.
  \end{thm}
  \begin{proof}
    This is \cite[Theorem 3.8]{De:2008uq}.%
    \endnote{
      In this theorem, strongly rich sets are the same as our
      $C$-sets. 
      There is nothing to prove here since the choice is just a matter
      of terminology.
      Currently, $C$-sets is the preferred term. 
    }
  \end{proof}

Since central sets where originally defined using notions of
topological dynamics, it is natural to investigate whether $C$-sets
admit a dynamical characterization.
We will prove such a characterization in Section \ref{sec:dyncsets}.
Before we can begin we will need to define what we mean by a dynamical
system, and provide a connection to $\beta S$.

\section{Preliminaries}
  \begin{defn}
    \label{defn:semiact}
    Let $S$ be a set. 
    A triple $(X, S, \pi)$ is a \textsl{semigroup action of $S$ on
      $X$} (or more shortly, \textsl{$S$ acts on $X$}) if and only if 
      \begin{itemize}
        \item[(1)] $S$ is a semigroup; and
        
        \item[(2)] $\pi : S \times X \to X$ is a function such that
          for all $s$, $t \in S$ and for every $x \in X$,
          \[ \pi\bigl(s, \pi(t,x)\bigr) = \pi(st, x). \]
      \end{itemize}
  \end{defn}
  
  \begin{rmk}
    The usual (and sometimes confusing) convention is to write the value
    $\pi(s,x)$ as $s \cdot x$.
    Following this convention the condition on $\pi$ becomes $s \cdot (t
    \cdot x) = st \cdot x$.
    We will not follow this convention since we will soon introduce
    better notation that will suit our purposes.
  \end{rmk}

Eventually we will want to extend our action of $S$ on $X$ to an
action of $\beta S$ on $X$.
To produce this extension we will be using \cite[Lemma
3.30]{Hindman:1998fk} (and also \cite[Corollary
4.22]{Hindman:1998fk}). 
Before we can use these results directly we will need to characterize
semigroup actions into a more convenient form. 

  \begin{prop} 
    \label{prop:semiact}
    Let $X$ be a set, $S$ a semigroup and $\pi : S \times X \to X$.
    The triple $(X, S, \pi)$ is a semigroup action if and only if
    there exists a semigroup homomorphism $T : S \to \setfunc{X}{X}$ such
    that for all $s \in S$ and every $x \in X$,
      \[ T(s)(x) = \pi(s,x). \]
  \end{prop}
  \begin{proof}
    First observe that $(\setfunc{X}{X}, \circ)$ is a semigroup.
    Now assume that the triple $(X, S, \pi)$ is a semigroup action. 
    Define the map $T : S \to \setfunc{X}{X}$ by $T(s)(x) =
    \pi(s,x)$.
    To see that $T$ is a semigroup homomorphism, let $s$, $t \in S$
    and $x \in X$. 
    Then 
      \begin{align*}
        T(st)(x) = \pi(st,x) &= \pi\bigl(s, \pi(t,x)\bigr), \\
        &= \pi\bigl(s, T(t)(x)\bigr), \\
        &= T(s)\bigl(T(t)(x)\bigr), \\
        &= \bigl(T(s) \circ T(t)\bigr) (x).
      \end{align*}
    Hence $T(st) = T(s) \circ T(t)$.

    Conversely, suppose $T$ is a semigroup homomorphism. 
    Let $s$, $t \in S$ and $x \in X$.
    Then 
      \begin{align*}
        \pi(st, x) &= T(st)(x), \\
        &= \bigl(T(s) \circ T(t)\bigr) (x), \\
        &= \pi\bigl(s, T(t)(x)\bigr), \\
        &= \pi\bigl(s, \pi(t,x)\bigr).
      \end{align*}
    Hence $(X, S, \pi)$ is a semigroup action.
  \end{proof}

With this Proposition we can now effectively forgot about our original
Definition \ref{defn:semiact} and simply consider semigroup actions
as elements of $\hom(S, \setfunc{X}{X})$. 
However, in this chapter we are not just concerned with any type of
semigroup action.
After all we are taking $\beta S$ to be a compact right topological
semigroup, and so we would like to apply this topological algebra to
our action in some way. 
We shall soon see that the notion of a dynamical system is one way to
accomplish this goal.

  \begin{defn}
    A pair $\ds$ is a \textsl{dynamical system} if and only if
      \begin{itemize}
        \item[(1)] $X$ is a compact Hausdorff space;
        \item[(2)] $S$ is a semigroup;
        \item[(3)] $T_s : X \to X$ is continuous for all $s \in S$;
          and
        \item[(4)] $T_s \circ T_t = T_{st}$ for all $s$, $t \in S$.%
        \endnote{
          In our definition of a dynamical system, the topological space
          $X$ is taken to be compact Hausdorff to ensure that
          $\setfunc{X}{X}$ is compact Hausdorff and limits along
          ultrafilters exist and are unique. 
          (Compactness guarantees existence, and Hausdorffness guarantees
          uniqueness.) 
        }
% I'm sure that there are dynamical systems where the phase space is
% not compact Hausdorff. Would be nice to mention were the reader can
% go to read about such dynamical systems. 
      \end{itemize}
  \end{defn}

  \begin{rmk}
    Let $\ds$ be a dynamical system and define $T : S \to
    \setfunc{X}{X}$ by $T(s) = T_s$.
    Then by Proposition \ref{prop:semiact} we are justified in saying
    that \textsl{$S$ acts on $X$ via $\la T_s \ra_{s \in S}$}.
  \end{rmk}

Using this remark we can extend the action of the dynamical system to
$\beta S$ as follows.
Giving $\setfunc{X}{X}$ the product topology and, as usual, taking $S$ to be 
discrete, we have that the function $T :
S \to \setfunc{X}{X}$ is a continuous semigroup
homomorphism into a compact space.
Therefore by \cite[Theorem 3.27]{Hindman:1998fk} we can produce a
continuous extension $\widetilde{T}$ of $T$.
(More directly, $\widetilde{T}$ is defined for each $p \in \beta S$ by
$\widetilde{T}(p) \in \bigcap \{\, c\ell\bigl( T[A] \bigr) : A \in p \,\}$, where
the closure is in $\setfunc{X}{X}$.)
Now by \cite[Theorem 2.22(a)]{Hindman:1998fk} the space $\setfunc{X}{X}$ is
a compact right-topological semigroup.
In order to show that $\widetilde{T}$ is a semigroup homomorphism it
suffices by, \cite[Corollary 4.22]{Hindman:1998fk}, to show that for all
$s \in S$, the
map $\lambda_{T(s)}$ is continuous.
However by \cite[Theorem 2.2(b)]{Hindman:1998fk} we have that
$\lambda_{T(s)}$ is continuous if and only if $T(s)$ is continuous. 
Since $\ds$ is a dynamical system and $T(s) = T_s$ we know by
definition that $T(s)$ is continuous. 
Hence $\widetilde{T} : \beta S \to \setfunc{X}{X}$ is a continuous semigroup
homomorphism.

  \begin{rmk}
    By using the map $\widetilde{T} : \beta S \to X$, we can define
    $T_p : X \to X$, for $p \in \beta S$, as $T_p =
    \widetilde{T}(p)$. 
    Since $\widetilde{T}$ is a semigroup homomorphism we immediately
    conclude that $T_p \circ T_q = T_{pq}$ for all $p$, $q \in \beta
    S$.
   \end{rmk}

It's important to note that in general $(\beta S, \la T_p \ra_{p
  \in \beta S})$ is not a dynamical system. 
The next example shows that $T_p$ may not be continuous for $p \in S^*$.
 

  \begin{example}
    We have that $(\beta\bbN, \la \lambda_s \ra_{s\in\bbN})$ is a
    dynamical system, but if $p \in \bbN^*$, then $\lambda_p$ is not
    continuous.
    This is proved in \cite[Theorem 6.10 and Remark 6.11]{Hindman:1998fk}.
   \end{example}

This example is somewhat disappointing since we lose the ``dynamical
part'' when extending the dynamical system to $\beta S$.
However even with this lost we will be able to prove something
intelligible about certain dynamical systems. 
Intuitively, the points of $S^*$ can be thought of as points at
infinity. 
Since in dynamical systems we are often concerned with the
``long-run'' behavior of our maps $T_s$ we will often be able to
correspond any interesting long run behaviors with a point at infinity
in $S^*$.

To make this idea precise, we will be using the notion of a limit
along an ultrafilter. 

  \begin{defn}
    \label{defn:plim}
    Let $S$ be a discrete space, $p \in \beta S$, $X$ a compact
    Hausdorff topological space, $\la x_s : s \in S \ra$ a family
    of points in $X$, and $y \in X$.
    Then \hbox{$p$-$\displaystyle\lim_{s \in S} x_s = y$} if and only
    if for every
    neighborhood $U$ of $y$ we have $\{\, s \in S : x_s \in U \,\} \in p$.
  \end{defn}

  \begin{rmk}
    Our definition of a \hbox{$p$-limit} is a bit more restrictive
    then usual.
    Normally the definition only takes $X$ be any topological space.
    We placed these extra restrictions on $X$ to ensure that every
    \hbox{$p$-limit} exists and is unique.
  \end{rmk}

  \begin{prop}
    \label{prop:dsplim}
    Let $\ds$ be a dynamical system.
    Then for every $p \in \beta S$ and each $x \in X$ we have $T_p(x)
    = \hbox{$p$--$\lim_{s \in S} T_s(x)$}$.
  \end{prop}
  \begin{proof}
    By definition, we need to show that for all neighborhoods $U$ of
    $T_p(x)$, we have $\{\, s \in S : T_s(x) \in U \,\} \in p$. 
    Define $\pi_x : \setfunc{X}{X} \to X$ by $\pi_x(f) = f(x)$ and let
    $U$ be a neighborhood of $T_p(x)$.
    Note that $\pi_x^{-1}[U]$ is a neighborhood of $T_p$.
    By definition of $T_p$, we have that for all $A \in p$,
    $\pi_x^{-1}[U] \cap \{\, T_s : s \in A \,\} \ne \emptyset$. 
    Hence for all $A \in p$, there exists $s \in A$ such that $T_s(x)
    \in U$. 

    Suppose there exists a neighborhood $U$ of $T_p(x)$ such that
    $\{\, s \in S : T_s(x) \in U \,\} \not\in p$. 
    Put $A = \{\, s \in S : T_s(x) \not\in U\,\}$.
    Then $A \in p$.
    Therefore there exists $s \in A$ such that $T_s(x) \in U$ (by our
    first paragraph) and $T_s(x) \not\in U$ (by our definition of
    $A$), a contradiction.
  \end{proof}

With these preliminaries out of the way, we can now provide a
dynamical characterization of $C$-sets.

\section{Dynamical Characterization of $C$-sets}
\label{sec:dyncsets}
 \begin{defn}
    \label{defn:JSUR}
    Let $\ds$ be a dynamical system, and let $x$, $y \in X$. 
    The pair $(x,y)$ is \textsl{jointly sparsely uniformly recurrent}
    (we'll abbreviate this to JSUR) if and only if $\{\, s \in S :
    \hbox{$T_s(x) \in U$ and $T_s(y) \in U$} \,\}$ is a $J$-set for every
    neighborhood $U$ of $y$.%
    \endnote{
      In this section Definition \ref{defn:JSUR}, Lemma \ref{lem:JSUR},
      and Theorem \ref{thm:dyncsets} and their proofs are all, essentially,
      minor modifications of \cite[Definition 3.1]{Burns:2007uq},
      \cite[Lemma 3.3]{Burns:2007uq}, \cite[Theorem 3.4]{Burns:2007uq}
      respectively. 
    }
  \end{defn}


  \begin{lem}
    \label{lem:JSUR}
    Let $\ds$ be a dynamical system, and let $x$, $y \in X$.
    The following statements are equivalent.
    \begin{itemize}
      \item[(a)] The pair $(x, y)$ is JSUR.
      \item[(b)] There exists $r \in J(S)$ such that $T_r(x) = y = T_r(y)$.
      \item[(c)] There exists an idempotent $r \in J(S)$ such that $T_r(x)
        = y = T_r(y)$. 
    \end{itemize}
  \end{lem}
  \begin{proof}
    (a) $\Rightarrow$ (b). 
    For each neighborhood $U$ of $y$, put 
      \[  
        B_U = \{\, s \in S : \hbox{$T_s(x) \in U$ and $T_s(y) \in U$}
        \,\}.
      \]
    By assumption each $B_U$ is a $J$-set. 
    We now show that the collection $\{\, B_U : \hbox{$U$ is a
      neighborhood of $y$} \,\}$ is closed under finite intersection
    by showing that, for all neighborhoods $U$ and $V$ of $y$ we have $B_{U
      \cap V} = B_U \cap B_V$.
    Let $s \in S$, then 
      \begin{align*}
        s \in B_{U \cap V} &\iff \hbox{$T_s(x) \in U \cap V$ and $T_s(y) \in U
        \cap V$}, \\
      &\iff \hbox{$T_s(x) \in U$, $T_s(x) \in V$, $T_s(y) \in U$, and
        $T_s(y) \in V$}, \\
      &\iff s \in B_U \cap B_V.
      \end{align*}
    
    By \cite[Theorem 2.14]{Hindman:2010fk}, we know that every $J$-set
    of $S$ is partition regular. 
    Therefore by \cite[Theorem 3.11 (b)]{Hindman:1998fk},  we can pick 
    $r \in J(S)$ such that $\{\, B_U : \hbox{$U$ is a neighborhood of
      $y$}\,\} \subseteq r$. 
    
    Now for all neighborhoods $U$ of $y$, we have $B_U \subseteq \{\,
    s \in S : T_s(x) \in U \,\}$ and $B_U \subseteq \{\, s \in S :
    T_s(y) \in U \,\}$. 
    Therefore $\{\, s \in S : T_s(x) \in U\,\} \in r$ and $\{\, s \in
    S : T_s(y) \in U \,\} \in r$. 
    By Definition \ref{defn:plim}, we can conclude that
    $r$-$\displaystyle\lim_{s\in S} T_s(x) = y$ and
    $r$-$\displaystyle\lim_{s \in S} T_s(y) = y$. 
    Hence by Proposition \ref{prop:dsplim}, we have $T_r(x) =
    r$-$\displaystyle\lim_{s \in S} T_s(x) = y = r$-$\displaystyle\lim_{s \in S}
    T_s(y) = T_r(y)$.
  
    (b) $\Rightarrow$ (c).
    Put $M  = \{\, r \in J(S) : T_r(x) = y = T_r(y) \,\}$. 
    We'll show that $M$ is a nonempty compact subsemigroup of $J(S)$. 
    If we can show this, then we can pick an idempotent in $M$ and our
    result follows.
    The fact that $M \ne \emptyset$ follows from our assumption. 
    To see that $M$ is compact it suffices to show that $M$ is
    closed. 
    Let $r \not\in M$, then either $T_r(x) \ne y$ or $T_r(y) \ne y$. 
    First, assume that $T_r(x) \ne y$. 
    By Definition \ref{defn:plim} and Proposition \ref{prop:dsplim},
    pick a
    neighborhood $U$ of $y$ such that 
    $\{\, s \in S : T_s(x) \in U \,\} \not\in r$.
    Put $A = \{\, s \in S : T_s(x) \in U \,\}$ and note that $S
    \setminus A \in r$ and $\overline{S \setminus A} \cap M =
    \emptyset$.
    [If $p \in \overline{S \setminus A} \cap M$, then $A = \{\, s \in
    S : T_s(x) \in U \,\} \in p$ by Definition \ref{defn:plim} and
    Proposition \ref{prop:dsplim}.
    We have also have that $S \setminus A \in p$.
    Hence $\emptyset = A \cap (S \setminus A) \in p$, a
    contradiction.]
    Now assume that $T_r(y) \ne y$. 
    By Definition \ref{defn:plim} and \ref{prop:dsplim}, pick a
    neighborhood $U$ of $y$ such that
    $\{\, s \in S : T_s(y) \in U \,\} \not\in r$.
    Put $A = \{\, s \in S : T_s(y) \in U \,\}$ and note that $S
    \setminus A \in r$ and $\overline{S \setminus A} \cap M =
    \emptyset$.
    Hence $M$ is a nonempty closed subset of $J(S)$.

    To see that $M$ is a subsemigroup, let $q$, $r \in M$.
    Then $T_{qr}(x) = T_q\bigl(T_r(x)\bigr) = T_q(y) =
    T_q\bigl(T_r(y)\bigr) = T_{qr}(y)$ and $T_q(y) = y$. 
    Hence $qr \in M$. 
    
    (c) $\Rightarrow$ (a).
    Pick $r$ as guaranteed in (c). 
    Let $U$ be a neighborhood of $y$. 
    Then $\{\, s \in S : T_s(x) \in U \,\} \in r$ and $\{\,  s \in S :
    T_s(y) \in U \,\} \in r$.
    Hence $\{\, s \in S : \hbox{$T_s(x) \in U$ and $T_s(y) \in U$}
    \,\} \in r$.  
  \end{proof}

  \begin{thm}
    \label{thm:dyncsets}
    Let $(S,\cdot)$ be a semigroup and $A \subseteq S$. 
    Then $A$ is a $C$-set if and only if there exist a dynamical
    system $\ds$ with points $x$, $y \in X$ where $(x,y)$ is JSUR, and
    a neighborhood $U$ of $y$ such that $A = \{\, s \in S : T_s(x) \in
    U \,\}$.
  \end{thm}
  \begin{proof}
    ($\Rightarrow$) Let $A \subseteq S$ be our $C$-set, and by
    Theorem \ref{thm:csetid} pick an idempotent $r \in J(S)$ such that
    $A \in r$. 
    Let $R = S \cup \{e\}$ be the semigroup with an identity $e$
    adjoined to S. 
    (For expository convenience, we still add this new identity even
    if $S$ already contains an identity.)
    Give $\{0,1\}$ the discrete topology, and take
    $\setfunc{R}{\{0,1\}}$ to have
    the product topology, and put $X = \setfunc{R}{\{0,1\}}$.
    Hence $X$ is a compact Hausdorff space.
    For each $s \in S$, define $T_s : X \to X$ by $T_s(f) = f \circ
    \rho_s$. 
    % Perhaps I should sketch the argument here?
    By \cite[Theorem 19.14]{Hindman:1998fk}, $\ds$ is a dynamical
    system. 
    
    Now let $x = \cchi_A$ be the characteristic function of $A$, and
    put $y = T_r(x)$.
    % Perhaps I should provide a proof since there is not one
    % explicitly given in the book?
    Then by \cite[Remark 19.13]{Hindman:1998fk}, we have that $T_r(y)
    = T_r\bigl(T_r(x)\bigr) = T_{rr}(x) = T_r(x) = y$.
    Therefore by (c) in Lemma \ref{lem:JSUR}, the pair $(x, y)$ is JSUR.

    Put $U = \{\, w \in X : w(e) = y(e) \,\}$, and note that $U =
    \pi^{-1}\bigl[\{y(e)\}\bigr]$ and so $y \in U$. 
    Hence $U$ is a (subbasic) open neighborhood of $y$.
    To help us show that $U$ is the neighborhood of $y$ we are looking
    for we will show that $y(e) = 1$.
    Since $y = T_r(x)$ we have that $\{\, s \in S : T_s(x) \in U \,\}
    \in r$ by Definition \ref{defn:plim} and Proposition \ref{prop:dsplim}.
    Since $A \in r$, we can pick $s \in A$ such that $T_s(x) \in U$. 
    Then by definition of $U$ and our choice of $T_s(x) \in U$, we
    have $y(e) = T_s(x)(e) = x\bigl(\rho_s(e)\bigr) = x(es) = x(s) =
    \chi_{A}(s) = 1$. 
    Finally, given $s \in S$, we have
      \begin{align*}
        s \in A &\iff \chi_{A}(s) = 1, \\
                &\iff x(s) = 1, \\
                &\iff x(es) = 1, \\
                &\iff (x \circ \rho_s)(e) = 1, \\
                &\iff T_s(x)(e) = 1 = y(e), \\
                &\iff T_s(x) \in U.
      \end{align*}
   Hence $A = \{\, s \in S : T_s(x) \in U \,\}$. 
   
   ($\Leftarrow$) Let $\ds$ and let the points $x$, $y \in X$ be given as
   guaranteed.
   By Theorem \ref{thm:csetid}, pick an idempotent $r \in J(S)$
   such that $T_r(x) = y = T_r(y)$. 
   Since $U$ is a neighborhood of $y$ and $T_r(x) = y$, we have that
   $A \in r$ by Definition \ref{defn:plim} and Proposition \ref{prop:dsplim}. 
 \end{proof}

% Thinking about cutting this section.
\section{Open Problems}
  \begin{enumerate}
    \item Produce a dynamical characterization of rich sets.
    \item Produce a dynamical characterization of strongly central sets.
  \end{enumerate}

\theendnotes

\bibliographystyle{plain}
\bibliography{references}
\end{document}
