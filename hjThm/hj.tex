% Notes on Hales-Jewett Theorem

\documentclass[12pt]{article}
\usepackage{amsthm, amssymb, amsmath}
\usepackage{endnotes}
\usepackage{todonotes}
% \usepackage[doublespacing]{setspace}
\usepackage{url}

\newtheoremstyle{plain}{3mm}{3mm}{\slshape}{}{\bfseries}{.}{.5em}{}
\theoremstyle{plain}
\newtheorem{thm}{Theorem}
\newtheorem*{vdw}{Van der Waerden's Theorem}
\newtheorem*{hj}{Hales-Jewett Theorem}
\newtheorem*{ramsey}{Ramsey's Theorem}
\newtheorem{cor}[thm]{Corollary}
\newtheorem{prop}[thm]{Proposition}
\newtheorem{lem}[thm]{Lemma}
\newtheorem{claim}[thm]{Claim}
\newtheorem{ques}[thm]{Question}
\newtheorem{conj}[thm]{Conjecture}
\newtheorem{fact}[thm]{Fact}

\theoremstyle{definition}
\newtheorem{defn}[thm]{Definition}
\newtheorem{example}[thm]{Example}
\newtheorem{rmk}[thm]{Remark}


\newcommand{\la}{\langle}
\newcommand{\ra}{\rangle}
\newcommand{\bbN}{\mathbb{N}}
\newcommand{\bbZ}{\mathbb{Z}}
\newcommand{\AP}{\mathcal{AP}}
\newcommand{\AL}{\mathcal{AL}}
\newcommand{\calG}{\mathcal{G}}
\newcommand{\calI}{\mathcal{I}}
\newcommand{\calJ}{\mathcal{J}}
\newcommand{\calP}{\mathcal{P}}
\newcommand{\calR}{\mathcal{R}}
\newcommand{\calS}{\mathcal{S}}
\newcommand{\calT}{\mathcal{T}}
\newcommand{\Pf}{\mathcal{P}_f}

\newcommand{\setfunc}[2]{\hbox{${}^{\hbox{$#1$}}\hskip -1 pt #2$}}

\font\bigmath=cmsy10 scaled \magstep 3
\newcommand{\bigtimes}{\hbox{\bigmath \char'2}}

\newcommand{\cchi}{\raise 2 pt \hbox{$\chi$}}

\begin{document}
\begin{defn}
  Let $X$ be a set, $\calS \subseteq \calP(X)$, and $\kappa$ a
  cardinal number.
  \begin{itemize}
    \item[(a)] We call $\calS$ \textsl{\mbox{$\kappa$-regular} in $X$}
      if and only if whenever $X = \bigcup_{i=1}^\kappa C_i$, there
      exist $i \in \kappa+1$ and $A \in \calS$ such that $A \subseteq
      C_i$.

    \item[(b)] We call $\calS$ \textsl{regular in $X$} if and only if
      $\calS$ is \mbox{$n$-regular} for all $n \in \bbN$.
  \end{itemize}
\end{defn}

\begin{example}
  Let $X = \{1, 2, \ldots, n(m-1) + 1\}$ and $\calS = \{\, A \subseteq
  X : |A| = m \,\}$.
  Then $\calS$ is \mbox{$n$-regular} in $X$ but not
  \mbox{$(n+1)$-regular}.
\end{example}
\begin{proof}
  We first show that $\calS$ is \mbox{$n$-regular} in $X$.
  Let $X = \bigcup_{i=1}^n C_i$.
  If each $C_i$ has at most $m-1$ elements, then $|X| \le n(m-1)$, a
  contradiction. 
  Hence there exist $i \in \{1, 2, \ldots, n\}$ and $A \in \calS$ such
  that $A \subseteq C_i$.

  We now show that $\calS$ is not \mbox{$(n+1)$-regular} in $X$.
  For each $i \in \{1, 2, \ldots, n\}$ pick $C_i \subseteq X$ such
  that $|C_i| = m-1$ and $C_i \cap C_j = \emptyset$ for $i \ne j$.
  Put $C_{n+1} = X \setminus \bigl(\bigcup_{i=1}^n C_i\bigr)$, then
  $|C_{n+1}| = 1$. 
  Moreover, $X = \bigcup_{i=1}^{n+1} C_i$ and no $C_i$ contains a
  \mbox{$m$-element} set.
\end{proof}

\begin{ramsey}
  For all $k$, $m$, $n \in \bbN$, there exists $p \in \bbN$ such that
  if $A = \{1, 2, \ldots, p\}$, then $\bigl\{\, [B]^k : B \in [A]^m
  \,\bigr\}$ is \mbox{$n$-regular} in $[A]^k$.
\end{ramsey}

\begin{defn}
  Let $\calS$ and $\calT$ both be collection of sets.
  Define
  \[
    \calS \otimes \calT = \{\, A \times B : \mbox{$A \in \calS$ and
      $B \in \calT$} \,\}.
  \]
\end{defn}

\begin{lem}
  \label{lem:TensorReg}
  Let $\kappa$ and $\lambda$ be cardinal numbers and let $X$ be a set
  with $|X| = \kappa$. 
  Let $\calS$ be \mbox{$\lambda$-regular} in $X$ and let $\calT$ be
  \mbox{$\lambda^\kappa$-regular} in $Y$.
  Then $\calS \otimes \calT$ is \mbox{$\lambda$-regular} in $X \times
  Y$.
\end{lem}
\begin{proof}
  Let $f \colon X \times Y \to \lambda$ and note that this function is
  equivalent to a partition of $X \times Y$ into $\lambda$ pieces.
  For each $y \in Y$ define the function $f_y \colon X \to \lambda$ by
  $f_y(x) = f(x,y)$. 
  Since the number of functions from $X$ to $\lambda$ is
  $\kappa^\lambda$, we can consider the mapping $y \mapsto f_y$ as a
  partition of $Y$ into $\lambda^\kappa$ pieces.
  
  Now $\calT$ is \mbox{$\lambda^\kappa$-regular} in $Y$, so we can
  pick $T \in \calT$ such that for all $y$, $y' \in T$, $f_y =
  f_{y'}$, that is, for all $y$, $y' \in T$ and for every $x \in
  X$, we have $f(x,y) = f(x',y')$.

  Let $y \in T$, then $f_y$ partitions $X$ into $\lambda$ pieces.
  Since $\calS$ is \mbox{$\lambda$-regular} in $X$, we can pick $S \in
  \calS$ such that for all $x$, $x' \in S$, $f_y(x) = f_y(x')$, that
  is, $f(x, y) = f(x', y)$.
  Then it follows that for all $y$, $y' \in T$ and for every $s$, $s'
  in S$, we have that $f(x,y) = f(x', y) = f(x',y')$.
  It follows that $\calS \otimes \calT$ is \mbox{$\lambda$-regular} in
  $X \times Y$.
\end{proof}

\begin{defn}
  Let $X$ and $Y$ be sets with $\calS \subseteq \calP(X)$ and $\calT
  \subseteq \calP(Y)$.
  We call $f \colon X \to Y$ \textsl{provincial with respect to
    $\calS$ and $\calT$} if and only if for every $A \in \calS$, there
  exists $B \in \calT$ such that $B \subseteq f[A]$.
\end{defn}

\begin{lem}
  \label{lem:provin}
  Let $f \colon X \to Y$ be provincial with respect to $\calS$ and
  $\calT$.
  Let $\kappa$ be a cardinal.
  If $\calS$ is \mbox{$\kappa$-regular} in $X$, then $\calT$ is
  \mbox{$\kappa$-regular} in $Y$.
\end{lem}
\begin{proof}
  Let $g \colon Y \to \kappa$, then $g \circ f \colon X \to \kappa$.
  Pick $A \in \calS$ such that for all $a$, $a' \in A$, $(g \circ
  f)(a) = (g \circ f)(a')$.
  Since $f$ is provincial, pick $B \in \calT$ such that $B \subseteq
  f[A]$.
  Then for all $b$, $b' \in B$, there exist $a$, $a' \in A$ such that
  $b = f(a)$, $b' = f(a')$, and $g(b) = (g \circ f)(a) = (g \circ
  f)(a') = g(b')$.
\end{proof}

\begin{lem}
  Let $(S, \cdot)$ be a semigroup and $\calS \subseteq \calP(S)$.
  Suppose that for all $k \in \bbN$, there exists $F \in \Pf(S)$ such
  that $\calS$ is \mbox{$k$-regular} in $F$.
  Then for each $n \in \bbN$,
  \[
    \calS_n = \{\, A_1A_2 \cdots A_n : A_i \in \calS \,\}
  \]
  is regular in $S$.
\end{lem}
\begin{proof}
  We proceed by induction on $n$. 
  Observe that since $\calS_1 = \calS$ we are done, by hypothesis,
  when $n = 1$.
  Let $n > 1$ and assume that $\calS_{n-1}$ is regular in $S$.

  Let $k \in \bbN$.
  Then there exist $m \in \bbN$ and $T \in [S]^m$ such that $\calS$ is
  regular in $T$.
  Since $\calS_{n-1}$ is regular in $S$, we have that $\calS_{n-1}$ is
  \mbox{$k^m$-regular} in $S$.
  By Lemma \ref{lem:TensorReg}, $\calS \otimes \calS_{n-1}$ is
  \mbox{$k$-regular} in $T \times S \subseteq S \times S$.
  We claim that the mapping $(x,y) \mapsto xy$ is provincial with
  respect to $\calS \otimes \calS_{n-1}$ and $\calS_n$. 
  If this claim is true, then it follows by Lemma \ref{lem:provin}
  that $\calS_n$ is regular.

  To that $(x,y) \mapsto xy$ is provincial with respect to $\calS
  \otimes \calS_{n-1}$ and $\calS_n$, let $C \in \calS$ and $D \in
  \calS_{n-1}$, then $C \times D \in \calS \otimes \calS_{n-1}$.
  The image of $C$ and $D$ under the map is $CD$ which is an element
  of $\calS_n$.
\end{proof}

\begin{defn}
  Let $S$ be the free semigroup on the alphabet $A$ and let $\star
  \not\in A$.
  \begin{itemize}
    \item[(a)] A \textsl{variable word} $w(\star)$ is a word in the
      free semigroup on $A \cup \{\star\}$ in which $\star$ occurs.

    \item[(b)] $S(\star) = \{\, w(\star) : \mbox{$w(\star)$ is a
        variable word} \,\}$.
    
    \item[(c)] Given a $w(\star) \in S(\star)$ and $a \in A$, let
      $w(a)$ represent the word in which every occurrence of $\star$
      is replaced by $a$.

    \item[(d)] Given $B \subseteq A$ and $w(\star) \in S(\star)$ we
      call the set $w[B] = \{\, w(a) : a \in B \,\}$ a
      \textsl{combinatorial line}.
  \end{itemize}
\end{defn}

\begin{prop}
  Let $A$ be a set, $S$ the free semigroup on $A$, $B \subseteq A$,
  and $n \in \bbN$.
  If $w_1(\star)$, $w_2(\star)$, \dots, $w_n(\star) \in S(\star)$,
  then there exists $w(\star) \in S(\star)$ such that for all $a \in
  B$, $w_1(a)w_2(a)\cdots w_n(a) = w(a)$.
\end{prop}

\begin{hj}
  Let $A$ be a finite set and let $S$ be the free semigroup on $A$. 
  Then $\{\, w[A] : w(\star) \in S(\star) \,\}$ is regular in $S$. 
\end{hj}
\bibliographystyle{amsplain}
\bibliography{../references}
\end{document}

