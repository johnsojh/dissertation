% Notes on Hales-Jewett Theorem

\documentclass[12pt]{article}
\usepackage{amsthm, amssymb, amsmath}
\usepackage{endnotes}
\usepackage{todonotes}
% \usepackage[doublespacing]{setspace}
\usepackage{url}

\newtheoremstyle{plain}{3mm}{3mm}{\slshape}{}{\bfseries}{.}{.5em}{}
\theoremstyle{plain}
\newtheorem{thm}{Theorem}
\newtheorem*{vdw}{Van der Waerden's Theorem}
\newtheorem*{hj}{Hales-Jewett Theorem}
\newtheorem*{ramsey}{Ramsey's Theorem}
\newtheorem{cor}[thm]{Corollary}
\newtheorem{prop}[thm]{Proposition}
\newtheorem{lem}[thm]{Lemma}
\newtheorem{claim}[thm]{Claim}
\newtheorem{ques}[thm]{Question}
\newtheorem{conj}[thm]{Conjecture}
\newtheorem{fact}[thm]{Fact}

\theoremstyle{definition}
\newtheorem{defn}[thm]{Definition}
\newtheorem{example}[thm]{Example}
\newtheorem{rmk}[thm]{Remark}


\newcommand{\la}{\langle}
\newcommand{\ra}{\rangle}
\newcommand{\bbN}{\mathbb{N}}
\newcommand{\bbZ}{\mathbb{Z}}
\newcommand{\AP}{\mathcal{AP}}
\newcommand{\AL}{\mathcal{AL}}
\newcommand{\calG}{\mathcal{G}}
\newcommand{\calI}{\mathcal{I}}
\newcommand{\calJ}{\mathcal{J}}
\newcommand{\calP}{\mathcal{P}}
\newcommand{\calR}{\mathcal{R}}
\newcommand{\calS}{\mathcal{S}}
\newcommand{\calT}{\mathcal{T}}
\newcommand{\Pf}{\mathcal{P}_f}

\newcommand{\setfunc}[2]{\hbox{${}^{\hbox{$#1$}}\hskip -1 pt #2$}}

\font\bigmath=cmsy10 scaled \magstep 3
\newcommand{\bigtimes}{\hbox{\bigmath \char'2}}

\newcommand{\cchi}{\raise 2 pt \hbox{$\chi$}}

\begin{document}
\begin{defn}
  Let $X$ be a set, $\calS \subseteq \calP(X)$, and $\kappa$ a
  cardinal number.
  \begin{itemize}
    \item[(a)] We call $\calS$ \textsl{\mbox{$\kappa$-regular} in $X$}
      if and only if whenever $X = \bigcup_{i=1}^\kappa C_i$, there
      exist $i \in \kappa+1$ and $A \in \calS$ such that $A \subseteq
      C_i$.

    \item[(b)] We call $\calS$ \textsl{regular in $X$} if and only if
      $\calS$ is \mbox{$n$-regular} for all $n \in \bbN$.
  \end{itemize}
\end{defn}

\begin{example}
  Let $X = \{1, 2, \ldots, n(m-1) + 1\}$ and $\calS = \{\, A \subseteq
  X : |A| = m \,\}$.
  Then $\calS$ is \mbox{$n$-regular} in $X$ but not
  \mbox{$n+1$-regular}.
\end{example}
\begin{proof}
  
\end{proof}

\begin{ramsey}
  For all $k$, $m$, $n \in \bbN$, there exists $p \in \bbN$ such that
  if $A = \{1, 2, \ldots, p\}$, then $\bigl\{\, [B]^k : B \in [A]^m
  \,\bigr\}$ is \mbox{$n$-regular} in $[A]^k$.
\end{ramsey}

\begin{defn}
  Let $\calS$ and $\calT$ both be collection of sets.
  Define
  \[
    \calS \otimes \calT = \{\, A \times B : \mbox{$A \in \calS$ and
      $B \in \calT$} \,\}.
  \]
\end{defn}

\begin{lem}
  Let $\kappa$ and $\lambda$ be cardinal numbers and let $X$ be a set
  with $|X| = \kappa$. 
  Let $\calS$ be \mbox{$\lambda$-regular} in $X$ and let $\calT$ be
  \mbox{$\lambda^\kappa$-regular} in $Y$.
  Then $\calS \otimes \calT$ is \mbox{$\lambda$-regular} in $X \times
  Y$.
\end{lem}
\begin{proof}
  
\end{proof}

\begin{defn}
  Let $X$ and $Y$ be sets with $\calS \subseteq \calP(X)$ and $\calT
  \subseteq \calP(Y)$.
  We call $f \colon X \to Y$ \textsl{provincial with respect to
    $\calS$ and $\calT$} if and only if for every $A \in \calS$, there
  exists $B \in \calT$ such that $B \subseteq f[A]$.
\end{defn}

\begin{lem}
  Let $f \colon X \to Y$ be provincial with respect to $\calS$ and
  $\calT$.
  Let $\kappa$ be a cardinal.
  If $\calS$ is \mbox{$\kappa$-regular} in $X$, then $\calT$ is
  \mbox{$\kappa$-regular} in $Y$.
\end{lem}
\begin{proof}
  
\end{proof}

\begin{lem}
  Let $(S, \cdot)$ be a semigroup and $\calS \subseteq \calP(S)$.
  Suppose that for all $k \in \bbN$, there exists $F \in \Pf(S)$ such
  that $\calS$ is \mbox{$k$-regular} in $F$.
  Then for each $n \in \bbN$,
  \[
    \calS_n = \{\, A_1A_2 \cdots A_n : A_i \in \calS \,\}
  \]
  is regular in $S$.
\end{lem}
\begin{proof}
  
\end{proof}
\bibliographystyle{amsplain}
\bibliography{../references}
\end{document}

