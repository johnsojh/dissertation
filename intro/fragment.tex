% Draft of Introduction for my Dissertation

\documentclass[12pt]{article}
\usepackage{amsthm, amssymb, amsmath}
\usepackage{endnotes}
\usepackage{todonotes}
% \usepackage[doublespacing]{setspace}
\usepackage{url}

\newtheoremstyle{plain}{3mm}{3mm}{\slshape}{}{\bfseries}{.}{.5em}{}
\theoremstyle{plain}
\newtheorem{thm}{Theorem}[section]
\newtheorem{CSTv1}{Original Central Sets Theorem}
\newtheorem{vdw}[thm]{Van der Waerden's Theorem}
\newtheorem{FST}[thm]{Hindman's Theorem}
\newtheorem{MBR}[thm]{Multiple Birkhoff Recurrence Theorem}
\newtheorem{recur}[thm]{Recurrence Theorem}
\newtheorem{OCST}[thm]{Furstenburg's Original Central Sets Theorem}
\newtheorem{cst}[thm]{Central Sets Theorem}
\newtheorem{cor}[thm]{Corollary}
\newtheorem{prop}[thm]{Proposition}
\newtheorem{lem}[thm]{Lemma}
\newtheorem{claim}[thm]{Claim}
\newtheorem{ques}[thm]{Question}
\newtheorem{conj}[thm]{Conjecture}
\newtheorem{fact}[thm]{Fact}

\theoremstyle{definition}
\newtheorem{defn}[thm]{Definition}
\newtheorem{example}[thm]{Example}
\newtheorem{rmk}[thm]{Remark}


\newcommand{\la}{\langle}
\newcommand{\ra}{\rangle}
\newcommand{\bbN}{\mathbb{N}}
\newcommand{\bbZ}{\mathbb{Z}}
\newcommand{\calI}{\mathcal{I}}
\newcommand{\calJ}{\mathcal{J}}
\newcommand{\calP}{\mathcal{P}}
\newcommand{\calR}{\mathcal{R}}
\newcommand{\calT}{\mathcal{T}}
\newcommand{\Pf}{\mathcal{P}_f}

\newcommand{\setfunc}[2]{\hbox{${}^{\hbox{$#1$}}\hskip -1 pt #2$}}

\font\bigmath=cmsy10 scaled \magstep 3
\newcommand{\bigtimes}{\hbox{\bigmath \char'2}}

\newcommand{\cchi}{\raise 2 pt \hbox{$\chi$}}

\begin{document}
\begin{defn}
Let $S$ be a semigroup and put $\calT = \setfunc{\bbN}{S}$. 
  \begin{itemize}
    \item[(a)] For each $m \in \bbN$, define
      \[
        \calJ_m = \{\, (t_1, t_2, \ldots, t_m) \in \bbN^m : t_1 < t_2 < \cdots < t_m \,\}.
      \]

    \item[(b)] For each $m \in \bbN$, $a \in S^{m+1}$, $t \in \calJ_m$, and $f \in \calT$, define
      \[
        x(m, a, t, f) = \Bigl(\prod_{i=1}^m \bigl( a(i) f(t_i) \bigr) \Bigr) a(m+1).
      \]

    \item[(c)] We call $A \subseteq S$ a \textsl{$J$-set (in $S$)} if and only if for every $F \in \Pf(\calT)$ there exist $m \in \bbN$, $a \in S^{m+1}$, and $t \in \calJ_m$ such that for all $f \in F$, $x(m, a, t, f) \in A$.
  \end{itemize}
\end{defn}

\begin{lem}
  Let $S$ be a semigroup.
  Let $n \in \bbN$, $c \in S^{n+1}$, $H \in \calI_n$, and $f \in \calT$.
  Fix $b \in S$ and define $g \in \calT$ by $g(t) = f(t)b$.
  Then there exists $m \in \bbN$, $a \in S^{m+1}$, and $t \in \calJ_m$ such that $x(m, a, t, f) = \bigl(\prod_{i=1}^n ( c(i)\prod_{t \in H(i)} g(t)) \bigr) c(n+1)$.
\end{lem}
\begin{proof}
  Put $H(0) = \emptyset$, for $s \in \{0, 1, \ldots, n\}$ define $h_s = \sum_{i=0}^s |H(i)|$, put $m = h_n$, and enumerate $\bigcup_{i=1}^n H(i)$ as a strictly increasing sequence $t_1 < t_2 < \cdots < t_m$ in $\bbN$. 
  We will adopt some temporary terminology and say that $\bigl(\prod_{i=1}^n ( c(i)\prod_{t \in H(i)} g(t)) \bigr) c(n+1)$ has \textsl{proper representation} if and only if $\bigl(\prod_{i=1}^n ( c(i)\prod_{t \in H(i)} g(t)) \bigr) c(n+1) = x(m, a, t, f)$ where $a \in S^{m+1}$ is defined as follows: 
  \begin{quote}
    For $j \in \{1, 2, \ldots, m+1\}$ put
    \[
      a(j) = 
      \begin{cases}
        c(j) & \mbox{if $j=1$;} \\
        b    & \mbox{if $s \in \{0, 1, \ldots, n-1\}$ and $2+h_s \le j \le h_{s+1}$; and } \\
        bc(s+1) & \mbox{if $s \in \{1, 2, \ldots, n\}$ and $j = 1+h_s$.}
      \end{cases}
    \]
  \end{quote}
  (To see how our $a$ was derived let's consider the example with $n = 3$, $H(1) = \{3, 5\}$, $H(2) = \{7\}$, and $H(3) = \{9, 11, 15\}$. 
  In this example,
  \[
    \prod_{i=1}^n\bigl( c(i) \prod_{t \in H(i)} g(t) \bigr) c(n+1) = c(1)f(2)bf(3)b c(2) f(7)bc(3)f(9)bf(11)bf(15)bc(4).
  \]
  Therefore $m = 6$, $a(1) = c(1)$, $a(2) = b$, $a(3) = bc(2)$, $a(4) = bc(3)$, $a(5) = b$, $a(6) = b$, and $a(7) = bc(4)$.)
  
  We prove that $\bigl(\prod_{i=1}^n ( c(i)\prod_{t \in H(i)} g(t)) \bigr) c(n+1)$ has proper representation by induction on $n$.

  First suppose that $n=1$, then
  \[
    c(1)\prod_{t \in H(i)} g(t) c(2) = c(1)f(t_1)bf(t_2)b \cdots f(t_m)bc(2).
  \]
  In this case, $h_0 = 0$, $h_1 = m$, and $s$ can only be 0 or 1.
  Now if $2+h_0 \le j \le h_1$, then by definition of $a$ we have $a(j) = b$ for all $j \in \{2, 3, \ldots, m\}$. 
  Also since $1+h_1 = m+1$, we have $a(m+1) = bc(1+1)$.
  Therefore
  \[
    c(1)f(t_1)bf(t_2)b \cdots f(t_m)bc(2) = a(1)f(t_1)a(2)f(t_2) \cdots f(t_m)a(m+1),
  \]
and so $c(1)\prod_{t \in H(i)} g(t) c(2)$ has proper representation. 

  Let $n > 1$, put $m' = h_{n-1}$, and assume that $\bigl(\prod_{i=1}^{n-1} ( c(i)\prod_{t \in H(i)} g(t)) \bigr) c(n)$ has proper representation with $\bigl(\prod_{i=1}^{n-1} ( c(i)\prod_{t \in H(i)} g(t)) \bigr) c(n) = x(m', a, t, f)$ where $a$ is regarded as an element of $S^{m'+1}$.
  By our consideration of the base case of our induction, we have that 
  \[
    a(m'+1)f(t_{m'+1})bf(t_{m'+2})b \cdots f(t_m)bc(n+1)
  \]
has proper representation. 
  Translating the indices for the proper representation of $a(m'+1)f(t_{m'+1})bf(t_{m'+2})b \cdots f(t_m)bc(n+1)$ shows that $x(m, a, t, f) = \bigl(\prod_{i=1}^{n} ( c(i)\prod_{t \in H(i)} g(t)) \bigr) c(n+1)$, that is, $\bigl(\prod_{i=1}^{n} ( c(i)\prod_{t \in H(i)} g(t)) \bigr) c(n+1)$ has proper representation. 
\end{proof}

\begin{cor}
  \label{cor:stJsets}
  Let $n \in \bbN$, $c \in S^{n+1}$, $H \in \calI_m$, and $F \in
  \Pf(\calT)$. 
  Fix $b \in S$ and for each $f \in F$ define $g_f \in \calT$ by
  $g_f(t) = f(t)b$. 
  Then there exists $m \in \bbN$, $a \in S^{m+1}$, and $t \in \calJ_m$
  such that for all $f \in F$, $x(m, a, t, f) = \bigl(\prod_{i=1}^n(
  c(i) \prod_{t \in H(i)} g_f(t)\bigr) c(n+1)$
\end{cor}

Our definition of a $J$-set is superficially stronger than the
definitions given in \cite[Definition 2.3(d)]{Hindman:2010fk} and
\cite[Definition 3.3(e)]{De:2008uq}. 
To show that our definition and \cite[Definition
2.3(d)]{Hindman:2010fk} are equivalent we will make use of a technical
lemma.
\begin{prop}
  Let $S$ be a semigroup and let $A \subseteq S$ be a set which
  satisfies \cite[Definition 2.3(d)]{Hindman:2010fk}, then $A$ is a $J$-set.
\end{prop}
\begin{proof}
  Let $F \in \Pf(\calT)$ and fix $b \in S$. 
  For each $f \in F$, define $g_f \in \calT$ by $g_f(t) = f(t)b$. 
  Since $A$ satisfies \cite[Definition 2.3(d)]{Hindman:2010fk}, there
  exist $n \in \bbN$, $c \in S^{n+1}$, and $H \in \calI_m$ such that
  for all $f \in F$, $\bigl(\prod_{i=1}^n( c(i) \prod_{t \in H(i)}
  g_f(t)\bigr) c(n+1) \in A$.
  By Corollary \ref{cor:stJsets}, pick $m \in \bbN$, $a \in S^{m+1}$,
  and $t \in \calJ_m$ such that for all $f \in F$, $x(m, a, t, f) =
  \bigl(\prod_{i=1}^n( c(i) \prod_{t \in H(i)} g_f(t)\bigr) c(n+1)$.
\end{proof}


We next show that our definition of a $J$-set is equivalent to
\cite[Definition 3.3(e)]{De:2008uq}.

\begin{prop}
  Let $S$ be a semigroup and let $A \subseteq S$ be a $J$-set. 
  the for each $F \in \Pf(\calT)$ and $n \in \bbN$, we can pick $m \in
  \bbN$, $a \in S^{m+1}$, and $t \in \calJ_m$ such that $n < t_1$ and
  for each $f \in F$, $x(m, a, t, f) \in A$.
\end{prop}
\begin{proof}
  For each $f \in F$, define $g_f \in \calT$ by $g_f(t) = f(t + n)$. 
  Then pick $m \in \bbN$, $a \in S^{m+1}$, and $s \in \calJ_m$ such
  that for all $f \in F$, $x(m, a, s, g_f) \in A$. 
  Define $t \in \calJ_m$ by $t_i = s_i + n$ for each $i \in \{1, 2,
  \ldots, m\}$. 
  Then $t_1 = s_1 + n > n$, and $x(m, a, t, f) = x(m, a, s, g_f)$. 
\end{proof}

\begin{thm}
  Let $S$ be a semigroup. 
  If $J(S) \ne \emptyset$, then $J(S)$ is a compact two-sided ideal of
  $\beta S$. 
\end{thm}
\begin{proof}
  To show that $J(S)$ is compact it suffices to show that $J(S)$ is
  topologically closed in $\beta S$.
  Let $p \not\in J(S)$, then pick $A \in p$ such that $A$ is not a
  $J$-set. 
  By definition of $J(S)$, we have that $\overline{A} \cap J(S) =
  \emptyset$. 
  Moreover, $\overline{A}$ is a (basic) open neighborhood of $p$. 

  Now let $p \in J(S)$ and $q \in \beta S$. 
  We show that $pq \in J(S)$ and $qp \in J(S)$. 

  Let $F \in \Pf(\calT)$, let $A \in pq$, and let $B = \{\, x \in S :
  x^{-1}A \in q \,\}$. 
  Then $B \in p$. 
  Since $B$ is a $J$-set, pick $m \in \bbN$, $a \in S^{m+1}$, and $t
  \in \calJ_m$ such that for all $f \in F$, $x(m, a, t, f) \in B$. 
  By definition of $B$, this means that for all $f \in F$, $x(m, a, t,
  f)^{-1} A \in q$.
  Hence $\bigcap_{f \in F} x(m, a, t, f)^{-1} A \in q$ and so we may
  pick $b \in \bigcap_{f \in F} x(m, a, t, f)^{-1} A$.
  Define $c \in S^{m+1}$ by
  \[
    c(j) = 
    \begin{cases}
      a(j) & \mbox{if $j \in \{1, 2, \ldots, m\}$}, \\
      a(m+1)b & \mbox{if $j = m+1$}.
    \end{cases}
  \]
  Then for each $f \in F$, $x(m, c, t, f) \in A$. 

  Now let $F \in \Pf(\calT)$, $A \in qp$, and let $B = \{\, x \in S :
  x^{-1}A \in p \,\}$.
  Then $B \in q$ and so we may pick $b \in B$ such that $b^{-1}A \in
  p$. 
  Since $b^{-1}A$ is a $J$-set, pick $m \in \bbN$, $a \in S^{m+1}$,
  and $t \in \calJ_m$ such that for all $f \in F$, $x(m, a, t, f) \in
  b^{-1}A$. 
  Define $c \in S^{m+1}$ by
  \[
    c(j) = 
    \begin{cases}
      ba(1) & \mbox{if $j = 1$}, \\
      a(j) & \mbox{if $j \in \{2, \ldots, m+1\}$}.
    \end{cases}
  \]
  Then $x(m, c, t, f) \in A$ for all $f \in F$.
\end{proof}

In order to show that $J(S)$ is nonempty we now focus on showing that
$J$-sets are partition regular. 

\begin{thm}
  Let $S$ be a semigroup, $A \subseteq S$ a $J$-set, and $A = A_1 \cup
  A_2$. 
  Then either $A_1$ is a $J$-set or $A_2$ is a $J$-set. 
\end{thm}
\begin{proof}
  Let $F \in \Pf(\calT)$, put $k = |F|$, and enumerate $F = \{ f_1, f_2, \ldots, f_k \}$.
  By the Hales-Jewett Theorem, pick $n \in \bbN$, such that whenever $[k]^n$ is 2-colored there exists a variable word $w(\star)$ which begins and ends with a constant letter such that $\{\, w(s) : s \in \{1, 2, \ldots, k\}$ is monochromatic. 

  Fix $b \in S$ and for each $w = (x_1, x_2, \ldots, x_n) \in [k]^n$ define $g_w \in \calT$ by $g_w(t) = \prod_{i=1}^n \bigl( f_{x_i}(nt + i)b \bigr)$.
  Since $\{\, g_w : w \in [k]^n \,\} \in \Pf(\calT)$ and $A$ is a $J$-set, pick $m \in \bbN$, $a \in S^{m+1}$, and $t \in \calJ_m$ such that for all $w \in [k]^n$, $x(m, a, t, g_w) \in A$.
  Define $\varphi \colon [k]^n \to \{1, 2\}$ by 
  \[
      \varphi(w) = 
      \begin{cases}
        1 & \mbox{if $x(m,a,t,g_w) \in A_1$ and} \\
        2 & \mbox{otherwise.}
      \end{cases}
  \]
Pick a variable word $w(\star)$ that begins and ends with a constant such that $\{\, w(s) : s \in \{1, 2, \ldots, k\} \,\}$ is a monochromatic combinatorial line with respect to $\varphi$.
Without loss of generality we suppose that $\varphi\bigl( w(s) \bigr) = 1$ for all $s \in \{1, 2, \ldots, k\}$. 

  Let $w(\star) = (x_1, x_2, \ldots, x_n)$ where each $x_i \in \{1, 2, \ldots, k\} \cup \{\star\}$ and some $x_i = \star$.
  Put $r = |\bigl\{\, i \in \{1, 2, \ldots, n-1\} : \mbox{$x_i = \star$ and $x_{i+1} \in \{1, 2, \ldots, k\}$} \,\bigr\}|$.
  (The number $r$ is the number of variable blocks in $w(\star)$.
  For instance if $k = 3$, $n = 10$, and $w(\star) = (2, \star, \star, 3, 1, \star, \star, \star, 1, 2)$, then $r = 2$.)
  Put $L \in \calI_{r+1}$ and $M \in \calI_r$ such that for each $i \in \{1, 2, \ldots, r\}$ we have $\max L(i) < \min M(i)$, $\max M(i) < \min L(i+1)$, and
  \begin{align*}
    \bigcup_{i=1}^{r+1} L(i) &= \bigl\{\, i \in \{1, 2, \ldots, n \} :
    x_i \in \{1, 2, \ldots, k\} \,\bigr\}, \\
    \bigcup_{i=1}^r M(i) &= \bigl\{\, i \in \{1, 2, \ldots, n \} : x_i
    = \star \,\bigr\}.
  \end{align*}
  (For example, if $k = 3$, $n = 10$, and $w(\star) = (2, \star, \star, 3, 1, \star, \star, \star, 1, 2)$, then $L(1) = \{1\}$, $M(1) = \{2, 3\}$, $L(2) = \{4, 5\}$, $M(2) = \{6, 7, 8\}$, $L(3) = \{9, 10\}$.)

  We now define $c \in S^{r+1}$ as follows, for $j \in \{1, 2, \ldots, r+1\}$,
  \[
    c(j) = 
    \begin{cases}
      \prod_{i \in L(1)} f_{x_i}(nt + i)b & \mbox{if $j = 1$,} \\
      b\prod_{i \in L(r+1)} f_{x_i}(nt + i)b & \mbox{if $j = r+1$,} \\
      b & \mbox{something else,} \\
      b\prod_{i \in L(u)} f_{x_i}(nt+i)b & \mbox{something else here too.}
    \end{cases}
  \]
\end{proof}
\begin{proof}
  Let $F \in \Pf(\calT)$, $k = |F|$, and enumerate $F = \{ f_1, f_2,
  \ldots, f_k \}$. 
  By the Hales-Jewett Theorem, pick $n \in \bbN$, such that whenever
  $[k]^n$ is 2-colored there exists a monochromatic combinatorial
  line.
  
  Fix $b \in S$ and for each $w = (x_1, x_2, \ldots, x_n) \in [k]^n$
  define $g_w \in \calT$ by $g_w(t) = \prod_{i=1}^n
  \bigr(f_{x_i}(nt+i)b\bigr)$.
  Since $\{\, g_w : w \in [k]^n \,\} \in \Pf(\calT)$ and $A$ is a
  $J$-set, pick $m \in \bbN$, $a \in S^{m+1}$, and $t \in \calJ_m$
  such that for all $w \in [k]^n$, $x(m, a, t, g_w) \in A$.
  Define $\varphi \colon [k]^n \to \{1, 2\}$ by $\varphi(w) = 1 $ if
  $x(m, a, t, g_w) \in A_1$, and $\varphi(w) = 2$ otherwise.
  Pick a variable word $w(\star)$ that begins and ends with a constant
  such that $\bigl\{\, w(s) : s \in
  \{1, 2, \ldots, k\} \,\bigr\}$ is a monochromatic combinatorial line
  with respect to $\varphi$.
  Without loss of generality we suppose that $\varphi\bigl(w(s)\bigr)
  = 1$ for all $s \in \{1, 2, \ldots, k\}$. 

  Let $w(\star) = (x_1, x_2, \ldots, x_n)$ where each $x_i \in \{1, 2,
  \ldots, \} \cup \{\star\}$ and some $x_i = \star$. 
  Let $r$ be the number of variable blocks in $w(\star)$, more
  formally, let $r = \bigm|\bigl\{\, (i, i+1) \in \{1, 2, \ldots, n-1\} :
  x_i = \star, x_{i+1} \in \{1, 2, \ldots, k\} \,\bigr\}\bigm|$.
  Pick $L \in \calI_{r+1}$ and $M \in \calI_r$ such that for each $i
  \in \{1, 2, \ldots, r\}$ , we have $\max L(i) < \min M(i)$, $\max
  M(i) < \min L(i+1)$, and 
  \begin{align*}
    \bigcup_{i=1}^{r+1} L(i) &= \bigl\{\, i \in \{1, 2, \ldots, n \} :
    x_i \in \{1, 2, \ldots, k\} \,\bigr\}, \\
    \bigcup_{i=1}^r M(i) &= \bigl\{\, i \in \{1, 2, \ldots, n \} : x_i
    = \star \,\bigr\}.
  \end{align*}
  (For example, if $k = 3$, $n = 8$, and $w(\star) = (2, \star, \star,
  3, 1, \star, 1, 2)$, then $r = 2$, $L(1) = \{1\}$, $M(1) = \{2,
  3\}$, $L(2) = \{4, 5\}$, $M(2) = \{6\}$, and $L(3) = \{7, 8\}$.)
  
  Looking at the expression of $x(m, a, t, g_w(s))$ it's intuitively
  clear that there exists $p \in \bbN$, $c \in S^{p+1}$, and $u \in
  \calJ_p$ such that $x(m, a, t, g_{w(s)}) = x(p, c, u, f_s)$ for all
  $u \in \{1, 2, \ldots, k\}$. 
  If $A_1$ was not a $J$-set, then  we can go back and select $F$ such
  that $A_1$ fails the definitions of a $J$-set with this $F$. 
  However, then would get a contradiction. 
\end{proof}

\section{Open Questions and Problems}

% Notes section produced by the 'endnotes' package
%\theendnotes

% Things referenced in the introduction. Eventually this will placed
% in a separate file so the References appear at the end.
\bibliographystyle{amsplain}
\bibliography{../references}
\end{document}

