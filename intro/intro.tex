% Draft of Introduction for my Dissertation

\documentclass[12pt]{article}
\usepackage{amsthm, amssymb, amsmath}
\usepackage{endnotes}
% \usepackage[doublespacing]{setspace}

\newtheoremstyle{plain}{3mm}{3mm}{\slshape}{}{\bfseries}{.}{.5em}{}
\theoremstyle{plain}
\newtheorem{thm}{Theorem}[section]
\newtheorem{CSTv1}{Original Central Sets Theorem}
\newtheorem{vdw}[thm]{Van der Waerden's Theorem}
\newtheorem{FST}[thm]{Hindman's Theorem}
\newtheorem{MBR}[thm]{Multiple Birkhoff Recurrence Theorem}
\newtheorem{recur}[thm]{Recurrence Theorem}
\newtheorem{OCST}[thm]{Furstenburg's Original Central Sets Theorem}
\newtheorem{cst}[thm]{Central Sets Theorem}
\newtheorem{cor}[thm]{Corollary}
\newtheorem{prop}[thm]{Proposition}
\newtheorem{lem}[thm]{Lemma}
\newtheorem{claim}[thm]{Claim}
\newtheorem{ques}[thm]{Question}
\newtheorem{conj}[thm]{Conjecture}
\newtheorem{fact}[thm]{Fact}

\theoremstyle{definition}
\newtheorem{defn}[thm]{Definition}
\newtheorem{example}[thm]{Example}
\newtheorem{rmk}[thm]{Remark}


\newcommand{\la}{\langle}
\newcommand{\ra}{\rangle}
\newcommand{\bbN}{\mathbb{N}}
\newcommand{\bbZ}{\mathbb{Z}}
\newcommand{\calP}{\mathcal{P}}
\newcommand{\calT}{\mathcal{T}}
\newcommand{\Pf}{\mathcal{P}_f}

\newcommand{\setfunc}[2]{\hbox{${}^{\hbox{$#1$}}\hskip -1 pt #2$}}

\font\bigmath=cmsy10 scaled \magstep 3
\newcommand{\bigtimes}{\hbox{\bigmath \char'2}}

\newcommand{\cchi}{\raise 2 pt \hbox{$\chi$}}

\begin{document}
\section{Overview}
Using notions from topological dynamics, Furstenberg defined the
concept of a central set of natural numbers and proved a powerful
combinatorial result that has since come to be called the Central Sets
Theorem \cite[Proposition 8.21]{Furstenberg:1981fk}. 
To give a brief demonstration on the versatility of the Central Sets
Theorem we show how we may easily derive van der Waerden's Theorem
from the Central Sets Theorem.
To aid us in our derivation we temporarily assume the following facts:
\begin{itemize}
  \item A central subset of $\bbN$ is a type of ``large'' subset of $\bbN$.
  \item $\bbN$ is a central set.
  \item If $C$ is a central subset of $\bbN$ and $C = C_1 \cup C_2$,
    then either $C_1$ is central or $C_2$ is central.
    For $r \in \bbN$ it follows (by induction on $r$) that if $C$ is a
    central set and $C = \bigcup_{i=1}^r C_i$, then there exists $i
    \in \{1, 2, \ldots, r\}$ such that $C_i$ is central.
  \item The Central Sets Theorem is true.
\end{itemize}
In what follows, if $X$ is a nonempty set, we let $\Pf(X)$ denote the
set of all nonempty finite subsets of $X$.
\begin{cst}
  Let $C \subseteq \bbN$ be central and let $l \in \bbN$.
  For each $i \in \{1, 2, \ldots, l\}$ let $\la y_{i,n}
  \ra_{n=1}^\infty$ be a sequence in $\bbZ$.
  Then there exist sequences $\la a_n \ra_{n=1}^\infty$ in $\bbN$ and
  $\la H_n \ra_{n=1}^\infty$ in $\Pf(\bbN)$ such that 
  \begin{itemize}
    \item[(1)] for every $n \in \bbN$, $\max H_n < \min H_{n+1}$; and
    \item[(2)] for every $G \in \Pf(\bbN)$ and $i \in \{1, 2, \ldots,
      l\}$ we have
      \[
        \sum_{n \in G}\Bigl(a_n + \sum_{t \in H_n} y_{i,t}\Bigr) \in C.
      \]
  \end{itemize}
\end{cst}

\begin{vdw}
  If $r \in \bbN$ and $\bbN = \bigcup_{i=1}^r C_i$, then for each $l
  \in \bbN$ there exist $i \in \{1, 2, \ldots, r\}$ and $a$, $d \in
  \bbN$ such that $\{\, a, a+d, \ldots, a+ld \,\} \subseteq C_i$.
\end{vdw}
\begin{proof}
  By our temporary assumption we may pick $i \in \{1, 2, \ldots, r\}$
  such that $C_i$ is central. 
  Fix a sequence $\la x_n \ra_{n=1}^\infty$ in $\bbN$ and let $l \in
  \bbN$.
  For each $j \in \{0, 1, \ldots, l\}$ defined the sequence $\la
  y_{j,n} \ra_{n=1}^\infty$ in $\bbZ$ by $y_{j,n} = j \cdot x_n$. 
  Pick sequences $\la a_n \ra_{n=1}^\infty$ in $\bbN$ and $\la H_n
  \ra_{n=1}^\infty$ in $\Pf(\bbN)$ as guaranteed by the Central Sets
  Theorem. 
  Put $a = a_1$ and $d = \sum_{t \in H_1} x_t$. 
  Then for $j \in \{0, 1, \ldots, l\}$ we have $a+jd = a_1 +
  j\cdot\sum_{t \in H_1} x_t = a_1 + \sum_{t \in H_1} y_{j,t} \in C_i$.
  This completes the proof.
\end{proof}
\begin{rmk}
  By a suitable choice of our sequence $\la x_n \ra_{n=1}^\infty$ in
  the proof we may strengthen the conclusion of this theorem by
  stating that there are ``many'' $l+1$-term arithmetic progressions
  in $C_i$. 
\end{rmk}

Van der Waerden's proof of his theorem \cite{Van-der-Waerden:1927fk}
uses a complicated, but elementary, combinatorial argument based on
the pigeonhole principle and ``double induction''.%
\endnote{
  I'm not sure of the exact form of van der Waerden's argument since I
  have never read the paper \cite{Van-der-Waerden:1927fk}. 
  This statement is based on my opinion that all known combinatorial
  proofs of van der Waerden's Theorem are not simple, Khinchin's
  opinion on the difficulty of the original proof \cite[expressed in
  Section 1, pages 11--12]{Khinchin:1998fk}, and van der Waerden's
  description on how he found his proof in
  \cite{Van-der-Waerden:1971fk}. 
  The interested reader can find a combinatorial proof of van der
  Waerden's Theorem in Section 2 of \cite{Hindman:1979fk}.
}
Obviously using partition regularity of central sets and the Central
Sets Theorem makes van der Waerden's Theorem easy to derive.
At this point the reader may suspect that the ``true'' difficulties
then must lie in proving these facts. 
However we shall soon see that hardest part will be absorbing all of
the needed definitions. 

\section{Algebraic and Topological Preliminaries}
A \textsl{semigroup} is a pair $(S, \cdot)$ such that $S$ is a
nonempty set and $\cdot$ is an associative binary operation on $S$,
that is, for all $x$, $y$, and $z \in S$, $x \cdot (y \cdot z) = (x
\cdot y) \cdot z$.
As usual in mathematics we will often suppress the explicit mention of
our binary operation $\cdot$ and simply write that $S$ is a semigroup.
Clearly any group is also a semigroup, but the canonical
example of a semigroup that is not a group is the set of
natural numbers under addition.
Another class of examples can be obtained by considering the
multiplicative part of a ring. 

In analogy with the definition of a commutative ring, we define a
\textsl{commutative semigroup} as a semigroup $S$ such that for all
$x$ and $y \in S$, $xy = yx$.
Also mimicking ring theory, we define certain ideals in semigroups.
Given a semigroup $S$ and a $\emptyset \ne I \subseteq S$, we call $I$
a \textsl{left ideal (of S)} if and only if $SI \subseteq I$; we call
$I$ a \textsl{right ideal of (S)} if and only if $IS \subseteq I$;
and, finally, we call $I$ a \textsl{(two-sided) ideal (of S)} if and
only if $I$ is both a left ideal and right ideal.
(The phrases in parentheses will often be omitted if they are clear
from context.)

The study of ideals in semigroups is one of the main differences
between semigroup theory and group theory.
(A semigroup is a group if and only if the only left and right ideals
is the semigroup itself.)
Moreover, semigroup theory is different from ring theory too, as the
following simple example shows that not every semigroup can be
embedded as the multiplicative semigroup of some ring.

\begin{example}
  We call $(S, \cdot)$ a \textsl{right-zero semigroup} if and only if
  for every $x$, $y \in S$, $x \cdot y = y$.
  A simple check shows that if $S$ is nonempty, this definition really
  produces a semigroup.
  The name ``right-zero'' is derived from the fact that, in the
  natural numbers, multiplying on the right by zero yields zero.
  Hence in a right-zero semigroup every element is ``zero'' when
  multiplied on the right.

  Suppose that $S$ is a right-zero semigroup with $|S| > 2$. 
  We show that $S$ cannot be the multiplicative semigroup of any
  ring. 
  Suppose, to the contrary, that there exists such a ring.
  Then for all $a$, $b$, and $c \in S$, we have that $(b+c) \cdot a =
  b \cdot a + c \cdot a$ by the ring axioms.
  Since $S$ is right-zero, $(b+c) \cdot a = a$, $b \cdot a = a$, and
  $c \cdot a = a$.
  Hence $a = a + a$ and this implies that $a = 0$. 
  Then $|S| > 2$ implies there exists $a$, $a' \in S$ with $a \ne a'$
  and $a = 0 = a'$.
\end{example}

We call a left ideal $L$ of a semigroup \textsl{minimal} if and only
if $L$ doesn't contain any proper left ideals.
We define the concept of a \textsl{minimal right ideal} dually, and we
can an ideal $I$ \textsl{minimal} if it doesn't contain any proper
ideal. 
The following result proves that a semigroup contains at most one
minimal ideal.

\begin{prop}
  Let $S$ be a semigroup and let $K$ be a minimal ideal.
  If $I$ is an ideal, then $K \subseteq I$.
\end{prop}
\begin{proof}
  It suffices to show that $K \cap I$ is an ideal.  
  Then the conclusion will follow from the minimality of $K$ and the
  fact that $K \cap I \subseteq I$.

  To see that $K \cap I \ne \emptyset$, let $x \in K$ and $y \in I$.
  Then since $K$ and $I$ are ideals, $xy \in K$ and $xy \in I$.

  Now let $x \in K \cap I$ and $y \in S$.
  Since $K$ and $I$ are ideals, $xy \in K$, $xy \in I$, $yx \in K$,
  and $yx \in I$.
\end{proof}

With this result we are justified in making the following important
definition.
\begin{defn}
  Let $S$ be a semigroup.
  If $S$ has a minimal ideal, then we call this ideal the
  \textsl{smallest ideal (of S)} and denote this ideal by $K(S)$.%
  \endnote{
  }
\end{defn}
\begin{rmk}
  Not all semigroups possess a smallest ideal.
  For instance, $(\bbN, +)$ and $(\bbN, \cdot)$ both don't have
  smallest ideals. 
\end{rmk}

Given a semigroup $S$, for each $x \in S$ we define the functions
$\lambda_x \colon S \to S$ and $\rho_x \colon S \to S$ by
$\lambda_x(y) = xy$ and $\rho_x(y) = yx$. 
For $A \subseteq S$ and $x \in S$, we use the special notation
$x^{-1}A = x^{-1}A = \lambda_x^{-1}[A]$. 
(Hence $x^{-1}A = \{\, y \in S : xy \in A \,\}$.)
Finally, we call an element $e \in S$ an \textsl{idempotent} if and
only if $ee = e$.

We shall not go further into the (purely) algebraic theory of
semigroups; but, the interested reader can consult the article
\cite{Hollings:2007uq} for a brief introduction or the monograph
\cite{Clifford:1961fk} for more extensive information. 
For our purposes the semigroup theory contained in \cite[Chapters 1
and 2]{Hindman:1998fk} will suffice. 

We now define how topology can interact with our semigroup operation
in various ways.  




\section{Algebraic Preliminaries}
The concept of a central set has been generalized to semigroups, and
admits a simple definition using the algebraic structure of the
Stone-\v{C}ech compactification of a discrete semigroup.
We chose to give this algebraic technology priority for several
reasons: proving that central sets are partition regular automatically
falls out of our definition, proving the Central Sets Theorem requires
a fairly easy induction argument, and strengthen its conclusion also
becomes easy. 
\begin{defn}
  Let $S$ be a nonempty set, let $* \colon S \times S \to S$ be
  a binary operation on $S$, and let $J \subseteq S$ be a nonempty
  subset of $S$.
  \begin{itemize}
    \item[(a)] We call the pair $(S, *)$ a \textsl{semigroup} if
      and only if for all $x$, $y$, and $z \in S$, $(x * y) *
      z = x * (y * z)$.
    \item[(b)] For each $x \in S$ define the functions $\lambda_x
      \colon S \to S$ and $\rho_x \colon S \to S$ by $\lambda_x(y) = x
      * y$ and $\rho_x(y) = y * x$, respectively.
    \item[(c)] We call an element $x \in S$ \textsl{idempotent} if and
      only if $x*x = x$. 
    \item[(d)] We say $J$ is a \textsl{left ideal (of S)} if and only
      if $S * J \subseteq J$.
      A left ideal is a \textsl{minimal left ideal} if and only if the
      only left ideal it contains is itself.
    \item[(e)] We say $J$ is a \textsl{right ideal (of S)} if and only
      if $J * S \subseteq J$.
      A right ideal is a \textsl{minimal right ideal} if and only if the
      only right ideal it contains is itself.
    \item[(f)] We say $J$ is a \textsl{(two-sided) ideal (of S)} if and only
      if $J$ is both a left ideal and right ideal.
      An ideal is a \textsl{minimal ideal} if and only if the only
      ideal it contains is itself.
  \end{itemize}
\end{defn}
\begin{rmk}
  As usual in mathematics, we will often denote the binary operation
  on our semigroup by $\cdot$ and often we will omit writing it at
  all.
\end{rmk}

The following easy results show that a semigroup contains at most
one minimal ideal, and that a minimal ideal is contained in every ideal.
\begin{prop}
  Let $S$ be a semigroup.
  If $J_1$ and $J_2$ are minimal ideals of $S$, then $J_1 = J_2$.
\end{prop}
\begin{proof}
  We show that $J_1 \cap J_2$ is an ideal of $S$. 
  The conclusion then follows from the minimality of $J_1$ and $J_2$.
  
  To see that $J_1 \cap J_2 \ne \emptyset$, let $x \in J_1$ and
  $y \in J_2$. 
  Since $J_1$ is a right ideal, $xy \in J_1$. 
  Since $J_2$ is a left ideal, $xy \in J_2$. 
  Hence $xy \in J_1 \cap J_2$. 

  To see that $J_1 \cap J_2$ is an ideal, let $x \in J_1 \cap J_2$ and
  $y \in S$. 
  Since $J_1$ and $J_2$ are both left ideals, we have $xy \in J_1$ and
  $xy \in J_2$.
  Similarly since $J_1$ and $J_2$ are both right ideals, we have $yx
  \in J_1$ and $yx \in J_2$. 
  Therefore $J_1 \cap J_2$ is an ideal.

  Finally, since $J_1 \cap J_2 \subseteq J_1$ and $J_1 \cap J_2
  \subseteq J_2$ it follows by minimality of $J_1$ and $J_2$ that $J_1
  = J_1 \cap J_2 = J_2$.
\end{proof}
\begin{prop}
  Let $S$ be a semigroup with the minimal ideal $J$. 
  If $I$ is an ideal of $S$, then $J \subseteq I$.
\end{prop}
\begin{proof}
  Similar to the proof above, $J \cap I$ is an ideal of $S$. 
  By minimality of $J$ we have $J \cap I = J$. 
  Therefore it follows that $J \subseteq I$.
\end{proof}

With these results we are justified in making the following definition. 
\begin{defn}
  Let $(S,\cdot)$ be a semigroup. 
  We call the minimal ideal of $S$ the \textsl{smallest ideal} and
  denote it by $K(S)$.%
  \endnote{
    The notation of $K(S)$ denoting the smallest ideal comes from the
    Russian mathematician Suschkewitsch.
    He called the smallest ideal the \textsl{kernel} of the
    semigroup.
    Suschkewitsch, based off of the description given in
    \cite{Hollings:2009uq}, in \cite{Suschkewitsch:1928kx} proved that
    all finite semigroups have a smallest ideal and using this fact
    was able to produce a structure theorem for finite semigroups. 
  }
\end{defn}
Unfortunately not all semigroups have a smallest ideal or
idempotents. 
However, we will only be concerned with a certain class of topological
semigroups that has both of these objects.

\begin{defn}
  A \textsl{compact right-topological semigroup} is a triple $(S,
  \calT, \cdot)$ such that $(S, \calT)$ is a compact Hausdorff space, $(S, \cdot)$ is
  a semigroup, and for every $x \in S$, $\rho_x \colon S
  \to S$ is continuous. 
\end{defn}

\begin{thm}
  Compact right-topological semigroups contains idempotents and
  the smallest ideal.
\end{thm}
\begin{proof}
  The existence of idempotents and the smallest ideal is proved in \cite[Theorem
  2.5]{Hindman:1998fk} and \cite[Theorem 2.8]{Hindman:1998fk}, respectively.
\end{proof}
The importance of the smallest ideal for us is that with a suitably
constructed semigroup we will be able to define central sets in terms
of idempotents of the smallest ideal.

\begin{defn}
  Let $S$ be a nonempty set.
  A nonempty set $p \subseteq \calP(S)$ is called an
  \textsl{ultrafilter} if and only if $p$ satisfies the following
  four properties: 
  \begin{itemize}
    \item[(1)] If $A$, $B \in p$, then $A \cap B \in p$.
    \item[(2)] If $A \in p$ and $A \subseteq B \subseteq S$, then $B
      \in p$.
    \item[(3)] $\emptyset \not\in p$.
    \item[(4)] For all $A \subseteq S$, either $A \in p$ or $S
      \setminus A \in p$. 
  \end{itemize}
\end{defn}

Given a discrete semigroup $(S,\cdot)$, we let $\beta S$ be the
collection of all ultrafilters on $S$, and identify the principal
ultrafilters with the points of $S$. 
For all $A \subseteq S$, we let $\overline{A} = \{\, p \in \beta S : A
\in p \,\}$. 
Then $\{\, \overline{A} : A \subseteq \,\}$ is a basis of open sets in
$\beta S$.
We can extend the semigroup operation on $S$ to $\beta S$ in the
following manner.

\section{Commutative Central Sets Theorem}
\section{Central Sets Theorem}
\section{Open Questions and Problems}

% Notes section produced by the 'endnotes' package
\theendnotes

% Things referenced in the introduction. Eventually this will placed
% in a separate file so the References appear at the end.
\bibliographystyle{amsplain}
\bibliography{../references}
\end{document}

