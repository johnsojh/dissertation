% Draft of Introduction for my Dissertation

\documentclass[12pt]{article}
\usepackage{amsthm, amssymb, amsmath}
\usepackage{endnotes}
\usepackage{todonotes}
\usepackage[doublespacing]{setspace}
\usepackage{url}

\newtheoremstyle{plain}{3mm}{3mm}{\slshape}{}{\bfseries}{.}{.5em}{}
\theoremstyle{plain}
\newtheorem{thm}{Theorem}[section]
\newtheorem{CSTv1}{Original Central Sets Theorem}
\newtheorem*{vdw}{Van der Waerden's Theorem}
\newtheorem*{hj}{Hales-Jewett Theorem}
\newtheorem{FST}[thm]{Hindman's Theorem}
\newtheorem{MBR}[thm]{Multiple Birkhoff Recurrence Theorem}
\newtheorem{recur}[thm]{Recurrence Theorem}
\newtheorem{OCST}[thm]{Furstenburg's Original Central Sets Theorem}
\newtheorem{cst}[thm]{Central Sets Theorem}
\newtheorem{cor}[thm]{Corollary}
\newtheorem{prop}[thm]{Proposition}
\newtheorem{lem}[thm]{Lemma}
\newtheorem{claim}[thm]{Claim}
\newtheorem{ques}[thm]{Question}
\newtheorem{conj}[thm]{Conjecture}
\newtheorem{fact}[thm]{Fact}

\theoremstyle{definition}
\newtheorem{defn}[thm]{Definition}
\newtheorem{example}[thm]{Example}
\newtheorem{rmk}[thm]{Remark}


\newcommand{\la}{\langle}
\newcommand{\ra}{\rangle}
\newcommand{\bbN}{\mathbb{N}}
\newcommand{\bbZ}{\mathbb{Z}}
\newcommand{\calG}{\mathcal{G}}
\newcommand{\calI}{\mathcal{I}}
\newcommand{\calJ}{\mathcal{J}}
\newcommand{\calP}{\mathcal{P}}
\newcommand{\calT}{\mathcal{T}}
\newcommand{\Pf}{\mathcal{P}_f}

\newcommand{\setfunc}[2]{\hbox{${}^{\hbox{$#1$}}\hskip -1 pt #2$}}

\font\bigmath=cmsy10 scaled \magstep 3
\newcommand{\bigtimes}{\hbox{\bigmath \char'2}}

\newcommand{\cchi}{\raise 2 pt \hbox{$\chi$}}

\begin{document}
\section{Algebraic and Topological Preliminaries}
We start by giving a, possibly laconic, introduction to the basic
definitions and results we need from algebraic semigroup theory and the
algebraic structure of the Stone-\v{C}ech compactification.
The interested reader can consult the article \cite{Hollings:2007uq}
for a more expansive, but still brief, introduction to algebraic
semigroup theory or the monographs \cite{Clifford:1961fk} and
\cite{Clifford:1967fk} for more extensive information.
\textsl{The} standard reference for detailed information about the
algebraic structure of the Stone-\v{C}ech compactification is
\cite{Hindman:1998fk}, along with the survey papers \cite{Hindman:2002kx} and
\cite{Hindman:2010uq} which contains the latest information on this
algebraic structure. 

\subsection{Algebraic Semigroup Theory}
A \textsl{semigroup} is a pair $(S, \cdot)$ such that $S$ is a
nonempty set
and $\cdot$ is an associative binary operation on $S$, that is, $x \cdot
(y \cdot z) = (x \cdot y) \cdot z$ for all $x$, $y$, and $z \in S$.
As usual in mathematics, we will often simply write that `$S$ is a
semigroup' and suppress the explicit mention
of our associative binary operation $\cdot$ in algebraic formulas. 
We call an element $e \in S$ an \textsl{idempotent} if and only if $ee
= e$. 
Clearly any group and the multiplicative part of a ring are examples
of semigroups.
However, semigroup theory is different from group theory and ring
theory since there are semigroups which cannot naturally be embedded
in any group, and there are semigroups which cannot form the
multiplicative part of any ring.
For instance, a semigroup may have many idempotents but a group only
has one.

Given a semigroup $S$ for each $x \in S$, we define functions
$\lambda_x \colon S \to S$ and $\rho_x \colon S \to S$ by
$\lambda_x(y) = xy$ and $\rho_x(y) = yx$. 
For $A \subseteq S$ and $x \in S$, we use the special notation
$x^{-1}A$ to mean $\lambda_x^{-1}[A]$ and $Ax^{-1}$ to mean
$\rho_x^{-1}[A]$, that is, $x^{-1}A = \{\, y \in S : xy \in A \,\}$
and $Ax^{-1} = \{\, y \in S : yx \in A \,\}$.

Let $\emptyset \ne I \subseteq S$, where $S$ is a semigroup.
We call $I$ a \textsl{left ideal (of S)} if and only if for every $x
\in S$, $\lambda_x[I] \subseteq I$; we call $I$ a \textsl{right ideal
  (of S)} if and only if for every $x \in S$, $\rho_x[I] \subseteq
I$; and finally, we call $I$ a \textsl{(two-sided) ideal (of S)} if
and only $I$ is both a left ideal and right ideal. 
(The phrases in parenthesis will often be omitted if they are clear
from context.)
A semigroup can contain, at most one, smallest ideal, that is, a
two-sided ideal contained in every ideal.
We denote the smallest ideal of a semigroup $S$, if it exists, by
$K(S)$. 
We have the following important fact about the smallest ideal.
\begin{thm}
  Let $\la S_i : i \in I \ra$ be a family of semigroups and let $S =
  \bigtimes_{i \in I} S_i$ and define an associative binary operation
  on $S$ coordinatewise in terms of the operations on each $S_i$.
  Suppose that, for each $i \in I$, $S_i$ has a smallest ideal.
  Then $S$ has a smallest ideal and $K(S) = \bigtimes_{i \in I}
  K(S_i)$. 
\end{thm}
\begin{proof}
  A proof of this fact is in \cite[Theorem 2.23]{Hindman:1998fk}.
\end{proof}

\subsection{Compact Right-Topological Semigroups}
A \textsl{right-topological semigroup} $S$ is a semigroup which is
also a topological space such that for every $x \in S$, the function
$\rho_x$ is continuous. 
From now on, unless otherwise noted, we will assume that all mentioned
topological spaces are Hausdorff. 
We this assumption we automatically gain the following facts about
compact right-topological semigroups

\begin{thm}
  \label{thm:IdK_S}
  Let $S$ be a compact right-topological semigroup.
  Then $S$ contains an idempotent and the smallest ideal of $S$ exists.
\end{thm}
\begin{proof}
  This conclusion is proved in \cite[Theorem2.5]{Hindman:1998fk}
  \cite[Theorem 2.8]{Hindman:1998fk},
\end{proof}

\begin{thm}
  Let $\la S_i : i \in I \ra$ be a family of right-topological
  semigroups and let $S = \bigtimes_{i \in I} S_i$.
  Give $S$ the product topology and define a binary operation on $S$
  coordinatewise. 
  \begin{itemize}
    \item[(a)] If each $S_i$ is compact, then $S$ is compact.

    \item[(b)] If $\vec{x} \in S$ and for each $i \in I$,
      $\lambda_{x_i} \colon S_i \to S_i$ is continuous, then
      $\lambda_{\vec{x}} \colon S \to S$ is continuous.
  \end{itemize}
\end{thm}
\begin{proof}
  The proof of (a) follows directly from Tychonoff's Theorem,
  and the proof of (b) is a straightforward exercise.
\end{proof}

\subsection{Algebra in the Stone-\v{C}ech Compactification}
\begin{defn}
  Let $S$ be a nonempty set.
  A set $\emptyset \ne p \subseteq \calP(S)$ is called an
  \textsl{ultrafilter (on S)} if and only if $p$ satisfies the
  following four conditions:
  \begin{itemize}
    \item[(1)] $\emptyset \not\in p$.
    \item[(2)] If $A$ and $B \in p$, then $A \cap B \in p$.
    \item[(3)] If $A \in p$ and $A \subseteq B \subseteq S$, then $B
      \in p$.
    \item[(4)] For all $A \subseteq S$, either $A \in p$ or $S
      \setminus A \in p$. 
  \end{itemize}
\end{defn}
We call an ultrafilter $p$ on $S$ a \textsl{principal ultrafilter} if and
only if there exists $a \in S$ such that $p = \{\, A \subseteq S : a
\in A \,\}$.
Nonprincipal ultrafilters, on an infinite set, can be shown to exist
by Zorn's Lemma.%
\endnote{
  In fact, the existence of nonprincipal ultrafilters requires some
  weak from of choice.  
  See section 2 of \cite{Rav:1977ys} for an interesting collection of
  statements which are essentially combinatorial formulations of the
  Ultrafilter Principle.  
  (The Ultrafilter Principle states that every filter can
  be extended to an ultrafilter.  
  This principle is known to be stronger than the statement that every
  infinite set has a nonprincipal ultrafilter.) 
  These combinatorial statements are essentially variants of Rado's
  Selection Lemma.
  
  However the reader probably shouldn't read too much into fact that
  there are combinatorial statements equivalent to the Ultrafilter
  Principle, since this principle is still a ``nonconstructive''
  statement. 
  See Chapter 6 in \cite{Schechter:1997fk} for a discussion on this
  point. 
} 

Given a set $S$, we let $\beta S = \{\, p : \mbox{$p$ is an
  ultrafilter on $S$} \,\}$.
If $S$ is also a discrete topological space, then we can topologize
$\beta S$ as follows. 
For each $A \subseteq S$, define $\overline{A} = \{\, p \in \beta S :
A \in p \,\}$. 
Then $\{\, \overline{A} : A \subseteq S \,\}$ forms a basis for a
compact Hausdorff
topology on $\beta S$.
In fact, by \cite[Theorem 3.27]{Hindman:1998fk} this topology is the
Stone-\v{C}ech compactification of $S$. 

Finally, if $S$ is a semigroup with the discrete topology, then we can
extend the semigroup operation on $S$ to $\beta S$ \cite[Theorem
4.1]{Hindman:1998fk}.
This turns $\beta S$ into a compact right-topological semigroup with
the property that for all $s \in S$, $\lambda_s$ is continuous. 
In particular, $\beta S$ has at least one idempotent by Theorem
\ref{thm:IdK_S} (it typically has more) and the smallest ideal
$K(\beta S)$.

We shall constantly use the following theorem.
\begin{thm}
  Let $S$ be a semigroup and $A \subseteq S$.
  \begin{itemize}
    \item[(a)] For any $p$, $q \in \beta S$, $A \in pq$ if and only if
      $\{\, s \in S : s^{-1}A \in q \,\} \in p$. 
    
    \item[(b)] For any $p \in \beta S$ and $s \in S$, $A \in sp$ if
      and only if $s^{-1}A \in p$.

    \item[(c)] For any $p \in \beta S$ and $s \in S$, $A \in ps$ if
      and only if $As^{-1} \in p$.
  \end{itemize}
\end{thm}
\begin{proof}
  \cite[Theorem 4.12]{Hindman:1998fk} contains a proof for (a) and
  (b).
  Statement (c) is a straight-forward exercise (once you have seen the
  proof of (a) and (b)).
\end{proof}

\section{Combinatorial Motivation} 
Let $S$ be a set and let $\calG \subseteq \calP(S)$.
We call the pair $(S, \calG)$ \textsl{partition regular} if and only
if for every $r \in \bbN$, if $S = \bigcup_{i=1}^r C_i$, then there
exist $i \in \{1, 2, \ldots, r\}$ and $G \in \calG$ such that $G
\subseteq C_i$. 
With this bit of terminology (originally due to Rado in
\cite{Rado:1945kx}), we can roughly define Ramsey Theory as the
  study and classification of partition regular sets. 
With such a ``definition'', it's a fact that studying a partition
regular set $(S, \calG)$ is equivalent to studying certain ultrafilters on
$S$. 
(See \cite[Theorem 5.7]{Hindman:1998fk} for a proof of this fact.)
For an example of a partition regular set, take $S$ be an infinite set
and let $\calG = \{\, A \subseteq S : \mbox{$A$ is infinite} \,\}$, then the
pigeonhole principle implies that $(S, \calG)$ is partition regular.

In this dissertation we will mainly consider partition regular $(S,
\calG)$, where $S$ is a semigroup and elements of $\calG$ are defined
in terms of the algebraic structure of $S$.
Moreover, if we give $S$ the discrete topology we can give $\beta S$,
the Stone-\v{C}ech compactification of $S$, the structure of a Hausdorff
compact right-topological semigroup 
such that elements in $\calG$ are associated with certain algebraic
structures of $\beta S$. 
(A \textsl{right-topological semigroup} is a semigroup that is also a
topological space such that for every element $y$ in our semigroup the
function $x \mapsto xy$ is continuous.)
Hence studying $(S, \calG)$, which is a purely algebraic and
combinatorial object, is equivalent to studying the algebraic and
topological structure of $\beta S$. 
In this section, we focus purely on the algebraic and combinatorial
side of this equivalence.

\begin{rmk}
  In what follows we shall intentionally use the term ``partition
  regular'' rather loosely by not explicitly mentioning the set
  $\calG$. 
  For instance, we often say that property $P$ is partition regular if
  and only if whenever $S$ has property $P$ and $S = \bigcup_{i=1}^r
  C_i$, then $C_i$ has property $P$ for some $i \in \{1, 2, \ldots,
  r\}$. 
\end{rmk}

We begin by giving one of the earliest Ramsey Theoretic result that
still forms the basis for much of the contemporary research in the
field.
This theorem states that the property of containing arbitrarily long
finite arithmetic progressions is partition regular. 

\begin{vdw}
  Let $r \in \bbN$.
  If $\bbN = \bigcup_{i=1}^r C_i$, then there exist $i \in \{1, 2,
  \ldots, r\}$ such that for all $\ell \in \bbN$, there exist $a$, $d
  \in \bbN$ with $\{\, a, a+d, \ldots, a+\ell d \,\} \subseteq C_i$. 
\end{vdw}

\begin{rmk}
  Van der Waerden's proof of his theorem in
  \cite{Van-der-Waerden:1927fk} uses a complicated, but elementary,
  combinatorial argument based on the pigeonhole principle and ``double
  induction.''%
  \endnote{
    I'm not sure of the exact form of van der Waerden's argument since I
    have never read the paper \cite{Van-der-Waerden:1927fk}. 
    This statement is based on my opinion that all known combinatorial
    proofs of van der Waerden's Theorem are not simple, Khinchin's
    opinion on the difficulty of the original proof \cite[expressed in
    Section 1, pages 11--12]{Khinchin:1998fk}, and van der Waerden's
    description on how he found his proof in
    \cite{Van-der-Waerden:1971fk}. 
    (Van der Waerden's description in \cite{Van-der-Waerden:1971fk}
    seems to imply that his proof is similar to the one in
    \cite[Chapter 1]{Khinchin:1998fk}.)
    The interested reader can find a combinatorial proof of van der
    Waerden's Theorem in Section 2 of \cite{Hindman:1979fk} or
    \cite[Chapter 1]{Khinchin:1998fk}.
  }  
\end{rmk}

Van der Waerden's Theorem can be rewritten to apply to any semigroup,
and, as observed in the Introduction of \cite{Bergelson:1992fk}, this
fact follows directly from van der Waerden's Theorem itself.

\begin{thm}
  Let $S$ be a semigroup and let $r \in \bbN$.
  If $S = \bigcup_{i=1}^r C_i$, then there exists $i \in \{1, 2,
  \ldots, r\}$ such that for all $\ell \in \bbN$, there exist $a$, $d
  \in S$ with $\{\, a , ad, \ldots, ad^\ell \,\} \subseteq C_i$. 
\end{thm}
\begin{proof}
  Fix $x \in S$.
  We shall consider the set $\{\, x^n : n \in \bbN \,\}$.
  For each $i \in \{1, 2, \ldots, r\}$, define $B_i = \{\, n \in \bbN
  : x^n \in C_i \,\}$.
  Then $\bbN = \bigcup_{i=1}^r B_i$, and we can apply van der
  Waerden's Theorem to pick $i \in \{1, 2, \ldots, r\}$ such that for
  all $\ell \in \bbN$ there exist $b$, $c \in \bbN$ with $\{\, b, b+c,
  \ldots, b+\ell c \,\} \subseteq B_i$. 
  By definition of $B_i$, we have that $\{\, x^b, x^{b+c}, \ldots,
  x^{b+ \ell c} \,\} \subseteq C_i$. 
  To complete the proof, simply put $a = x^b$ and $d = x^c$. 
\end{proof}

However as also observed in the Introduction of
\cite{Bergelson:1992fk} there are two potential problems with such a
straightforward approach:
Our increment $d$ may be a right identity element in our semigroup;
and, even if $d$ is not a right identity element, there is no
guarantee that all the elements in $\{\, a, ad, \ldots, ad^{\ell}
\,\}$ are distinct. 
(See the Introduction of \cite{Bergelson:1992fk} for an example of
each potential problem.)
Moreover, we shall find that when we consider the algebraic structure
of $\beta S$, that such a straightforward generalization doesn't
interact nicely with the algebraic structure of $\beta S$ when $S$ is
a noncommutative semigroup. 

One of the first proper generalization of a set that contains
arbitrarily long finite arithmetic progression to a semigroup is the
notion of a piecewise syndetic set. 
However, since the definition of a piecewise syndetic set is, initially,
somewhat difficult to understand, we choose to give some 
motivation on how this concept can be derived before explicitly
defining it. 
Consequently, we start by proving that a certain statement is
equivalent to van der Waerden's Theorem. 

In what follows, if $A \subseteq \bbN$ and $x \in \bbN$, we define $-x
+ A = \{\, y \in \bbN : x + y \in A \,\}$.

\begin{thm}
  \label{thm:vdwEqSyn}
  The following statements are equivalent.
  \begin{itemize}
    \item[(a)] Let $r \in \bbN$.
      If $\bbN = \bigcup_{i=1}^r C_i$, then there exists $i \in \{1, 2,
      \ldots, r\}$ such that for all $\ell \in \bbN$, there exist $a$,
      $d \in \bbN$ with $\{\, a, a+d, \ldots, a+\ell d \,\} \subseteq
      C_i$. 

    \item[(b)] Let $\la x_n \ra_{n=1}^\infty$ be a strictly increasing
      sequence in $\bbN$ with $\max\{\, x_{n+1} - x_n : n \in \bbN
      \,\}$ finite.
      Then for all $\ell \in \bbN$, the set $\{\, x_n : n \in \bbN
      \,\}$ contains a \mbox{$\ell$-term} arithmetic progression.
  \end{itemize}
\end{thm}
\begin{proof}
  \textsl{(a) $\implies$ (b).}
  Let $\la x_n \ra_{n=1}^\infty$ be a strictly increasing sequence in
  $\bbN$ with $\max\{\, x_{n+1} - x_n : n \in \bbN \,\}$ finite. 
  Put $b =\max\bigl(\{\, x_{n+1} - x_n : n \in \bbN \,\} \cup
  \{x_1\}\bigr)$ and $A = \{\, x_n : n \in \bbN \,\}$. 
  We claim that $\bbN = \bigcup_{t=1}^b -t+A$. 
  Suppose temporarily that this claim is true.
  We can then apply (a) to pick $i \in \{1, 2, \ldots, b\}$ such that
  for all $\ell \in \bbN$, the set $-i + A$ contains a
  \mbox{$\ell$-term} arithmetic progression. 
  From this situation it follows that $A$ contains a
  \mbox{$\ell$-term} arithmetic progression for every $\ell \in
  \bbN$. 
  We now prove our claim.
  
  Clearly, $\bigcup_{t=1}^b -t+A \subseteq \bbN$.
  Let $x \in \bbN$.
  First, assume that $x < x_1$, then since $x_1 \le b$, we can pick $t
  \in \{1, 2, \ldots, b\}$ such that $t + x = x_1$, that is, $x \in
  -t+A$.
  Now assume that $x_1 \le x$.
  Since $\la x_n \ra_{n=1}^\infty$ is a strictly increasing sequence,
  we can pick $n \in \bbN$ such that $x_n \le x < x_{n+1}$.
  Moreover, since $x_{n+1} - x_n \le b$, we can pick $t \in \{1, 2,
  \ldots, b\}$ such that $t + x = x_{n+1}$, that is, $x \in -t + A$.
  This completes the proof for this direction. 

  \textsl{(b) $\implies$ (a).}
  Let $r \in \bbN$, and $\bbN = \bigcup_{i=1}^r C_i$. 
  By the pigeonhole principle, we can pick $i \in \{1, 2, \ldots, r\}$
  such that $C_i$ is infinite.
  Without loss of generality suppose that $C_1$ is infinite. 
  Enumerate $C_1$ as a strictly increasing sequence $\la x_n
  \ra_{n=1}^\infty$.
  If $\max\{\, x_{n+1} - x_n : n \in \bbN \,\}$ is finite, then we are
  done by (b). 
  Hence suppose instead that $\max\{\, x_{n+1} - x_n : n \in \bbN
  \,\}$ is infinite, that is, for all $\ell \in \bbN$, there exists $n
  \in \bbN$ with $x_{n+1} - x_n \ge \ell + 2$. 
  Therefore, for all $\ell \in \bbN$, there exists $n \in \bbN$ such
  that $\{\, x_n + 1, x_n + 2, \ldots, x_n + (\ell + 1) \,\} \subseteq
  \bigcup_{i=2}^r C_i$. 
  This means that $\bigcup_{i=2}^r C_i$ contains a \mbox{$\ell$-term}
  arithmetic progression for every $\ell \in \bbN$.
  In particular, $\bigcup_{i=2}^r C_i$ is infinite and so by the
  pigeonhole principle, we may pick $i \in \{2, 3, \ldots, r\}$ such
  that $C_i$ is infinite.
  Without loss of generality, suppose that $C_2$ is infinite.

  Enumerate $C_2$ as a strictly increasing sequence $\la x_n
  \ra_{n=1}^\infty$.
  If $\max\{\, x_{n+1} - x_n : n \in \bbN \,\}$ is finite, then we are
  done by (b).
  Suppose that $\max\{\, x_{n+1} - x_n : n \in \bbN \,\}$ is infinite,
  then, as above, for every $\ell \in \bbN$, there exists $n \in
  \bbN$, such that $\{\, x_n +1, x_n + 2, \ldots, x_n + (\ell +1) \,\}
  \subseteq \bigcup_{i=3}^r C_i$. 
  Since $\bigcup_{i=3}^r C_i$, we may continue as before.
  Eventually, we will stop at a set with bounded gaps or a set that
  contains arbitrarily long arithmetic progressions.%
  \endnote{
    \cite[Theorem 2.5]{Landman:2004fk} contains a more complete list
    of statements equivalent to van der Waerden's Theorem.
  }
\end{proof}

We wish to translate statement Theorem \ref{thm:vdwEqSyn} (b) into
the language of semigroups.
Of course, given an arbitrary semigroup, there is no guarantee that 
the terms ``strictly increasing'' and ``distance between two
elements'' makes sense.
It is the goal of the next result to show that we can forgo these
terms.

In what follows, if $X$ is a set, we let $\Pf(X)$ denote the set of
all nonempty finite subsets of $X$. 

\begin{prop}
  \label{prop:syn}
  Let $A \subseteq \bbN$ be infinite.
  The following statements are equivalent.
  \begin{itemize}
    \item[(a)] Enumerate $A$ as a strictly increasing sequence $\la
      x_n \ra_{n=1}^\infty$.
      Then $\max\{\, x_{n+1} - x_n : n \in \bbN \,\}$ is finite.

    \item[(b)] There exists $G \in \Pf(\bbN)$ such that $\bbN =
      \bigcup_{t \in G} -t + A$. 
  \end{itemize}
\end{prop}
\begin{proof}
  \textsl{(a) $\implies$ (b).}
  Let $b = \max\{\, x_{n+1} - x_n : n \in \bbN \,\} \cup \{x_1\}$ and
  put $G = \{1, 2, \ldots, b\}$. 
  Clearly, $\bigcup_{t \in G} -t + A \subseteq \bbN$ so let $x \in
  \bbN$. 
  First, assume that $x < x_1$, then since $x_1 \le b$, we can pick $t
  \in G$ such that $t + x = x_1$, that is $x \in -t + A$.
  Now assume that $x_1 \le x$.
  Since $\la x_n \ra_{n=1}^\infty$ is strictly increasing, there
  exists $n \in \bbN$ such that $x_n \le x < x_{n+1}$.  
  Then $1 \le x_{n+1} - x \le x_{n+1} - x_n \le b$, that is, $x_{n+1}
  - x \in G$.
  Hence $(x_{n+1} - x) + x = x_{n+1} \in A$, and this completes the
  forward direction of the proof.

  \textsl{(b) $\implies$ (a).}
  Enumerate $A$ a strictly increasing sequence $\la x_n
  \ra_{n=1}^\infty$. 
  Pick $b \in \bbN$ such that $G \subseteq \{1, 2, \ldots, b\}$.
  We will show that $\max\{\, x_{n+1} - x_n : n \in \bbN \,\} \le b$. 

  Let $n \in \bbN$.
  Since $\bbN = \bigcup_{t \in G} -t+A$ and $x_n \in \bbN$, pick $t
  \in G$ such that $t + x_n \in A$.
  Hence pick $m \in \bbN$ such that $x_m = t + x_n$, that is, $x_m -
  x_m = t \le b$. 
  Observe that $x_{n+1} \le x_m$.
  Therefore it follows that $x_{n+1} - x_n \le x_m - x_n \le b$, and 
  this completes the proof of the reverse direction.
\end{proof}

\begin{defn}
  Let $(S, \cdot)$ be a semigroup.
  \begin{itemize}
    \item[(a)] For all $x \in S$ and $A \subseteq S$, define $x^{-1}A = \{\, y
      \in S : x \cdot y \in A \,\}$.
    
    \item[(b)] For all $A$, $B \subseteq S$, define $AB = \{\, a\cdot
      b : \mbox{$a \in A$ and $b \in B$} \,\}$. 
      If $A = \{a\}$ or $B = \{b\}$, then we write $\{a\}B$ as $aB$
      and $A\{b\}$ as $Ab$, respectively. 

    \item[(c)] We call $A$ a \textsl{syndetic (set in $S$)} if and
      only if there exist $G \in \Pf(S)$ such that for all $S =
      \bigcup_{t \in G} t^{-1}A$. 
  \end{itemize}
\end{defn}
\begin{rmk}
  It's important to note that in (a) we are not assuming that $x$ has
  an inverse in $S$.
  (In particular, $x^{-1}A$ may even be empty.)
  If for every $x \in S$ we define the function $\lambda_x \colon S
  \to S$ by $\lambda_x(y) = xy$, then $x^{-1}A = \lambda_x^{-1}[A]$. 

  Analogous to group theory, if $S$ is a commutative semigroup we
  write the semigroup operation as $+$. 
  Under this case, we write $x^{-1}A$ as $-x + A$ and $AB$ as $A +
  B$. 
\end{rmk}

Even though we have successfully translated Theorem
\ref{thm:vdwEqSyn}(b) into algebraic terms it's an unfortunate fact
that the notion of syndetic set is \textsl{not} partition regular. 
(See the Introduction in \cite{Bergelson:2001fk} for an example in
$\bbN$.)
In order to recover partition regularity, we weaken the definition of
a syndetic set to obtain a piecewise syndetic set.
 
For a first definition of a piecewise syndetic set we introduce some
temporary terminology and notation.
Given $F \in \Pf(\bbN)$ with $|F| = n$, enumerate $F$ as a strictly
increasing sequence $x_1 < x_2 < \cdots < x_n$. 
We define the \textsl{gap size of $F$} as $g(F) = \max\bigr\{\,
x_{j+1} - x_j : j \in \{1, 2, \ldots, n-1\} \,\bigl\}$.
We call a set $A \subseteq \bbN$ \textsl{piecewise syndetic} if and
only if there exists $b \in \bbN$ such that for all $N \in \bbN$,
there exists $B \in \Pf(A)$ with $|B| \ge N$ and $g(B) \le b$.

It's clear that a syndetic set in $\bbN$ is a piecewise syndetic set
in $\bbN$.
It's also clear that  parts of this definition may not make since
for arbitrarily semigroups.
It's is the purpose of this next result to show that we can dispense
with the potentially troublesome parts.

\begin{prop}
  Let $A \subseteq \bbN$.
  The following statements are equivalent.
  \begin{itemize}
    \item[(a)] $A$ is piecewise syndetic.
    
    \item[(b)] There exists $G \in \Pf(\bbN)$ such that for all $F \in
      \Pf(\bbN)$, there exists $x \in \bbN$ such that $F+x \subseteq
      \bigcup_{t \in G} -t+A$. 
  \end{itemize}
\end{prop}
\begin{proof}
  \textsl{(a) $\implies$ (b).}
  Pick $b \in \bbN$ as guaranteed by the definition. 
  Put $G = \{1, 2, \ldots, b\}$.
  We show that for all $F \in \Pf(\bbN)$, there exists $x \in \bbN$
  such that $F+x \subseteq \bigcup_{t \in G} -t + A$.

  Let $F \in \Pf(\bbN)$ and put $N = 3 |F| \max F$. 
  Pick $B \in \Pf(A)$ with $|B| \le N$ and $g(B) \le b$. 
  Put $n = |B|$ and enumerate $B$ as a strictly increasing sequence
  $x_1 < x_2 < \cdots  < x_N$
  It suffices to show that there exists $x \in \bbN$ such
  that $F + x \subseteq \bigcup_{t \in G} -t + B$.
  (Since $\bigcup_{t \in G} -t + B \subseteq \bigcup_{t \in G} -t +
  A$.)
  Pick $x \in \bbN$ such that $\{1, 2, \ldots, |F| \max F, |F| \max F
  + 1, \ldots, 2|F| \max F\} + x \subseteq \{x_1, x_1 + 1,
  \ldots, x_n\}$.
  (We can do this since $n = |B| \le 3 |F| \max F$.)
  Let $f \in F$, then $f + x \in \{x_1, x_1 + 1, \ldots, x_n
  \}$ and we can pick $m \in \{1, 2, \ldots, n\}$ with $x_m \le f+x < x_{m+1}$. 
  Since $x_{m+1} - x_m \le b$, there exists $t \in G$ such that $t +
  f + x = x_{m+1}$. 
  This completes the forward direction of the proof.  
  
  \textsl{(b) $\implies$ (a).}
  Pick $b \in \bbN$ such that $G \subseteq \{1, 2, \ldots, b\}$. 
  Let $N \in \bbN$ and put $F = \{1, 2, \ldots, 2bN\}$. that $|F| \ge N$. 
  Pick $x \in \bbN$ such that $F + x \subseteq \bigcup_{t \in G} -t +
  A$. 
  For each $i \in \{1, 2, \ldots, 2bN\}$, pick $t_i \in G$ such that
  $t_i + i + x \in A$. 
  Put $B = \bigl\{\, t_i + i + x : f \in \{1, 2, \ldots, 2bN\} \,\bigr\}$. 
  We show that $g(B) \le b$. 
  For $i \in \{1, 2, \ldots, 2bN-1\}$ we have that $t_{i+1} + (i+1) +
  x - (t_i + i + x) = t_{i+1} - t_i + 1 \le b - 1 + 1 = b$.
  This completes the reverse direction of the proof.
\end{proof}

With this equivalence we now can define the notion of a piecewise
syndetic set in a semigroup. 
Moreover, unlike syndetic sets, piecewise syndetic sets are partition
regular.%
\endnote{
  In \cite[Lemma 1]{Brown:1971fk}, proves that if the natural numbers
  are finitely partitioned, then at least one cell of the partition is
  piecewise syndetic.
  It's an interesting question as to whether, Brown's Lemma can be
  extended to prove Proposition \ref{prop:psReg}.
}

\begin{defn}
  Let $(S, \cdot)$ be a semigroup and $A \subseteq S$.
  We say $A$ is \textsl{piecewise syndetic (set in S)} if and only if there
  exists $G \in \Pf(S)$ such that for all $F \in \Pf(S)$ there exists
  $x \in S$ with $Fx \subseteq \bigcup_{t \in G} t^{-1}A$.
\end{defn}

\begin{prop}
  \label{prop:psReg}
  Let $S$ be a semigroup and $A \subseteq S$ piecewise syndetic.
  If $A = C_1 \cup C_2$, then either $C_1$ is piecewise syndetic or
  $C_2$ is piecewise syndetic.
\end{prop}
\begin{proof}
  Pick $G \in \Pf(S)$ such that for all $F \in \Pf(S)$, there exists
  $x \in S$ such that $Fx \subseteq \bigcup_{t \in G} t^{-1}A$. 
  Suppose, to the contrary, that $C_1$ is not piecewise syndetic and
  $C_2$ is not piecewise syndetic. 
  Since $C_1$ is not piecewise syndetic, we can pick $F \in
  \Pf(S)$ such that for all $x \in S$, $Fx \setminus (\bigcup_{t \in
    G} t^{-1}C_1) \ne \emptyset$. 
  Since $C_2$ is not piecewise syndetic, we can pick $H \in \Pf(S)$
  such that for all $x \in S$, $Hx \setminus (\bigcup_{t \in GF}
  t^{-1}C_1) \ne emptyset$.
  Since $FH \in \Pf(S)$ pick $x \in S$ such that 
  \[
    FHx \subseteq \bigcup_{t \in G} t^{-1}(C_1 \cup C_2). 
  \]
  
  Pick $y \in H$ such that $yx \not\in \bigcup_{t \in GF} t^{-1}B$. 
  Now $Fyx \subseteq \bigcup_{t \in G} t^{-1}(C_1 \cup C_2)$,
  $yx \in S$, and $C_1$ is not piecewise syndetic, so we have that 
  \[
    Fyz \setminus (\bigcup_{t \in G} t^{-1}C_1) \ne \emptyset. 
  \]
  Pick $z \in F$ such that $zyx \not\in \bigcup_{t \in G} t^{-1}C_1$. 
  Now $zyx \in \bigcup_{t \in G} t^{-1}(C_1 \cup C_2)$, so pick $t \in
  G$ such that $tzyx \in C_1 \cup C_2$. 
  If $tzyx \in C_1$, then $zyx \in t^{-1}C_1$, a contradiction.
  If $tzyx \in C_2$, then $yx \in (tz)^{-1}C_2$, a contradiction.
  Hence we conclude that either $C_1$ is piecewise syndetic or $C_2$
  is piecewise syndetic.
\end{proof}

There is another statement we can import into a general semigroup, the
Hales-Jewett Theorem.
To state this theorem we require some terminology. 

\begin{defn}
  Let $S$ be the free semigroup on the finite alphabet $A$, and let
  $\star$ denote an element not in $A$.
  \begin{itemize}
    \item[(a)] A \textsl{variable word (on $A$)} $w(\star)$ is a word in the
      free semigroup on the alphabet $A \cup \{\star\}$ in which
      $\star$ occurs.

    \item[(b)] Given a variable word $w(\star)$ on $A$ and $a \in A$,
      we let $w(a)$ denote the word in $S$ where each occurrence of
      $\star$ in $w(\star)$ is replaced by $a$.

    \item[(c)] Given a variable word $w(\star)$ on $A$ we call the set
      $\{\, w(a) : a \in A \,\}$ a \textsl{combinatorial line}.
  \end{itemize}
\end{defn}

\begin{hj}
  Let $A$ be a finite nonempty alphabet, $S$ be a free semigroup on
  $A$, and let $r \in \bbN$.
  If $S = \bigcup_{i=1}^r C_i$, then there exist $i \in \{1, 2,
  \ldots, r\}$ such that $C_i$ contains a combinatorial line. 
\end{hj}

We will define a certain specific type of ``combinatorial line'' type
structure in a general semigroup called a \textsl{$J$-set}.

\begin{defn}
  Let $S$ be a semigroup. 
  \begin{itemize}
    \item[(a)] Define $\calT(S) = \setfunc{\bbN}{S}$.
      If $S$ is clear from context, we simply write $\calT(S)$ as
      $\calT$.

    \item[(b)] For each $m \in \bbN$, define
      \[
        \calJ_m = \{\, (t_1, t_2, \ldots, t_m) \in \bbN^m : t_1 < t_2
        < \cdots < t_m \,\}.
      \]

    \item[(c)] For each $m \in \bbN$, $a \in S^{m+1}$, $t \in
      \calJ_m$, and $f \in \calT$, define
      \[
        x(m, a, t, f) = \biggl(\prod_{i=1}^m \bigl( a(j) f(t_j) \bigr)
        \biggr) a(m+1).
      \]

    \item[(d)] We call $A \subseteq S$ a \textsl{$J$-set}%
      \endnote{
        Despite appearances to the contrary I did \textsl{not} name
        this concept.
        The term $J$-set is first coined in \cite[Definition
        2.4(b)]{Hindman:1996uq} as a $J_Y$-set.
      }
      \textsl{(in S)} if and only if for every $F \in \Pf(\calT)$,
      there exist $m \in \bbN$, $a \in S^{m+1}$, and $t \in \calJ_m$
      such that for all $f  \in F$, $x(m, a, t, f) \in A$. 
  \end{itemize}
\end{defn}

Observe that our definition of a $J$-set is a certain type of
``combinatorial line'' in a general semigroup.
We don't wish to push this analogy too far since we don't completely
understand it yet.
Now we will look at some of the combinatorial results connected with
$J$-sets.

\section{$J$-sets are Partition Regular}
Our definition of a $J$-set is superficially stronger than the
definitions given in \cite[Definition 2.3(d)]{Hindman:2010fk} and
\cite[Definition 3.3(e)]{De:2008uq}. 
To show that our definition and \cite[Definition
2.3(d)]{Hindman:2010fk} are equivalent we will make use of a technical
lemma.

\begin{lem}
  Let $S$ be a semigroup and $A \subseteq S$.
  The following statements are equivalent.
  \begin{itemize}
    \item[(a)] $A$ is a $J$-set.
    
    \item[(b)] For all $F \in \calT$, there exists $m \in \bbN$, $a
      \in S^{m+1}$, and $\bigl( H(1), H(2), \ldots, H(m) \bigr) \in
      \bigr(\Pf(\bbN)\bigl)^m$ such that 
      \begin{itemize}
        \item[(i)] For all $i \in \{1, 2, \ldots, m-1 \}$, $\max H(i) <
        \min H(i+1)$.
       
        \item[(ii)] For all $f \in F$, 
          \[
            \biggl( \prod_{i=1}^m \bigl( a(i) \prod_{t \in H(i)} f(t)
            \bigr) \biggr) a(m+1) \in A.
          \]
      \end{itemize}
  \end{itemize}
\end{lem}
\begin{proof}
  \textsl{(a) $\implies$ (b).} 
  This direction is clear after a moment's thought. 

  \textsl{(b) $\implies$ (a).}
  Let $F \in \calT$ and fix $b \in S$. 
  For all $f \in F$, define $g_f(t) = f(t)b$ and put $G = \{\, g_f : f
  \in F \,\}$.
  Since $G \in \calT$ and (b) is assumed to hold, we can pick $m \in
  \bbN$, $a \in S^{m+1}$, and $\bigl( H(1), H(2), \ldots, H(m) \bigr)
  \in \bigl( \Pf(\bbN) \bigr)^m$ such that $\bigl( H(1), H(2), \ldots,
  H(m) \bigr)$ satisfies condition (i) of (b), and for all $f \in F$,
  \[
    \biggl( \prod_{i=1}^m \bigl( a(i) \prod_{t \in H(i)} g_f(t) \bigr)
    \biggr) a(m+1) \in A.
  \]

  For notation, convenience put $H(0) = \emptyset$ and for each $\ell \in
  \{0, 1, \ldots, m\}$ define $h_\ell = 1 + \sum_{i=0}^\ell |H(i)|$. 
  Put $n = \sum_{i=1}^m |H(i)|$ and enumerate $\bigcup_{i=1}^m H(i)$
  as a strictly increasing sequence $\la t_i \ra_{i=1}^n$. 
  Define $c \in S^{n+1}$ as follows:
  \[
    c(j) = 
    \begin{cases}
      a(1) & \mbox{if $j = 1$,}\\
      b    & \mbox{if $\ell \in \{0, 1, \ldots, m-1\}$ } \\
      & \hspace{2em}\mbox{and $j \in \{h_\ell +1, h_\ell + 2, \ldots,
        h_\ell + h_{\ell +1} - 1 \}$, and}\\
      b a(\ell + 1) &
    \end{cases}
  \]
\end{proof}


\section{Overview}
Using notions from topological dynamics, Furstenberg defined the
concept of a central set of natural numbers and proved a powerful
combinatorial result that has since come to be called the Central Sets
Theorem \cite[Proposition 8.21]{Furstenberg:1981fk}. 
To give a brief demonstration on the versatility of the Central Sets
Theorem we show how we may easily derive van der Waerden's Theorem
from the Central Sets Theorem.
To aid us in our derivation we temporarily assume the following facts:
\begin{itemize}
  \item A central subset of $\bbN$ is a type of ``large'' subset of $\bbN$.
  \item $\bbN$ is a central set.
  \item If $C$ is a central subset of $\bbN$ and $C = C_1 \cup C_2$,
    then either $C_1$ is central or $C_2$ is central.
    For $r \in \bbN$ it follows (by induction on $r$) that if $C$ is a
    central set and $C = \bigcup_{i=1}^r C_i$, then there exists $i
    \in \{1, 2, \ldots, r\}$ such that $C_i$ is central.
  \item The Central Sets Theorem is true.
\end{itemize}
In what follows, if $X$ is a nonempty set, we let $\Pf(X)$ denote the
set of all nonempty finite subsets of $X$.
\begin{cst}
  Let $C \subseteq \bbN$ be central and let $l \in \bbN$.
  For each $i \in \{1, 2, \ldots, l\}$ let $\la y_{i,n}
  \ra_{n=1}^\infty$ be a sequence in $\bbZ$.
  Then there exist sequences $\la a_n \ra_{n=1}^\infty$ in $\bbN$ and
  $\la H_n \ra_{n=1}^\infty$ in $\Pf(\bbN)$ such that 
  \begin{itemize}
    \item[(1)] for every $n \in \bbN$, $\max H_n < \min H_{n+1}$; and
    \item[(2)] for every $G \in \Pf(\bbN)$ and $i \in \{1, 2, \ldots,
      l\}$ we have
      \[
        \sum_{n \in G}\Bigl(a_n + \sum_{t \in H_n} y_{i,t}\Bigr) \in C.
      \]
  \end{itemize}
\end{cst}

\begin{vdw}
  If $r \in \bbN$ and $\bbN = \bigcup_{i=1}^r C_i$, then for each $l
  \in \bbN$ there exist $i \in \{1, 2, \ldots, r\}$ and $a$, $d \in
  \bbN$ such that $\{\, a, a+d, \ldots, a+ld \,\} \subseteq C_i$.
\end{vdw}
\begin{proof}
  By our temporary assumption we may pick $i \in \{1, 2, \ldots, r\}$
  such that $C_i$ is central. 
  Fix a sequence $\la x_n \ra_{n=1}^\infty$ in $\bbN$ and let $l \in
  \bbN$.
  For each $j \in \{0, 1, \ldots, l\}$ defined the sequence $\la
  y_{j,n} \ra_{n=1}^\infty$ in $\bbZ$ by $y_{j,n} = j \cdot x_n$. 
  Pick sequences $\la a_n \ra_{n=1}^\infty$ in $\bbN$ and $\la H_n
  \ra_{n=1}^\infty$ in $\Pf(\bbN)$ as guaranteed by the Central Sets
  Theorem. 
  Put $a = a_1$ and $d = \sum_{t \in H_1} x_t$. 
  Then for $j \in \{0, 1, \ldots, l\}$ we have $a+jd = a_1 +
  j\cdot\sum_{t \in H_1} x_t = a_1 + \sum_{t \in H_1} y_{j,t} \in C_i$.
  This completes the proof.
\end{proof}
\begin{rmk}
  By a suitable choice of our sequence $\la x_n \ra_{n=1}^\infty$ in
  the proof we may strengthen the conclusion of this theorem by
  stating that there are ``many'' $l+1$-term arithmetic progressions
  in $C_i$. 
\end{rmk}

Van der Waerden's proof of his theorem \cite{Van-der-Waerden:1927fk}
uses a complicated, but elementary, combinatorial argument based on
the pigeonhole principle and ``double induction''.%
\endnote{
  I'm not sure of the exact form of van der Waerden's argument since I
  have never read the paper \cite{Van-der-Waerden:1927fk}. 
  This statement is based on my opinion that all known combinatorial
  proofs of van der Waerden's Theorem are not simple, Khinchin's
  opinion on the difficulty of the original proof \cite[expressed in
  Section 1, pages 11--12]{Khinchin:1998fk}, and van der Waerden's
  description on how he found his proof in
  \cite{Van-der-Waerden:1971fk}. 
  The interested reader can find a combinatorial proof of van der
  Waerden's Theorem in Section 2 of \cite{Hindman:1979fk}.
}
Obviously using partition regularity of central sets and the Central
Sets Theorem makes van der Waerden's Theorem easy to derive.
At this point the reader may suspect that the ``true'' difficulties
then must lie in proving these facts. 
However we shall soon see that hardest part will be absorbing all of
the needed definitions. 

\section{Algebraic and Topological Preliminaries}
A \textsl{semigroup} is a pair $(S, \cdot)$ such that $S$ is a
nonempty set and $\cdot$ is an associative binary operation on $S$,
that is, for all $x$, $y$, and $z \in S$, $x \cdot (y \cdot z) = (x
\cdot y) \cdot z$.
As usual in mathematics we will often suppress the explicit mention of
our binary operation $\cdot$ and simply write that $S$ is a semigroup.
Clearly any group is also a semigroup, but the canonical
example of a semigroup which is not a group is the set of
natural numbers under addition.
Another class of examples can be obtained by considering the
multiplicative part of a ring. 

In analogy with the definition of a commutative ring, we define a
\textsl{commutative semigroup} as a semigroup $S$ such that for all
$x$ and $y \in S$, $xy = yx$.
Also mimicking ring theory, we define certain ideals in semigroups.
Given a semigroup $S$ and a $\emptyset \ne I \subseteq S$, we call $I$
a \textsl{left ideal (of S)} if and only if $SI \subseteq I$; we call
$I$ a \textsl{right ideal of (S)} if and only if $IS \subseteq I$;
and, finally, we call $I$ a \textsl{(two-sided) ideal (of S)} if and
only if $I$ is both a left ideal and right ideal.
(The phrases in parentheses will often be omitted if they are clear
from context.)

The study of ideals in semigroups is one of the main differences
between semigroup theory and group theory.
(A semigroup is a group if and only if the only left and right ideals
is the semigroup itself.
\cite[Example 1.28]{Hindman:1998fk} shows that a semigroup with only
one ideal need not be a group.)
Moreover, semigroup theory is different from ring theory too --- as the
following simple example shows that not every semigroup can be
embedded into the multiplicative semigroup of some ring.

\begin{example}
  We call $(S, \cdot)$ a \textsl{right-zero semigroup} if and only if
  for every $x$, $y \in S$, $x \cdot y = y$.
  A simple check shows that if $S$ is nonempty, this definition really
  produces a semigroup.
  The name ``right-zero'' is derived from the fact that, in the
  natural numbers, multiplying on the right by zero yields zero.
  Hence in a right-zero semigroup every element is a ``zero'' when
  multiplied on the right.

  Suppose that $S$ is a right-zero semigroup with $|S| > 2$. 
  We show that $S$ cannot be the multiplicative semigroup of any
  ring. 
  Suppose, to the contrary, that there exists such a ring.
  Then for all $a$, $b$, and $c \in S$, we have that $(b+c) \cdot a =
  b \cdot a + c \cdot a$ by the ring axioms.
  Since $S$ is right-zero, $(b+c) \cdot a = a$, $b \cdot a = a$, and
  $c \cdot a = a$.
  Hence $a = a + a$ and this implies that $a = 0$. 
  Then $|S| > 2$ implies there exists $a$, $a' \in S$ with $a \ne a'$
  and $a = 0 = a'$.
\end{example}

We call a left ideal $L$ of a semigroup \textsl{minimal} if and only
if $L$ doesn't contain any proper left ideals.
We define the concept of a \textsl{minimal right ideal} dually, and, we
call an ideal $I$ \textsl{minimal} if it doesn't contain any proper
ideal. 
In contrast to ring theory, the following result proves that a
semigroup contains at most one
minimal ideal.

\begin{prop}
  Let $S$ be a semigroup and let $K$ be a minimal ideal.
  If $I$ is an ideal, then $K \subseteq I$.
\end{prop}
\begin{proof}
  It suffices to show that $K \cap I$ is an ideal.  
  Then the conclusion will follow from $K \cap I \subseteq I$, $K\cap
  I \subseteq K$, and the minimality of $K$.

  To see that $K \cap I \ne \emptyset$, let $x \in K$ and $y \in I$.
  Then since $K$ and $I$ are ideals, $xy \in K$ and $xy \in I$.
  Now let $x \in K \cap I$ and $y \in S$.
  Since $K$ and $I$ are ideals, $xy \in K$, $xy \in I$, $yx \in K$,
  and $yx \in I$.
\end{proof}

With this result we are justified in making the following important
definition.
\begin{defn}
  Let $S$ be a semigroup.
  If $S$ has a minimal ideal, then we call this ideal the
  \textsl{smallest ideal (of S)} and denote this ideal by
  $K(S)$. \todo{Turn endnotes into hyperlinks.}%
  \endnote{
    The use of $K(S)$ to denote the smallest ideal is derived from
    early semigroup research done by Suschkewitsch in
    \cite{Suschkewitsch:1928kx}. 
    In this paper, Suschkewitsch investigates the structure of finite
    semigroups.
    He proves that a finite semigroup has minimal left ideals; minimal
    right ideals; a smallest ideal which he denotes by $\mathfrak{K}$
    (fraktur $K$)
    and calls the \textsl{kernel}; and that all of these ideals are
    related in nice ways.
    In particular, he proves that the smallest ideal is the union of
    all minimal left ideals, the union of all minimal right ideals,
    and a structure theorem on $K(S)$.

    N.B. 
    I haven't personally read this Suschkewitsch paper.
    This summary is based on \cite[Appendix A]{Clifford:1961fk},
    the brief biography of Suschkewitsch in \cite{Hollings:2009uq},
    and the unpublished ``summary translation'' of \cite{Hollings:2005vn}.
  }
\end{defn}
\begin{rmk}
  Not all semigroups possess a smallest ideal.
  For instance, $(\bbN, +)$ and $(\bbN, \cdot)$ both don't have
  smallest ideals. 
\end{rmk}

Given a semigroup $S$, for each $x \in S$ we define the functions
$\lambda_x \colon S \to S$ and $\rho_x \colon S \to S$ by
$\lambda_x(y) = xy$ and $\rho_x(y) = yx$. 
For $A \subseteq S$ and $x \in S$, we use the special notation
$x^{-1}A = \lambda_x^{-1}[A]$. 
(Hence $x^{-1}A = \{\, y \in S : xy \in A \,\}$.)
Finally, we call an element $e \in S$ an \textsl{idempotent} if and
only if $ee = e$.

We shall not go further into the (purely) algebraic theory of
semigroups; but, the interested reader can consult the article
\cite{Hollings:2007uq} for a brief introduction or the monographs 
\cite{Clifford:1961fk} and \cite{Clifford:1967fk} for more extensive
information. 
For our purposes the semigroup theory contained in \cite[Chapters 1
and 2]{Hindman:1998fk} will suffice. 

We now define how topology can interact with our semigroup operation
in various ways.  




\section{Algebraic Preliminaries}
The concept of a central set has been generalized to semigroups, and
admits a simple definition using the algebraic structure of the
Stone-\v{C}ech compactification of a discrete semigroup.
We chose to give this algebraic technology priority for several
reasons: proving that central sets are partition regular automatically
falls out of our definition, proving the Central Sets Theorem requires
a fairly easy induction argument, and strengthen its conclusion also
becomes easy. 
\begin{defn}
  Let $S$ be a nonempty set, let $* \colon S \times S \to S$ be
  a binary operation on $S$, and let $J \subseteq S$ be a nonempty
  subset of $S$.
  \begin{itemize}
    \item[(a)] We call the pair $(S, *)$ a \textsl{semigroup} if
      and only if for all $x$, $y$, and $z \in S$, $(x * y) *
      z = x * (y * z)$.
    \item[(b)] For each $x \in S$ define the functions $\lambda_x
      \colon S \to S$ and $\rho_x \colon S \to S$ by $\lambda_x(y) = x
      * y$ and $\rho_x(y) = y * x$, respectively.
    \item[(c)] We call an element $x \in S$ \textsl{idempotent} if and
      only if $x*x = x$. 
    \item[(d)] We say $J$ is a \textsl{left ideal (of S)} if and only
      if $S * J \subseteq J$.
      A left ideal is a \textsl{minimal left ideal} if and only if the
      only left ideal it contains is itself.
    \item[(e)] We say $J$ is a \textsl{right ideal (of S)} if and only
      if $J * S \subseteq J$.
      A right ideal is a \textsl{minimal right ideal} if and only if the
      only right ideal it contains is itself.
    \item[(f)] We say $J$ is a \textsl{(two-sided) ideal (of S)} if and only
      if $J$ is both a left ideal and right ideal.
      An ideal is a \textsl{minimal ideal} if and only if the only
      ideal it contains is itself.
  \end{itemize}
\end{defn}
\begin{rmk}
  As usual in mathematics, we will often denote the binary operation
  on our semigroup by $\cdot$ and often we will omit writing it at
  all.
\end{rmk}

The following easy results show that a semigroup contains at most
one minimal ideal, and that a minimal ideal is contained in every ideal.
\begin{prop}
  Let $S$ be a semigroup.
  If $J_1$ and $J_2$ are minimal ideals of $S$, then $J_1 = J_2$.
\end{prop}
\begin{proof}
  We show that $J_1 \cap J_2$ is an ideal of $S$. 
  The conclusion then follows from the minimality of $J_1$ and $J_2$.
  
  To see that $J_1 \cap J_2 \ne \emptyset$, let $x \in J_1$ and
  $y \in J_2$. 
  Since $J_1$ is a right ideal, $xy \in J_1$. 
  Since $J_2$ is a left ideal, $xy \in J_2$. 
  Hence $xy \in J_1 \cap J_2$. 

  To see that $J_1 \cap J_2$ is an ideal, let $x \in J_1 \cap J_2$ and
  $y \in S$. 
  Since $J_1$ and $J_2$ are both left ideals, we have $xy \in J_1$ and
  $xy \in J_2$.
  Similarly since $J_1$ and $J_2$ are both right ideals, we have $yx
  \in J_1$ and $yx \in J_2$. 
  Therefore $J_1 \cap J_2$ is an ideal.

  Finally, since $J_1 \cap J_2 \subseteq J_1$ and $J_1 \cap J_2
  \subseteq J_2$ it follows by minimality of $J_1$ and $J_2$ that $J_1
  = J_1 \cap J_2 = J_2$.
\end{proof}
\begin{prop}
  Let $S$ be a semigroup with the minimal ideal $J$. 
  If $I$ is an ideal of $S$, then $J \subseteq I$.
\end{prop}
\begin{proof}
  Similar to the proof above, $J \cap I$ is an ideal of $S$. 
  By minimality of $J$ we have $J \cap I = J$. 
  Therefore it follows that $J \subseteq I$.
\end{proof}

With these results we are justified in making the following definition. 
\begin{defn}
  Let $(S,\cdot)$ be a semigroup. 
  We call the minimal ideal of $S$ the \textsl{smallest ideal} and
  denote it by $K(S)$.%
  \endnote{
    The notation of $K(S)$ denoting the smallest ideal comes from the
    Russian mathematician Suschkewitsch.
    He called the smallest ideal the \textsl{kernel} of the
    semigroup.
    Suschkewitsch, based off of the description given in
    \cite{Hollings:2009uq}, in \cite{Suschkewitsch:1928kx} proved that
    all finite semigroups have a smallest ideal and using this fact
    was able to produce a structure theorem for finite semigroups. 
  }
\end{defn}
Unfortunately not all semigroups have a smallest ideal or
idempotents. 
However, we will only be concerned with a certain class of topological
semigroups that has both of these objects.

\begin{defn}
  A \textsl{compact right-topological semigroup} is a triple $(S,
  \calT, \cdot)$ such that $(S, \calT)$ is a compact Hausdorff space, $(S, \cdot)$ is
  a semigroup, and for every $x \in S$, $\rho_x \colon S
  \to S$ is continuous. 
\end{defn}

\begin{thm}
  Compact right-topological semigroups contains idempotents and
  the smallest ideal.
\end{thm}
\begin{proof}
  The existence of idempotents and the smallest ideal is proved in \cite[Theorem
  2.5]{Hindman:1998fk} and \cite[Theorem 2.8]{Hindman:1998fk}, respectively.
\end{proof}
The importance of the smallest ideal for us is that with a suitably
constructed semigroup we will be able to define central sets in terms
of idempotents of the smallest ideal.

\begin{defn}
  Let $S$ be a nonempty set.
  A nonempty set $p \subseteq \calP(S)$ is called an
  \textsl{ultrafilter} if and only if $p$ satisfies the following
  four properties: 
  \begin{itemize}
    \item[(1)] If $A$, $B \in p$, then $A \cap B \in p$.
    \item[(2)] If $A \in p$ and $A \subseteq B \subseteq S$, then $B
      \in p$.
    \item[(3)] $\emptyset \not\in p$.
    \item[(4)] For all $A \subseteq S$, either $A \in p$ or $S
      \setminus A \in p$. 
  \end{itemize}
\end{defn}

Given a discrete semigroup $(S,\cdot)$, we let $\beta S$ be the
collection of all ultrafilters on $S$, and identify the principal
ultrafilters with the points of $S$. 
For all $A \subseteq S$, we let $\overline{A} = \{\, p \in \beta S : A
\in p \,\}$. 
Then $\{\, \overline{A} : A \subseteq \,\}$ is a basis of open sets in
$\beta S$.
We can extend the semigroup operation on $S$ to $\beta S$ in the
following manner.

\section{Commutative Central Sets Theorem}
\section{Central Sets Theorem}
\begin{defn}
  Let $S$ be a semigroup and $\calT = \setfunc{\bbN}{S}$.
  \begin{itemize}
    \item[(a)] For each $m \in \bbN$, define
      \[
        \calJ_m = \{\, (t_1, t_2, \ldots, t_m) \in \bbN^m : t_1 < t_2
        < \cdots < t_m \,\}.
      \]

    \item[(b)] For each $m \in \bbN$, $a \in S^{m+1}$, $t \in
      \calJ_m$, and $f \in \calT$, define
      \[
        x(m, a, t, f) = \Bigl(\prod_{i=1}^m\bigl( a(j) f(t_j) \bigr)\Bigr) a(m+1)
      \]

    \item[(c)] We call $A \subseteq S$ a \textsl{$J$-set (in S)} if
      and only if for every $F \in \Pf(\calT)$, there exist $m \in
      \bbN$, $a \in S^{m+1}$, and $t \in \calJ_m$ such that for all $f
      \in F$, $x(m, a, t, f) \in A$.

    \item[(d)] $J(S) = \{\, p \in \beta S : \mbox{for all $A \in p$,
        $A$ is a $J$-set} \,\}$.

    \item[(e)] We call $C \subseteq S$ a \textsl{$C$-set} if and only
      if there exists functions $m \colon \Pf(\calT) \to \bbN$,
      $\alpha \in \bigtimes_{F \in \Pf(\calT)} S^{m(F) + 1}$, and $t
      \in \bigtimes_{F \in \Pf(\calT)} \calJ_m$ such that
      \begin{itemize}
        \item[(1)] if $F$, $G \in \Pf(\calT)$ with $F \subsetneq G$,
          then $t(F)\bigl( m(F) \bigr) < t(G)(1)$; and
        
        \item[(2)] whenever $n \in \bbN$, $G_1$, $G_2$, \ldots, $G_n
          \in \Pf(\calT)$ with $G_1 \subsetneq G_2 \subsetneq \cdots
          \subsetneq G_n$, and $\la f_i \ra_{i=1}^n \in
          \bigtimes_{i=1}^n G_i$, we have
          $\prod_{i=1}^n x(m(G_i), \alpha(G_i), t(G_i), f_i) \in C$. 
      \end{itemize}
  \end{itemize}
\end{defn}
Our definition of a $J$-set is superficially stronger than the
definitions given in \cite[Definition 2.3(d)]{Hindman:2010fk} and
\cite[Definition 3.3(e)]{De:2008uq}. 
To show that our definition and \cite[Definition
2.3(d)]{Hindman:2010fk} are equivalent we will make use of a technical
lemma.
\begin{lem}
  Let $n \in \bbN$, $c \in S^{n+1}$, $H \in \calI_m$, and $f \in
  \calT$. 
  Fix $b \in S$ and define $g \in \calT$ by $g(t) = f(t)b$. 
  Then there exists $m \in \bbN$, $a \in S^{m+1}$, and $t \in \calJ_m$
  such that $x(m, a, t, f) = \bigl(\prod_{i=1}^n( c(i) \prod_{t \in H(i)}
  g(t)\bigr) c(n+1)$
\end{lem}
\begin{proof}
  Put $m = \sum_{i=1}^n |H(i)|$ and enumerate $\bigcup_{i=1}^n H(i)$
  as a strictly increasing sequence $\la t_i \ra_{i=1}^m$. 
  Put $H(0) = \emptyset$ and define $a \in S^{m+1}$ as follows:
  \[
    a(j) =
    \begin{cases}
      c(1) & \mbox{if $j = 1$,} \\
      b    & \mbox{if $s \in \{0, 1, \ldots, n-1\}$ and $j \in \{1 +
        \sum_{i=0}^s |H(i)| + 1,$}\\
      & \hspace{1em}\mbox{$1 + \sum_{i=0}^s |H(i)| + 2, \ldots,
        1 + \sum_{i=0}^s |H(i)| + |H(s+1)| - 1\}$, and} \\
      bc(s+1) & \mbox{if $s \in \{1, 2, \ldots, n\}$ and $j = 1 +
        \sum_{i=0}^s |H(i)|$.} 
    \end{cases}
  \]
  We will show by induction on $n$ that $x(m, a, t, f) =
  \bigl(\prod_{i=1}^n( c(i) \prod_{t \in H(i)} g(t)\bigr) c(n+1)$. 

  (To see how our $a$ was derived consider the example with $n = 3$,
  $H(1) = \{3, 5\}$, $H(2) = \{7\}$, and $H(3) = \{9, 11, 15\}$.
  In this case, $\bigl(\prod_{i=1}^n( c(i) \prod_{t \in H(i)}
  g(t)\bigr) c(n+1) =
  c(1)f(2)bf(5)bc(2)f(7)bc(3)f(9)bf(11)bf(15)bc(4)$.
  Hence $m = 6$ and $a(1) = c(1)$, $a(2) = b$, $a(3) = bc(2)$, $a(4) =
  bc(3)$, $a(5) = b$, $a(6) = b$, and $a(7) = bc(4)$.)

  First suppose $n = 1$, then
  \begin{align*}
    c(1) \prod_{t \in H(1)} g(t) c(2) &= c(1) \prod_{i=1}^m (f(t_i) b)
    c(2). \\
    &\mbox{Induction argument on $|H(1)|$
      too \ldots}
  \end{align*}
\end{proof}
\begin{cor}
  \label{cor:stJsets}
  Let $n \in \bbN$, $c \in S^{n+1}$, $H \in \calI_m$, and $F \in
  \Pf(\calT)$. 
  Fix $b \in S$ and for each $f \in F$ define $g_f \in \calT$ by
  $g_f(t) = f(t)b$. 
  Then there exists $m \in \bbN$, $a \in S^{m+1}$, and $t \in \calJ_m$
  such that for all $f \in F$, $x(m, a, t, f) = \bigl(\prod_{i=1}^n(
  c(i) \prod_{t \in H(i)} g_f(t)\bigr) c(n+1)$
\end{cor}

\begin{prop}
  Let $S$ be a semigroup and let $A \subseteq S$ be a set which
  satisfies \cite[Definition 2.3(d)]{Hindman:2010fk}, then $A$ is a $J$-set.
\end{prop}
\begin{proof}
  Let $F \in \Pf(\calT)$ and fix $b \in S$. 
  For each $f \in F$, define $g_f \in \calT$ by $g_f(t) = f(t)b$. 
  Since $A$ satisfies \cite[Definition 2.3(d)]{Hindman:2010fk}, there
  exist $n \in \bbN$, $c \in S^{n+1}$, and $H \in \calI_m$ such that
  for all $f \in F$, $\bigl(\prod_{i=1}^n( c(i) \prod_{t \in H(i)}
  g_f(t)\bigr) c(n+1) \in A$.
  By Corollary \ref{cor:stJsets}, pick $m \in \bbN$, $a \in S^{m+1}$,
  and $t \in \calJ_m$ such that for all $f \in F$, $x(m, a, t, f) =
  \bigl(\prod_{i=1}^n( c(i) \prod_{t \in H(i)} g_f(t)\bigr) c(n+1)$.
\end{proof}

We next show that our definition of a $J$-set is equivalent to
\cite[Definition 3.3(e)]{De:2008uq}.

\begin{prop}
  Let $S$ be a semigroup and let $A \subseteq S$ be a $J$-set. 
  the for each $F \in \Pf(\calT)$ and $n \in \bbN$, we can pick $m \in
  \bbN$, $a \in S^{m+1}$, and $t \in \calJ_m$ such that $n < t_1$ and
  for each $f \in F$, $x(m, a, t, f) \in A$.
\end{prop}
\begin{proof}
  For each $f \in F$, define $g_f \in \calT$ by $g_f(t) = f(t + n)$. 
  Then pick $m \in \bbN$, $a \in S^{m+1}$, and $s \in \calJ_m$ such
  that for all $f \in F$, $x(m, a, s, g_f) \in A$. 
  Define $t \in \calJ_m$ by $t_i = s_i + n$ for each $i \in \{1, 2,
  \ldots, m\}$. 
  Then $t_1 = s_1 + n > n$, and $x(m, a, t, f) = x(m, a, s, g_f)$. 
\end{proof}

\begin{thm}
  Let $S$ be a semigroup. 
  If $J(S) \ne \emptyset$, then $J(S)$ is a compact two-sided ideal of
  $\beta S$. 
\end{thm}
\begin{proof}
  To show that $J(S)$ is compact it suffices to show that $J(S)$ is
  topologically closed in $\beta S$.
  Let $p \not\in J(S)$, then pick $A \in p$ such that $A$ is not a
  $J$-set. 
  By definition of $J(S)$, we have that $\overline{A} \cap J(S) =
  \emptyset$. 
  Moreover, $\overline{A}$ is a (basic) open neighborhood of $p$. 

  Now let $p \in J(S)$ and $q \in \beta S$. 
  We show that $pq \in J(S)$ and $qp \in J(S)$. 

  Let $F \in \Pf(\calT)$, let $A \in pq$, and let $B = \{\, x \in S :
  x^{-1}A \in q \,\}$. 
  Then $B \in p$. 
  Since $B$ is a $J$-set, pick $m \in \bbN$, $a \in S^{m+1}$, and $t
  \in \calJ_m$ such that for all $f \in F$, $x(m, a, t, f) \in B$. 
  By definition of $B$, this means that for all $f \in F$, $x(m, a, t,
  f)^{-1} A \in q$.
  Hence $\bigcap_{f \in F} x(m, a, t, f)^{-1} A \in q$ and so we may
  pick $b \in \bigcap_{f \in F} x(m, a, t, f)^{-1} A$.
  Define $c \in S^{m+1}$ by
  \[
    c(j) = 
    \begin{cases}
      a(j) & \mbox{if $j \in \{1, 2, \ldots, m\}$}, \\
      a(m+1)b & \mbox{if $j = m+1$}.
    \end{cases}
  \]
  Then for each $f \in F$, $x(m, c, t, f) \in A$. 

  Now let $F \in \Pf(\calT)$, $A \in qp$, and let $B = \{\, x \in S :
  x^{-1}A \in p \,\}$.
  Then $B \in q$ and so we may pick $b \in B$ such that $b^{-1}A \in
  p$. 
  Since $b^{-1}A$ is a $J$-set, pick $m \in \bbN$, $a \in S^{m+1}$,
  and $t \in \calJ_m$ such that for all $f \in F$, $x(m, a, t, f) \in
  b^{-1}A$. 
  Define $c \in S^{m+1}$ by
  \[
    c(j) = 
    \begin{cases}
      ba(1) & \mbox{if $j = 1$}, \\
      a(j) & \mbox{if $j \in \{2, \ldots, m+1\}$}.
    \end{cases}
  \]
  Then $x(m, c, t, f) \in A$ for all $f \in F$.
\end{proof}

In order to show that $J(S)$ is nonempty we now focus on showing that
$J$-sets are partition regular. 

\begin{thm}
  Let $S$ be a semigroup, $A \subseteq S$ a $J$-set, and $A = A_1 \cup
  A_2$. 
  Then either $A_1$ is a $J$-set or $A_2$ is a $J$-set. 
\end{thm}
\begin{proof}
  Let $F \in \Pf(\calT)$, $k = |F|$, and enumerate $F = \{ f_1, f_2,
  \ldots, f_k \}$. 
  By the Hales-Jewett Theorem, pick $n \in \bbN$, such that whenever
  $[k]^n$ is 2-colored there exists a monochromatic combinatorial
  line.
  
  Fix $b \in S$ and for each $w = (x_1, x_2, \ldots, x_n) \in [k]^n$
  define $g_w \in \calT$ by $g_w(t) = \prod_{i=1}^n
  \bigr(f_{x_i}(nt+i)b\bigr)$.
  Since $\{\, g_w : w \in [k]^n \,\} \in \Pf(\calT)$ and $A$ is a
  $J$-set, pick $m \in \bbN$, $a \in S^{m+1}$, and $t \in \calJ_m$
  such that for all $w \in [k]^n$, $x(m, a, t, g_w) \in A$.
  Define $\varphi \colon [k]^n \to \{1, 2\}$ by $\varphi(w) = 1 $ if
  $x(m, a, t, g_w) \in A_1$, and $\varphi(w) = 2$ otherwise.
  Pick a variable word $w(\star)$ that begins and ends with a constant
  such that $\bigl\{\, w(s) : s \in
  \{1, 2, \ldots, k\} \,\bigr\}$ is a monochromatic combinatorial line
  with respect to $\varphi$.
  Without loss of generality we suppose that $\varphi\bigl(w(s)\bigr)
  = 1$ for all $s \in \{1, 2, \ldots, k\}$. 

  Let $w(\star) = (x_1, x_2, \ldots, x_n)$ where each $x_i \in \{1, 2,
  \ldots, \} \cup \{\star\}$ and some $x_i = \star$. 
  Let $r$ be the number of variable blocks in $w(\star)$, more
  formally, let $r = \bigm|\bigl\{\, (i, i+1) \in \{1, 2, \ldots, n-1\} :
  x_i = \star, x_{i+1} \in \{1, 2, \ldots, k\} \,\bigr\}\bigm|$.
  Pick $L \in \calI_{r+1}$ and $M \in \calI_r$ such that for each $i
  \in \{1, 2, \ldots, r\}$ , we have $\max L(i) < \min M(i)$, $\max
  M(i) < \min L(i+1)$, and 
  \begin{align*}
    \bigcup_{i=1}^{r+1} L(i) &= \bigl\{\, i \in \{1, 2, \ldots, n \} :
    x_i \in \{1, 2, \ldots, k\} \,\bigr\}, \\
    \bigcup_{i=1}^r M(i) &= \bigl\{\, i \in \{1, 2, \ldots, n \} : x_i
    = \star \,\bigr\}.
  \end{align*}
  (For example, if $k = 3$, $n = 8$, and $w(\star) = (2, \star, \star,
  3, 1, \star, 1, 2)$, then $r = 2$, $L(1) = \{1\}$, $M(1) = \{2,
  3\}$, $L(2) = \{4, 5\}$, $M(2) = \{6\}$, and $L(3) = \{7, 8\}$.)
  
  Looking at the expression of $x(m, a, t, g_w(s))$ it's intuitively
  clear that there exists $p \in \bbN$, $c \in S^{p+1}$, and $u \in
  \calJ_p$ such that $x(m, a, t, g_{w(s)}) = x(p, c, u, f_s)$ for all
  $u \in \{1, 2, \ldots, k\}$. 
  If $A_1$ was not a $J$-set, then  we can go back and select $F$ such
  that $A_1$ fails the definitions of a $J$-set with this $F$. 
  However, then would get a contradiction. 
\end{proof}

\section{Open Questions and Problems}

% Notes section produced by the 'endnotes' package
\theendnotes

% Things referenced in the introduction. Eventually this will placed
% in a separate file so the References appear at the end.
\bibliographystyle{amsplain}
\bibliography{../references}
\end{document}

